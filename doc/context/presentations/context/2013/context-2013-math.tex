% \enablemode[print]

\usemodule[present-stepwise,present-tiles,abr-02]

\definecolor[maincolor] [darkgray]
\definecolor[othercolor][b=.3]

\setupinteractionscreen
  [option=max]

% \setupalign
%   [flushleft,tolerant]

\defineframed
  [conclusion]
  [location=low,
   width=max,
   align=flushleft,
   background=color,
   backgroundcolor=white,
   foregroundcolor=othercolor]

\setupbodyfont[15pt]

\startdocument
  [title={Math:\\\\progress or standing still},
  %subtitle={Hans Hagen\\TUG Conference\\October 2013}]
   subtitle={Hans Hagen\\\ConTeXt\ Meeting\\September 2013}]

\StartSteps \starttopic [title={Math as script}]

    \startitem math can be input using the \TEX\ syntax, \MATHML, calculator like sequences, \unknown \FlushStep \stopitem
    \startitem but apart from content \MATHML\ all stay close to good old \TEX \FlushStep \stopitem
    \startitem although not officially a script, \OPENTYPE\ treats it as such, but without control \FlushStep \stopitem

    \blank[2*big]

    \starttyping
    $ ( (x + 1) / a + 1 )^2  = (x - 1) / b $
    \stoptyping

    \FlushStep

    \starttyping
    $ \left( \frac{x + 1}{a}  + 1 \right)^2  = \frac{x - 1}{b} $
    \stoptyping

    \FlushStep

    \starttyping
    <mfenced open="(" close = ")">
      <mfrac>...</mfrac> <mo>+</mo> <mn>1</mn>
    </mfenced>
    \stoptyping

    \FlushStep

    \starttyping
    <mrow>
      <mo>(</mo> <mfrac>...</mfrac> <mo>+</mo> <mn>1</mn> <mo>)</mo>
    </mrow>
    \stoptyping

    \FlushStep

    \vfilll \conclusion{There is recognition of math as a proper (but not standardized) script.} \FlushStep

\stoptopic \StopSteps

\StartSteps \starttopic [title={Alphabets}]

    \startitem the shape (style) of a character determines its meaning \FlushStep \stopitem
    \startitem but in most cases an type {a} is entered as \ASCII\ character \FlushStep \stopitem
    \startitem and tagged with some rendering directive, often indicating a font style \FlushStep \stopitem
    \startitem in traditional \TEX\ we have alphabets in different fonts, so we're talking switches \FlushStep \stopitem
    \startitem in \UNICODE\ and \OPENTYPE\ we have alphabets with standardized code points (but gaps too) \FlushStep \stopitem
    \startitem this has big advantages for communicating, transferring data etc \FlushStep \stopitem
    \startitem but a math engine still has to deal with \ASCII\ input as well \FlushStep \stopitem
    \startitem multiple axis: types, alphabets, styles, variants, shapes, modifiers \FlushStep \stopitem

    \vfilll \conclusion{We're off better but the gaps are an anomality.} \FlushStep

\stoptopic \StopSteps

\StartSteps \starttopic [title={Heavy bold}]

    \startitem for titles and captions we might need bolder math \FlushStep \stopitem
    \startitem bold symbols in math have special meaning \FlushStep \stopitem
    \startitem so when going full bold they should become heavy \FlushStep \stopitem
    \startitem heavy math involves boldening everything, including extensibles \FlushStep \stopitem
    \startitem there are currently no fonts that have such complete heavy companions \FlushStep \stopitem

    \vfilll \conclusion{We need proper bold fonts, but they need to be relatively complete.} \FlushStep

\stoptopic \StopSteps

\StartSteps \starttopic [title={Radicals}]

    \startitem this always has been (and still is) a combination of vertical extensibles and horizontal rules \FlushStep \stopitem
    \startitem it is the only two dimensional extensible so always a bit of an exception \FlushStep \stopitem
    \startitem in the wide engines we now have more direct support primitive for that (no macro needed) \FlushStep \stopitem
    \startitem in practice (at least in \MKIV) we still use macros because we want control \FlushStep \stopitem

    \vfilll \conclusion{Native support for radicals is nice to have and makes coding cleaner.} \FlushStep

\stoptopic \StopSteps

\StartSteps \starttopic [title={Primes}]

    \startitem this is a special case as we (sort of) have upto two superscripts  \FlushStep \stopitem
    \startitem and also need to handle an optional subscript of the base symbol \FlushStep \stopitem
    \startitem and in order to be visually okay, we need to collect multiple primes \FlushStep \stopitem
    \startitem some fonts have primes raised, some have them flying high \FlushStep \stopitem
    \startitem maybe at some point the upcoming math pre- and postscripts will help \FlushStep \stopitem

    \vfilll \conclusion{Supporting primes will always be a bit of a pain but I stay on top of it.} \FlushStep

\stoptopic \StopSteps

\StartSteps \starttopic [title={Accents}]

    \startitem they can go on top or below one or more characters (also in combination) \FlushStep \stopitem
    \startitem accents have some hard codes positional properties \FlushStep \stopitem
    \startitem the wide engines have more direct support for this \FlushStep \stopitem
    \startitem fonts provide a limited set of sizes, such accents cannot extend (by design) \FlushStep \stopitem

    \vfilll \conclusion{Engine support for accents is better now but maybe fonts need to have more sizes.} \FlushStep

\stoptopic \StopSteps

\StartSteps \starttopic [title={Stackers}]

    \startitem arrows (and other horizontal extensibles) traditionally were made from snippets \FlushStep \stopitem
    \startitem we need them also for chemistry, in rather flexible ways \FlushStep \stopitem
    \startitem in upcoming math fonts they are become real extensibles \FlushStep \stopitem
    \startitem but then we still need to deal with existing fonts that lack them (one font in the end) \FlushStep \stopitem
    \startitem there will be native support for so called character leaders \FlushStep \stopitem

    \vfilll \conclusion{Stackers are more easily implemented although fonts pose some challenges.} \FlushStep

\stoptopic \StopSteps

\StartSteps \starttopic [title={Fences}]

    \startitem these go left and right (or in the middle) of things \FlushStep \stopitem
    \startitem there need to be a matching pair else we get an error \FlushStep \stopitem
    \startitem they have to adapt their size to what they wrap \FlushStep \stopitem
    \startitem \TEX ies can take care of that in their input \FlushStep \stopitem
    \startitem but in for instance \MATHML\ checking all this is a bit of a pain \FlushStep \stopitem
    \startitem this is still the domain of macros \FlushStep \stopitem
    \startitem but we could make the engines a bit more tolerant (hard to do) \FlushStep \stopitem

    \vfilll \conclusion{Matching fences will always be a bit of a problem.} \FlushStep

\stoptopic \StopSteps

\StartSteps \starttopic [title={Directions}]

    \startitem bidirectional math is mostly a matter of the availability of fonts \FlushStep \stopitem
    \startitem there need to be some agreement (at the macro package level) of control \FlushStep \stopitem
    \startitem it's (for me) a visually interesting challenge \FlushStep \stopitem
    \startitem there are some \TEX ies working on these matters (quite some research is done already) \FlushStep \stopitem

    \vfilll \conclusion{Right to left math will show up thanks to pioneers.} \FlushStep

\stoptopic \StopSteps

\StartSteps \starttopic [title={Structure}]

    \startitem demand for tagging also means that we need to carry a bit more info around \FlushStep \stopitem
    \startitem this puts a little more burden on the user \FlushStep \stopitem
    \startitem in the end it largely is a macro package issue \FlushStep \stopitem
    \startitem better tagging of input can also help rendering \FlushStep \stopitem
    \startitem detailed control at the \TEX\ level makes that users can spoil the game \FlushStep \stopitem

    \vfilll \conclusion{In these times structure gets more important so minimal coding is less an option.} \FlushStep

\stoptopic \StopSteps

\StartSteps \starttopic [title={Italic correction}]

    \startitem in traditional \TEX\ fonts this was used for spacing as well as special purposed \FlushStep \stopitem
    \startitem across fonts there was never much correction \FlushStep \stopitem
    \startitem \OPENTYPE\ doesn't have this concept \FlushStep \stopitem
    \startitem \OPENTYPE\ math has some of if but also more powerful kerning \FlushStep \stopitem
    \startitem generally speaking: we can ignore italic corrections \FlushStep \stopitem

    \vfilll \conclusion{We need to accept that old concepts die and new onces show up.} \FlushStep

\stoptopic \StopSteps

\StartSteps \starttopic [title={Big}]

    \startitem normally extensible fences are chosen automatically \FlushStep \stopitem
    \startitem but macro packages provide tricks to choose a size \FlushStep \stopitem
    \startitem extensible steps are unpredictable but still several mechanisms can be provided \FlushStep \stopitem

    \vfilll \conclusion{Users will always want control and no engine can provide that but macros can.} \FlushStep

\stoptopic \StopSteps

\StartSteps \starttopic [title={Macros}]

    \startitem some special symbols were constructed by macros (and using special font properties) \FlushStep \stopitem
    \startitem these are mostly gone (the diagonal dots) \FlushStep \stopitem
    \startitem if it is ever needed again, we should extend the fonts \FlushStep \stopitem

    \vfilll \conclusion{Thanks to new font technologies and wide engines need less dirty tricks.} \FlushStep

\stoptopic \StopSteps

\StartSteps \starttopic [title={Unscripting}]

    \startitem you can bet on those funny \UNICODE\ super and subscripts showing up in input \FlushStep \stopitem
    \startitem it's a somewhat limited and unuseable lot for math (a modifier would have made more sense) \FlushStep \stopitem
    \startitem it's one of these legacies that we need to deal with \FlushStep \stopitem
    \startitem so the macro package needs to intercept them and map them onto proper math \FlushStep \stopitem

    \vfilll \conclusion{We always need to deal with weird input, if only because standards lack.} \FlushStep

\stoptopic \StopSteps

\StartSteps \starttopic [title={Combining fonts}]

    \startitem we can expect math fonts to be rather complete and if not, one should choose another one \FlushStep \stopitem
    \startitem but sometimes (for simple math) you want to swap in alphabets and digits that match the text font \FlushStep \stopitem
    \startitem given that we talk of ranges this is easy to support at the macro package level \FlushStep \stopitem

    \vfilll \conclusion{Although fonts are more complete, occasional combinations should remain possible.} \FlushStep

\stoptopic \StopSteps

\StartSteps \starttopic [title={Tracing}]

    \startitem there are lots of symbols involved \FlushStep \stopitem
    \startitem and we have those extensibles too \FlushStep \stopitem
    \startitem the larger the fonts get the more checking we need to do \FlushStep \stopitem
    \startitem so macro packages need to provide some tracing options (or tables in print) \FlushStep \stopitem

    \vfilll \conclusion{We keep an eye on things.} \FlushStep

\stoptopic \StopSteps

\stopdocument
