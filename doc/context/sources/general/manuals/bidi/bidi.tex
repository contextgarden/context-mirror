% language=uk

% \showglyphs

\dontcomplain

\startbuffer[preamble-fonts]
\definefontfallback
  [Serif] [scheherazaderegular*arabic sa 1.5]
  [arabic] [force=yes]
\definefontfallback
  [SerifBold] [scheherazadebold*arabic sa 1.5]
  [arabic] [force=yes]
\definefontfallback
  [SerifItalic] [scheherazaderegular*arabic sa 1.5]
  [arabic] [force=yes]
\definefontfallback
  [SerifBoldItalic] [scheherazadebold*arabic sa 1.5]
  [arabic] [force=yes]

\definefontfallback
  [Serif] [sileot*hebrew sa 1.0]
  [hebrew] [force=yes]
\definefontfallback
  [SerifBold] [sileot*hebrew sa 1.0]
  [hebrew] [force=yes]
\definefontfallback
  [SerifItalic] [sileot*hebrew sa 1.0]
  [hebrew] [force=yes]
\definefontfallback
  [SerifBoldItalic] [sileot*hebrew sa 1.0]
  [hebrew] [force=yes]

\definefontfeature[fakemono][mode=node,fakemono=yes]

% \definefontfallback
%   [Mono] [scheherazaderegular*fakemono sa 1.5]
%   [arabic] [force=yes,factor=1] % factor forces a monospace

\definefontfallback
  [Mono] [sileot*fakemono sa 1.0]
  [hebrew] [force=yes,factor=1] % factor forces a monospace

\setupbodyfont
  [dejavu,10pt]
\stopbuffer

\startbuffer[preamble-languages]
\setuplanguage[ar][font=arabic,bidi=right]
\setuplanguage[he][font=hebrew,bidi=right]
\stopbuffer

\getbuffer[preamble-fonts]
\getbuffer[preamble-languages]

\setuplayout
  [backspace=15mm,
   topspace=15mm,
   footer=0pt,
   width=middle,
   height=middle]

\setuptyping
  [color=middleblue]

\setuptype
  [color=middleblue]

\definecolor
  [maincolor]
  [middleblue]

\setupwhitespace
  [big]

%%%%%%%%%%%%%%%%%%%%%%%%%%%%%%%%%%%%%%%%%%%%%%%%%%%%%%%%%%%%%%%%%%%%%%%%%%%%%%%%%%%%%%%%%%%%%%%%%

\startluacode
    local report = logs.reporter("directions","check")
    local line   = 0
    function nodes.tracers.checkdirections(head)
        line = line + 1
        report("line: %i",line)
        for n in nodes.traverse_id(nodes.nodecodes.dir,head) do
            report("  %s (%i,%i)",n.dir,n.subtype,n.direction)
        end
        return head, false
    end

    nodes.tasks.appendaction("contributers","after","nodes.tracers.checkdirections")
    nodes.tasks.disableaction("contributers","nodes.tracers.checkdirections")
\stopluacode

\installtextracker
   {directions.check}
   {\ctxlua{nodes.tasks.enableaction("contributers","nodes.tracers.checkdirections")}}
   {\ctxlua{nodes.tasks.disableaction("contributers","nodes.tracers.checkdirections")}}

%%%%%%%%%%%%%%%%%%%%%%%%%%%%%%%%%%%%%%%%%%%%%%%%%%%%%%%%%%%%%%%%%%%%%%%%%%%%%%%%%%%%%%%%%%%%%%%%%

\starttext

\startMPpage

    picture p, q, r, s ;

    p := textext("l2r") xsized .9PaperWidth ;
    q := textext("r2l") xsized .9PaperWidth ;
    r := textext("a few tips") xsized .9PaperWidth ;
    s := textext("hans\quad\quad hagen") xsized .5bbheight(p);

    p := p shifted - llcorner p ;
    q := q shifted - llcorner q ;
    r := r shifted - llcorner r ;
    s := s shifted - llcorner s ;

    fill Page withcolor "darkyellow" ;

    p := p shifted (.05PaperWidth,ypart .5[ulcorner Page, urcorner Page]-1.1bbheight(p)) ;
    q := q shifted (.05PaperWidth,ypart .5[ulcorner Page, urcorner Page]-1.1bbheight(p)-1.15bbheight(q)) ;
    r := r shifted (.05PaperWidth,ypart .5[llcorner Page, lrcorner Page]+0.3bbheight(r)) ;
    s := s shifted (.66PaperWidth,ypart .5[llcorner Page, lrcorner Page]+1.5bbheight(s)) ;

    draw p withcolor "lightgray" ;
    draw q withcolor "lightgray" ;
    draw r withcolor "middleblue" ;
    draw s withcolor "middleblue" ;

\stopMPpage

\startchapter[title=Introduction]

With \CONTEXT\ you can typeset in two directions: from left to right and from
right to left. In fact you can also combine these two directions, like this:

\startbuffer
There are many {\righttoleft \maincolor \bf scripts in use} and some run into the
other direction. However, there is {\righttoleft \maincolor \bf no fixed relation
{\lefttoright \black \tf between the} direction of the script} and cars being
driven left or right of the road.
\stopbuffer

\typebuffer

% \enabletrackers[directions.check]

\getbuffer

% \disabletrackers[directions.check]

This manual is written by a left to right user so don't expect a manual on
semitic typesetting (Hebrew and Arabic). Also don't expect a (yet) complete
manual. I'll add whatever comes to mind. So let's see how Arabic comes out:

\startbuffer
The sentence \quotation {I have no clue what this means.} is translated (by
Google Translate) into \quotation {\ar \righttoleft ليس لدي أي فكرة عما يعنيه هذا.}
which is then translated back to \quotation {I have no idea what this means.} so
maybe arabic has no clue what a clue is. The suggested Arabic pronunciation is
\quotation {\ar lays laday 'ayu fikrat eamaa yaenih hadha}. Hebrew also likes ideas
more: \quotation {\hr \righttoleft אין לי מושג מה זה אומר}.
\stopbuffer

\typebuffer

\getbuffer

The \CONTEXT\ (or any \TEX) ecosystem deals with languages and fonts. Languages
(that relate to scripts) have specific characteristics, like running from right
to left, and fonts provide a repertoire of glyphs and features. There is no real
(standard) relationship between these. In for instance browsers, there are
automatic fallback systems for missing characters in a font: another font is
taken. These fallbacks are often not easy to tweak.

In this document we use Dejavu and although that font has Arabic shapes in its
monospace variant, the serifs come without them (at least when I write this
down). Before we actually define the bodyfont we hook in some fallbacks. The
typescript for Dejavu has lines like this:

\starttyping
\definefontsynonym
  [SerifBoldItalic]
  [name:dejavuserifbolditalic]
  [features=default,
   fallbacks=SerifBoldItalic]
\stoptyping

This permits us to do this:

\typebuffer[preamble-fonts]

In addition we set up the languages:

\typebuffer[preamble-languages]

The following example demonstrates what the effects of these commands are:

\startbuffer
{ليس لدي أي فكرة عما يعنيه هذا.}
{אין לי מושג מה זה אומר.}
{\righttoleft ليس لدي أي فكرة عما يعنيه هذا.}
{\righttoleft אין לי מושג מה זה אומר.}
{\ar \righttoleft ليس لدي أي فكرة عما يعنيه هذا.}
{\he \righttoleft אין לי מושג מה זה אומר.}
{\ar ليس لدي أي فكرة عما يعنيه هذا.}
{\he אין לי מושג מה זה אומר.}
\stopbuffer

\typebuffer

\startlines
\getbuffer
\stoplines

In principle you can also rely on automatic direction changes, for instance
by using the following command:

\starttyping
\setupdirections
  [bidi=global,
   method=three]
\stoptyping

But that doesn't do a font switch for you, nor does it do any of the other
language related settings. It really helps if you properly tag your
document content, as in:

\starttyping
{\ar ليس لدي أي فكرة عما يعنيه هذا.}
{\he אין לי מושג מה זה אומר.}
\stoptyping

One reason to set the \type {font} parameter for a language is that it will
activate the right features in a font. Instead of falling back on some default,
we can be very specific in what we want to enable.

\stopchapter

\stoptext
