% language=uk

\startcomponent cld-specialcommands

\environment cld-environment

\startchapter[title=Special commands]

\index{tracing}

There are a few functions in the \type {context} namespace that are no
macros at the \TEX\ end.

\starttyping
context.runfile("somefile.cld")
\stoptyping

Another useful command is:

\starttyping
context.settracing(true)
\stoptyping

There are a few tracing options that you can set at the \TEX\ end:

\starttyping
\enabletrackers[context.files]
\enabletrackers[context.trace]
\stoptyping

A few macros have special functions at the \LUA\ end. One of them is \type
{\char}. The function makes sure that the characters ends up right. The same is
true for \type {\chardef}. So, you don't need to mess around with \type {\relax}
or trailing spaces as you would do at the \TEX\ end in order to tell the scanner
to stop looking ahead.

\starttyping
context.char(123)
\stoptyping

Other examples of macros that have optimized functions are \type {\par},
\type{\bgroup} and \type {\egroup}.

\stopchapter

\stopcomponent
