% language=uk

\environment details-environment

\startcomponent details-tuningformulas

\startchapter[title={Tuning math formulas}]

Because of its look and feel, a math formula can look too widely spaced when put
on a grid. There are a few ways to control this. First of all, the default grid
option bound to math is already more tolerant. But you can control it locally
too. Take the following formula:

\startbuffer
\startformula
  a = b^c
\stopformula
\stopbuffer

\blank \fakeline \getbuffer \fakeline \blank

This has been entered as:

\typebuffer

and because it is just a line of math it comes out as expected. The next code

\startbuffer
\startformula
  a = \frac {a} {b}
\stopformula
\stopbuffer

\typebuffer

produces a higher line:

\blank \fakeline \getbuffer \fakeline \blank

as does:

\startbuffer
\startformula
  a = \frac {\frac {b} {c}} {\frac {d} {e}}
\stopformula
\stopbuffer

\typebuffer

\blank \fakeline \getbuffer \fakeline \blank

We will now demonstrate three ways to compensate fo rexcessive spacing. The first
variant just sets a grid parameter:

\startbuffer
\startformula[grid=math:-halfline]
    a = \frac {\frac {b} {c}} {\frac {d} {e}}
\stopformula
\stopbuffer

\typebuffer

\blank \fakeline \getbuffer \fakeline \blank

You can also pass this as an option. Only a few such grid related options are
defined: \type {halfline}, \type {line}, \type {-halfline} and \type {-grid}.

\startbuffer
\startformula[-halfline]
    a = \frac {\frac {b} {c}} {\frac {d} {e}}
\stopformula
\stopbuffer

\typebuffer

\blank \fakeline \getbuffer \fakeline \blank

If you need to compensate frequently you can consider defining an instance:

\startbuffer
\defineformula[tight][grid=math:-halfline]

\starttightformula
    a = \frac {\frac {b} {c}} {\frac {d} {e}}
\stoptightformula
\stopbuffer

\typebuffer

\blank \fakeline \getbuffer \fakeline \blank

The result can be somewhat unexpected at the top and bottom of a page. When we
subtract half a line from the height we can end up above the text area. This is
where the \type {split} directive comes in. So, the compensations are actually
defined as

\starttabulate[|TCT{blue}|T|]
\NC math            \NC \theexpandedsnapperset{math} \NC \NR
\NC math:line       \NC \theexpandedsnapperset{math:line} \NC \NR
\NC math:halfline   \NC \theexpandedsnapperset{math:halfline} \NC \NR
\NC math:-line      \NC \theexpandedsnapperset{math:-line} \NC \NR
\NC math:-halfline  \NC \theexpandedsnapperset{math:-halfline} \NC \NR
\stoptabulate

You can define your own variants building on top of an existing one:

\starttyping
\definegridsnapping[math:my][math,....]
\stoptyping

We demonstrate the effect of the \type {split} directive here. It triggers a
check at the page boundaries but you need to keep in mind that this is not always
robust as such boundaries themselves can be triggered by and inject anything.

\startbuffer[a]
\dorecurse {15} {
    \startformula[grid={math,-halfline}]
        a = \frac {\frac {b} {c}} {\frac {d} {e}}
        (\hbox{top #1 default})
    \stopformula
    \blank[samepage]
    \fakeline
}
\stopbuffer

\startbuffer[b]
\dorecurse {15} {
    \startformula[grid={math,-halfline,split}]
        a = \frac {\frac {b} {c}} {\frac {d} {e}}
        (\hbox{top #1 compensated})
    \stopformula
    \blank[samepage]
    \fakeline
}
\stopbuffer

\getbuffer[a]
\getbuffer[b]

As said, the compensation is achieved with the \type {page} directive. The
previous pages were rendered using:

\typebuffer[a]

and

\typebuffer[b]

In order to get a consistent result we keep the depth of the formula the same but
effectively shift it down a bit, still honouring the grid. So what about the
bottom.

We can decide that the snapped formula doesn't fit and force a new page but we
can also accept that it sticks out to the bottom, which is less worse than the
top|-|of|-|the|-|page case.

\startbuffer[a]
\dorecurse{45}{\fakeline}
\startformula[grid={math,-halfline}]
    a = \frac {\frac {b} {c}} {\frac {d} {e}}
    (\hbox{bottom default})
\stopformula
\stopbuffer

\startbuffer[b]
\dorecurse{45}{\fakeline}
\startformula[grid={math,-halfline,split}]
    a = \frac {\frac {b} {c}} {\frac {d} {e}}
    (\hbox{bottom compensated})
\stopformula
\stopbuffer

\page \getbuffer[a] % fits on the page
\page \getbuffer[b] % forces a new page

These mechanisms might be improved over time but as we don't use it frequently
that might take a while.

The following formula was posted at the \CONTEXT\ mailing list in a grid snapping
thread and we will use it to demonstrate how you can mess a bit with the
snapping.

\startbuffer
g(x_{*}) = \lim_{n\to\infty} g(a_{n}) \leq 0 \leq \lim_{n\to\infty} g(b_{n}) = g(x_{*})
\stopbuffer

\typebuffer

We show the given grid parameter as well as its expansion into the low level grid
directives.

\unexpanded\def\SampleFormula#1%
  {\definegridsnapping[math:temp][#1]
   \blank
   \type{grid=#1} \hfill expanded: \normalexpanded{\type{\theexpandedsnapperset{math:temp}}}
   \blank[samepage]
   \fakeline
   \blank[samepage]
   \startformula[grid={#1}]
     \getbuffer
   \stopformula
   \blank[samepage]
   \fakeline
   \blank}

\SampleFormula{math}
\SampleFormula{low,halfline}
\SampleFormula{math,nodepth}

\stopchapter

\stopcomponent
