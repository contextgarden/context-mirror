% language=us

\environment lowlevel-style

\startdocument
  [title=paragraphs,
   color=middlecyan]

\startsection[title=Introduction]

This manual is mostly discussing a few wrappers around low level \TEX\ features.
Its writing is triggered by an update to the \METAFUN\ and \LUAMETAFUN\ manuals
where we mess a bit with shapes. It gave a good reason to also cover some more
paragraph related topics but it might take a while to complete. Remind me if you
feel that takes too much time.

\stopsection

\startsection[title=Properties]

A paragraph is just a collection of lines that result from one input line that
got broken. This process of breaking into lines is influenced by quite some
parameters. In traditional \TEX\ and also in \LUAMETATEX\ by default the values
that are in effect when the end of the paragraph is met are used. So, when you
change them in a group and then ends the paragraph after the group, the values
you've set in the group are not used.

However, in \LUAMETATEX\ we can optionally store them with the paragraph. When
that happens the values current at the start are frozen. You can still overload
them but that has to be done explicitly then. The advantage is that grouping no
longer interferes with the line break algorithm. The magic primitive is \type
{\snapshotpar}.

\starttabulate
\NC \type {\hsize}                 \NC \NC \NR
\NC \type {\leftskip}              \NC \NC \NR
\NC \type {\rightskip}             \NC \NC \NR
\NC \type {\hangindent}            \NC \NC \NR
\NC \type {\hangafter}             \NC \NC \NR
\NC \type {\parindent}             \NC \NC \NR
\NC \type {\parfillleftskip}       \NC \NC \NR
\NC \type {\parfillrightskip}      \NC \NC \NR
\NC \type {\adjustspacing}         \NC \NC \NR
\NC \type {\protrudechars}         \NC \NC \NR
\NC \type {\pretolerance}          \NC \NC \NR
\NC \type {\tolerance}             \NC \NC \NR
\NC \type {\emergencystretch}      \NC \NC \NR
\NC \type {\looseness}             \NC \NC \NR
\NC \type {\lastlinefit}           \NC \NC \NR
\NC \type {\linepenalty}           \NC \NC \NR
\NC \type {\interlinepenalty}      \NC \NC \NR
\NC \type {\clubpenalty}           \NC \NC \NR
\NC \type {\widowpenalty}          \NC \NC \NR
\NC \type {\displaywidowpenalty}   \NC \NC \NR
\NC \type {\brokenpenalty}         \NC \NC \NR
\NC \type {\adjdemerits}           \NC \NC \NR
\NC \type {\doublehyphendemerits}  \NC \NC \NR
\NC \type {\finalhyphendemerits}   \NC \NC \NR
\NC \type {\parshape}              \NC \NC \NR
\NC \type {\interlinepenalties}    \NC \NC \NR
\NC \type {\clubpenalties}         \NC \NC \NR
\NC \type {\widowpenalties}        \NC \NC \NR
\NC \type {\displaywidowpenalties} \NC \NC \NR
\NC \type {\baselineskip}          \NC \NC \NR
\NC \type {\lineskip}              \NC \NC \NR
\NC \type {\lineskiplimit}         \NC \NC \NR
\NC \type {\adjustspacingstep}     \NC \NC \NR
\NC \type {\adjustspacingshrink}   \NC \NC \NR
\NC \type {\adjustspacingstretch}  \NC \NC \NR
\NC \type {\hyphenationmode}       \NC \NC \NR
\stoptabulate

There are more paragraph related parameters than in for instance \PDFTEX\ and
\LUATEX\ and these are (to be) explained in the \LUAMETATEX\ manual. You can
imagine that keeping this around with the paragraph adds some extra overhead to
the machinery but most users won't notice that because is is compensated by gains
elsewhere.

In \LMTX\ taking these snapshots is turned on by default and because it thereby
fundamentally influences the par builder, users can run into compatibility issues
but in practice there has been no complaints (and this feature has been in use
quite a while before this document was written). One reason for users not
noticing is that one of the big benefits is probably handled by tricks mentioned on the
mailing list. Imagine that you have this:

\starttyping[option=TEX]
{\bf watch out:} here is some text
\stoptyping

In this small example the result will be as expected. But what if something magic
with the start of a paragraph is done? Like this:

\starttyping[option=TEX]
\placefigure[left]{A cow!}{\externalfigure[cow.pdf]}

{\bf watch out:} here is some text ... of course much more is needed to
    get a flow around the figure!
\stoptyping

The figure will hang at the left side of the paragraph but it is put there when
the text starts and that happens inside the bold group. It means that the
properties we set in order to get the shape around the figure are lost as soon as
we're at \quote{\type {here is some text}} and definitely is wrong when the
paragraph ends and the par builder has to use them to get the shape right. We get
text overlapping the figure. A trick to overcome this is:

\starttyping[option=TEX]
\dontleavehmode {\bf watch out:} here is some text ... of course much
    more is needed to get a flow around the figure!
\stoptyping

where the first macro makes sure we already start a paragraph before the group is
entered (using a \type {\strut} also works). It's not nice and I bet users have
been bitten by this and by now know the tricks. But, with snapshots such fuzzy
hacks are not needed any more! The same is true with this:

\starttyping[option=TEX]
{\leftskip 1em some text \par}
\stoptyping

where we had to explicitly end the paragraph inside the group in order to retain
the skip. I suppose that users normally use the high level environments so they
never had to worry about this. It's also why users probably won't notice that
this new mechanism has been active for a while. Actually, when you now change a
parameter inside the paragraph will not be applied (unless you prefix it with
\type {\frozen}) but no one did that anyway.

{\em todo: freeze categories, overloading, turning on and off, etc}

\stopsection

\startsection[title=Wraping up]

In \CONTEXT\ \LMTX\ we have a mechanism to exercise macros (or content) before a
paragraph ends. This is implemented using the \type {\wrapuppar} primitive. The
to be wrapped up material is bound to the current paragraph which in order to
get this done has to be started when this primitive is used.

Although the high level interface has been around for a while it still needs a
bit more testing (read: use cases are needed). In the few cases where we already
use it application can be different because again it relates to snapshots. This
because in the past we had to use tricks that also influenced the user interface
of some macros (which made them less natural as one would expect). So the
question is: where do we apply it in old mechanisms and where not.

{\em todo: accumulation, interference, where applied, limitations}

% \vbox   {vbox    : \wrapuppar{1}test\par x\wrapuppar{2}test}\blank
% \vtop   {vtop    : \wrapuppar{1}test\par x\wrapuppar{2}test}\blank
% \vcenter{vcenter : \wrapuppar{1}test\par x\wrapuppar{2}test}\blank
% $$x = \vcenter{vcenter : \wrapuppar{1}test\par x\wrapuppar{2}test}$$\blank
% x\vadjust{vadjust : \wrapuppar{1}test\par x\wrapuppar{2}test}x\blank

\stopsection

\startsection[title=Shapes]

In \CONTEXT\ we don't use \type {\parshape} a lot. It is used in for instance
side floats but even then not in all cases. It's more meant for special
applications. This means that in \MKII\ and \MKIV\ we don't have some high level
interface. However, when \METAFUN\ got upgraded to \LUAMETAFUN, and the manual
also needed an update, one of the examples in that manual that used shapes also
got done differently (read: nicer). And that triggered the arrival of a new high
level shape mechanism.

One important property of the \type {\parshape} mechanism is that it works per
paragraph. You define a shape in terms of a left margin and width of a line. The
shape has a fixed number of such pairs and when there is more content, the last one
is used for the rest of the lines. When the paragraph is finished, the shape is
forgotten.

{\em Not discussed here is a variant that will end up in \LUAMETATEX\ that works
with the progression, i.e.\ takes the height of the content so far into account.
This is somewhat tricky because for that to work vertical skips need to be
frozen, which is no real big deal but has to be done careful in the code.}

The high level interface is a follow up on the example in the \METAFUN\ manual and
uses shapes that carry over to the next paragraph. In addition we can cycle over
a shape. In this interface shapes are defined using keyword. Here are some
examples:

\starttyping[option=TEX]
\startparagraphshape[test]
    left 1mm right 1mm
    left 5mm right 5mm
\stopparagraphshape
\stoptyping

This shape has only two entries so the first line will have a 1mm margin while
later lines will get 5mm margins. This translates into a \type {\parshape} like:

\starttyping[option=TEX]
\parshape 2
    1mm \dimexpr\hsize-1mm\relax
    5mm \dimexpr\hsize-5mm\relax
\stoptyping

Watch the number \type {2}: it tells how many specification lines follow. As you
see, we need to calculate the width.

\starttyping[option=TEX]
\startparagraphshape[test]
    left 1mm right 1mm
    left 5mm right 5mm
    repeat
\stopparagraphshape
\stoptyping

This variant will alternate between 1mm and 5mm margins. The repeating feature is
translated as follows. Maybe at some point I will introduce a few more options.

\starttyping[option=TEX]
\parshape 2 options 1
    1mm \dimexpr\hsize-1mm\relax
    5mm \dimexpr\hsize-5mm\relax
\stoptyping

A shape can have some repetition, and we can save keystrokes by copying the last
entry. The resulting \type {\parshape} becomes rather long.

\starttyping[option=TEX]
\startparagraphshape[test]
    left 1mm right 1mm
    left 2mm right 2mm
    left 3mm right 3mm
    copy 8
    left 4mm right 4mm
    left 5mm right 5mm
    left 5mm hsize 10cm
\stopparagraphshape
\stoptyping

Also watch the \type {hsize} keyword: we don't calculate the hsize from the \type
{left} and \type {right} values but explicitly set it.

\starttyping[option=TEX]
\startparagraphshape[test]
    left 1mm right 1mm
    right 3mm
    left 5mm right 5mm
    repeat
\stopparagraphshape
\stoptyping

When a \type {right} keywords comes first the \type {left} is assumed to be zero.
In the examples that follow we will use a couple of definitions:

\startbuffer[setup]
\startparagraphshape[test]
    both 1mm both 2mm both 3mm both 4mm both 5mm both 6mm
    both 7mm both 6mm both 5mm both 4mm both 3mm both 2mm
\stopparagraphshape
\stopbuffer

\startbuffer[setup-repeat]
\startparagraphshape[test-repeat]
    both 1mm both 2mm both 3mm both 4mm both 5mm both 6mm
    both 7mm both 6mm both 5mm both 4mm both 3mm both 2mm
    repeat
\stopparagraphshape
\stopbuffer

\typebuffer[setup,setup-repeat][option=TEX]

The last one could also be defines as:

\starttyping[option=TEX]
\startparagraphshape[test-repeat]
    \rawparagraphshape{test} repeat
\stopparagraphshape
\stoptyping

In the previous code we already introduced the \type {repeat} option. This will
make the shape repeat at the engine level when the shape runs out of specified
lines. In the application of a shape definition we can specify a \type {method}
to be used and that determine if the next paragraph will start where we left off
and discard afterwards (\type {shift}) or that we move the discarded lines up
front so that we never run out of lines (\type {cycle}). It sounds complicated
but just keep in mind that \type {repeat} is part of the \type {\parshape} and
act within a paragraph while \type {shift} and \type {cycle} are applied when a
new paragraph is started.

\startbuffer[demo]
\startshapedparagraph[list=test]
    \dorecurse{8}{\showparagraphshape\samplefile{tufte} \par}
\stopshapedparagraph
\stopbuffer

\startbuffer[demo-repeat]
\startshapedparagraph[list=test-repeat]
    \dorecurse{8}{\showparagraphshape\samplefile{tufte} \par}
\stopshapedparagraph
\stopbuffer

In \in {figure} [fig:shape:discard] you see the following applied:

\typebuffer[demo,demo-repeat][option=TEX]

\startplacefigure[title=Discarded shaping,reference=fig:shape:discard]
\startcombination[nx=2,ny=2]
    {\typesetbuffer[setup,demo][page=1,width=.4\textwidth,frame=on]}               {discard, finite shape,    page 1}
    {\typesetbuffer[setup,demo][page=2,width=.4\textwidth,frame=on]}               {discard, finite shape,    page 2}
    {\typesetbuffer[setup-repeat,demo-repeat][page=1,width=.4\textwidth,frame=on]} {discard, repeat in shape, page 1}
    {\typesetbuffer[setup-repeat,demo-repeat][page=2,width=.4\textwidth,frame=on]} {discard, repeat in shape, page 2}
\stopcombination
\stopplacefigure

In \in {figure} [fig:shape:shift] we use this instead:

\startbuffer[demo]
\startshapedparagraph[list=test,method=shift]
    \dorecurse{8}{\showparagraphshape\samplefile{tufte} \par}
\stopshapedparagraph
\stopbuffer

\startbuffer[demo-shift]
\startshapedparagraph[list=test-repeat,method=shift]
    \dorecurse{8}{\showparagraphshape\samplefile{tufte} \par}
\stopshapedparagraph
\stopbuffer

\typebuffer[demo,demo-repeat][option=TEX]

\startplacefigure[title=Shifted shaping,,reference=fig:shape:shift]
\startcombination[nx=2,ny=2]
    {\typesetbuffer[setup,demo][page=1,width=.4\textwidth,frame=on]}              {shift, finite shape,    page 1}
    {\typesetbuffer[setup,demo][page=2,width=.4\textwidth,frame=on]}              {shift, finite shape,    page 2}
    {\typesetbuffer[setup-repeat,demo-shift][page=1,width=.4\textwidth,frame=on]} {shift, repeat in shape, page 1}
    {\typesetbuffer[setup-repeat,demo-shift][page=2,width=.4\textwidth,frame=on]} {shift, repeat in shape, page 2}
\stopcombination
\stopplacefigure

Finally, in \in {figure} [fig:shape:cycle] we use:

\startbuffer[demo]
\startshapedparagraph[list=test,method=cycle]
    \dorecurse{8}{\showparagraphshape\samplefile{tufte} \par}
\stopshapedparagraph
\stopbuffer

\startbuffer[demo-cycle]
\startshapedparagraph[list=test-repeat,method=cycle]
    \dorecurse{8}{\showparagraphshape\samplefile{tufte} \par}
\stopshapedparagraph
\stopbuffer

\typebuffer[demo,demo-repeat][option=TEX]

\startplacefigure[title=Cycled shaping,reference=fig:shape:cycle]
\startcombination[nx=2,ny=2]
    {\typesetbuffer[setup,demo][page=1,width=.4\textwidth,frame=on]}              {cycle, finite shape,    page 1}
    {\typesetbuffer[setup,demo][page=2,width=.4\textwidth,frame=on]}              {cycle, finite shape,    page 2}
    {\typesetbuffer[setup-repeat,demo-cycle][page=1,width=.4\textwidth,frame=on]} {cycle, repeat in shape, page 1}
    {\typesetbuffer[setup-repeat,demo-cycle][page=2,width=.4\textwidth,frame=on]} {cycle, repeat in shape, page 2}
\stopcombination
\stopplacefigure

These examples are probably too small to see the details but you can run them
yourself or zoom in on the details. In the margin we show the values used. Here
is a simple example of (non) poetry. There are other environments that can be
used instead but this makes a good example anyway.

\startbuffer
\startparagraphshape[test]
    left 0em right 0em
    left 1em right 0em
    repeat
\stopparagraphshape

\startshapedparagraph[list=test,method=cycle]
    verse line 1.1\crlf verse line 2.1\crlf
    verse line 3.1\crlf verse line 4.1\par
    verse line 1.2\crlf verse line 2.2\crlf
    verse line 3.2\crlf verse line 4.2\crlf
    verse line 5.2\crlf verse line 6.2\par
\stopshapedparagraph
\stopbuffer

\typebuffer[option=TEX]

\getbuffer

{\em todo: move the new (still in {\em \type {meta-imp-txt.mkxl})} code into the
core and integrate it in {\em \type {\startshapedparagraph}} as method {\em \type
{mp}} in which case the list is a list of graphics.}

\starttyping[option=TEX]
\startshapedparagraph[list={test 1,test 2,test 3,test 4},method=mp]
    .....
\stopshapedparagraph
\stoptyping

{\em So methods then become kind of plugins.}

A mechanism like this is often never completely automatic in the sense that you
need to keep an eye on the results. Depending on user demands more features can
be added. With weird shapes you might want to set up the alignment to be \type
{tolerant} and have some \type {stretch}.

\stopsection

% \startsection[title=Linebreaks]
\startsection[title=Modes]

% \ruledvbox{1\ifhmode\writestatus{!}{HMODE 1}\fi}                               % hsize
% \ruledvbox{\hbox{\strut 2}\ifhmode\writestatus{!}{HMODE 2}\fi}                 % fit
% \ruledvbox{\hbox{\strut 3}\hbox{\strut 3}\ifhmode\writestatus{!}{HMODE 3}\fi}  % fit
% \ruledvbox{\hbox{\strut 4}4\ifhmode\writestatus{!}{HMODE 4}\fi}                % hsize
% \ruledvbox{\hbox{\strut 5}5\hbox{\strut 5}\ifhmode\writestatus{!}{HMODE 5}\fi} % hsize
% \ruledvbox{6\hbox{\strut 6}\ifhmode\writestatus{!}{HMODE 6}\fi}                % hsize

{\em todo: some of the side effects of so called modes}

\stopsection

\startsection[title=Normalization]

{\em todo: users don't need to bother about this but it might be interesting anyway}

\stopsection

\stopdocument

