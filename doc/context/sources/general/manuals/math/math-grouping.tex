% language=us runpath=texruns:manuals/math

\environment math-layout

\startbuffer[oldcolor]
    \disableexperiments[simplegroups]%
    % when we no longer have it as experiment
    \unprotect \pushoverloadmode
    \permanent\protected\def\color[#1]%
      {\bgroup % \beginsimplegroup
       \edef\currentcolorname{#1}%
       \ifempty\currentcolorprefix
         \colo_helpers_activate_nop
       \else
         \colo_helpers_activate_yes
       \fi
       \let\nexttoken}%
    \popoverloadmode \protect
\stopbuffer

\startbuffer[newcolor]
    \enableexperiments[simplegroups]%
\stopbuffer

\startcomponent math-grouping

\startchapter[title=Grouping]

\startsection[title=Some details]

In \TEX\ there are all kind of groups. When you start with a curly brace, you
often enter a group but when you start a box or table cell you also do that. When
you enter math a math group is started. Assignments are, unless explicitly done
global, nearly always local to the group. Here we discuss the following two
cases:

\starttyping
{ .... }
\bgroup .... \egroup
\begingroup .... \endgroup
\stoptyping

We say two cases, not three, because the first two are equivalent: the two macros
in the second line are not primitives but aliases to the curly braces. There is
however one fundamental difference between them. The verbose \type {\begingroup}
starts a so called simple group so let's call the other complex. A complex group
is bounded by equivalents to the two characters (braces) that have catcodes that
begin and end these groups. So, the following is valid.

\starttyping
{ .... }
{ .... \egroup
\bgroup .... }
\bgroup .... \egroup
\stoptyping

This means that a macro like this is okay:

\starttyping
\def\foo{\bgroup\bf\let\next}  \foo{text}
\stoptyping

The \type {\foo} will start a complex group, switch font and then pick up the
brace. The group will be closed by the matching right brace or an equivalent.
The following two cases are invalid:

\starttyping
\bgroup .... \endgroup
\begingroup .... \egroup
\stoptyping

which means that:

\starttyping
\def\foo{\begingroup\bf\let\next}  \foo{text}
\stoptyping

will trigger an error message. This is rather unfortunate because using braces
to wrap an argument in curly braces is rather convenient. The way out is this:

\starttyping
\def\foo#1{\begingroup\bf#1\endgroup}  \foo{text}
\stoptyping

which is perfectly valid apart from the fact that the argument is first picked up
and then fed back into the input. Apart from a small performance hit and using
more memory it also adds noise to tracing.

\stopsection

\startsection[title=Side effects]

In math mode matters are complicates by the fact that complex groups (the ones
started with the curly brace) start a math list. And that has side effects because
the spacing between math elements depends on what we deal with: math symbol, lists,
fenced material. The following example shows a whole lot of this:

\startbuffer[example]
\starttabulate[|j1|j1||]
    \NC $\sin{\left(xxx\right)}$
    \NC $f   {\left(xxx\right)}$
    \NC $x   {\left(xxx\right)}$
    \NR
    \NC $\sin\begingroup\left(xxx\right)\endgroup$
    \NC $f   \begingroup\left(xxx\right)\endgroup$
    \NC $x   \begingroup\left(xxx\right)\endgroup$
    \NR
    \NC $\sin\left(xxx\right)$\par
    \NC $f   \left(xxx\right)$\par
    \NC $x   \left(xxx\right)$\par
    \NR
    \NC $\sin\color[darkgreen]{(xxx)}$\par
    \NC $f   \color[darkgreen]{(xxx)}$\par
    \NC $x   \color[darkgreen]{(xxx)}$\par
    \NR
    \NC $\sin\startcolor[darkblue](xxx)\stopcolor$\par
    \NC $f   \startcolor[darkblue](xxx)\stopcolor$\par
    \NC $x   \startcolor[darkblue](xxx)\stopcolor$\par
    \NR
    \NC $\sin(xxx)$\par
    \NC $f   (xxx)$\par
    \NC $x   (xxx)$\par
    \NR
    \NC $\sin{\color[darkyellow]{(xxx)}}$\par
    \NC $f   {\color[darkyellow]{(xxx)}}$\par
    \NC $x   {\color[darkyellow]{(xxx)}}$\par
    \NR
    \NC $\sin\begingroup\color[darkred]{(xxx)}\endgroup$\par
    \NC $f   \begingroup\color[darkred]{(xxx)}\endgroup$\par
    \NC $x   \begingroup\color[darkred]{(xxx)}\endgroup$\par
    \NR
\stoptabulate
\stopbuffer

\typebuffer[example]

A valid question is why we would want to add curly braces. One reason is that we
we want to apply something to a few characters, in this case color the argument
to the sinus. Now, when a color command is defined as the \type {\foo} before, we
start a complex group and that influences spacing! This is demonstrated in the
left part of \in {figure} [fig:grouping:sin] (you can zoom in on the table to
see the details; we also report the kind of spacing applied).

\startplacefigure[reference=fig:grouping:sin,title=Grouping influencing math spacing (list).]
    \startcombination[nx=2,ny=1,distance=2em,style=bold]
        \startcontent
            \showmakeup[math,fontkern,italic]% laps into the margin hence the -2em
            \getbuffer[oldcolor]%
            \scale[width=\dimexpr\measure{combination}-2em\relax]{\getbuffer[example]}%
        \stopcontent
        \startcaption
            pseudo grouping
        \stopcaption
        \startcontent
            \showmakeup[math,fontkern,italic]% laps into the margin hence the -2em
            \getbuffer[newcolor]%
            \scale[width=\dimexpr\measure{combination}-1em\relax]{\getbuffer[example]}%
        \stopcontent
        \startcaption
            grouping
        \stopcaption
    \stopcombination
\stopplacefigure

The right variant in that figure uses a different way of grouping, one that is
equivalent to:

\starttyping
\def\foo{\beginsimplegroup\bf\let\next}
\stoptyping

This time we effectively do \type {\begingroup} but permits both \type
{\endgroup} or a curly brace (or \type {\egroup} to wrap up the group. That means
that we don't get the side effect of starting a math list.

\startbuffer[example]
$ x = y \quad x \color[red]{=} y$
\stopbuffer

This example shows the effect of coloring a single character (the result is shown
in \in {figure} [fig:grouping:char]):

\typebuffer[example]

\startplacefigure[reference=fig:grouping:char,title=Grouping influencing math spacing (symbol).]
    \vkern4ex
    \startcombination[nx=2,ny=1,distance=4em,style=bold]
        \startcontent
            \showmakeup[math,fontkern,italic]% laps into the margin hence the -2em
            \getbuffer[oldcolor]%
            \scale[width=\dimexpr\measure{combination}-2em\relax]{\getbuffer[example]}%
        \stopcontent
        \startcaption
            pseudo grouping
        \stopcaption
        \startcontent
            \showmakeup[math,fontkern,italic]% laps into the margin hence the -2em
            \getbuffer[newcolor]%
            \scale[width=\dimexpr\measure{combination}-2em\relax]{\getbuffer[example]}%
        \stopcontent
        \startcaption
            grouping
        \stopcaption
    \stopcombination
    \vkern2ex
\stopplacefigure

\stopsection

\stopchapter

\stopcomponent
