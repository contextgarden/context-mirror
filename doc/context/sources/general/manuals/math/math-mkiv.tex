% language=uk

\setupbodyfont
  [pagella]

\setupwhitespace
  [big]

\setuppagenumbering
  [alternative=doublesided]

\setuplayout
  [backspace=2cm,
   width=middle,
   cutspace=2cm,
   topspace=2cm,
   header=2cm,
   height=middle,
   footer=0pt,
   bottomspace=2cm]

\usemodule[x][setups-basics]

\usemodule[visual]

\loadsetups[i-context]

\starttext

\startMPpage

StartPage;
    fill Page withcolor .35magenta ;
    picture p ; p := image ( draw outlinetext.b
        ("\textdollar")
        (withcolor .35green)
        (withcolor .5yellow withpen pencircle scaled .1) ;
    ) ;
    p := p xysized(PaperWidth - 20pt,PaperHeight - 20pt ) ;
    p := p shifted - llcorner p shifted (10pt,10pt) ;
    draw p ;
    picture p ; p := image ( draw outlinetext.b
        ("math")
        (withcolor .35green)
        (withcolor .5yellow withpen pencircle scaled .2) ;
    ) ;
    p := p xsized(PaperWidth/2 - 40pt) ;
    p := p shifted - llcorner p shifted (10pt,10pt) ;
    draw p shifted (PaperWidth/2 + 20pt,20pt);
StopPage;

\stopMPpage

\page[empty] \setuppagenumber[number=1]

\setuplayout
  [backspace=2cm,
   topspace=2cm,
   header=2cm,
   height=middle,
   width=middle]

\setuphead
  [chapter]
  [header=high,
   style=\bfc,
   color=darkmagenta]

\setuphead
  [section]
  [style=\bfb]

\setuphead
  [subsection]
  [style=\bfa]

\starttitle[title=Introduction]

This manual is not a systematic discussion about math in \CONTEXT\ but more a
collection of wrap|-|ups. The file also serves as testcase. The content can
change over time and can also serve as a trigger for discussions on the mailing
list. Feel free to contribute.

\startlines
Hans Hagen
Hasselt NL
% May 2016
\stoplines

\stoptitle

\startchapter[title=Vertical spacing]

The low level way to input inline math in \TEX\ is

\starttyping
$ e = mc^2 $
\stoptyping

while display math can be entered like:

\starttyping
$$ e = mc^2 $$
\stoptyping

The inline method is still valid, but for display math the \type {$$} method
should not be used. This has to do with the fact that we want to control spacing
in a consistent way. In \CONTEXT\ the vertical spacing model is rather stable
although in \MKIV\ the implementation is quite different. It has always been a
challenge to let this mechanism work well with space round display formulas. This
has to do with the fact that (in the kind of documents that we have to produce)
interaction with already present spacing is somewhat tricky.

Of course much can be achieved in \TEX\ but in \CONTEXT\ we need to have control
over the many mechanisms that can interact. Given the way \TEX\ handles space
around display math there is no real robust solution possible that gives visually
consistent space in all cases so that is why we basically disable the existing
spacing model. Disabling is easier in \LUATEX\ and recent versions of \MKIV\ have
been adapted to that.

In pure \TEX\ what happens is this:

\startbuffer
$$ x $$
\stopbuffer

\typebuffer \par \start \showboxes \getbuffer \par \stop

A horizontal box get added which triggers a baselineskip. Then the formula is put
below it. We can get rid of that box with \type {\noindent}:

\startbuffer
\noindent $$ x $$
\stopbuffer

\typebuffer \par \start \showboxes \getbuffer \par \stop

In addition (not shown here) vertical space is added before and after the formula
and left- and rightskip on the edges. In fact typesetting display math goes like this:

\startitemize[packed]
    \startitem
        typeset the formula using display mode and wrap it in a box
    \stopitem
    \startitem
        add an equation number, if possible in the same line, otherwise on a line
        below
    \stopitem
    \startitem
        in the process center the formula using the available display width and
        required display indentation
    \stopitem
    \startitem
        add vertical space above and below (depending also in displays being
        short in relation to the previous line
    \stopitem
    \startitem
        at the same time also add penalties that determine the break across
        pages
    \stopitem
\stopitemize

Apart from the spacing around the formula and the equation number, typesetting is
not different from:

\starttyping
\hbox {$ \displaystyle x $}
\stoptyping

This spacing around math is a sensitive issue. Because math itself can have a
narrow band, for instance a lone $x$, or relative much depth, as with $y$, or
both depth and height as in $(1,2)$ and $x^2 + y_2$ and because a preceding line
can have no or little depth and a following line little height, the visual
appearance can become inconsistent. The default approach is to force consistent
spacing, but when needed we can implement variants.

Spacing around display math is set up with \type {\setupformulas}:

\starttyping
  \setupformulas
    [spacebefore=big,
     spaceafter=big]
\stoptyping

When the whitespace is larger that setting wins because as usual the larger
of blanks or whitespace wins.

% \showdefinition[setupformula]
% \showdefinition[setupmathematics]

In \in {figures} [whitespace-no], \in {figures} [whitespace-medium] \in {and}
[whitespace-big] we see how things interact. We show lines with and without
maximum line height and depth (enforced by struts) alongside.

% no whitespace

\startbuffer[demo-1]
\disablemode[with-struts]
\showmakeup
\setupformulas[spacebefore=none,spaceafter=none]
\setupwhitespace[none]
\input math-spacing-001.tex
\stopbuffer

\startbuffer[demo-2]
\enablemode[with-struts]
\showmakeup
\setupformulas[spacebefore=none,spaceafter=none]
\setupwhitespace[none]
\input math-spacing-001.tex
\stopbuffer

\startbuffer[demo-3]
\disablemode[with-struts]
\showmakeup
\setupformulas[spacebefore=medium,spaceafter=medium]
\setupwhitespace[none]
\input math-spacing-001.tex
\stopbuffer

\startbuffer[demo-4]
\enablemode[with-struts]
\showmakeup
\setupformulas[spacebefore=medium,spaceafter=medium]
\setupwhitespace[none]
\input math-spacing-001.tex
\stopbuffer

\startplacefigure[location=page,reference=whitespace-no,title={No whitespace.}]
    \startcombination[2*2]
        {\typesetbuffer[demo-1][height=.45\textheight]} {\tttf natural + none   + ws none}
        {\typesetbuffer[demo-2][height=.45\textheight]} {\tttf strut   + none   + ws none}
        {\typesetbuffer[demo-3][height=.45\textheight]} {\tttf natural + medium + ws none}
        {\typesetbuffer[demo-4][height=.45\textheight]} {\tttf strut   + medium + ws none}
    \stopcombination
\stopplacefigure

% whitespace medium same as medium spacing around math

\startbuffer[demo-1]
\disablemode[with-struts]
\showmakeup
\setupformulas[spacebefore=none,spaceafter=none]
\setupwhitespace[medium]
\input math-spacing-001.tex
\stopbuffer

\startbuffer[demo-2]
\enablemode[with-struts]
\showmakeup
\setupformulas[spacebefore=none,spaceafter=none]
\setupwhitespace[medium]
\input math-spacing-001.tex
\stopbuffer

\startbuffer[demo-3]
\disablemode[with-struts]
\showmakeup
\setupformulas[spacebefore=medium,spaceafter=medium]
\setupwhitespace[medium]
\input math-spacing-001.tex
\stopbuffer

\startbuffer[demo-4]
\enablemode[with-struts]
\showmakeup
\setupformulas[spacebefore=medium,spaceafter=medium]
\setupwhitespace[medium]
\input math-spacing-001.tex
\stopbuffer

\startplacefigure[location=page,reference=whitespace-medium,title={Whitespace the same as display spacing.}]
    \startcombination[2*2]
        {\typesetbuffer[demo-1][height=.45\textheight]} {\tttf natural + none   + ws medium}
        {\typesetbuffer[demo-2][height=.45\textheight]} {\tttf strut   + none   + ws medium}
        {\typesetbuffer[demo-3][height=.45\textheight]} {\tttf natural + medium + ws medium}
        {\typesetbuffer[demo-4][height=.45\textheight]} {\tttf strut   + medium + ws medium}
    \stopcombination
\stopplacefigure

% whitespace big wins from medium spacing around math

\startbuffer[demo-1]
\disablemode[with-struts]
\showmakeup
\setupformulas[spacebefore=none,spaceafter=none]
\setupwhitespace[big]
\input math-spacing-001.tex
\stopbuffer

\startbuffer[demo-2]
\enablemode[with-struts]
\showmakeup
\setupformulas[spacebefore=none,spaceafter=none]
\setupwhitespace[big]
\input math-spacing-001.tex
\stopbuffer

\startbuffer[demo-3]
\disablemode[with-struts]
\showmakeup
\setupformulas[spacebefore=medium,spaceafter=medium]
\setupwhitespace[big]
\input math-spacing-001.tex
\stopbuffer

\startbuffer[demo-4]
\enablemode[with-struts]
\showmakeup
\setupformulas[spacebefore=medium,spaceafter=medium]
\setupwhitespace[big]
\input math-spacing-001.tex
\stopbuffer

\startplacefigure[location=page,reference=whitespace-big,title={Whitespace larger than display spacing.}]
    \startcombination[2*2]
        {\typesetbuffer[demo-1][height=.45\textheight]} {\tttf natural + none   + ws big}
        {\typesetbuffer[demo-2][height=.45\textheight]} {\tttf strut   + none   + ws big}
        {\typesetbuffer[demo-3][height=.45\textheight]} {\tttf natural + medium + ws big}
        {\typesetbuffer[demo-4][height=.45\textheight]} {\tttf strut   + medium + ws big}
    \stopcombination
\stopplacefigure

Because we want to have control over the placement of the formula number but also
want to be able to align the formula with the left or right edge of the text
area, we don't use the native display handler by default. We still have a way to
force this, but this is only for testing purposes. By default a formula is placed
centered relative to the current text, including left and right margins.

\startbuffer
\fakewords{20}{40}

\startitemize
    \startitem
        \fakewords{20}{40}
        \placeformula
            \startformula
                \fakeformula
            \stopformula
    \stopitem
    \startitem
        \fakewords{20}{40}
    \stopitem
\stopitemize

\fakewords{20}{40}\epar
\stopbuffer

\typebuffer

\start \getbuffer \stop

In the next examples we explicitly align formulas to the left (\type
{flushleft}), center (\type {middle}) and right (\type {flushright}):

\startbuffer[demo]
\setupformulas[align=flushleft]
\startformula\fakeformula\stopformula
\setupformulas[align=middle]
\startformula\fakeformula\stopformula
\setupformulas[align=flushright]
\startformula\fakeformula\stopformula
\stopbuffer

\typebuffer[demo]

The three cases show up as:

\start \getbuffer[demo] \stop

You can also set a left and|/|or right margin:

\startbuffer[setting]
\setupformulas
  [leftmargin=3cm,
   rightmargin=3cm]
\stopbuffer

\start \getbuffer[setting] \getbuffer[demo] \stop

With formula numbers these formulas look as follows:

\startbuffer[demo]
\setupformulas[align=flushleft]
\placeformula \startformula\fakeformula\stopformula
\setupformulas[align=middle]
\placeformula \startformula\fakeformula\stopformula
\setupformulas[align=flushright]
\placeformula \startformula\fakeformula\stopformula
\stopbuffer

\start \getbuffer[demo] \stop

and the same with margins:

\start \getbuffer[setting] \getbuffer[demo] \stop

\page

When the \type {margin} option is set to \type {standard} or \type {yes} the
current indentation (when set) or left skip is added to the left side.

\startbuffer
\setupformulas[align=flushleft]
\startformula \fakeformula \stopformula
\placeformula \startformula \fakeformula \stopformula
\stopbuffer

\typebuffer \start \getbuffer \stop

\startbuffer
\setupformulas[align=flushleft,margin=standard]
\startformula \fakeformula \stopformula
\placeformula \startformula \fakeformula \stopformula
\stopbuffer

\typebuffer \start \getbuffer \stop

The distance between the formula and the number is only applied when the formula
is left or right aligned.

\startbuffer
\setupformulas[align=flushright,distance=0pt]
\startformula \fakeformula \stopformula
\placeformula \startformula \fakeformula \stopformula
\stopbuffer

\typebuffer \start \getbuffer \stop

\startbuffer
\setupformulas[align=flushright,distance=2em]
\startformula \fakeformula \stopformula
\placeformula \startformula \fakeformula \stopformula
\stopbuffer

\typebuffer \start \getbuffer \stop

\stopchapter

% \startchapter[title=Numbering]

% \stopchapter

\startchapter[title=Framing]

The \type {\framed} macro is one of the core constructors in \CONTEXT\ and it's
used all over the place. This macro is unlikely to change its behaviour and as it
has evolved over years it comes with quite some options and some can interfere
with the expectations one has. In general using this macro works out well but you
need to keep an eye on using struts and alignment.

\startbuffer
\framed{$e=mc^2$}
\stopbuffer

\typebuffer

The outcome of this is:

\startlinecorrection \getbuffer \stoplinecorrection

There is a bit of offset (that you can set) but also struts are added as can be
seen when we visualize them:

\startlinecorrection \showstruts \getbuffer \stoplinecorrection

These struts can be disabled:

\startbuffer
\framed[strut=no]{$e=mc^2$}
\stopbuffer

\typebuffer

Now the result is more tight.

\startlinecorrection \showstruts \getbuffer \stoplinecorrection

These struts are the way to get a consistent look and feel and are used
frequently in \CONTEXT. We mention these struts because they get in the way when
we frame a display formula. Let's first look at what happens when we
just package a formula in a box:

\startbuffer
\vbox\bgroup
    \startformula
        e = mc^2
    \stopformula
\egroup
\stopbuffer

\typebuffer

We get:

\startlinecorrection \start \showmakeup \getbuffer \stop \stoplinecorrection

Now there are a few properties of displaymath that one needs to keep in mind when
messing around with them this way. First of all display math is meant to be used
as part of the page stream. This means that spacing above and below is adapted to
what comes before and after. It also means that, because formulas can be numbered,
we have some settings that relate to horizontal placement.

The default vertical spacing is easy to get rid of:

\startbuffer
\vbox\bgroup
    \startformula[packed]
        e = mc^2
    \stopformula
\egroup
\stopbuffer

\typebuffer

This gives:

\startlinecorrection \start \showmakeup \getbuffer \stop \stoplinecorrection

Another handy keyword is \type {tight}:

\startbuffer
\vbox\bgroup
    \startformula[tight]
        e = mc^2
    \stopformula
\egroup
\stopbuffer

\typebuffer

This gives:

\startlinecorrection \start \showmakeup \getbuffer \stop \stoplinecorrection

We can combine these two:

\startbuffer
\vbox\bgroup
    \startformula[packed,tight]
        e = mc^2
    \stopformula
\egroup
\stopbuffer

\typebuffer

This gives:

\startlinecorrection \start \showmakeup \getbuffer \stop \stoplinecorrection

Just in case you wonder why we need to go through these troubles: keep in mind
that we are wrapping something (math) that normally goes in a vertical list with
text above and below.

The \type {packed} and \type {tight} options can help when we want to wrap
a formula in a frame:

\startbuffer
\framed
    [strut=no]
    {
        \startformula[packed,tight]
            e = mc^2
        \stopformula
    }
\stopbuffer

\typebuffer

which renders as:

\startlinecorrection \getbuffer \stoplinecorrection

There is a dedicated math framed instance that is tuned to give better results
and automatically switches to math mode:

\startbuffer
\mframed {
    e = mc^2
}
\stopbuffer

\typebuffer

becomes:

\startlinecorrection \getbuffer \stoplinecorrection

Multiple formulas can be combined:

\startbuffer
\startformulas
    \startformula
        a + b = c
    \stopformula
    \startformula
        d - e = f
    \stopformula
\stopformulas
\stopbuffer

\typebuffer

We might add some more control in the future, but this is what you get:

\startlinecorrection \getbuffer \stoplinecorrection

Framing a formula is also supported as a option, where the full power of framed can
be applied to the formula. We will illustrate this in detail on the next pages. For this
we use the following sample:

\typefile{math-framing-001.tex}

In \in {figure} [framing-flushleft], \in [framing-middle] \in {and}
[framing-flushright] you see some combinations. You can run this example on your
machine and see the details.

\startplacefigure[location=page,reference=framing-flushleft,title={Framed formulas flushed left.}]
    \startcombination[2*2]
        {\typesetfile[math-framing-001.tex][page=01,height=.45\textheight]} {\tttf right + flushleft}
        {\typesetfile[math-framing-001.tex][page=02,height=.45\textheight]} {\tttf right + flushleft}
        {\typesetfile[math-framing-001.tex][page=07,height=.45\textheight]} {\tttf left  + flushleft + tight}
        {\typesetfile[math-framing-001.tex][page=08,height=.45\textheight]} {\tttf left  + flushleft + tight}
    \stopcombination
\stopplacefigure

\startplacefigure[location=page,reference=framing-middle,title={Framed formulas centered.}]
    \startcombination[2*2]
        {\typesetfile[math-framing-001.tex][page=03,height=.45\textheight]} {\tttf right + middle}
        {\typesetfile[math-framing-001.tex][page=04,height=.45\textheight]} {\tttf right + middle}
        {\typesetfile[math-framing-001.tex][page=09,height=.45\textheight]} {\tttf left  + middle + tight}
        {\typesetfile[math-framing-001.tex][page=10,height=.45\textheight]} {\tttf left  + middle + tight}
    \stopcombination
\stopplacefigure

\startplacefigure[location=page,reference=framing-flushright,title={Framed formulas flushed right.}]
    \startcombination[2*2]
        {\typesetfile[math-framing-001.tex][page=05,height=.45\textheight]} {\tttf right + flushright}
        {\typesetfile[math-framing-001.tex][page=06,height=.45\textheight]} {\tttf right + flushright}
        {\typesetfile[math-framing-001.tex][page=11,height=.45\textheight]} {\tttf left  + flushright + tight}
        {\typesetfile[math-framing-001.tex][page=12,height=.45\textheight]} {\tttf left  + flushright + tight}
    \stopcombination
\stopplacefigure

\stopchapter

With each formula class a framed variants is automatically created:

\startbuffer
\defineformula
  [foo]

\setupformulaframed
  [foo]
  [frame=on,
   framecolor=red]

\startfooformula[frame]
    e=mc^2
\stopfooformula
\stopbuffer

\typebuffer

This time you get a red frame:

\getbuffer

You can also frame the number, as in:

\startbuffer
\setupformulaframed[framecolor=red,frame=on,offset=1ex]
\setupformula[option=frame,color=blue]
\setupformula[numbercommand={\inframed[framecolor=green]}]

\startplaceformula
    \startformula
        2 + 2 = 2x
    \stopformula
\stopplaceformula
\stopbuffer

\typebuffer

The boxes get properly aligned:

\start \showboxes \getbuffer \stop

\stoptext

% when we compare these tables with the one that the ff loader produces we notice
% some differences: this is because (deduced from source) that ff does some kind
% of interpolation for missing heights for the last kern (for n=2 adding 100 and
% for adding 1 the max height) ... this is probably a side effect if missing specs
% here we don't do that
% \enabledirectives[fontgoodies.mathkerning]

% \startTEXpage[offset=2mm]
%     \startcombination[4*2]
%         {\hbox to 50pt{\hss\showboxes\switchtobodyfont  [modern]$V_i^i = W_i^i$\hss}} {\infofont modern}
%         {\hbox to 50pt{\hss\showboxes\switchtobodyfont [cambria]$V_i^i = W_i^i$\hss}} {\infofont cambria}
%         {\hbox to 50pt{\hss\showboxes\switchtobodyfont[lucidaot]$V_i^i = W_i^i$\hss}} {\infofont lucida}
%         {\hbox to 50pt{\hss\showboxes\switchtobodyfont  [dejavu]$V_i^i = W_i^i$\hss}} {\infofont dejavu}
%         {\hbox to 50pt{\hss\showboxes\switchtobodyfont [pagella]$V_i^i = W_i^i$\hss}} {\infofont pagella}
%         {\hbox to 50pt{\hss\showboxes\switchtobodyfont  [termes]$V_i^i = W_i^i$\hss}} {\infofont termes}
%         {\hbox to 50pt{\hss\showboxes\switchtobodyfont   [bonum]$V_i^i = W_i^i$\hss}} {\infofont bonum}
%         {\hbox to 50pt{\hss\showboxes\switchtobodyfont  [schola]$V_i^i = W_i^i$\hss}} {\infofont schola}
%     \stopcombination
% \stopTEXpage
