% language=uk

\startcomponent introduction

\environment mk-environment

\chapter{Introduction}

In this document I will keep track of the transition of \CONTEXT\
from \MKII\ to \MKIV, the latter being the \LUA\ aware version.

The development of \LUATEX\ started with a few email exchanges
between me and Hartmut Henkel. I had played a bit with \LUA\ in
\SCITE\ and somehow felt that it would fit into \TEX\ quite well.
Hartmut made me a version of \PDFTEX\ which provided a \type
{\lua} command. After exploring this road a bit Taco Hoek\-water
took over and we quickly reached a point where the \PDFTEX\
development team could agree on following this road to the future.

The development was boosted by a substantial grant from Colorado
State University in the context of the Oriental \TEX\ Project of
Idris Samawi Hamid. This project aims at bringing features into
\TEX\ that will permit \CONTEXT\ to do high quality Arabic
typesetting. Due to this grant Taco could spent substantial time
on development, which in turn meant that I could start playing
with more advanced features.

This document is not so much a users manual as a history
of the development. Consider it a collection of articles, and some
chapters indeed have ended up in the journals of user groups. Things
may evolve and the way things are done may change, but it felt right
to keep track of the process this way. Keep in mind that some features
may have changed while \LUATEX\ matured.

Just for the record: development in the \LUATEX\ project is done
by Taco Hoekwater, Hartmut Henkel and Hans Hagen. Eventually, the
stable versions will become \PDFTEX\ version~2 and other members
of the \PDFTEX\ team will be involved in development and
maintenance. In order to prevent problems due to new and maybe
even slightly incompatible features, \PDFTEX\ version~1 will be kept
around as well, but no fundamentally new features will be added to
it. For practical reasons we use \LUATEX\ as the name of the
development version but also for \PDFTEX~2. That way we can use
both engines side by side.

This document is also one of our test cases. Here we use traditional
\TEX\ fonts (for math), \TYPEONE\ and \OPENTYPE\ fonts. We use color
and include test code. Taco and I always test new versions of
\LUATEX\ (the program) and \MKIV\ (the macros and \LUA\ code) with
this document before a new version is released. It also means that
there can be temporary flaws in the rendering. Keep tuned,

\blank

Hans Hagen, Hasselt NL,\crlf August 2006\endash\currentdate[year]

\blank

\type {http://www.luatex.org}

\stopcomponent
