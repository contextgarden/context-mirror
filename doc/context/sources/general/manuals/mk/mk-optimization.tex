% language=uk

\startcomponent mk-arabic

\environment mk-environment

\chapter{Optimization}

\subject{quality of code}

How good is the \MKIV\ code? Well, as good as I can make it. When you browse
the code you will probably notice differences in coding style and this is a
related to the learning curve. For instance the \type {luat-inp} module needs
some cleanup, for instance hiding local function from users.

Since benchmarking has been done right from the start there is probably not
that much to gain, but who knows. When coding in \LUA\ you should be careful
with defining global variables, since they may override something. In \MKIV\
we don't guarantee that the name you use for variable will not be used at
some point. Therefore, best operate in a dedicated \LUA\ instance, or operate
in userspace.

\starttyping
do
    -- your code
end
\stoptyping

If you want to use your data later on, think of working this way (the example
is somewhat silly):

\starttyping
userdata['your.name'] = userdata['your.name'] or { }

do
    local mydata = userdata['your.name']

    mydata.data = {}

    local function foo() return 'bar' end

    function mydata.dothis()
        mydata[foo] = foo()
    end


end
\stoptyping

In this case you can always access your user data while temporary
variables are hidden. The \type {userdata} table is predefined. As is
\type {thirddata} for modules that you may write. Of course this
assumes that you create a namespace within these global tables.

A nice test for checking global cluttering is the following:

\starttyping
for k, v in pairs(_G) do
    print(k, v)
end
\stoptyping

When you incidentally define global variables like \type {n} or \type {str}
they will show up here.

\subject{clean or dirty}

Processing the first 120 pages of this document (16 chapters) takes some 23.5
seconds on a dell M90 (2.3GHZ, 4GB mem, Windows Vista Ultimate). A rough estimate
of where \LUA\ spends its time is:

\starttabulate[|l|c|]
\NC \bf acticvity             \NC \bf sec \NC \NR
\NC input load time           \NC 0.114   \NC \NR
\NC fonts load time           \NC 6.692   \NC \NR
\NC mps conversion time       \NC 0.004   \NC \NR
\NC node processing time      \NC 0.832   \NC \NR
\NC attribute processing time \NC 3.376   \NC \NR
\stoptabulate

Font loading takes some time, which is nu surprise because we load huge Zapfino, Arabic
and \CJK\ fonts and define many instances of them. Some tracing learns that there
are some 14.254.041 function calls, of which 13.339.226 concern functions that are
called more than 5.000 times. A total of 62.434 function is counted, which is
a result of locally defined ones.

A rough indication of this overhead is given by the following test code:

\starttyping
local a,b,c,d,e,f = 1,2,3,4,5,6

function one  (a)           local n = 1 end
function three(a,b,c)       local n = 1 end
function six  (a,b,c,d,e,f) local n = 1 end

for i=1,14254041 do one  (a)           end
for i=1,14254041 do three(a,b,c)       end
for i=1,14254041 do six  (a,b,c,d,e,f) end
\stoptyping

The runtime for these tests (excluding startup) is:

\starttabulate[|l|l|]
\NC one argument    \NC 1.8 seconds \NC \NR
\NC three arguments \NC 2.0 seconds \NC \NR
\NC six arguments   \NC 2.3 seconds \NC \NR
\stoptabulate

So, the of the total runtime for this document we easily spend a couple
of seconds on function calls, especially in node processing and attribute
resolving. Does this mean that we need to change the code and follow a more
inline approach? Eventually we may optimize some code, but for the moment
we keep things as readable as possible, and even then much code is still
quite complex. Font loading is often constant for a document anyway, and
independent of the number of pages. Time spent on node processing depends on
the script, and often processing intense scripts are typeset in a larger font and
since they are less verbose than latin, this does not really influence
the average time spent on typesetting a page. Attribute handling is probably
the most time consuming activity, and for large documents the time spent on this
is large compared to font loading and node processing. But then, after a few
\MKIV\ development cycles the picture may be different.

When we turned on tracing of function calls, if becomes clear where currently
the time is spent in a document like this which demands complex Zapfino
contextual analysis as well as Arabic analysis and feature application (both
fonts demand node insertion and deletion). Of course using color also has a
price. Handling weighted and conditional spacing (new in \MKIV) involves
just over 10.000 calls to the main handler for 120 pages of this document.
Glyph related processing of node lists needs 42.000 calls, and contextual
analysis of \OPENTYPE\ fonts is good for 11.000 calls. Timing \LUA\ related
tasks involves 2 times 37.000 calls to the stopwatch. Collapsing \UTF\ in
the input lines equals the number of lines: 7700.

However, at the the top of the charts we find calls to attribute related
functions. 97.000 calls for handling special effects, overprint, transparency
and alike, and another 24.000 calls for combined color and colorspace handling.
These calls result in over 6.000 insertions of \PDF\ literals (this number is
large because we show Arabic samples with color based tracing enabled). In
case you wonder if the attribute handler can be made more efficient (we're
talking seconds here), the answer is \quotation {possibly not}. This action
is needed for each shipped out object and each shipped out page. If we divide
the 24.000 (calls) by 120 (pages) we get 200 calls per page for color processing
which is okay if you keep in mind that we need to recurse in nested horizontal
and vertical lists of the completely made op page.

\subject{serialization}

When serializing tables, we can end up with very large tables, especially
when dealing with big fonts like \quote{arabtype} or \quote {zapfino}. When
serializing tables one has to find a compromise between speed of writing,
effeciency of loading and readability. First we had (sub)tables like:

\starttyping
boundingbox = {
    [1] = 0,
    [2] = 0,
    [3] = 100,
    [4] = 200
}
\stoptyping

I mistakingly assumed that this would generate an indexed table, but at \TUG\ 2007
Roberto Ierusalimschy explained to me that this was not that efficient, since this
variant boils down to the following byte code:

\starttyping
1       [1]     NEWTABLE        0 0 4
2       [2]     SETTABLE        0 -2 -3 ; 1 0
3       [3]     SETTABLE        0 -4 -3 ; 2 0
4       [4]     SETTABLE        0 -5 -6 ; 3 100
5       [5]     SETTABLE        0 -7 -8 ; 4 200
6       [6]     SETGLOBAL       0 -1    ; boundingbox
7       [6]     RETURN          0 1
\stoptyping

This creates a hashed table. The following variant is better:

\starttyping
boundingbox = { 0, 0, 100, 200 }
\stoptyping

This results in:

\starttyping
1       [1]     NEWTABLE        0 4 0
2       [2]     LOADK           1 -2    ; 0
3       [3]     LOADK           2 -2    ; 0
4       [4]     LOADK           3 -3    ; 100
5       [6]     LOADK           4 -4    ; 200
6       [6]     SETLIST         0 4 1   ; 1
7       [6]     SETGLOBAL       0 -1    ; boundingbox
8       [6]     RETURN          0 1
\stoptyping

The resulting tables are not only smaller in terms of bytes, but also
are less memory hungry when loaded. For readability we write tables with
only numbers, strings or boolean values in an inline||format:

\starttyping
boundingbox = { 0, 0, 100, 200 }
\stoptyping

The serialized tables are somewhat smaller, depending on how
many subtables are indexed (boundary boxes, lookup sequences, etc.)

\starttabulate[|r|r|l|]
\NC \bf normal \NC \bf compact \NC \bf filename \NC \NR
\NC 34.055.092 \NC 32.403.326 \NC arabtype.tma                \NC \NR
\NC  1.620.614 \NC  1.513.863 \NC lmroman10-italic.tma        \NC \NR
\NC  1.325.585 \NC  1.233.044 \NC lmroman10-regular.tma       \NC \NR
\NC  1.248.157 \NC  1.158.903 \NC lmsans10-regular.tma        \NC \NR
\NC    194.646 \NC    153.120 \NC lmtypewriter10-regular.tma  \NC \NR
\NC  1.771.678 \NC  1.658.461 \NC palatinosanscom-bold.tma    \NC \NR
\NC  1.695.251 \NC  1.584.491 \NC palatinosanscom-regular.tma \NC \NR
\NC 13.736.534 \NC 13.409.446 \NC zapfinoextraltpro.tma       \NC \NR
\stoptabulate

Since we compile the tables to bytecode, the effects are more
spectacular there.

\starttabulate[|r|r|l|]
\NC \bf normal \NC \bf compact \NC \bf filename \NC \NR
\NC 13.679.038 \NC 11.774.106 \NC arabtype.tmc                \NC \NR
\NC    886.248 \NC    754.944 \NC lmroman10-italic.tmc        \NC \NR
\NC    729.828 \NC    466.864 \NC lmroman10-regular.tmc       \NC \NR
\NC    688.482 \NC    441.962 \NC lmsans10-regular.tmc        \NC \NR
\NC    128.685 \NC     95.853 \NC lmtypewriter10-regular.tmc  \NC \NR
\NC    715.929 \NC    582.985 \NC palatinosanscom-bold.tmc    \NC \NR
\NC    669.942 \NC    540.126 \NC palatinosanscom-regular.tmc \NC \NR
\NC  1.560.588 \NC  1.317.000 \NC zapfinoextraltpro.tmc       \NC \NR
\stoptabulate

Especially when a table is partially indexed and hashed, readability is a bit
less than normal but in practice one will seldom consult such tables in its verbose
form.

After going beta, users reported problems with scaling of the the Latin Modern and
\TeX-Gyre fonts. The troubles originate in the fact that the \OPENTYPE\ versions of
these fonts lack a design size specification and it happens that the Latin Modern
fonts do have design sizes other than 10 points. Here the power of a flexible
\TEX\ engine shows \unknown\ we can repair this when we load the font. In \MKIV\
we can now define patches:

\starttyping
do
    local function patch(data,filename)
        if data.design_size == 0 then
            local ds = (file.basename(filename)):match("(%d+)")
            if ds then
                logs.report("load otf",string.format("patching design size (%s)",ds))
                data.design_size = tonumber(ds) * 10
            end
        end
    end

    fonts.otf.enhance.patches["^lmroman"] = patch
    fonts.otf.enhance.patches["^lmsans"]  = patch
    fonts.otf.enhance.patches["^lmmono"]  = patch
end
\stoptyping

Eventually such code will move to typescripts instead of in the kernel code.


\stopcomponent
