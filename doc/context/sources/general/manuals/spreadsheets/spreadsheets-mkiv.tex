% language=uk

% author    : Hans Hagen
% copyright : PRAGMA ADE & ConTeXt Development Team
% license   : Creative Commons Attribution ShareAlike 4.0 International
% reference : pragma-ade.nl | contextgarden.net | texlive (related) distributions
% origin    : the ConTeXt distribution
%
% comment   : Because this manual is distributed with TeX distributions it comes with a rather
%             liberal license. We try to adapt these documents to upgrades in the (sub)systems
%             that they describe. Using parts of the content otherwise can therefore conflict
%             with existing functionality and we cannot be held responsible for that. Many of
%             the manuals contain characteristic graphics and personal notes or examples that
%             make no sense when used out-of-context.
%
% comment   : Some chapters might have been published in TugBoat, the NTG Maps, the ConTeXt
%             Group journal or otherwise. Thanks to the editors for corrections. Also thanks
%             to users for testing, feedback and corrections.

\usemodule[spreadsheet]
\usemodule[art-01,abr-02]

\definecolor[darkred]  [r=.4]
\definecolor[darkgreen][g=.4]
\definecolor[darkblue] [b=.4]

\definecolor[maincolor] [darkred]
\definecolor[extracolor][darkblue]

\setuptyping
  [color=extracolor]

\setuptype
  [color=extracolor]

\setuphead
  [section]
  [color=maincolor]

\setupbodyfont
  [10pt]

\setupinteraction
  [hidden]

% \setupnumbering
%   [alternative=doublesided]

\startdocument
  [metadata:author=Hans Hagen,
   metadata:title=Simple Spreadsheets,
   author=Hans Hagen,
   affiliation=PRAGMA ADE,
   location=Hasselt NL,
   title=Simple Spreadsheets,
   extra=ConTeXt MkIV,
   support=www.contextgarden.net,
   website=www.pragma-ade.nl]

\startMPpage

    StartPage;

    numeric n, m ; n := 3 * 4 ; m := 4 * 4 ;
    numeric w, h ; w := PaperWidth/n ; h := PaperHeight/m ;

    for i=1 upto n :
        for j=1 upto m :
            fill
                unitsquare
                xysized (w,h)
                shifted ((i-1)*w,(j-1)*h)
                withcolor .5[i*red/n,j*blue/m]
            ;
        endfor ;
    endfor ;

    for i=1 upto n :
        for j=1 upto m :
            draw
                textext("\tt" & char(64+i) & if j < 10 : "0" else : "" fi & decimal j)
                xysized (.7*w,.7*h)
                shifted (i*w-.5w,(m+1-j)*h-.5h)
                withcolor .5[(n+1-i)*green/n,(m+1-j)*yellow/m]
            ;
        endfor ;
    endfor ;

    draw
        textext.llft("\ssbf{\documentvariable{title}}")
        xsized (PaperHeight-h)
        rotated 90
        shifted (PaperWidth-1.75w,PaperHeight-h/2)
        withcolor white
    ;

    draw
        textext.llft("\ssbf{\documentvariable{extra}}")
        ysized (h/2)
        shifted (PaperWidth-2.5w,3.75h)
        withcolor white
    ;


    draw
        textext.llft("\ssbf{\documentvariable{author}}")
        ysized (h/2)
        shifted (PaperWidth-2.5w,2.75h)
        withcolor white
    ;

    draw
        textext.llft("\ssbf \currentdate")
        ysized (h/2)
        shifted (PaperWidth-2.5w,1.75h)
        withcolor white
    ;

    StopPage;

\stopMPpage

% \page[empty] \setuppagenumber[start=1]

\startsubject[title={Contents}]

\placelist[section][criterium=all,interaction=all]

\stopsubject

\startsection [title={Introduction}]

Occasionally a question pops up on the \CONTEXT\ mailing list and answering it
becomes a nice distraction from a boring task at hand. The spreadsheet module is
the result of such a diversion. As with more support code in \CONTEXT, this is
not a replacement for \quote {the real thing} but just a nice feature for simple
cases. The module is loaded with

\starttyping
\usemodule[spreadsheet]
\stoptyping

So this is (at least currently) not one of the core functionalities but an
add||on. Of course some useful extensions might appear in the future.

\stopsection

\startsection [title={Spreadsheet tables}]

We can use \LUA\ in each cell, because under the hood it is all \LUA. There is
some basic parsing applied so that we can use the usual \type {A..Z} variables to
access cells.

\startbuffer[demo]
\startspreadsheettable[test]
  \startrow
    \startcell 1.1         \stopcell
    \startcell 2.1         \stopcell
    \startcell A[1] + B[1] \stopcell
  \stoprow
  \startrow
    \startcell 2.1         \stopcell
    \startcell 2.2         \stopcell
    \startcell A[2] + B[2] \stopcell
  \stoprow
  \startrow
    \startcell A[1] + B[1] \stopcell
    \startcell A[2] + B[2] \stopcell
    \startcell A[3] + B[3] \stopcell
  \stoprow
\stopspreadsheettable
\stopbuffer

\typebuffer[demo]

The rendering is shown in \in {figure} [spreadsheet:1]. Keep in mind that in
\LUA\ all calculations are done using floats, at least in \LUA\ versions with
version numbers preceding 5.3.

\placefigure
  [here]
  [spreadsheet:1]
  {A simple spreadsheet.}
  {\getbuffer[demo]}

The last cell can also look like this:

\starttyping
\startcell
function()
  local s = 0
  for i=1,2 do
    for j=1,2 do
      s = s + dat[i][j]
    end
  end
  return s
end
\stopcell
\stoptyping

The content of a cell is either a number or a function. In this example
we just loop over the (already set) cells and calculate their sum. The
\type {dat} variable accesses the grid of cells.

\starttyping
\startcell
function()
  local s = 0
  for i=1,2 do
    for j=1,2 do
      s = s + dat[i][j]
    end
  end
  tmp.total = s
end
\stopcell
\stoptyping

In this variant we store the sum in the table \type {tmp} which is local to the
current sheet. Another table is \type {fnc} where we can store functions. This
table is shared between all sheets. There are two predefined functions:

\starttyping
sum(columnname,firstrow,lastrow)
fmt(specification,n)
\stoptyping

The \type {sum} function works top||down in columns, and roughly looks like
this:

\starttyping
function sum(currentcolumn,firstrow,lastrow)
  local r = 0
  for i = firstrow, lastrow do
    r = r + cells[currentcolumn][i]
  end
  return r
end
\stoptyping

The last two arguments are optional:

\starttyping
sum(columnname,lastrow)
\stoptyping

This is equivalent to:

\starttyping
function sum(currentcolumn,lastrow)
  local r = 0
  for i = 1, lastrow do
    r = r + cells[currentcolumn][i]
  end
  return r
end
\stoptyping

While:

\starttyping
sum(columnname)
\stoptyping

boils down to:

\starttyping
function sum(currentcolumn)
  local r = 0
  for i = 1, currentrow do
    r = r + cells[currentcolumn][i]
  end
  return r
end
\stoptyping

Empty cells or cells that have no numbers are skipped. Let's now see these
functions in action:

\startbuffer[demo]
\startspreadsheettable[test]
  \startrow
    \startcell 1.1 \stopcell \startcell 2.1 \stopcell
  \stoprow
  \startrow
    \startcell 2.1 \stopcell \startcell 2.2 \stopcell
  \stoprow
  \startrow
    \startcell
      function()
        local s = 0
        for i=1,2 do
          for j=1,2 do
            s = s + dat[i][j]
          end
        end
        context.bold(s)
      end
    \stopcell
    \startcell
      function()
        local s = 1
        for i=1,2 do
          for j=1,2 do
            s = s * dat[i][j]
          end
        end
        context.bold(fmt("@.1f",s))
      end
    \stopcell
  \stoprow
\stopspreadsheettable
\stopbuffer

\typebuffer[demo]

The result is shown in \in {figure} [spreadsheet:2]. Watch the \type {fmt} call:
we use an at sign instead of a percent to please \TEX.

\placefigure
  [here]
  [spreadsheet:2]
  {Cells can be (complex) functions.}
  {\getbuffer[demo]}

Keep in mind that we're typesetting and that doing complex calculations is not
our main objective. A typical application of this module is in making bills, for
which you can combine it with the correspondence modules. We leave that as an
exercise for the reader and stick to a simple example.

\startbuffer[demo]
\startspreadsheettable[test]
  \startrow
    \startcell[align=flushleft,width=8cm] "item one" \stopcell
    \startcell[align=flushright,width=3cm] @ "0.2f EUR" 3.50 \stopcell
  \stoprow
  \startrow
    \startcell[align=flushleft] "item two" \stopcell
    \startcell[align=flushright] @ "0.2f EUR" 8.45 \stopcell
  \stoprow
  \startrow
    \startcell[align=flushleft] "tax 19\percent" \stopcell
    \startcell[align=flushright] @ "0.2f EUR" 0.19 * (B[1]+B[2]) \stopcell
  \stoprow
  \startrow
    \startcell[align=flushleft] "total 1" \stopcell
    \startcell[align=flushright] @ "0.2f EUR" sum(B,1,3) \stopcell
  \stoprow
  \startrow
    \startcell[align=flushleft] "total 2" \stopcell
    \startcell[align=flushright] @ "0.2f EUR" B[1] + B[2] + B[3] \stopcell
  \stoprow
  \startrow
    \startcell[align=flushleft] "total 3" \stopcell
    \startcell[align=flushright] @ "0.2f EUR" sum(B) \stopcell
  \stoprow
\stopspreadsheettable
\stopbuffer

\typebuffer[demo]

Here (and in \in {figure} [spreadsheet:8]) you see a quick and more
readable way to format cell content. The \type {@} in the template is
optional, but needed in cases like this:

\starttyping
@ "(@0.2f) EUR" 8.45
\stoptyping

A \type {@} is only prepended when no \type {@} is given in the template.

\placefigure
  [here]
  [spreadsheet:8]
  {Cells can be formatted by using \type {@} directives.}
  {\getbuffer[demo]}

In practice this table we can be less specific and let \type {\sum} behave more
automatical. That way the coding can be simplified (see \in {figure}
[spreadsheet:7]) and also look nicer.

\startbuffer[demo]
\startspreadsheettable[test][frame=off]
  \startrow
    \startcell[align=flushleft,width=8cm] "The first item" \stopcell
    \startcell[align=flushright,width=3cm] @ "0.2f EUR" 3.50 \stopcell
  \stoprow
  \startrow
    \startcell[align=flushleft] "The second item" \stopcell
    \startcell[align=flushright] @ "0.2f EUR" 8.45 \stopcell
  \stoprow
  \startrow
    \startcell[align=flushleft] "The third item" \stopcell
    \startcell[align=flushright] @ "0.2f EUR" 5.90 \stopcell
  \stoprow
  \startrow[topframe=on]
    \startcell[align=flushleft] "VAT 19\percent" \stopcell
    \startcell[align=flushright] @ "0.2f EUR" 0.19 * sum(B) \stopcell
  \stoprow
  \startrow[topframe=on]
    \startcell[align=flushleft] "\bf Grand total" \stopcell
    \startcell[align=flushright] @ "0.2f EUR" sum(B) \stopcell
  \stoprow
\stopspreadsheettable
\stopbuffer

\typebuffer[demo]

\placefigure
  [here]
  [spreadsheet:7]
  {The \type {sum} function accumulates stepwise.}
  {\getbuffer[demo]}

There are a few more special start characters. This is demonstrated in \in
{figure} [spreadsheet:9]. An \type {=} character is ignored. \footnote {Taco
suggested to support this because some spreadsheet programs use that character to
flush a value.} When we start with an \type {!}, the content is not typeset.
Strings can be surrounded by single or double quotes and are not really
processed.

\startbuffer[demo]
\startspreadsheettable[test][offset=1ex]
  \startrow
    \startcell[align=flushleft] "first" \stopcell
    \startcell[align=flushleft] '\type{@ "[@i]" 1}' \stopcell
    \startcell[align=flushright,width=3cm] @ "[@i]" 1 \stopcell
  \stoprow
  \startrow
    \startcell[align=flushleft] "second" \stopcell
    \startcell[align=flushleft] '\type{= 2}' \stopcell
    \startcell[align=flushright] = 2 \stopcell
  \stoprow
  \startrow
    \startcell[align=flushleft] "third" \stopcell
    \startcell[align=flushleft] '\type{! 3}' \stopcell
    \startcell[align=flushright] ! 3 \stopcell
  \stoprow
  \startrow
    \startcell[align=flushleft] "fourth" \stopcell
    \startcell[align=flushleft] '\type{4}' \stopcell
    \startcell[align=flushright] 4 \stopcell
  \stoprow
  \startrow
    \startcell[align=flushleft] "\bf total one" \stopcell
    \startcell[align=flushleft] '\type{sum(C)}' \stopcell
    \startcell[align=flushright] sum(C) \stopcell
  \stoprow
  \startrow
    \startcell[align=flushleft] "\bf total two" \stopcell
    \startcell[align=flushleft] '\type{= sum(C)}' \stopcell
    \startcell[align=flushright] = sum(C) \stopcell
  \stoprow
\stopspreadsheettable
\stopbuffer

\typebuffer[demo]

The \type {sum} function is clever enough not to include itself in the
summation. Only preceding cells are taken into account, given that they
represent a number.

\placefigure
  [here]
  [spreadsheet:9]
  {Cells can be hidden by \type {!} and can contain strings only.}
  {\getbuffer[demo]}

\stopsection

\startsection [title={Normal tables}]

In the previous examples we used \TEX\ commands for structuring the sheet but
the content of cells is \LUA\ code. It is also possible to stick to a regular
table and use specific commands to set and get cell data.

\startbuffer[demo]
\bTABLE[align=middle]
  \bTR
    \bTD \getspr{100} \eTD \bTD test \setspr{30} \eTD
  \eTR
  \bTR
    \bTD \getspr{20} \eTD \bTD \getspr{4+3} \eTD
  \eTR
  \bTR
    \bTD \getspr{A[1] + A[2]} \eTD
    \bTD \getspr{B1 + B2} \eTD
  \eTR
  \bTR
    \bTD[nx=2] \bf \getspr{(A[3] + B[3]) /100} \eTD
  \eTR
  \bTR
    \bTD[nx=2] \bf \getspr{fmt("@0.3f",(A[3] + B[3]) /100)} \eTD
  \eTR
  \bTR
    \bTD[nx=2] \bf \getspr{fmt("@0.3f",(sum(A,1,2)) / 10)} \eTD
  \eTR
\eTABLE
\stopbuffer

\typebuffer[demo]

The method to use depends on the complexity of the table. If there is
more text than data then this method is probably more comfortable.

\placefigure
  [here]
  [spreadsheet:3]
  {A sheet can be filled and accessed from regular tables.}
  {\getbuffer[demo]}

% \setupspreadsheet[mysheet]
%
% \startspreadsheet[mysheet]
%
% \bTABLE[align=middle]
%   \bTR
%     \bTD \getspr{100} \eTD \bTD test \setspr{30} \eTD
%   \eTR
%   \bTR
%     \bTD \getspr{20} \eTD \bTD \getspr{4+3.5} \eTD
%   \eTR
%   \bTR
%     \bTD \getspr{A[1] + A[2]} \eTD
%     \bTD \getspr{B[1] + B[2]} \eTD
%   \eTR
%   \bTR
%     \bTD[nx=2] \bf \getspr{A[3] + B[3]} \eTD
%   \eTR
% \eTABLE
%
% \stopspreadsheet

\stopsection

\startsection[title={A few settings}]

It's possible to influence the rendering. The following example demonstrates
this. We don't use any formatting directives.

\startbuffer[demo]
\startspreadsheettable[test]
  \startrow
    \startcell   123456.78 \stopcell
  \stoprow
  \startrow
    \startcell  1234567.89 \stopcell
  \stoprow
  \startrow
    \startcell A[1] + A[2] \stopcell
  \stoprow
\stopspreadsheettable
\stopbuffer

\typebuffer[demo]

\placefigure
  [here]
  [spreadsheet:4]
  {Formatting (large) numbers.}
  {\getbuffer[demo]}

\in {Figure} [spreadsheet:4] demonstrates how this gets rendered by
default. However, often you want numbers to be split in parts separated by
periods and commas. This can be done as follows:

\startbuffer[setup]
\definehighlight[BoldAndRed]  [style=bold,color=darkred]
\definehighlight[BoldAndGreen][style=bold,color=darkgreen]

\setupspreadsheet
  [test]
  [period={\BoldAndRed{.}},
   comma={\BoldAndGreen{,}},
   split=yes]
\stopbuffer

\typebuffer[setup] \getbuffer[setup]

\placefigure
  [here]
  [spreadsheet:5]
  {Formatting (large) numbers with style and color.}
  {\getbuffer[setup,demo]}

\stopsection

\startsection[title={The \LUA\ end}]

You can also use spreadsheets from within \LUA. The following example is
rather straightforward:

\startbuffer[demo-a]
\startluacode
context.startspreadsheettable { "test" }
  context.startrow()
    context.startcell() context("123456.78")   context.stopcell()
  context.stoprow()
  context.startrow()
    context.startcell() context("1234567.89")  context.stopcell()
  context.stoprow()
  context.startrow()
    context.startcell() context("A[1] + A[2]") context.stopcell()
  context.stoprow()
context.stopspreadsheettable()
\stopluacode
\stopbuffer

\typebuffer[demo-a]

However, even more \LUA|-|ish is the next variant:

\startbuffer[demo-b]
\startluacode
  local set = moduledata.spreadsheets.set
  local get = moduledata.spreadsheets.get

  moduledata.spreadsheets.start("test")
    set("test",1,1,"123456.78")
    set("test",2,1,"1234567.89")
    set("test",3,1,"A[1] + A[2]")
  moduledata.spreadsheets.stop()

  context.bTABLE()
    context.bTR()
      context.bTD() context(get("test",1,1)) context.eTD()
    context.eTR()
    context.bTR()
      context.bTD() context(get("test",2,1)) context.eTD()
    context.eTR()
    context.bTR()
      context.bTD() context(get("test",3,1)) context.eTD()
    context.eTR()
  context.eTABLE()
\stopluacode
\stopbuffer

\typebuffer[demo-b]

Of course the second variant does not make much sense as we can do this way
more efficient by not using a spreadsheet at all:

\startbuffer[demo-c]
\startluacode
  local A1, A2 = 123456.78, 1234567.89
  context.bTABLE()
    context.bTR()
      context.bTD() context(A1)    context.eTD()
    context.eTR()
    context.bTR()
      context.bTD() context(A2)    context.eTD()
    context.eTR()
    context.bTR()
      context.bTD() context(A1+A2) context.eTD()
    context.eTR()
  context.eTABLE()
\stopluacode
\stopbuffer

\typebuffer[demo-c]

You can of course use format explicitly. Here we use the normal percent
directives because we're in \LUA, and not in \TEX, where percentage
signs are a bit of an issue.

\startbuffer[demo-d]
\startluacode
  local A1, A2 = 123456.78, 1234567.89
  local options = { align = "flushright" }
  context.bTABLE()
    context.bTR()
      context.bTD(options)
        context("%0.2f",A1)
      context.eTD()
    context.eTR()
    context.bTR()
      context.bTD(options)
        context("%0.2f",A2)
      context.eTD()
    context.eTR()
    context.bTR()
      context.bTD(options)
        context("%0.2f",A1+A2)
      context.eTD()
    context.eTR()
  context.eTABLE()
\stopluacode
\stopbuffer

\typebuffer[demo-d]

As expected and shown in \in {figure} [spreadsheet:6], only the first and last
variant gets the numbers typeset nicely.

\placefigure
  [here]
  [spreadsheet:6]
  {Spreadsheets purely done as \CONTEXT\ \LUA\ Document.}
  {\startcombination[4*1]
     {\getbuffer[demo-a]} {}
     {\getbuffer[demo-b]} {}
     {\getbuffer[demo-c]} {}
     {\getbuffer[demo-d]} {}
   \stopcombination}

\stopsection

\startsection[title={Helper macros}]

There are two helper macros that you can use to see what is stored in a
spreadsheet:

\starttyping
\inspectspreadsheet[test]
\showspreadsheet   [test]
\stoptyping

The first command reports the content of \type {test} to the console, and
the second one typesets it in the running text:

\blank
\showspreadsheet[test]
\blank

Another helper function is \type {\doifelsespreadsheetcell}, You can use this
one to check if a cell is set.

\startbuffer[demo]
(1,1): \doifelsespreadsheetcell[test]{1}{1}{set}{unset}
(2,2): \doifelsespreadsheetcell[test]{2}{2}{set}{unset}
(9,9): \doifelsespreadsheetcell[test]{9}{9}{set}{unset}
\stopbuffer

\typebuffer[demo]

This gives:

\startlines
\getbuffer[demo]
\stoplines

There is not much more to say about this module, apart from that it is a
nice example of a \TEX\ and \LUA\ mix. Maybe some more (basic) functionality
will be added in the future but it all depends on usage.

\stopsection

\startsubject[title={Colofon}]

\starttabulate[|B|p|]
\NC author    \NC \getvariable{document}{author}, \getvariable{document}{affiliation}, \getvariable{document}{location} \NC \NR
\NC version   \NC \currentdate \NC \NR
\NC website   \NC \getvariable{document}{website} \endash\ \getvariable{document}{support} \NC \NR
\NC copyright \NC \symbol[cc][cc-by-sa-nc] \NC \NR
\stoptabulate

\stopsubject

\stopdocument
