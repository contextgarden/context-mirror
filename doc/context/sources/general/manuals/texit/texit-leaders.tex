\environment texit-style

\startcomponent texit-leaders

\startchapter[title={Leaders}]

The following example comes from a question on the \CONTEXT\ list. It
exhibits a few low level tricks. For the purpose of this example we
use \type {\ruledhbox} instead of \type {\hbox}. We start with a simple
command that puts something at the end of a line:

\startbuffer
\starttexdefinition MyFill #1
    \removeunwantedspaces
    \hfill
    \ruledhbox{#1}
\stoptexdefinition
\stopbuffer

\typebuffer[option=TEX] \getbuffer

We use this in:

\startbuffer[sample]
\startitemize[packed,joinedup][rightmargin=3em]
    \startitem
        \samplefile{ward}\MyFill{DW}
    \stopitem
\stopitemize
\stopbuffer

\typebuffer[sample][option=TEX]

and get:

\getbuffer[sample]

But, the requirement was that we move the number towards the right margin, so
instead we need something:

\startbuffer
\starttexdefinition MyFill #1
    \removeunwantedspaces
    \hfill
    \rlap{\ruledhbox to \rightskip{\hss#1}}
\stoptexdefinition
\stopbuffer

\typebuffer[option=TEX] \getbuffer

This already looks more like it:

\getbuffer[sample]

But also part of the requirements was that there should be dots between the end
of the last sentence and the number. In low level \TEX\ speak that means using
leaders: repeated boxed content where the repitition is driven by a glue
specification. Let's naively use leaders now:

\startbuffer
\starttexdefinition MyFill #1
    \leaders
        \ruledhbox to 1em{\hss.\hss}
        \hfill
    \ruledhbox{#1}
\stoptexdefinition
\stopbuffer

\typebuffer[option=TEX] \getbuffer

Let's see what we get:

\getbuffer[sample]

Again we need to move the number to the right. This time we need a different
solution because we need to fill the space in between. When \TEX\ ends a
paragraph it adds \type {\parfillskip} so we will now manipulate that parameter.

\startbuffer
\starttexdefinition MyFill #1
    \parfillskip-1\rightskip plus 1fil\relax
    \leaders
        \ruledhbox to 1em{\hss.\hss}
        \hfill
    \ruledhbox{#1}
\stoptexdefinition
\stopbuffer

\typebuffer[option=TEX] \getbuffer

Does it look better?

\getbuffer[sample]

Indeed it does, but watch this:

\startbuffer[sample]
\startitemize[packed,joinedup][rightmargin=8.5em]
    \startitem
        \samplefile{ward}\MyFill{DW}\par
        \samplefile{ward}\par
        \samplefile{ward}\MyFill{DW}
    \stopitem
\stopitemize
\stopbuffer

\typebuffer[sample][option=TEX]

The first \type {\MyFill} will set the \type {\parfillskip} to a value that will
also be used later on.

\getbuffer[sample]

The way out is the following

\startbuffer
\starttexdefinition MyFill #1
    \begingroup
    \parfillskip-1\rightskip plus 1fil\relax
    \leaders
        \ruledhbox to 1em{\hss.\hss}
        \hfill
    \ruledhbox{#1}
    \par
    \endgroup
\stoptexdefinition
\stopbuffer

\typebuffer[option=TEX] \getbuffer

This looks more or less okay. The \type {\par} keeps the adaption local but for
it to work well, the \type {\par} must be inside the group.

\getbuffer[sample]

Now it's time to go for perfection! First of all, we get rid of any leading
spacing. If we need some we should inject it after a cleanup. We also use a
different leader command. Instead of \type {to} we use a \type {spread} so that
we get half the emwidth and not something slightly less due to the width of the
period.

\startbuffer
\starttexdefinition MyFill #1
    \removeunwantedspaces
    \begingroup
    \parfillskip-1\rightskip plus 1fil\relax
    \cleaders
        \ruledhbox spread 1em{\hss.\hss}
        \hfill
    \ruledhbox{#1}
    \par
    \endgroup
\stoptexdefinition
\stopbuffer

\typebuffer[option=TEX] \getbuffer

So, we end up here:

\startbuffer[sample]
\startitemize[packed,joinedup][rightmargin=5em]
    \startitem
        \samplefile{sapolsky}\MyFill{RS}\par
    \stopitem
\stopitemize
\stopbuffer

\getbuffer[sample]

For which we used this:

\typebuffer[sample][option=TEX]

Finally we get rid of the tracing:

\startbuffer
\starttexdefinition unexpanded MyFill #1
    \begingroup
    \parfillskip-1\rightskip plus 1fil\relax
    \leaders
        \hbox to \emwidth{\hss.\hss}
        \hfill
    \hbox{#1}
    \par
    \endgroup
\stoptexdefinition
\stopbuffer

\typebuffer[option=TEX] \getbuffer

Watch a few more details. It brings us to:

\getbuffer[sample]

\page

\startbuffer
\definefiller
  [MyFiller]
  [offset=.25\emwidth,
   method=middle]

\starttexdefinition unexpanded MyFill #1
    \begingroup
    \parfillskip-1\rightskip plus 1fil\relax
    \filler[MyFiller]%
    \hbox{#1}
    \par
    \endgroup
\stoptexdefinition
\stopbuffer

\typebuffer[option=TEX] \getbuffer

\getbuffer[sample]

When writing these examples I realized that it's rather trivial to add this
option to the already existing filler mechanism. The definition of such a filler
looks like this:

\startbuffer
\definefiller
  [MyFiller]
  [offset=.25\emwidth,
   rightmargindistance=-\rightskip,
   method=middle]
\stopbuffer

\typebuffer[option=TEX] \getbuffer

\startbuffer[sample]
\startitemize[packed,joinedup][rightmargin=5em]
    \startitem
        \samplefile{sapolsky}\fillupto[MyFiller]{RS}
    \stopitem
\stopitemize
\stopbuffer

The sample code now becomes:

\typebuffer[sample][option=TEX]

Ans as expected renders as:

\getbuffer[sample]

\stopcomponent
