% language=uk

% \usepath[jobfile:]
% \setupexternalfigures[directory=jobfile:]
% \usepath[toppath:]
% \setupexternalfigures[directory=toppath:]

\usemodule
  [abr-04]

\setupbodyfont
  [bookman,11pt]

\definecolor
  [maincolor]
  [s=.35]

\setuplayout
  [height=middle,
   width=middle,
   footer=0pt]

\setupwhitespace
  [big]

\setuphead
  [chapter]
  [style=\bfc,
   color=maincolor,
   header=high]

\setuphead
  [section]
  [style=\bfb,
   color=maincolor]

\setuptyping
  [color=maincolor]

\setuptype
  [color=maincolor]

% \showframe

\startdocument
  [metadata:author=Hans Hagen,
   metadata:title=Workflow support in context,
   author=Hans Hagen,
   affiliation=PRAGMA ADE,
   location=Hasselt NL,
   title=workflow,
   extra=support in context,
   support=www.contextgarden.net,
   website=www.pragma-ade.nl]

\definefontfeature[LatinModernMonoVariable][default][liga=no]
\definefont[LatinModernMonoVariable][LMTypewriterVarWd-Regular*LatinModernMonoVariable sa 1]

\startMPpage

    fill fullsquare xysized(PaperWidth,PaperHeight) withcolor .4white ;

    draw image (
        fill arrowhead fullcircle scaled .5 rotated   0 scaled 10 withcolor white ;
        fill arrowhead fullcircle scaled .5 rotated 120 scaled 10 withcolor white ;
        fill arrowhead fullcircle scaled .5 rotated 240 scaled 10 withcolor white ;
        fill arrowhead fullcircle scaled .5 rotated  60 scaled 10 withcolor white ;
        fill arrowhead fullcircle scaled .5 rotated 180 scaled 10 withcolor white ;
        fill arrowhead fullcircle scaled .5 rotated 300 scaled 10 withcolor white ;
    ) xsized (.9PaperWidth) shifted (0,.2PaperWidth) ;

    draw image (
        fill arrowhead fullcircle scaled .5 rotated   0 scaled 10 withcolor .4red     withtransparency (1,.5) ;
        fill arrowhead fullcircle scaled .5 rotated 120 scaled 10 withcolor .4green   withtransparency (1,.5) ;
        fill arrowhead fullcircle scaled .5 rotated 240 scaled 10 withcolor .4blue    withtransparency (1,.5) ;
        fill arrowhead fullcircle scaled .5 rotated  60 scaled 10 withcolor .4cyan    withtransparency (1,.5) ;
        fill arrowhead fullcircle scaled .5 rotated 180 scaled 10 withcolor .4magenta withtransparency (1,.5) ;
        fill arrowhead fullcircle scaled .5 rotated 300 scaled 10 withcolor .4yellow  withtransparency (1,.5) ;
    ) xsized (.9PaperWidth) shifted (0,.2PaperWidth) ;

    draw textext ("\LatinModernMonoVariable \documentvariable{title}") xsized (.9PaperWidth) shifted (0,-.425PaperWidth) withcolor white ;
    draw textext ("\LatinModernMonoVariable \documentvariable{extra}") xsized (.9PaperWidth) shifted (0,-.575PaperWidth) withcolor white ;

\stopMPpage

\startfrontmatter

\starttitle[title=Contents]

    \placelist[chapter]

\stoptitle

\stopfrontmatter

\startbodymatter

\startchapter[title=Introduction]

This manual contains some information about features that can help you to manage
workflows or \CONTEXT\ related processes. Because we use \CONTEXT\ ourselves all
that we need ends up in the distribution. When you discover something workflow
related that is not yet covered here, you can tell me. I simply forget about all
there is (especially if it's made for projects.)

\startlines
\documentvariable{author},
\documentvariable{affiliation}
\documentvariable{location}
\currentdate[month,year]
\stoplines

\stopsubject

\stopchapter

\startchapter[title=Accessing resources]

One of the benefits of \TEX\ is that you can use it in automated workflows
where large quantities of data is involved. A document can consist of
several files and normally also includes images. Of course there are styles
involved too. At \PRAGMA\ normally put styles and fonts in:

\starttyping
/data/site/context/tex/texmf-project/tex/context/user/<project>/...
/data/site/context/tex/texmf-fonts/data/<foundry>/<collection>/...
\stoptyping

alongside

\starttyping
/data/framework/...
\stoptyping

where the job management services are put, while we put resources in:

\starttyping
/data/resources/...
\stoptyping

The processing happens in:

\starttyping
/data/work/<uuid user space>/
\stoptyping

Putting styles (and resources like logos and common images) and fonts (if the
project has specific ones not present in the distribution) in the \TEX\ tree
makes sense because that is where such files are normally searched. Of course you
need to keep the distributions file database upto|-|date after adding files there.

Processing has to happen isolated from other runs so there we use unique
locations. The services responsible for running also deal with regular cleanup
of these temporary files.

Resources are somewhat special. They can be stable, i.e.\ change seldom, but more
often they are updated or extended periodically (or even daily). We're not
talking of a few files here but of thousands. In one project we have 20 thousand
resources, that can be combined into arbitrary books, and in another one, each
chapter alone is about 400 \XML\ and image files. That means we can have 5000
files per book and as we have at least 20 books, we end up with 100K files. In
the first case accessing the resources is easy because there is a well defined
structure (under our control) so we know exactly where each file sits in the
resource tree. In the 100K case there is a deeper structure which is in itself
predictable but because many authors are involved the references to these files
are somewhat instable (and undefined). It is surprising to notice that publishers
don't care about filenames (read: cannot control all the parties involved) which
means that we have inconsist use of mixed case in filenames, and spaces,
underscores and dashes creeping in. Because typesetting for paper is always at
the end of the pipeline (which nowadays is mostly driven by (limitations) of web
products) we need to have a robust and flexible lookup mechanism. It's a side
effect of the click and point culture: if objects are associated (filename in
source file with file on the system) anything you key in will work, and
consistency completely depends on the user. And bad things then happen when files
are copied, renamed, etc. In that stadium we can better be tolerant than try to
get it fixed. \footnote {From what we normally receive we often conclude that
copy|-|editing and image production companies don't impose any discipline or
probably simply lack the tools and methods to control this. Some of our workflows
had checkers and fixers, so that when we got 5000 new resources while only a few
needed to be replaced we could filter the right ones.  It was not uncommon to find
duplicates for thousands of pictures: similar or older variants.}

\starttyping
foo.jpg
bar/foo.jpg
images/bar/foo.jpg
images/foo.jpg
\stoptyping

The xml files have names like:

\starttyping
b-c.xml
a/b-c.jpg
a/b/b-c.jpg
a/b/c/b-c.jpg
\stoptyping

So it's sort of a mess, especially if you add arbitrary casing to this. Of course
one can argue that a wrong (relative) location is asking for problems, it's less
an issue here because each image has a unique name. We could flatten the resource
tree but having tens of thousands of files on one directory is asking for
problems when you want to manage them.

The typesetting (and related services) run on virtual machines. The three
directories:

\starttyping
/data/site
/data/resources
/data/work
\stoptyping

are all mounted as nfs shares on a network storage. For the styles (and binaries)
this is no big deal as normally these files are cached, but the resources are
another story. Scanning the complete (mounted) resource tree each run is no
option so there we use a special mechanism in \CONTEXT\ for locating files.

Already early in the development of \MKIV\ one of the locating mechanisms was
the following:

\starttyping
tree:////data/resources/foo/**/drawing.jpg
tree:////data/resources/foo/**/Drawing.jpg
\stoptyping

Here the tree is scanned once per run, which is normally quite okay when there
are not that many files and when the files reside on the machine itself. For a
more high performance approach using network shares we have a different
mechanism. This time it looks like this:

\starttyping
dirlist:/data/resources/**/drawing.jpg
dirlist:/data/resources/**/Drawing.jpg
dirlist:/data/resources/**/just/some/place/drawing.jpg
dirlist:/data/resources/**/images/drawing.jpg
dirlist:/data/resources/**/images/drawing.jpg?option=fileonly
dirfile:/data/resources/**/images/drawing.jpg
\stoptyping

The first two lookups are wildcard. If there is a file with that name, it will be
found. If there are more, the first hit is used. The second and third examples
are more selective. Here the part after the \type {**} has to match too. So here
we can deal with multiple files named \type {drawing.jpg}. The last two
equivalent examples are more tolerant. If no explicit match is found, a lookup
happens without being selective. The case of a name is ignored but when found, a
name with the right case is used.

You can hook a path into the resolver for source files, for example:

\starttyping
\usepath                       [dirfile://./resources/**]
\setupexternalfigures[directory=dirfile://./resources/**]
\stoptyping

You need to make sure that file(name)s in that location don't override ones in
the regular \TEX\ tree. These extra paths are only used for source file lookups
so for instance font lookups are not affected.

When you add, remove or move files the tree, you need to remove the \type
{dirlist.*} files in the root because these are used for locating files. A new
file will be generated automatically. Don't forget this!

\stopchapter

\startchapter[title=Graphics]

\startsection[title=Bad names]

After many years of using \CONTEXT\ in workflows where large amounts of source files
as well as graphics were involved we can safely say that it's hard for publishers to
control the way these are named. This is probably due to the fact that in a
click|-|and|-|point based desktop publishing workflow names don't matter as one stays on
one machine, and names are only entered once (after that these names become abstractions and
get cut and pasted). Proper consistent resource managament is simply not part of the flow.

This means that you get names like:

\starttyping
foo_Bar_01_03-a.EPS
foo__Bar-01a_03.eps
foo__Bar-01a_03.eps
foo BarA  01-03.eps
\stoptyping

Especially when a non proportional screen font is used multiple spaces can look
like one. In fancy screen fonts upper and lowercase usage might get obscured. It
really makes one wonder if copy|-|editing or adding labels to graphics isn't
suffering from the same problem.

Anyhow, as in an automated rendering workflow the rendering is often the last step you
can imagine that when names get messed up it's that last step that gets blamed. It's not
that hard to sanitize names of files on disk as well as in the files that refer to them,
and we normally do that we have complete control. This is no option when all the resources
are synchronzied from elsewhere. In that case the only way out is signaling potential
issues. Say that in the source file there is a reference:

\starttyping
foo_Bar_01_03-a.EPS
\stoptyping

and that the graphic on disk has the same name, but for some reason after an update
has become:

\starttyping
foo-Bar_01_03-a.EPS
\stoptyping

The old image is probably still there so the update is not reflected in the final
product. This is not that uncommon when you deal with tens of thousands of files,
many editors and graphic designers, and no strict filename policy.

For this we provide the following tracing option:

\starttyping
\enabletrackers[graphics.lognames]
\stoptyping

This will put information in the log file about included graphics, like:

\starttyping
system          > graphics > start names

used graphic    > asked   : cow.pdf
used graphic    > comment : not found
used graphic    > asked   : t:/sources/cow.pdf
used graphic    > format  : pdf
used graphic    > found   : t:/sources/cow.pdf
used graphic    > used    : t:/sources/cow.pdf

system          > graphics > stop names
\stoptyping

You can also information to the file itself:

\starttyping
\usemodule[s-figures-names]
\stoptyping

Of course that has to be done at the end of the document. Bad names are reported
and suitable action can be taken.

\stopsection

\stopchapter

\stopbodymatter

\stopdocument

\disabledirectives[resolvers.maxreadlevel]
