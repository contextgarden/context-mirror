%D \module
%D   [       file=catc-cys,
%D        version=2006.09.18,
%D          title=\CONTEXT\ Catcode Macros,
%D       subtitle=Extra Tables,
%D         author=Hans Hagen,
%D           date=\currentdate,
%D      copyright={PRAGMA / Hans Hagen \& Ton Otten}]
%C
%C This module is part of the \CONTEXT\ macro||package and is
%C therefore copyrighted by \PRAGMA. See mreadme.pdf for
%C details.

%D We prefer to define relevant catcode tables in this file instead
%D of everywhere around.

\ifdefined \ctxcatcodes  \else \newcatcodetable \ctxcatcodes  \fi
\ifdefined \mthcatcodes  \else \newcatcodetable \mthcatcodes  \fi % math, not used, too tricky
\ifdefined \xmlcatcodesn \else \newcatcodetable \xmlcatcodesn \fi % normal
\ifdefined \xmlcatcodese \else \newcatcodetable \xmlcatcodese \fi % entitle
\ifdefined \xmlcatcodesr \else \newcatcodetable \xmlcatcodesr \fi % reduce
\ifdefined \typcatcodesa \else \newcatcodetable \typcatcodesa \fi % { }
\ifdefined \typcatcodesb \else \newcatcodetable \typcatcodesb \fi % < >

\startcatcodetable \ctxcatcodes
    \catcode`\^^I = 10
    \catcode`\^^M =  5
  % \catcode`\^^J = 10 % new
    \catcode`\^^L =  5
    \catcode`\    = 10
    \catcode`\^^Z =  9
    \catcode`\\   =  0
    \catcode`\{   =  1
    \catcode`\}   =  2
    \catcode`\$   =  3
    \catcode`\&   =  4
    \catcode`\#   =  6
    \catcode`\^   =  7
    \catcode`\_   =  8
    \catcode`\%   = 14
    \catcode`\~   = 13
    \catcode`\|   = 13
\stopcatcodetable

\startcatcodetable \prtcatcodes
    \catcode`\^^I = 10
    \catcode`\^^M =  5
    \catcode`\^^L =  5
    \catcode`\    = 10
    \catcode`\^^Z =  9
    \catcode`\\   =  0
    \catcode`\{   =  1
    \catcode`\}   =  2
    \catcode`\$   =  3
    \catcode`\&   =  4
    \catcode`\#   =  6
    \catcode`\^   =  7
    \catcode`\_   =  8
    \catcode`\%   = 14
    \catcode`\@   = 11
    \catcode`\!   = 11
    \catcode`\?   = 11
    \catcode`\~   = 13
    \catcode`\|   = 13
\stopcatcodetable

\startcatcodetable \mthcatcodes
    \catcode`\^^I = 10
    \catcode`\^^M =  5
   %\catcode`\^^J = 10 % new
    \catcode`\^^L =  5
    \catcode`\    = 10
    \catcode`\^^Z =  9
    \catcode`\\   =  0
    \catcode`\{   =  1
    \catcode`\}   =  2
    \catcode`\$   =  3
    \catcode`\&   =  4
    \catcode`\#   =  6
    \catcode`\^   =  7
    \catcode`\_   =  8
    \catcode`\%   = 14
   %\catcode`\~   = 13
   %\catcode`\|   = 13
\stopcatcodetable

\startcatcodetable \xmlcatcodesn
    \catcode`\^^I = 10 % ascii tab is a blank space
    \catcode`\^^M =  5 % ascii return is end-line
    \catcode`\^^L =  5 % ascii form-feed
    \catcode`\    = 10 % ascii space is blank space
    \catcode`\^^Z =  9 % ascii eof is ignored
    \catcode`\&   = 13 % entity
    \catcode`\<   = 13 % element
    \catcode`\>   = 12
    \catcode`\"   = 12 % probably not needed any more
    \catcode`\/   = 12 % probably not needed any more
    \catcode`\'   = 12 % probably not needed any more
    \catcode`\~   = 12 % probably not needed any more
    \catcode`\#   = 12 % probably not needed any more
    \catcode`\\   = 12 % probably not needed any more
\stopcatcodetable

\startcatcodetable \xmlcatcodese
    \catcode`\^^I = 10 % ascii tab is a blank space
    \catcode`\^^M =  5 % ascii return is end-line
    \catcode`\^^L =  5 % ascii form-feed
    \catcode`\    = 10 % ascii space is blank space
    \catcode`\^^Z =  9 % ascii eof is ignored
    \catcode`\&   = 13 % entity
    \catcode`\<   = 13 % element
    \catcode`\>   = 12
    \catcode`\#   = 13
    \catcode`\$   = 13
    \catcode`\%   = 13
    \catcode`\\   = 13
    \catcode`\^   = 13
    \catcode`\_   = 13
    \catcode`\{   = 13
    \catcode`\}   = 13
    \catcode`\|   = 13
    \catcode`\~   = 13
\stopcatcodetable

\startcatcodetable \xmlcatcodesr
    \catcode`\^^I = 10 % ascii tab is a blank space
    \catcode`\^^M =  5 % ascii return is end-line
    \catcode`\^^L =  5 % ascii form-feed
    \catcode`\    = 10 % ascii space is blank space
    \catcode`\^^Z =  9 % ascii eof is ignored
    \catcode`\&   = 13 % entity
    \catcode`\<   = 13 % element
    \catcode`\>   = 12
    \catcode`\#   = 13
    \catcode`\$   = 13
    \catcode`\%   = 13
    \catcode`\\   = 13
    \catcode`\^   = 13
    \catcode`\_   = 13
    \catcode`\{   = 13
    \catcode`\}   = 13
    \catcode`\|   = 13
    \catcode`\~   = 13
\stopcatcodetable

\startcatcodetable \typcatcodesa
    \catcode`\^^I = 12
    \catcode`\^^M = 12
    \catcode`\^^L = 12
    \catcode`\    = 12
    \catcode`\^^Z = 12
    \catcode`\{   =  1
    \catcode`\}   =  2
\stopcatcodetable

\startcatcodetable \typcatcodesb
    \catcode`\^^I = 12
    \catcode`\^^M = 12
    \catcode`\^^L = 12
    \catcode`\    = 12
    \catcode`\^^Z = 12
    \catcode`\<   =  1
    \catcode`\>   =  2
\stopcatcodetable

\letcatcodecommand \ctxcatcodes  `\|   \relax
\letcatcodecommand \ctxcatcodes  `\~   \relax

%letcatcodecommand \prtcatcodes  `\|   \relax % falls back on ctx
%letcatcodecommand \prtcatcodes  `\~   \relax % falls back on ctx

\letcatcodecommand \xmlcatcodesn `\&   \relax
\letcatcodecommand \xmlcatcodesn `\<   \relax

\letcatcodecommand \xmlcatcodese `\&   \relax
\letcatcodecommand \xmlcatcodese `\<   \relax

\letcatcodecommand \xmlcatcodesr `\&   \relax
\letcatcodecommand \xmlcatcodesr `\<   \relax

\letcatcodecommand \xmlcatcodese `\#   \relax
\letcatcodecommand \xmlcatcodese `\$   \relax
\letcatcodecommand \xmlcatcodese `\%   \relax
\letcatcodecommand \xmlcatcodese `\\   \relax
\letcatcodecommand \xmlcatcodese `\^   \relax
\letcatcodecommand \xmlcatcodese `\_   \relax
\letcatcodecommand \xmlcatcodese `\{   \relax
\letcatcodecommand \xmlcatcodese `\}   \relax
\letcatcodecommand \xmlcatcodese `\|   \relax
\letcatcodecommand \xmlcatcodese `\~   \relax

\letcatcodecommand \xmlcatcodesr `\#   \relax
\letcatcodecommand \xmlcatcodesr `\$   \relax
\letcatcodecommand \xmlcatcodesr `\%   \relax
\letcatcodecommand \xmlcatcodesr `\\   \relax
\letcatcodecommand \xmlcatcodesr `\^   \relax
\letcatcodecommand \xmlcatcodesr `\_   \relax
\letcatcodecommand \xmlcatcodesr `\{   \relax
\letcatcodecommand \xmlcatcodesr `\}   \relax
\letcatcodecommand \xmlcatcodesr `\|   \relax
\letcatcodecommand \xmlcatcodesr `\~   \relax

    \catcodetable       \ctxcatcodes
\let\defaultcatcodetable\ctxcatcodes
\let\xmlcatcodes        \xmlcatcodesn % beware, in mkiv we use \notcatcodes

\endinput

% under consideration:
%
% \newcatcodetable\txtcatcodes
%
% \startcatcodetable \txtcatcodes
%     \catcode`\^^I = 10
%     \catcode`\^^M =  5
%     \catcode`\^^L =  5
%     \catcode`\    = 10
%     \catcode`\\   =  0
%     \catcode`\{   =  1
%     \catcode`\}   =  2
% \stopcatcodetable
%
% \newcount\relaxedcatcodedepth
%
% \def\startrelaxedcatcodes
%   {\global\chardef\relaxedcatcodeparent\catcodetable
%    \global\advance\relaxedcatcodedepth\plusone
%    \nonknuthmode\setcatcodetable\txtcatcodes}
%
% \def\stoprelaxedcatcodes
%   {\ifcase\relaxedcatcodedepth
%      % error
%    \or
%      \setcatcodetable\relaxedcatcodeparent
%      \global\relaxedcatcodedepth\zerocount
%    \else
%      \global\advance\relaxedcatcodedepth\minusone
%      \setcatcodetable\txtcatcodes
%    \fi}
%
% \starttext
%
% \startrelaxedcatcodes
%     \startcomment test \stopcomment
%     test $ test 10% whatever|test \mathematics{x^2=1}
% \stoprelaxedcatcodes
%
% $x^2=1$
%
% \stoptext
