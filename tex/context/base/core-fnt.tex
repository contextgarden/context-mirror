%D \module
%D   [       file=core-fnt,
%D        version=1995.10.10,
%D          title=\CONTEXT\ Core Macros,
%D       subtitle=Font Support,
%D         author=Hans Hagen,
%D           date=\currentdate,
%D      copyright={PRAGMA / Hans Hagen \& Ton Otten}]
%C
%C This module is part of the \CONTEXT\ macro||package and is
%C therefore copyrighted by \PRAGMA. Non||commercial use is
%C granted.

\writestatus{loading}{Context Core Macros / Font Support}

\unprotect

%D \macros
%D   {kap,KAP,Kap,Kaps,nokap}
%D
%D We already introduced \type{\kap} as way to capitalize
%D words. This command comes in several versions:
%D
%D \startbuffer
%D \kap {let's put on a \kap{cap}}
%D \kap {let's put on a \nokap{cap}}
%D \KAP {let's put on a \\{cap}}
%D \Kap {let's put on a \\{cap}}
%D \Kaps{let's put on a cap}
%D \stopbuffer
%D
%D \typebuffer
%D
%D Note the use of \type{\nokap}, \type{\\} and the nested
%D \type{\kap}.
%D
%D \startvoorbeeld
%D \startregels
%D \haalbuffer
%D \stopregels
%D \stopvoorbeeld
%D
%D These macros show te main reason why we introduced the
%D smaller \type{\tx} and \type{\txx}.
%D
%D \starttypen
%D \kap\romeins{1995}
%D \stoptypen
%D
%D This at first sight unusual capitilization is completely
%D legal.
%D 
%D \showsetup{\y!kap}
%D \showsetup{\y!Kap}
%D \showsetup{\y!KAP}
%D \showsetup{\y!Kaps}
%D \showsetup{\y!nokap}

\unexpanded\def\kap%
  {\futurelet\next\dokap}

% \def\dokap%
%   {\ifx\next\bgroup
%      \def\next{\dodokap\relax}%
%    \else
%      \def\next{\dodokap}%
%    \fi
%    \next}

\def\dokap%
  {\ifx\next\bgroup
     \expandafter\dodokap\expandafter\relax
   \else
     \expandafter\dodokap
   \fi}

\def\dodokap#1#2%
  {\ifmmode\hbox\fi
   \bgroup
   \tx 
   \the\everyuppercase
   \uppercase{#1{#2}}%
   \egroup}

\unexpanded\def\KAP#1%
  {{\def\\##1{\kap{##1}}#1}}

\unexpanded\def\Kap#1%
  {\KAP{\\#1}}

\def\nokap#1%
  {\lowercase{#1}}

\def\Kaps%
  {\let\processword=\Kap   
   \processwords}

%D \macros
%D   {Word, Words, WORD, WORDS, doprocesswords}
%D
%D This is probably not the right place to present the next set
%D of macros.
%D
%D \starttypen
%D \Word {far too many words}
%D \Words{far too many words}
%D \WORD {far too many words}
%D \WORDS{far too many words}
%D \stoptypen
%D
%D \typebuffer
%D
%D This calls result in:
%D
%D \startvoorbeeld
%D \startregels
%D \haalbuffer
%D \stopregels
%D \stopvoorbeeld
%D
%D \showsetup{\y!Word}
%D \showsetup{\y!Words}
%D \showsetup{\y!WORD}
%D \showsetup{\y!WORDS}

\def\doWord#1%
  {\bgroup
   \the\everyuppercase
   \uppercase{#1}%
   \egroup}

\def\Word#1%
  {\doWord#1}

\def\doprocesswords#1 #2\od%
  {\ConvertToConstant\doifnot{#1}{}
     {\processword{#1} %
      \doprocesswords#2 \od}}

\def\processwords#1%
  {\doprocesswords#1 \od\unskip}

\def\Words%
  {\let\processwords=\Word 
   \processwords}

\def\WORD#1%
  {\bgroup
   \def\kap#1{#1}%
   \edef\next{#1}%
   \the\everyuppercase
   \uppercase\expandafter{\next}%
   \egroup}

\def\WORDS#1%
  {\WORD{#1}}

%D \macros
%D   {stretched}
%D
%D Stretching characters in a word is a sort of typographical
%D murder. Nevertheless we support this manipulation for use in
%D for instance titles.
%D
%D \starttypen
%D \hbox to 5cm{\stretched{murder}}
%D \stoptypen
%D
%D \typebuffer
%D
%D or
%D
%D \startvoorbeeld
%D \haalbuffer
%D \stopvoorbeeld
%D
%D \showsetup{\y!stretched}

\def\stretched%
  {\ifvmode\hbox to \hsize\else\ifinner\else\hbox\fi\fi
   \processtokens\relax\hss\relax\normalspace}

%D \startbuffer
%D \stretched{Unknown Box}
%D \hbox to .5\hsize{\stretched{A Horizontal Box}}
%D \vbox to 2cm{\stretched{A Vertical Box}}
%D \hbox to 3cm{\stretched{sp{\'e}c{\`\i}{\"a}l}}
%D \stopbuffer
%D 
%D \haalbuffer 
%D 
%D The first line of this macros takes care of boxing. Normally
%D one will use an \type{\hbox} specification. The last line
%D shows how special characters should be passed. 
%D 
%D \typebuffer

%D \macros
%D   {underbar,underbars,overstrike,overstrikes,setupunderbar} 
%D
%D In the rare case that we need undelined words, for instance
%D because all font alternatives are already in use, one can
%D use \type{\underbar} and \type{\overstrike} and their plural
%D forms. 
%D 
%D \startbuffer
%D \underbars{drawing \underbar{bars} under words is a typewriter leftover}
%D \overstrikes{striking words makes them \overstrike{unreadable}}
%D \stopbuffer
%D 
%D \typebuffer
%D 
%D \startvoorbeeld
%D \startregels
%D \haalbuffer
%D \stopregels
%D \stopvoorbeeld
%D 
%D The next macros are derived from the \PLAIN\ \TEX\ one, but 
%D also supports nesting. The \type{$} keeps us in horizontal 
%D mode and at the same time applies grouping. 
%D
%D \showsetup{\y!underbar}
%D \showsetup{\y!underbars}
%D \showsetup{\y!overstrike}
%D \showsetup{\y!overstrikes} 
%D
%D Although underlining is ill advised, we permit some 
%D alternatives, that can be set up by: 
%D
%D \showsetup{\y!setupunderbar}
%D
%D The alternatives show up as 
%D   {\setupunderbar [variant=a]{alternative a},
%D   {\setupunderbar [variant=b]{alternative b},
%D   {\setupunderbar [variant=c]{alternative c}
%D and 
%D   {\setupunderbar [lijndikte=1pt]{1pt width},
%D   {\setupunderbar [lijndikte=2pt]{2pt width}, 
%D or whatever. Because \type{\overstrike} uses the same 
%D method, the settings also apply to that macro. 

\newcounter\underbarlevel
\newbox\underbarbox

\def\underbarmethoda#1#2#3% RULE 
  {\hbox to #1{\vrule\!!width#1\!!height#2\!!depth#3}}

\def\underbarmethodb#1#2#3% DASH 
  {\hbox to #1
     {\hskip-.25em 
      \xleaders
        \hbox{\hskip.25em\vrule\!!width.25em\!!height#2\!!depth#3}
        \hfil}}

\def\underbarmethodc#1#2#3% PERIOD
  {\hbox to #1
     {\dimen0=#3
      \advance\dimen0 by .2ex 
      \hskip-.25em 
      \xleaders
        \hbox{\hskip.25em\lower\dimen0\hbox{.}}
        \hfil}}

\def\dodounderbar#1#2#3% 
  {\bmath
   \setbox0=\hbox{#3}%
   \setbox2=\getvalue{underbarmethod\@@onvariant}{\wd0}{#1}{#2}%
   \wd0=\!!zeropoint
   \box0\box2
   \emath}

\unexpanded\def\dounderbar#1%
  {\bgroup
   \increment\underbarlevel
   \dimen0=1.5\normallineskip   % was \dimen0=1.5\lineskip
   \dimen0=\underbarlevel\dimen0
   \dimen2=\dimen0
   \advance\dimen2 by \@@onlijndikte
   \dodounderbar{-\dimen0}{\dimen2}{#1}%
   \egroup}

\unexpanded\def\underbar#1%
  {\bgroup
   \setbox\underbarbox=\hbox
     {\dounderbar{\hskip\fontdimen2\font}}%
   \def\betweenisolatedwords%
     {\nobreak
      \hskip\!!zeropoint\!!minus\fontdimen4\font
      \discretionary{}{}{\copy\underbarbox}}%
   \processisolatedwords{#1}\dounderbar
   \egroup}

\unexpanded\def\underbars#1%
  {\processisolatedwords{#1}\dounderbar}

\unexpanded\def\dooverstrike#1%
  {\bgroup
   \dimen0=2.5\lineskip
   \dimen2=\dimen0
   \advance\dimen2 by \@@onlijndikte
   \dodounderbar{\dimen2}{-\dimen0}{#1}%
   \egroup}

\unexpanded\def\overstrike#1%
  {\bgroup
   \setbox\underbarbox=\hbox
     {\dooverstrike{\hskip\fontdimen2\font}}%
   \def\betweenisolatedwords%
     {\nobreak
      \hskip\!!zeropoint\!!minus\fontdimen4\font
      \discretionary{}{}{\copy\underbarbox}}%
   \processisolatedwords{#1}\dooverstrike
   \egroup}

\unexpanded\def\overstrikes#1%
  {\processisolatedwords{#1}\dooverstrike}

\def\setupunderbar%
  {\dodoubleargument\getparameters[\??on]}

%D This module has only a few setups:

\setupunderbar
  [\c!variant=a,
   \c!lijndikte=\linewidth]
  
\protect 

\endinput
