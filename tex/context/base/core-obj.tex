%D \module
%D   [       file=core-obj,
%D        version=1998.01.15,
%D          title=\CONTEXT\ Core Macros,
%D       subtitle=Object Handling, 
%D         author=Hans Hagen,
%D           date=\currentdate,
%D      copyright={PRAGMA / Hans Hagen \& Ton Otten}]
%C
%C This module is part of the \CONTEXT\ macro||package and is
%C therefore copyrighted by \PRAGMA. Non||commercial use is
%C granted.

\writestatus{loading}{Context Core Macros / Object Handling}

\unprotect

\startmessages  dutch  library: references
     30: onbekend object --
\stopmessages

\startmessages  english  library: references
     30: unknown object --
\stopmessages

\startmessages  german  library: references
     30: unbekanntes Object --
\stopmessages

%D Boxes can be considered reuable objects. Unfortunaltely once
%D passed to the \DVI\ file, such objects cannot be reused. In
%D \PDF\ however, reusing is possible and sometimes even a
%D necessity. Therefore, \CONTEXT\ supports reusable objects. 
%D 
%D During the \TEX\ processing run, boxes can serve the purpose
%D of objects, and the \DVI\ driver module implements objects
%D using boxes. Only when \ETEX\ is widespread, and therefore
%D the limit on 256 boxes is removed, this becomes useful. 
%D 
%D The \PDF\ and \PDFTEX\ driver modules implement objects
%D using \PDF\ forms. There is no (real) restriction on the
%D number of objects there. 
%D 
%D The first application of objects in \CONTEXT\ concerned
%D \METAPOST\ graphics and fill||in form fields. The first
%D application can save lots of bytes, while the latter use is
%D more a necessity than byte saving. 
%D 
%D \starttypen
%D \setobject{name}=\somebox{}
%D \getobject{name}
%D \stoptypen
%D
%D Here \type{\somebox} can be whatever box specification 
%D suits \TEX. 

\def\setobject#1%
  {\dosetuppaper{\papierformaat}{\the\papierbreedte}{\the\papierhoogte}%
   \dowithnextbox
     {\setxvalue{\r!object#1}%
        {\noexpand\dogetobject{#1}
           {\ifhbox\nextbox\hbox\else\vbox\fi}
           {\the\wd\nextbox}{\the\ht\nextbox}{\the\dp\nextbox}}%
      \dostartobject{#1}% 
        {\number\wd\nextbox}{\number\ht\nextbox}{\number\dp\nextbox}%
      \box\nextbox
      \dostopobject}}

\def\dogetobject#1#2#3#4#5%
  {\dosetuppaper{\papierformaat}{\the\papierbreedte}{\the\papierhoogte}%
   \bgroup
   \setbox0=\vbox to #4{\vfill\doinsertobject{#1}}% 
   \setbox0=#2{\box0}%
   \wd0=#3\ht0=#4\dp0=#5\relax
   \box0
   \egroup}

\def\getobject#1%
  {\getvalue{\r!object#1}}

%D We keep track of object references by means of the cross 
%D reference mechanism. Normally, objects are defined before 
%D they are used, but forward referencing sometimes occurs. 
%D 
%D \starttypen
%D \dosetobjectreference {identifier} {reference value}
%D \dogetobjectreference {identifier} \csname
%D \stoptypen

\def\dosetobjectreference#1#2%
  {\checkreferences
   \bgroup
   \locationfalse % we don't want this to be a location 
   \textreference[\r!driver#1]{#2}%
   \egroup
   \setxvalue{\r!cross\r!driver#1}{\rt!cross{}{0}{#2}{0}}}

\def\dogetobjectreference#1#2%
  {\doifrawreferencefoundelse{\r!driver#1}
     {\global\let#2=\currenttextreference}
     {\showmessage{\m!references}{30}{[#1]}%
      \global\def#2{0}}}

\protect

\endinput 
