%D \module
%D   [       file=lang-chi,
%D        version=1998.10.10,
%D          title=\CONTEXT\ Language Macros,
%D       subtitle=Chinese,
%D         author=Hans Hagen,
%D           date=\currentdate,
%D    suggestions=Wang Lei,
%D      copyright={PRAGMA / Hans Hagen \& Ton Otten}]
%C
%C This module is part of the \CONTEXT\ macro||package and is
%C therefore copyrighted by \PRAGMA. See mreadme.pdf for
%C details.

\writestatus{loading}{Context Language Macros / Chinese}

%D This module is coded using the \UNICODE\ support built in
%D \CONTEXT. Therefore, \type {\uchar} is used instead of latin
%D characters.

\unprotect

\definesystemconstant {chinese}   \definesystemconstant {cn}

\installlanguage[\s!cn][\c!status=\v!start]

\setupheadtext [\s!cn]      [\v!inhoud=\uchar{196}{191}\uchar{194}{188}]
\setupheadtext [\s!cn]    [\v!tabellen=\uchar{196}{191}\uchar{177}{237}]
\setupheadtext [\s!cn]     [\v!figuren=\uchar{196}{191}\uchar{205}{188}]
\setupheadtext [\s!cn]   [\v!grafieken=Graphics]
\setupheadtext [\s!cn] [\v!intermezzos=Intermezzos]
\setupheadtext [\s!cn]       [\v!index=\uchar{203}{247}\uchar{210}{253}]
\setupheadtext [\s!cn] [\v!afkortingen=Abbreviations]
\setupheadtext [\s!cn]       [\v!logos=Logos]
\setupheadtext [\s!cn]    [\v!eenheden=Units]

\setuplabeltext [\s!cn]      [\v!tabel=\uchar{177}{237} ]
\setuplabeltext [\s!cn]     [\v!figuur=\uchar{205}{188} ]
\setuplabeltext [\s!cn] [\v!intermezzo=Intermezzo ]
\setuplabeltext [\s!cn]    [\v!grafiek=Illustration ]
\setuplabeltext [\s!cn]    [\v!bijlage=]
\setuplabeltext [\s!cn]       [\v!deel={\cnencoding\cnencodedintro,\cnencoding\cnencodedpart}]
\setuplabeltext [\s!cn]  [\v!hoofdstuk={\cnencoding\cnencodedintro,\cnencoding\cnencodedchapter}]
\setuplabeltext [\s!cn]  [\v!paragraaf={\cnencoding\cnencodedintro,\cnencoding\cnencodedsection}]
\setuplabeltext [\s!cn]      [\v!regel=line ]
\setuplabeltext [\s!cn]     [\v!regels=lines ]

\setuplabeltext [\s!cn]             [\v!sub\v!paragraaf=]
\setuplabeltext [\s!cn]       [\v!sub\v!sub\v!paragraaf=]
\setuplabeltext [\s!cn] [\v!sub\v!sub\v!sub\v!paragraaf=]

%D One can specify a split labeltext, as demonstrated in
%D the definition of the \type {part} label. Unfortunately
%D the glyphs of both part depend on the encoding. Therefore,
%D we have an encoding section here.

\def\cnencoding{\enableencoding[\chineseencoding]} % ugly and temporary

\startencoding[gbk]
  \definecommand cnencodedintro   {\uchar{181}{218}}
  \definecommand cnencodedpart    {\uchar{178}{191}\uchar{183}{214}}
  \definecommand cnencodedchapter {\uchar{213}{194}}
  \definecommand cnencodedsection {\uchar{189}{218}}
\stopencoding

\startencoding[big5]
  \definecommand cnencodedintro   {\uchar{178}{196}}
  \definecommand cnencodedpart    {\uchar{179}{161}\uchar{164}{192}}
  \definecommand cnencodedchapter {\uchar{179}{185}}
  \definecommand cnencodedsection {\uchar{184} {96}}
\stopencoding

%D From this definition one can deduce that language, input 
%D encoding, font encoding, and glyph meaning form a pretty 
%D complex four dimensional space.

\startlanguagespecifics[\s!cn]

\stelsectiein[\v!sectionlevel-1][\c!conversie=\s!chinese]
\stelsectiein[\v!sectionlevel-2][\c!conversie=\s!chinese]
\stelsectiein[\v!sectionlevel-3][\c!conversie=\s!chinese]
\stelsectiein[\v!sectionlevel-4][\c!conversie=\s!chinese]
\stelsectiein[\v!sectionlevel-5][\c!conversie=\s!chinese]
\stelsectiein[\v!sectionlevel-6][\c!conversie=\s!chinese]
\stelsectiein[\v!sectionlevel-7][\c!conversie=\s!chinese]

% \stelblokkopjesin[\c!conversie=\s!chinese]

\stoplanguagespecifics

\protect \endinput
