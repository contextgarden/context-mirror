%D \module
%D   [       file=lang-ini,
%D        version=1996.01.25,
%D          title=\CONTEXT\ Language Macros,
%D       subtitle=Initialization,
%D         author=Hans Hagen,
%D           date=\currentdate,
%D      copyright={PRAGMA / Hans Hagen \& Ton Otten}]
%C
%C This module is part of the \CONTEXT\ macro||package and is
%C therefore copyrighted by \PRAGMA. Non||commercial use is
%C granted.

%D This module implements the (for the moment still simple)
%D multi||language support of \CONTEXT, which should not be
%D confused with the multi||lingual interface. This support
%D will be extended when needed.

\writestatus{loading}{Context Language Macros / Initialization}

\unprotect

\startmessages  dutch  library: linguals
  title: taal
      1: afbreekpatronen voor -- geladen
      2: geen afbreekpatronen voor --
      3: afbreekdefinities voor -- geladen
      4: geen afbreekdefinities voor --
      5: afbreekpatronen voor -- niet geladen
      6: taal -- is niet gedefinieerd
      7: taal specifieke opties [--] introduceren een skip van --
      8: taal specifieke opties [--] naadloos toegevoegd
\stopmessages

\startmessages  english  library: linguals
  title: language
      1: patterns for -- loaded
      2: no patterns for --
      3: hyphenations for -- loaded
      4: no hyphenations for --
      5: patterns for -- not loaded
      6: language -- is undefined
      7: language specific options [--] introduce a -- skip
      8: language specific options [--] seamless appended
\stopmessages

\startmessages  german  library: linguals
  title: Sprache
      1: Trennmuster fuer -- geladen
      2: Keine Trennmuster fuer --
      3: Trenndefinitionen fuer -- geladen
      4: Keine Trenndefinitionen fuer --
      5: Trennmuster fuer -- nicht geladen
      6: Sprache -- ist undefiniert
      7: Sprachenspezifische Option [--] fuegt eine Luecke von -- ein
      8: Sprachenspezifische Option [--] nahtlos hinzugefuegt
\stopmessages

% dutch   : \lccode`\'=`\'
% english : \lccode`\'=0
% german  : \lccode`\'=`\'
% french  : \lccode`\'=`\'

%D When loading hyphenation patterns, \TEX\ assign a number to
%D each loaded table, starting with~0. Switching to a specific
%D table is done by assigning the relevant number to the
%D predefined \COUNTER\ \type{\language}. Unfortunately the
%D name of this command suits very well the name of the
%D language switching command we are to define, so let's save
%D this primitive under another name:

\let\normallanguage = \language

%D We keep track of the last loaded patterns by means of a
%D pseudo \COUNTER. This just one of those situations in which
%D we don't want to spent a real one.

\newcounter\loadedlanguage

%D \macros
%D   {currentlanguage}
%D   {}
%D
%D Instead of numbers,we are going to use symbolic names for
%D the languages. The current langage is saved in the macro
%D \type{\currentlanguage}.

\let\currentlanguage = \empty

%D \macros
%D   {installlanguage}
%D   {}
%D
%D Hyphenation patterns can only be loaded when the format file
%D is prepared. The next macro takes care of this loading. A
%D language is specified with
%D
%D \showsetup{\y!installlanguage}
%D
%D When \type{\c!status} equals \type{\v!start}, both patterns
%D and additional hyphenation specifications are loaded. These
%D files are seached for on the system path and are to be
%D named:
%D
%D \starttypen
%D \f!languageprefix-identifier.\f!patternsextension
%D \f!languageprefix-identifier.\f!hyhensextension
%D \stoptypen
%D
%D The \type{\c!spatiering} specifies how the spaces after
%D punctuation has to be handled. English is by tradition more
%D tolerant to inter||sentence spacing than other languages.
%D
%D This macro also defines \type{\identifier} as a shortcut
%D switch to the language. Furthermore the command defined as
%D being language specific, are executed. With
%D \type{\c!default} we can default to another language
%D (patterns) at format generation time. This default language
%D is overruled when the appropriate patterns are loaded (some
%D implementations support run time addition of patterns to a
%D preloaded format).

\def\dodoinstalllanguage#1%
  {\doifundefined{#1}{\setvalue{#1}{\language[#1]}}%
   \expanded{\noexpand\uppercase{\noexpand\edef\noexpand\ascii{#1}}}%
   \doifundefined{\ascii}{\setvalue{\ascii}{\language[#1]}}}      

\def\doinstalllanguage[#1][#2]%
  {\doifinstringelse{=}{#2}
     {\doifdefinedelse{\??la#1\c!nummer}%
        {\getparameters[\??la#1][#2]}
        {\setvalue{\l!prefix!#1}{#1}%
         \setevalue{\??la#1\c!nummer}{\loadedlanguage}%
         \increment\loadedlanguage
         \dodoinstalllanguage{#1}%
         \getparameters
           [\??la#1]
           [\s!lefthyphenmin=2,
            \s!righthyphenmin=2,
            \c!spatiering=\v!opelkaar,
            \c!leftsentence=---,
            \c!rightsentence=---,
            \c!leftsubsentence=---,
            \c!rightsubsentence=---,
            \c!leftquote=\upperleftsinglesixquote,
            \c!rightquote=\upperrightsingleninequote,
            \c!leftquotation=\upperleftdoublesixquote,
            \c!rightquotation=\upperrightdoubleninequote,
            \c!datum={\v!jaar,\ ,\v!maand,\ ,\v!dag},
            \c!status=\v!stop,
            \s!done=\v!nee,
            \c!default=#1,
            #2]}%
      \language=\getvalue{\??la#1\c!nummer}\relax
      \doifelsevalue{\??la#1\c!status}{\v!start}
        {\doifelsevalue{\??la#1\s!done}{\v!nee}
           {\readsysfile{\f!languageprefix#1.\f!patternsextension}
              {\getparameters[\??la#1][\s!done=\v!ja,\c!default=#1]%
               \showmessage{\m!linguals}{1}{#1}}
              {\showmessage{\m!linguals}{2}{#1}}%
            \readsysfile{\f!languageprefix#1.\f!hyphensextension}
              {\showmessage{\m!linguals}{3}{#1}}
              {\showmessage{\m!linguals}{4}{#1}}}
           {\showmessage{\m!linguals}{1}{#1}%
            \showmessage{\m!linguals}{3}{#1}}}
        {\showmessage{\m!linguals}{5}{#1}}}
     {\setvalue{\l!prefix!#1}{#2}%
      \dodoinstalllanguage{#1}}%
   }% \language[#1]} gave unwanted side effect of loading language specifics 

\def\installlanguage%
  {\dodoubleargument\doinstalllanguage}

%D When the second argument is a language identifier, a 
%D synonym is created. This feature is present because we 
%D used dutch mnemonics in the dutch version, but nowadays 
%D conform a standard.  

%D \macros 
%D   {setuplanguage}
%D
%D Quick and dirty, but useful: 
%D
%D \showsetup{\y!setuplanguage}

\def\setuplanguage[#1]%
  {\dodoubleargument\getparameters[\??la#1]}

%D The values \type {\c!leftsentence} and \type
%D {\c!rightsentence} can be (and are) used to implement
%D automatic subsentence boundary glyphs, like in {\fr
%D |<|french guillemots|>|} or {\de |<|german guillemots|>|} or
%D {\nl |<|dutch dashes|>|} like situations. Furthermore \type
%D {\c!leftquotation} and \type {\c!leftquote} come into view
%D \citaat {when we quote} or \citeer {quote} something.

%D \macros
%D  {currentdatespecification}
%D
%D Just to make things easy we can ask for the current date
%D specification by saying:

\def\currentdatespecification%
  {\getvalue{\??la\currentlanguage\c!datum}}

%D This command is not meant for users.

%D Carefull reading of these macros shows that it's legal to
%D say
%D
%D \starttypen
%D \installlanguage [du ] [de]
%D \stoptypen

%D \macros
%D   {language,mainlanguage}
%D   {}
%D
%D Switching to another language (actually another hyphenation
%D pattern) is done with:
%D
%D \starttypen
%D \language[identifier]
%D \stoptypen
%D
%D or with \type{\identifier}. Just to be compatible with
%D \PLAIN\ \TEX, we still support the original meaning, so
%D
%D \starttypen
%D \language=1
%D \stoptypen
%D
%D is a valid operation.
%D
%D \showsetup{\y!language}
%D \showsetup{\y!mainlanguage}
%D
%D Both commands take a predefined language identifier as
%D argument. We can use \type{\mainlanguage[identifier]} for
%D setting the (indeed) main language. This is the language
%D used for translating labels like {\em figure} and {\em
%D table}. The main language defaults to the current language.
%D
%D We take care of local as well as standardized language
%D switching (fr and fa, de and du, but nl and nl).

\def\docomplexlanguage[#1]%
  {\processaction
     [\getvalue{\??la#1\c!default}]
     [        #1=>\normallanguage=\getvalue{\??la#1\c!nummer},
      \s!default=>\normallanguage=\getvalue{\??la#1\c!nummer},
      \s!unknown=>\expanded{\language[\getvalue{\??la#1\c!default}]}]%
   \edef\currentlanguage{#1}%
   \enablelanguagespecifics[#1]%
   \lefthyphenmin =0\getvalue{\??la#1\s!lefthyphenmin}\relax
   \righthyphenmin=0\getvalue{\??la#1\s!righthyphenmin}\relax
   \processaction
     [\getvalue{\??la#1\c!spatiering}]
     [\v!opelkaar=>\frenchspacing,
          \v!ruim=>\nonfrenchspacing,
       \s!unknown=>\frenchspacing]}

\def\complexlanguage[#1]%
  {\doifdefinedelse{\l!prefix!#1}
     {\expanded{\docomplexlanguage[\getvalue{\l!prefix!#1}]}}
     {\showmessage{\m!linguals}{6}{#1}}}

\def\simplelanguage%
  {\normallanguage}

\definecomplexorsimple\language

\let\currentmainlanguage=\empty

\def\mainlanguage[#1]%
  {\edef\currentmainlanguage{#1}} % We expand indeed!

%D \macros
%D   {startlanguagespecifics,enablelanguagespecifics}
%D   {}
%D
%D Each language has its own typographic pecularities. Some of
%D those can be influenced by parameters, others are handled by
%D the interface, but as soon as specific commands come into
%D view we need another mechanism. In the macro that activates
%D a language, we call \type{\enablelanguagespecifics}. This
%D macro in return calls for the setup of language specific
%D macros. Such specifics are defined as:
%D
%D \starttypen
%D \startlanguagespecifics[de]
%D   \installcompoundcharacter "a {\"a}
%D   \installcompoundcharacter "e {\"e}
%D   \installcompoundcharacter "s {\SS}
%D \stoplanguagespecifics
%D \stoptypen
%D
%D Instead of \type{[du]} we can pass a comma separated
%D list, like \type{[du,nl]}. Next calls to this macro add the
%D specifics to the current list.
%D
%D Before we actually read the specifics, we first take some
%D precautions that will prevent spurious spaces to creep into
%D the list.

\def\startlanguagespecifics%
  {\bgroup
   \catcode`\^^I=\@@ignore
   \catcode`\^^M=\@@ignore
   \catcode`\^^L=\@@ignore
   \dostartlanguagespecifics}

%D The main macro looks quite complicated but actually does
%D nothing special. By embedding \type{\do} we can easily
%D append to the lists and also execute them at will. Just to
%D be sure, we check on spurious spaces.

\long\def\dostartlanguagespecifics[#1]#2\stoplanguagespecifics%
  {\egroup
   \long\def\docommando##1%
     {\doifdefinedelse{\??la##1\??la}
        {\long\def\do####1####2####3%
           {\setvalue{\??la####1\??la}{\do{####1}{####2####3}}}%
         \getvalue{\??la##1\??la}{#2}}
        {\setvalue{\??la##1\??la}{\do{##1}{#2}}}%
      \bgroup
      \setbox0=\hbox{\enablelanguagespecifics[##1]}%
      \ifdim\wd0>\!!zeropoint
        \showmessage{\m!linguals}{7}{##1,\the\wd0\space}\wait
      \else
        \showmessage{\m!linguals}{8}{##1}%
      \fi
      \egroup}%
   \processcommalist[#1]\docommando}

%D Enabling them is rather straightforward. We only have to
%D define \type{\do} in such a way that \type{{ }} is removed
%D and the language key is gobbled.

\def\enablelanguagespecifics[#1]%
  {\long\def\do##1##2{##2}%
   \getvalue{\??la#1\??la}}

%D \macros
%D   {leftguillemot,rightguillemot,leftsubguillemot,rightsubguillemot,
%D    ...single...quote,...double...quote}
%D   {}
%D
%D We assign logical names to all kind of quote and sentence
%D boundary characters.

\def\lowerleftsingleninequote  {\char44 }
\def\lowerleftdoubleninequote  {\char44\kern-.1em\char44 }
\def\upperleftsingleninequote  {\char39 }
\def\upperleftdoubleninequote  {\char34\kern-.1em}
\def\upperleftsinglesixquote   {\char96 }
\def\upperleftdoublesixquote   {\char96\kern-.1em\char96 }

\def\lowerrightsingleninequote {\char44 }
\def\lowerrightdoubleninequote {\char44\kern-.1em\char44 }
\def\upperrightsingleninequote {\char39 }
\def\upperrightdoubleninequote {\char34 }
\def\upperrightsinglesixquote  {\char96 }
\def\upperrightdoublesixquote  {\kern-.125em\char92 }

\unexpanded\def\leftguillemot%
  {\dontleavehmode\hbox{\raise.25ex\hbox{$\scriptscriptstyle\ll$}}}

\unexpanded\def\rightguillemot%
  {\hbox{\raise.25ex\hbox{$\scriptscriptstyle\gg$}}}

\unexpanded\def\leftsubguillemot%
  {\dontleavehmode\hbox{\raise.25ex\hbox{$\scriptscriptstyle<$}}}

\unexpanded\def\rightsubguillemot%
  {\hbox{\raise.25ex\hbox{$\scriptscriptstyle>$}}}

%D Just like with subsentence boundary symbols, quotes
%D placement depends on the current language, therefore we show
%D the defaults here.
%D
%D \def\ShowLanguageValues [#1] [#2] #3 #4
%D   {\blanko
%D    \startregelcorrectie
%D    \vbox\bgroup
%D    \language[#1]%
%D    \setbox0=\hbox to \hsize{\hss\bf#2 subsentence symbol and quotes\hss}
%D    \dp0=0pt
%D    \box0
%D    \vskip.5em
%D    \hrule
%D    \vskip.5em
%D    \hbox to \hsize
%D      {\hfil\citaat{#3 #4}\hfil\citeer{#2}\hfil\strut|<||<|#3|>|#4|>|\hfil}
%D    \vskip.5em
%D    \hrule
%D    \egroup
%D    \stopregelcorrectie
%D    \blanko}
%D
%D \ShowLanguageValues [nl] [dutch]      nederlandse zuinigheid
%D \ShowLanguageValues [en] [english]    engelse humor
%D \ShowLanguageValues [de] [german]     duitse degelijkheid
%D \ShowLanguageValues [fr] [french]     franse slag
%D \ShowLanguageValues [sp] [spanish]    spaans benauwd
%D \ShowLanguageValues [it] [italian]    italiaanse ...
%D \ShowLanguageValues [da] [danish]     deense ...
%D \ShowLanguageValues [pt] [portuguese] portugese ...
%D \ShowLanguageValues [sv] [swedish]    zweedse ...
%D \ShowLanguageValues [pl] [polish]     poolse ...
%D \ShowLanguageValues [fi] [finnish]    finse ...
%D \ShowLanguageValues [af] [afrikaans]  afrikaanse ...
%D \ShowLanguageValues [no] [norwegian]  noorse ...
%D \ShowLanguageValues [tr] [turkish]    turks fruit

%D We support a lot of languages. These are specified and 
%D loaded in separate files, according to their roots. Here 
%D we only take care of (postponed) setting of the current 
%D language. 
%D 
%D \unprotect
%D \plaatstabel{The germanic languages (\type{lang-ger})}
%D \starttabel[||||]
%D \HL 
%D \NC \bf mnemonic \NC \bf language \NC \bf group \NC\SR
%D \HL 
%D \NC \s!nl        \NC dutch        \NC germanic  \NC\FR  
%D \NC \s!en        \NC english      \NC germanic  \NC\MR  
%D \NC \s!de        \NC german       \NC germanic  \NC\MR    
%D \NC \s!da        \NC danish       \NC germanic  \NC\MR    
%D \NC \s!sv        \NC swedish      \NC germanic  \NC\MR    
%D \NC \s!af        \NC afrikaans    \NC germanic  \NC\MR    
%D \NC \s!no        \NC norwegian    \NC germanic  \NC\LR    
%D \HL 
%D \stoptabel
%D \protect
%D 
%D \unprotect
%D \plaatstabel{The italic languages (\type{lang-ita})}
%D \starttabel[||||]
%D \HL 
%D \NC \bf mnemonic \NC \bf language \NC \bf group \NC\SR
%D \HL 
%D \NC \s!fr        \NC french       \NC italic    \NC\FR
%D \NC \s!sp        \NC spanish      \NC italic    \NC\MR    
%D \NC \s!it        \NC italian      \NC italic    \NC\MR    
%D \NC \s!pt        \NC portuguese   \NC italic    \NC\LR    
%D \HL 
%D \stoptabel
%D \protect
%D 
%D \unprotect
%D \plaatstabel{The slavic languages (\type{lang-sla})}
%D \starttabel[||||]
%D \HL 
%D \NC \bf mnemonic \NC \bf language \NC \bf group \NC \bf file \NC\SR
%D \HL 
%D \NC \s!pl        \NC polish       \NC slavic    \NC lang-sla \NC\SR    
%D \HL 
%D \stoptabel
%D \protect
%D \unprotect
%D 
%D \plaatstabel{The altaic languages (\type{lang-alt})}
%D \starttabel[||||]
%D \HL 
%D \NC \bf mnemonic \NC \bf language \NC \bf group \NC\SR
%D \HL 
%D \NC \s!tr        \NC turkish      \NC altaic    \NC\SR    
%D \HL 
%D \stoptabel
%D
%D \plaatstabel{The uralic languages (\type{lang-ura})}
%D \starttabel[||||]
%D \HL 
%D \NC \bf mnemonic \NC \bf language \NC \bf group \NC\SR
%D \HL 
%D \NC \s!fi        \NC finnish      \NC uralic    \NC\SR    
%D \HL 
%D \stoptabel
%D \protect

%D We default to the language belonging to the interface. This
%D is one of the few places outside the interface modules where
%D \type{\startinterface} is used. 

%  \language[\s!en] \mainlanguage[\currentlanguage]

\startinterface dutch      \appendtoks \language[\s!nl]\to \everyjob \stopinterface
\startinterface english    \appendtoks \language[\s!en]\to \everyjob \stopinterface
\startinterface german     \appendtoks \language[\s!de]\to \everyjob \stopinterface
\startinterface french     \appendtoks \language[\s!fr]\to \everyjob \stopinterface
\startinterface spanish    \appendtoks \language[\s!sp]\to \everyjob \stopinterface
\startinterface italian    \appendtoks \language[\s!it]\to \everyjob \stopinterface
\startinterface danish     \appendtoks \language[\s!da]\to \everyjob \stopinterface
\startinterface portuguese \appendtoks \language[\s!pt]\to \everyjob \stopinterface
\startinterface swedish    \appendtoks \language[\s!sv]\to \everyjob \stopinterface
\startinterface polish     \appendtoks \language[\s!pl]\to \everyjob \stopinterface
\startinterface finish     \appendtoks \language[\s!fi]\to \everyjob \stopinterface
\startinterface afrikaans  \appendtoks \language[\s!af]\to \everyjob \stopinterface
\startinterface norwegian  \appendtoks \language[\s!no]\to \everyjob \stopinterface
\startinterface turkish    \appendtoks \language[\s!tr]\to \everyjob \stopinterface

\appendtoks \mainlanguage[\currentlanguage] \to \everyjob

\protect

\endinput
