%D \module 
%D   [       file=main-001,
%D        version=1997.03.31,
%D          title=\CONTEXT\ Core Macros,
%D       subtitle=1A (to be split),
%D         author=Hans Hagen,
%D           date=\currentdate,
%D      copyright={PRAGMA / Hans Hagen \& Ton Otten}]
%C
%C This module is part of the \CONTEXT\ macro||package and is
%C therefore copyrighted by \PRAGMA. See mreadme.pdf for
%C details.

%D This module is still to be split and documented. 

\writestatus{loading}{Context Core Macros (1)}

\newevery \everybodyfont \Everybodyfont % just to be sure

\appendtoks \presetnormallineheight \to \everybodyfont
\appendtoks \setnormalbaselines     \to \everybodyfont
\appendtoks \setstrut               \to \everybodyfont
\appendtoks \settopskip             \to \everybodyfont
\appendtoks \setmaxdepth            \to \everybodyfont
\appendtoks \stelinspringenin       \to \everybodyfont
\appendtoks \stelblankoin           \to \everybodyfont
\appendtoks \stelwitruimtein        \to \everybodyfont
%\appendtoks\setupfootnotes         \to \everybodyfont
\appendtoks \stelspatieringin       \to \everybodyfont % nieuw
\appendtoks \setdisplayskips        \to \everybodyfont % nieuw

% \appendtoks .. \to \everypar
% \appendtoks .. \to \everypar
% \appendtoks .. \to \everypar

% kan elders ook worden gebruikt i.i.g ongeveer
% \v!tekst EN \c!tekst etc checken

\unprotect

\def\gobbleassigndimen#1\\{}

\def\assigndimen#1#2%
  {\afterassignment\gobbleassigndimen#1=#2\!!zeropoint\\}

\protect

\unprotect

\startmessages  dutch  library: systems
  title: systeem
      1: laden hulpfile uitgesteld (typemode)
      2: -- geladen
      3: probeer LaTeX eens
      4: commando -- is al gedefinieerd
      5: macro's uit module -- geladen
      6: geen macro's in module -- gevonden
      7: macro's uit module -- reeds geladen
      8: nieuwe versie hulpfile, tweede run nodig
      9: -- niet gevonden/geplaatst
     10: gebruik geen em in --
     11: aanmaken basale hulpfile
     12: de hulpfile is niet gesorteerd, gebruik texutil
     13: markering -- gedefinieerd --
     14: geforceerde paginaovergang in lijst voor --
     15: wegschrijven buffer --
     16: inlezen buffer --
     17: verbatim inlezen buffer --
     18: synoniem -- -- bestaat niet
     19: betekenissen (synoniemen) van -- geladen
     20: betekenissen (sorteren) van -- geladen
     21: de hulpfile is niet geladen
     22: gebruik een goede hulpfile
     23: -- gearrangeerd op --
     24: Plaatsblokken
     25: Verwijzingen
     26: Registers
     27: Versie
\stopmessages

\startmessages  english  library: systems
  title: system
      1: loading utility-file postponed (typemode)
      2: -- loaded
      3: try LaTeX
      4: command -- is already defined
      5: macros of module -- loaded
      6: no macros found in module --
      7: macros of module -- already loaded
      8: new version of utility file, second pass needed
      9: -- not found/processed
     10: don't use em in --
     11: building simple util
     12: the utility-file is not sorted, use texutil
     13: mark -- defined --
     14: forced newpage in list at --
     15: saving buffer --
     16: typesetting buffer --
     17: typesetting verbatim buffer --
     18: synonym -- -- does not exist
     19: meaning (synonyms) of -- loaded
     20: meaning (sorts) of -- loaded
     21: no utility data is loaded
     22: use a valid utilityfile
     23: -- arranged at --
     24: Floatblocks
     25: References
     26: Registers
     27: Version
\stopmessages

\startmessages  german  library: systems
  title: system
      1: Laden der Hilfsdatei verschoben (tippenmodus)
      2: -- geladen
      3: Versuche LaTeX
      4: Befehl -- ist bereits definiert
      5: Makros aus Modul -- geladen
      6: Keine Makros in Modul -- gefunden
      7: Makros aus Modul -- bereits geladen
      8: Neue Version der Hilfsdatei, zweiter Durchlauf benoetigt
      9: -- nicht gefunden/verarbeitet
     10: Benutzte kein em in --
     11: Erstelle einfache Hilfdatei
     12: Die Hilfdatei ist nicht sortiert, verwende texutil
     13: Beschriftung -- definiert --
     14: Erzwungendes Seitenumbruch in Liste bei --
     15: Speichere Buffer --
     16: Setzte Buffer --
     17: Setzte tippen-Buffer --
     18: Synonym -- -- existiert nicht
     19: Bedeutung (synonyme) von -- geladen
     20: Bedeutung (sortieren) von -- geladen
     21: Die Hilfsdatei ist nicht geladen
     22: Benoetige gueltige Hilfsdateie
     23: -- angeordnet auf --
     24: Fliessbloecke
     25: Referenzen
     26: Register
     27: Version
\stopmessages

\startmessages  czech  library: systems
  title: system
      1: nacteni pomocneho souboru odlozeno (typemode)
      2: -- nacteno
      3: zkuste LaTeX
      4: prikaz -- je jiz definovan
      5: makra z -- nactena
      6: zadna makra v -- nenalezena
      7: makra z -- jsou jiz nactena
      8: nova verze pomocneho souboru, je treba druheho behu
      9: -- nenalezeno/nezpracovano
     10: nepouzivejte em v --
     11: vytvarim jednoduchy pomocny soubor
     12: pomosny soubor neni setriden, pouzijte texutil
     13: znacka -- definovana --
     14: vynucena nova stranka v seznamu na --
     15: uklada se buffer --
     16: sazi se buffer --
     17: sazi se doslovny (verbatim) buffer --
     18: synonymum -- -- neexistuje
     19: vyznam (synonyma) -- nacten
     20: vyznam (trideni) -- nacten
     21: pomocny soubor necten
     22: pouzijte platny pomocny soubor
     23: -- upraveno na --
     24: plovouci bloky
     25: reference
     26: registry
     27: verze
\stopmessages

\startmessages  italian  library: systems
  title: sistema
      1: caricamento dei file supplementari posticipato (typemode)
      2: -- caricato
      3: provare LaTeX
      4: comando -- gi� definito
      5: macro del modulo -- caricate
      6: nessuna macro trovata nel modulo --
      7: macro del modulo -- gi� caricate
      8: nuova versione del file supplementare, seconda passata necessaria
      9: -- non trovato/elaborato
     10: non usare em in --
     11: costruzione di un semplice supplemento
     12: file di supplemento non ordinato, usare texutil
     13: marcatura -- definita --
     14: nuova pagina obbligata in lista a --
     15: salvataggio del buffer --
     16: composizione del buffer --
     17: composizione verbatim del buffer --
     18: sinonimo -- -- non esistente
     19: significato (sinonimi) di -- caricato
     20: significato (specie) di -- caricato
     21: nessuna informazione supplementare caricata
     22: usare un file supplementare valido
     23: -- sistemato a --
     24: Oggetti mobili
     25: Riferimenti
     26: Registri
     27: Versione
\stopmessages

\startmessages  norwegian  library: systems
  title: system
      1: innlesning av hjelpefila utsatt (typemode)
      2: -- er lest inn
      3: fors�ker LaTeX
      4: kommando -- er allerede definert
      5: makroene i modul  -- er lest inn
      6: ingen makroer funnet i modul ---
      7: makroene i modul -- er allerede lest inn
      8: ny versjon av hjelpefil, andre gjennomkj�ring n�dvendig
      9: -- ikke funnet/behandlet
     10: ikke bruk em i --
     11: lager enkel hjelpefil
     12: hjelpefila er ikke sortert, bruk texutil
     13: markering -- definert --
     14: tvunget sideskift i liste ved --
     15: lagrer Buffer --
     16: tegnsetter buffer --
     17: tegnsetter verbatim-buffer --
     18: synonym -- -- eksisterer ikke
     19: betydning (synonymer) av -- er lest inn
     20: betydning (sorterer) av -- er lest inn
     21: hjelpefila er ikke lest inn
     22: bruk en gyldig hjelpefil
     23: -- arrangert p� --
     24: Flytblokker
     25: Referanser
     26: Registere
     27: Versjon
\stopmessages

\startmessages  romanian  library: systems
  title: sistem
      1: se incarca utilitarul-fisierul este amanat (typemode)
      2: -- s-a incarcat
      3: incercati LaTeX
      4: comanda -- este deja definita
      5: macro-urile din modulul -- s-au incarcat
      6: nu s-au gasit macro-uri in modulul --
      7: macro-urile din modulul -- s-au incarcat deja
      8: o noua versiune de fisier utilitar, este necesara o noua trecere
      9: -- nu este gasit/procesat
     10: nu folositi em in --
     11: se creeaza un utilitar simplu
     12: fisierul utilitar nu este sortat, folositi texutil
     13: marcajul -- definit --
     14: s-a fortat trecere pa pagina noua in lista la --
     15: buffer salvat --
     16: buffer-ul -- s-a cules
     17: se culege buffer-ul verbatim --
     18: sinonimul -- -- nu exista
     19: intelesul (sinonimele) pentru -- incarcat
     20: intelesul (ordinea) pentru -- incarcat
     21: nici o data utilitara nu este incarcata
     22: folositi un fisier utilitar valid
     23: -- aranjat la --
     24: Blocuri
     25: Referinte
     26: Registri
     27: Versiune
\stopmessages

\startmessages  dutch  library: floatblocks
  title: plaatsblokken
      1: -- hernummerd / -- => --
      2: -- bewaard
      3: -- verplaatst
      4: -- geplaatst
      5: volgorde aangepast
      6: maximaal -- boven
      7: maximaal -- onder
      8: minder dan -- regels
      9: volgorde verstoord
     10: -- begrensd
     11: geen blok opgegeven
     12: niet gedefinieerd
\stopmessages

\startmessages  english  library: floatblocks
 title: floatblocks
      1: -- renumbered / -- => --
      2: -- saved
      3: -- moved
      4: -- placed
      5: order adapted
      6: n of top floats limited to --
      7: n of bottom floats limited to --
      8: less than -- lines
      9: order disturbed
     10: -- limited
     11: no block given
     12: undefined
\stopmessages

\startmessages  german  library: floatblocks
 title: Gleitobjektbloecke
      1: -- neu nummeriert / -- => --
      2: -- gespeichert
      3: -- verschoben
      4: -- plaziert
      5: Reihenfolge angepasst
      6: Anz. der oberen Gleitobjekte beschraengt auf --
      7: Anz. der unteren Gleitobjekte beschraengt auf  --
      8: weniger als -- zeilen
      9: Reigenfolge gestoert
     10: -- begrenzt
     11: kein Block gegeben
     12: undefiniert
\stopmessages

\startmessages  czech  library: floatblocks
 title: plovouciobjekty
      1: -- precislovano / -- => --
      2: -- ulozeno
      3: -- presunuto
      4: -- umisteno
      5: poradi prizpusobeno
      6: pocet hornich plovoucich objektu je omezen na --
      7: pocet spodnich plovoucich objektu je omezen na --
      8: radku je mene nez --
      9: poradi naruseno
     10: -- omezeno
     11: nedan zadny blok
     12: nedefinovano
\stopmessages

\startmessages  italian  library: floatblocks
 title: oggetti mobili
      1: -- rinumerato / -- => --
      2: -- salavto
      3: -- mosso
      4: -- sistemato
      5: ordine aggiustato
      6: n di top floats limitato a --
      7: n di bottom floats limitato a --
      8: meno di -- righe
      9: ordine disturbato
     10: -- limitato
     11: nessun oggetto specificato
     12: non definito
\stopmessages

\startmessages  norwegian  library: floatblocks
 title: flytblokker
      1: -- renummerert / -- => --
      2: -- lagret
      3: -- flyttet
      4: -- plassert
      5: rekkef�lge tilpasset
      6: maksimalt -- flytblokker �verst
      7: maksimalt -- flytblokker nederst
      8: mindre enn -- linjer
      9: rekkef�lge endret
     10: -- begrenset
     11: ingen blokk oppgitt
     12: udefinert
\stopmessages

\startmessages  romanian  library: floatblocks
 title: Blocuri
      1: -- renumerotat / -- => --
      2: -- salvat
      3: -- mutat
      4: -- plasat
      5: ordinea adaptata
      6: nr. cadrelor de sus limitat la  --
      7: nr. blocurilor de jos limitat la --
      8: mai putin de -- linii
      9: ordinea deranjata
     10: -- limitat
     11: nu este dat nici un bloc
     12: nedefinit
\stopmessages

\startmessages  dutch  library: layouts
  title: layout
      1: teksthoogte aangepast met -- op pagina --
      2: -- maal uitgestelde tekst tussengevoegd
      3: -- maal tekst plaatsen uitstellen
      4: margeblokken actief
      5: margeblokken inactief
      6: subpagina reeks -- verwerkt (aantal --)
      7: beeldmerken berekenen
      8: achtergronden berekenen
     10: -- en -- tellen niet op tot 1.0
     11: interlinie -- niet toegestaan in gridmode
\stopmessages

\startmessages  english  library: layouts
  title: layout
      1: textheight adapted with -- at page --
      2: -- times postponed text placed
      3: -- times text postponed
      4: marginblocks active
      5: marginblocks inactive
      6: subpage set -- processed (size --)
      7: calculating logospace
      8: calculating backgrounds
     10: -- and -- don't add up to 1.0
     11: spacing -- not permitted in gridmode
\stopmessages

\startmessages  german  library: layouts
  title: Layout
      1: Texthoehe angepasst mit -- auf Seite --
      2: -- mal verschobener Text plaziert
      3: -- mal Text verschoben
      4: marginalbloecke aktiv
      5: marginalbloecke inaktiv
      6: Unterseitenfolge -- verarbeitet (Groesse --)
      7: berechne Platz des Logo
      8: berechne Hintergrund
     10: -- und -- ergeben zusammen nicht 1.0
     11: Zwischenraum -- nicht im Grittermoduserlau
\stopmessages

\startmessages  czech  library: layouts
  title: layout
      1: vyska textu prizpusobena s -- na strane --
      2: -- krat odlozeny text umisten
      3: -- krat text odlozen
      4: okrajove bloky aktivni
      5: okrajove bloky neaktivni
      6: sada stran -- zpracovana (velikost --)
      7: pocita se misto pro logo
      8: pocita se pozadi
     10: -- a -- nedava dohromady 1.0
     11: svisla mezera -- neni povolena v pevnem radkovem rejstriku
\stopmessages

\startmessages  italian  library: layouts
  title: layout
      1: altezza del testo adattata con -- a pagina --
      2: posizionato testo posticipato -- volte
      3: testo posticipato -- volte
      4: blocchi in margine attivi
      5: blocchi in margine inattivi
      6: gruppo di sottopagine -- elaborato (dimensione --)
      7: calcolo dello spazio per logo
      8: calcolo dello sfondo
     10: -- e -- non sommano a 1.0
     11: spaziatura -- non permessa in modo griglia
\stopmessages

\startmessages  norwegian  library: layouts
  title: layout
      1: teksth�yde tilpasset med -- p� side --
      2: -- ganger forskj�vet tekst plassert
      3: -- ganger tekst forskj�vet 
      4: margblokker aktive
      5: margblokker inaktive
      6: delside sett -- behandlet (st�rrelse --)
      7: beregner plass for logo
      8: beregner bakgrunn
     10: -- og -- er ikke 1.0 til sammen
     11: mellomrom -- ikke tillatt i gridmodus
\stopmessages

\startmessages  romanian  library: layouts
  title: aranjamente
      1: textheight adaptat cu -- la pagina --
      2: textul amanat de -- ori a fost plasat
      3: textul amanat de -- ori 
      4: blocuri marginale active
      5: blocuri marginale inactive
      6: setul -- de subpagini procesat (dimensiunea --)
      7: se calculeaza spatiul pentru logo
      8: se calculeaza fundalurile
     10: -- si -- nu se adauga pana la 1.0
     11: spatierea -- nu este permisa in gridmode
\stopmessages

% \CONTEXTtrue % Now we know that we can use ConTeXt commands.

% \def\teststatus{stop}
%
% \def\doiftrue  {\iftrue}
% \def\doiffalse {\iffalse}
%
% \def\setstatus#1#2%
%   {\doifelse{\getvalue{#1\c!status}}{\v!start}
%      {\let#2=\doiftrue}
%      {\let#2=\doiffalse}}
%
% \setstatus{test}\iftest
%
% \iftest
%   \message{JA}
% \else
%   \message{NEE}
% \fi

\def\convertexpanded#1#2#3% watch the double \v!ja expansion ! 
  {\ExpandFirstAfter\processaction
     [\getvalue{#1\c!expansie}]
     [       \v!ja=>{{\honorunexpanded\xdef\@@globalexpanded{#2}%
                      \xdef\@@globalexpanded{\@@globalexpanded}}%
                     \dodoglobal\convertcommand\@@globalexpanded\to#3},
       \v!commando=>{\dodoglobal\convertcommand #2\to#3},
        \s!default=>{\dodoglobal\convertargument#2\to#3},
        \s!unknown=>{\dodoglobal\convertargument#2\to#3}]}

% om problemen te voorkomen:
%
% \ascii   => \@@ascii@@
% \asciiA  => \@@ascii@@A
% \asciiB  => \@@ascii@@B

% Nodig i.v.m. inspringen eerste alineas

\def\explicithmode%
  {\unhbox\voidb@x}

% Nodig voor gebruikers

\def\geentest{\donottest}

% Dit moet nog ergens een plaats krijgen:

\def\stelfactorenin%
  {\stelwitruimtein
   \stelblankoin
   \settopskip
   \setmaxdepth}

% Nog doen:
%
%  \goodbreak -> \allowbreak en \dosomebreak{..} in koppen
%
% bij koppen zowieso: \blanko[reset]

% Nog in commando verwerken:
%
% \voorkeur � la \blanko
%
% Om ongewenste witruimte te voorkomen kan met \dosomebreak{\break}
% een \penalty v��r witruimte worden geplaatst.

\def\removelastskip% a redefinition of plain 
  {\ifvmode\ifdim\lastskip=\z@\else\vskip-\lastskip\fi\fi}

\def\dosomebreak#1%
  {\skip0=\lastskip
   \removelastskip
   %\type{#1}%
   #1\relax
   \ifdim\skip0=\!!zeropoint
   \else
     \vskip\skip0
   \fi}

% beter, vooral in \vbox; nog in \pagina toepassen s!

\def\doifoutervmode#1%
  {\ifvmode\ifinner\else#1\fi\fi}

\def\dosomebreak#1%
  {\doifoutervmode
     {\skip0=\lastskip
      \removelastskip
      %\leavevmode\type{#1}%
      #1\relax
      \ifdim\skip0=\!!zeropoint % else interference with footnotes
      \else
        \vskip\skip0
      \fi}}

% Idem:
%
% \springin

%\def\noindentation% vervallen
%   {\EveryPar
%     {\ifdim\parindent=\!!zeropoint
%      \else
%        \bgroup
%        \setbox0=\lastbox
%        \egroup
%      \fi
%      \EveryPar{}}}

\newif\ifindentation \indentationtrue  % documenteren, naar buiten

\let\checkindentation=\relax

\def\donoindentation%
  {\ifdim\parindent=\!!zeropoint
   \else
     \bgroup
     \setbox0=\lastbox
     \egroup
   \fi}

\def\noindentation% made global
  {\ifinpagebody \else
     \global\indentationfalse
     \gdef\checkindentation%
       {\donoindentation
        \gdef\checkindentation{\global\indentationtrue}}%
   \fi}

\def\nonoindentation% bv bij floats
  {\ifinpagebody \else
     \global\indentationtrue
     \gdef\checkindentation{\global\indentationtrue}%
   \fi}

\def\indentation%
  {\ifvmode \ifdim\parindent=\!!zeropoint \else
     % was : \hskip\parindent 
     % can be: \indent  
     % but we test: 
     \noindent\hskip\parindent 
   \fi \fi}

% vergeten

\def\forgeteverypar%
  {\everypar{}}

\def\forgeteverypar%
  {\everypar{\the\neverypar}}

\def\forgetparindent%
  {\forgeteverypar
   \indentfirstparagraphtrue % recently added
   \stelinspringenin[\v!geen]}

\def\forgetparskip%
  {\stelwitruimtein[\v!geen]}

\def\forgetbothskips%
  {\tolerance=1500
   \leftskip\!!zeropoint
   \rightskip\!!zeropoint\relax}

\def\forgetspacing%
  {\emergencystretch\!!zeropoint\relax}

\def\forgetall%
  {\the\everyforgetall}

\appendtoks \let\par\endgraf    \to \everyforgetall % i.v.m. getpar etc
\appendtoks \notragged          \to \everyforgetall 
\appendtoks \forgetparskip      \to \everyforgetall  
\appendtoks \forgetparindent    \to \everyforgetall 
\appendtoks \forgetbothskips    \to \everyforgetall 
\appendtoks \forgetspacing      \to \everyforgetall % i.v.m. funny spacing in pagebody
\appendtoks \everypar\emptytoks \to \everyforgetall % indeed!

\def\localvbox#1#%
  {\vbox#1\bgroup
     \forgetparskip
     \setlocalhsize
     \hsize=\localhsize
     \forgetparindent
     \forgetbothskips
     \forgeteverypar
     \let\next=}

% ach ja

\unexpanded\def\dostartattributes#1#2#3%
  {\begingroup  % geen \bgroup, anders in mathmode lege \hbox
   \doifdefinedelse{#1#2}
     {\def\fontattribute{\getvalue{#1#2}}}
     {\let\fontattribute=\empty}%
   \doifdefinedelse{#1#3}
     {\def\colorattribute{\getvalue{#1#3}}}
     {\let\colorattribute=\empty}%
   \startcolor[\colorattribute]%
   \@EA\doconvertfont\@EA{\fontattribute}}

\unexpanded\def\dostopattributes%
  {\stopcolor
   \endgroup}

\unexpanded\def\doattributes#1#2#3#4%
  {\dostartattributes{#1}{#2}{#3}{#4}\dostopattributes}

% kan vaker worden toegepast:

\newskip\leftskipadaption

\def\doadaptleftskip#1%
  {\dosetleftskipadaption{#1}%
   \advance\leftskip by \leftskipadaption}

\def\dosetleftskipadaption#1%
  {\leftskipadaption\!!zeropoint
   \processaction[#1] % \ExpandFirstAfter
     [\v!standaard=>\leftskipadaption=
                    \ifdim\voorwit=\!!zeropoint\@@sllinks\else\voorwit\fi,
             \v!ja=>\leftskipadaption=
                    \ifdim\voorwit=\!!zeropoint\@@sllinks\else\voorwit\fi,
            \v!nee=>,
        \s!unknown=>\leftskipadaption=#1]}

\def\herhaal            {\dorepeat}
\def\herhaler           {\repeater}
\def\herhaalmetcommando {\dorepeatwithcommand}

% This permits things like ^\index{hans}^, where hans is
% duplicated in the text.

\newif\ifduplicate

\bgroup
\gdef\checkduplication%   in line with Knuth
  {\ifmmode
     \def\next{^}%
   \else
     \let\next=\startduplication
   \fi
   \next}
\gdef\insideduplication%
  {\ifmmode
     \def\next{^}%
   \else
     \let\next=\egroup
   \fi
   \next}
\catcode`\^=\@@active
\gdef\enableduplication%
  {\catcode`\^=\@@active
   \let^=\checkduplication}
\gdef\disableduplication%
  {\catcode`\^=\@@superscript}
\gdef\startduplication%
  {\bgroup
   \duplicatetrue
   \let^=\insideduplication}
\egroup

\def\verbatim#1%
  {\convertargument#1\to\ascii\ascii}

% mogelijke optimalisaties:
%
% \ifx ...\else ...\fi
% \ifvisible ... \fi

% De opbouw van deze file
%
% Deze file bevat naast de verschillende Pragma-Macro's ook
% helpinformatie bij deze macro's en templates. Een blok
% helpinformatie wordt gekenmerkt door een %I.
%
% Een blok kan zijn opgedeeld in pagina's. In dat geval is
% %I vervangen door %P. De eerste regel van een blok bevat
% de titel van de informatie.
%
% Een template (voorgedefinieerde structuur) wordt gekenmerkt
% door %T. Ook hier bevat de eerste regel een titel,
% eventueel gevolgd door een mnemonic.
%
% Zowel de helpinformatie als de templates zijn in het
% programma TeXEdit oproepbaar.
%
% Het programma TeXEdit kan t.z.t. worden ingesteld met
% behulp van de onderstaande, door %S voorafgegane,
% setupcommando's. Vooralsnog is een en ander 'hard' in het
% programma geprogrammeerd.

%S InputFile     \input
%S InputFile     \omgeving    \environment
%S InputFile     \projekt     \project
%S InputFile     \produkt     \product
%S InputFile     \onderdeel   \component
%S
%S CheckStrings  \start  \stop
%S CheckStrings  \begin  \end
%S CheckStrings  \begin  \eind
%S
%S CheckChars    { }
%S CheckChars    [ ]
%S CheckChars    ( )
%S
%S CheckChar     $

% Het <pagina>-karakter (FormFeed), wordt omgezet in \par

\edef\oldlinefeed{\the\catcode`\^^L}

\catcode`\^^L=\oldlinefeed

\catcode`\^^L=\@@endofline

\def\toonstruts%
  {\setteststrut}

% Hieronder volgen enkele instellingen en macro's ten behoeve
% van de interlinie en \strut. De waarden 2.8, 0.07, 0.72 en
% 0.28 zijn ooit eens ontleend aan INRS-TEX en moeten wellicht
% nog eens instelbaar worden.
%
%   \lineheight        : de hoogte van een regel
%   \spacing{getal}    : instellen interlinie
%   \normalbaselines   : instellen regelafstend
%
%   \setstrut          : instellen \strut
%   \setnostrut        : resetten \strut, \endstrut, \begstrut
%
%   \setteststrut      : instellen zichtbare struts
%   \resetteststrut    : instellen onzichtbare struts
%
%   \setfontparameters : instellen na fontset
%
% De hoogte van een regel (\lineheight) is gelijk aan de
% som van de hoogte (\ht) en diepte (\dp) van \strutbox.
%
%   \strut            : denkbeeldig blokje met hoogte en diepte
%
% Een \hbox kan als deze aan het begin van een regel staat
% een breedte \hsize krijgen. Dit is soms te voorkomen met het
% commando \leavevmode. Binnen een \vbox geeft dit echter
% niet altijd het gewenste resultaat, vandaar het commando
%
%   \leaveoutervmode

% Pas op: niet zomaar \topskip en \baselineskip aanpassen
% en zeker niet \widowpenalty. Dit kan ernstige gevolgen
% hebben voor kolommen.
%
% Enige glue kan op zich geen kwaad, echter als blanko=vast,
% dan moet ook de rek 0 zijn. Binnen kolommen is rek ook
% niet bepaald mooi. Een hele kleine waarde (0.025) voldoet,
% omdat een positieve glue eindeloos rekbaar is.

\newdimen\strutdimen
\newdimen\lineheight
\newdimen\openlineheight
\newdimen\openstrutheight
\newdimen\openstrutdepth
\def\strutheightfactor      {.72}
\def\strutdepthfactor       {.28}

\def\baselinefactor         {2.8}
\def\baselinegluefactor     {0}

\def\normallineheight       {\baselinefactor ex}
\def\minimallinedistance    {\lineskip} 

\def\strutheight            {0pt}
\def\strutdepth             {0pt}
\def\strutwidth             {0pt}

\def\spacingfactor          {1}

\def\topskipfactor          {1.0}
\def\maxdepthfactor         {0.5}

\def\systemtopskipfactor    {\topskipfactor}
\def\systemmaxdepthfactor   {\maxdepthfactor}

% De onderstaande definitie wordt in de font-module overruled

\ifx\globalbodyfontsize\undefined
  \newdimen\globalbodyfontsize
  \globalbodyfontsize=12pt
\fi
\ifx\normalizedbodyfontsize\undefined
  \def\normalizedbodyfontsize{12pt}
\fi

% door een \dimen. Dit is geen probleem omdat (1) de default
% korpsgrootte 12pt is en (2) de fonts nog niet geladen zijn
% en de instellingen bij het laden nogmaals plaatsvinden.

\def\topskipcorrection%
  {\ifdim\topskip>\openstrutheight
     \vskip\topskip
     \vskip-\openstrutheight
   \fi
   \vbox{\strut}
   \vskip-\openlineheight}

\def\settopskip% the extra test is needed for the lbr family
  {\topskip=\systemtopskipfactor\globalbodyfontsize
   \ifgridsnapping \else
     \ifr@ggedbottom\!!plus5\globalbodyfontsize\fi
   \fi
   \relax % the skip
   \ifdim\topskip<\strutheightfactor\openlineheight
     \topskip=\strutheightfactor\openlineheight\relax
   \fi}

\def\setmaxdepth%
  {\maxdepth=\systemmaxdepthfactor\globalbodyfontsize}

\def\normalbaselines%
  {\baselineskip\normalbaselineskip
   \lineskip\normallineskip
   \lineskiplimit\normallineskiplimit}

\def\setnormalbaselines%
  {\lineheight=\normallineheight
   \openlineheight=\spacingfactor\lineheight
\openstrutheight=\strutheightfactor\openlineheight
\openstrutdepth =\strutdepthfactor \openlineheight
   \normalbaselineskip=
     \openlineheight
     \!!plus\baselinegluefactor\openlineheight
     \!!minus\baselinegluefactor\openlineheight
   \normallineskip\minimallinedistance\relax % \!!onepoint\relax
   \normallineskiplimit\!!zeropoint\relax
   \normalbaselines}

\def\setspacingfactor#1\to#2\by#3\\%
  {\strutdimen=#2pt\relax
   \strutdimen=#3\strutdimen
   \edef#1{\withoutpt{\the\strutdimen}}}

\def\spacing#1%
  {\ifgridsnapping
     \edef\spacingfactor{1}%
     \showmessage{\m!layouts}{11}{#1}%
   \else
     \edef\spacingfactor{#1}%
   \fi
   \setspacingfactor\systemtopskipfactor\to\topskipfactor\by#1\\%
   \setspacingfactor\systemmaxdepthfactor\to\maxdepthfactor\by#1\\%
   \setnormalbaselines
   \setstrut}

\def\setstrutdimen#1#2#3%              % een strut is n.m maal ex
  {\strutdimen=\normallineheight       % wat niet per se \lineheight
   \strutdimen=#2\strutdimen           % is omdat een strut lokaal
   \strutdimen=#3\strutdimen           % kan afwijken van de globale
   \edef#1{\the\strutdimen}}           % strut

% plain definition:
%
% \def\strut{\relax\ifmmode\copy\strutbox\else\unhcopy\strutbox\fi}
%
% could be: 
%
% \def\strut{\relax\ifmmode\copy\else\unhcopy\fi\strutbox}

\let\normalstrut=\strut 

% The double \hbox construction enables us to \backtrack
% boxes.

% \def\setstrut%
%   {\setstrutdimen\strutheight\strutheightfactor\spacingfactor
%    \setstrutdimen\strutdepth \strutdepthfactor \spacingfactor
%    \let\strut=\normalstrut
%    \setbox\strutbox=\normalhbox
%      {\normalhbox
%         {\vrule
%            \!!width  \strutwidth
%            \!!height \strutheight
%            \!!depth  \strutdepth
%            \normalkern-\strutwidth}}}

\def\setstrut%
  {\setstrutdimen\strutheight\strutheightfactor\spacingfactor
   \setstrutdimen\strutdepth \strutdepthfactor \spacingfactor
   \dosetstrut}

\def\setcharstrut#1%
  {\setbox\strutbox=\hbox{#1}% 
   \edef\strutheight{\the\ht\strutbox}%
   \edef\strutdepth {\the\dp\strutbox}%
   \dosetstrut}

\def\setcapstrut% could be M, but Q has descender
  {\setcharstrut{Q}} 

%D Centered looks nicer: 

\def\dosetstrut%
  {\let\strut=\normalstrut
   \setbox\strutbox=\normalhbox
     {\normalhbox to \!!zeropoint
        {% \hss % new, will be option 
         \vrule
           \!!width \strutwidth
           \!!height\strutheight
           \!!depth \strutdepth
         \hss}}}
%D Sometimes a capstrut comes in handy
%D
%D \starttabulatie[|Tl|l|l|]
%D \NC yes          \NC normal strut               \NC {\toonstruts\setupstrut[ja]\strut}  \NC \NR 
%D \NC no           \NC no strut                   \NC {\toonstruts\setupstrut[nee]\strut}  \NC \NR 
%D \NC kap          \NC a capital strut (i.e. Q)   \NC {\toonstruts\setupstrut[kap]\strut} \NC \NR 
%D \NC A B \unknown \NC a character strut (e.g. A) \NC {\toonstruts\setupstrut[A]\strut}   \NC \NR 
%D \NC              \NC a normal strut             \NC {\toonstruts\setupstrut\strut}      \NC \NR 
%D \stoptabulatie

\def\setupstrut%
  {\dosingleempty\dosetupstrut}

\def\dosetupstrut[#1]% yet undocumented 
  {\processaction
     [#1]
     [     \v!ja=>\setstrut,
          \v!nee=>\setnostrut,
          \v!kap=>\setcapstrut,
      \s!default=>\setstrut,
      \s!unknown=>\setcharstrut{\commalistelement}]}

\def\setteststrut%
  {\def\strutwidth{.8pt}%
   \setstrut}

\def\begstrut%
  {\relax\ifdim\ht\strutbox=\!!zeropoint\else
     \strut
     \normalpenalty\!!tenthousand
     \normalhskip\!!zeropoint
     \ignorespaces
   \fi}

\def\endstrut%
  {\relax\ifhmode\ifdim\ht\strutbox=\!!zeropoint\else
     \unskip\unskip\unskip
     \normalpenalty\!!tenthousand
     \normalhskip\!!zeropoint
     \strut
   \fi\fi}

\def\setnostrut%
  {\setbox\strutbox=\normalhbox{\normalhbox{}}%
   \let\strut=\empty
   \let\endstrut=\empty
   \let\begstrut=\empty}

% unsave:
%
% \def\pseudostrut%
%   {\bgroup
%    \setnostrut
%    \normalstrut
%    \egroup}
%
% try:
%
% \startchemie
%   \chemie[ONE,Z0,SB15,MOV1,SB15,Z0][C,C]
% \stopchemie
%
% so:

\def\pseudostrut%
  {\noindent} % better: \dontleavehmode

\let\pseudobegstrut\pseudostrut

\def\pseudoendstrut% removes all kind of signals 
  {\ifhmode\unskip\unskip\unskip\unskip\fi}

\def\resetteststrut%
  {\let\strutwidth=\!!zeropoint
   \setstrut}

\def\setfontparameters%
  {\the\everybodyfont}

%D Handy: 

\def\baselinedistance{\the\lineheight}

%D We need \type{\normaloffinterlineskip} because the new
%D definition contains an assignment, and |<|don't ask me
%D why|>| this assignment gives troubles in for instance the
%D visual debugger.

\ifx\undefined\normaloffinterlineskip
  \let\normaloffinterlineskip=\offinterlineskip % knuth's original
\fi

\def\offinterlineskip%
  {\ifdim\baselineskip>\!!zeropoint
     \edef\oninterlineskip%
       {\baselineskip=\the\baselineskip
        \lineskip=\the\lineskip
        \lineskiplimit=\the\lineskiplimit
        \noexpand\let\noexpand\offinterlineskip=\noexpand\normaloffinterlineskip}%
   \else
     \let\oninterlineskip=\setnormalbaselines
   \fi
   \normaloffinterlineskip}

\let\oninterlineskip=\relax

\def\leaveoutervmode%
  {\ifvmode\ifinner\else
     \leavevmode
   \fi\fi}

% We stellen enkele penalties anders in dan Plain TEX:

\def\defaultwidowpenalty{2000} % was: 1000
\def\defaultclubpenalty {2000} % was:  800

\widowpenalty=\defaultwidowpenalty\relax
\clubpenalty =\defaultclubpenalty \relax

% Bovendien definieren we enkele extra \fill's:

\def\hfilll%
  {\hskip\!!zeropoint\!!plus1filll\relax}

\def\vfilll%
  {\vskip\!!zeropoint\!!plus1filll\relax}

% De onderstaande hulpmacro's moeten nog eens instelbaar worden
% gemaakt.

\def\tfskipsize{1em\relax}

\def\tfkernsize{1ex\relax}

\def\tfskip%
  {{\tf\hskip\tfskipsize}}

\def\tfkern%
  {{\tf\kern\tfkernsize}}

% Dit hoort eigenlijk thuis onder het kopje boodschappen cq,
% meldingen.

\let\mindermeldingen\dontcomplain

% Maten
%
% De onderstaande instellingen worden gebruikt voor het
% vastleggen van de zetspiegel en marges.

\voffset=0pt % setting this to -1in let's go metapost crazy
\hoffset=0pt % setting this to -1in let's go metapost crazy

\newdimen\papierhoogte
\newdimen\papierbreedte

\newdimen\printpapierhoogte
\newdimen\printpapierbreedte

\newdimen\zethoogte
\newdimen\zetbreedte

\newdimen\teksthoogte
\newdimen\tekstbreedte

\newdimen\kopwit              \kopwit=2cm
\newdimen\rugwit              \rugwit=2cm

\newdimen\hoofdhoogte         \hoofdhoogte=2cm
\newdimen\voethoogte          \voethoogte=2cm

%\newdimen\kopkopwit           \kopkopwit=0cm

\newdimen\kopoffset           \kopoffset=\!!zeropoint
\newdimen\rugoffset           \rugoffset=\!!zeropoint

\newdimen\linkermargebreedte  \linkermargebreedte=3cm
\newdimen\rechtermargebreedte \rechtermargebreedte=\linkermargebreedte

\newdimen\linkerrandbreedte   \linkerrandbreedte=3cm
\newdimen\rechterrandbreedte  \rechterrandbreedte=\linkerrandbreedte

\newdimen\bovenhoogte         \bovenhoogte=1cm
\newdimen\onderhoogte         \onderhoogte=\bovenhoogte

\def\margeafstand%
  {\@@lymargeafstand}

\def\randafstand%
  {\@@lyrandafstand}

\def\margebreedte%
  {\@@lymarge}

\def\randbreedte%
  {\@@lyrand}

\def\linkerrandafstand%
  {\ifdim\!!zeropoint<\linkerrandbreedte
     \@@lylinkerrandafstand
   \else
     \!!zeropoint
   \fi}

\def\rechterrandafstand%
  {\ifdim\!!zeropoint<\rechterrandbreedte
     \@@lyrechterrandafstand
   \else
     \!!zeropoint
   \fi}

\def\linkermargeafstand%
  {\ifdim\!!zeropoint<\linkermargebreedte
     \@@lylinkermargeafstand
   \else
     \!!zeropoint
   \fi}

\def\rechtermargeafstand%
  {\ifdim\!!zeropoint<\rechtermargebreedte
     \@@lyrechtermargeafstand
   \else
     \!!zeropoint
   \fi}

\def\bovenafstand%
  {\ifdim\!!zeropoint<\bovenhoogte
     \@@lybovenafstand
   \else
     \!!zeropoint
   \fi}

\def\hoofdafstand%
  {\ifdim\!!zeropoint<\hoofdhoogte
     \@@lyhoofdafstand
   \else
     \!!zeropoint
   \fi}

\def\voetafstand%
  {\ifdim\!!zeropoint<\voethoogte
     \@@lyvoetafstand
   \else
     \!!zeropoint
   \fi}

\def\onderafstand%
  {\ifdim\!!zeropoint<\onderhoogte
     \@@lyonderafstand
   \else
     \!!zeropoint
   \fi}

\newif\ifdubbelzijdig
\dubbelzijdigfalse

\newif\ifenkelzijdig
\enkelzijdigtrue

\def\doifsometextlineelse#1#2#3% ! omgekeerd !
  {\doifinsetelse{\getvalue{\??tk#1\v!tekst\c!status}}{\v!geen,\v!hoog}
     {#3}{#2}}

% NOG EENS NAGAAN WANNEER NU GLOBAL EN WANNEER NIET

\def\calculatevsizes% global needed in \resetlayoutregel
  {\redoglobal\teksthoogte=\zethoogte
   \doifsometextlineelse{\v!hoofd}
     {\redoglobal\advance\teksthoogte by -\hoofdhoogte
      \redoglobal\advance\teksthoogte by -\hoofdafstand}
     {}%
   \doifsometextlineelse{\v!voet}
     {\redoglobal\advance\teksthoogte by -\voethoogte
      \redoglobal\advance\teksthoogte by -\voetafstand}
     {}%
   \resetglobal
   \setvsize}

\def\calculatereducedvsizes%
  {\teksthoogte=\zethoogte
   \doifsometextlineelse{\v!hoofd}
     {\advance\teksthoogte by -\hoofdhoogte
      \advance\teksthoogte by -\hoofdafstand}
     {\hoofdhoogte=\!!zeropoint}%
   \doifsometextlineelse{\v!voet}
     {\advance\teksthoogte by -\voethoogte
      \advance\teksthoogte by -\voetafstand}
     {\voethoogte=\!!zeropoint}}

\def\calculatehsizes%
  {\tekstbreedte=\zetbreedte
   \doifsomething{\@@lytekstbreedte}   % may be set to \tekstbreedte
     {\tekstbreedte=\@@lytekstbreedte} % which is tricky but ok
   \sethsize}

\def\sethsize%
  {\global\hsize=\tekstbreedte}

\def\setvsize%
  {\ifdim\vsize=\teksthoogte
   \else
     \bgroup
     \dimen0=-\vsize
     \advance\dimen0 by \teksthoogte
     \global\advance\vsize by \dimen0
     \ifdim\pagegoal<\maxdimen
       \advance\dimen0 by \pagegoal
       \global\pagegoal=\dimen0
     \fi
     \egroup
   \fi}

% Algemeen
%
% De Pragma-macros zijn samengesteld met behulp van de
% commandos van PlainTeX- en enkele TugBoat routines.
%
% Voor de volledigheid zijn in de definitie steeds de
% {}-haakjes vermeld. Deze haakjes zijn niet altijd
% nodig, Als bijvoorbeeld een paragraaf bewerkt wordt,
% kunnen ze achterwege blijven.
%
% Instellingen worden opgegeven tussen []-haakjes,
% meestal direct na het commando. Instellingen mogen
% soms achterwege blijven.
%
% Een aantal veelgebruikte macro's zijn in TeXEdit op
% naam en/of door middel van een mnemonic oproepbaar.

% De onderstaande macro voert commando's uit, afhankelijk van
% het karakter van het paginanummer.
%
% \doifonevenpaginaelse{then-commando}{else-commando}

% NB \userpageno vervangen door \realpageno

% \def\doifonevenpaginaelse#1#2%
%   {\ifodd\realpageno#1\else#2\fi}
% 
% \def\doifbothsidesoverruled#1\orsideone#2\orsidetwo#3\od%
%   {\ifdubbelzijdig
%      \ifodd\realpageno#2\relax\else#3\relax\fi
%    \else
%      #1\relax
%    \fi}
% 
% \def\doifbothsides#1\orsideone#2\orsidetwo#3\od%
%   {\ifdubbelzijdig
%      \ifenkelzijdig
%        #1\relax
%      \else
%        \ifodd\realpageno#2\relax\else#3\relax\fi
%      \fi
%    \else
%      #1\relax
%    \fi}

%D When we start at an even page, we need to swap the layout 
%D differently. We cannot adapt the real page number, since 
%D it is used in cross referencing. The next switch is set 
%D when we start at an even page. 

\newif\ifshiftedrealpageno 

\def\doifonevenpaginaelse#1#2%
  {\ifshiftedrealpageno
     \ifodd\realpageno#2\else#1\fi
   \else
     \ifodd\realpageno#1\else#2\fi
   \fi}

\def\doifbothsidesoverruled#1\orsideone#2\orsidetwo#3\od%
  {\ifdubbelzijdig
     \doifonevenpaginaelse{#2}{#3}\relax
   \else
     #1\relax
   \fi}

\def\doifbothsides#1\orsideone#2\orsidetwo#3\od%
  {\ifdubbelzijdig
     \ifenkelzijdig
       #1\relax
     \else
       \doifonevenpaginaelse{#2}{#3}\relax
     \fi
   \else
     #1\relax
   \fi}

\def\dostartglobaldefs#1#2%
  {\edef\!!stringa{\the\globaldefs}%
   \ifnum\globaldefs#10
     \globaldefs=-\globaldefs
   \fi
   \advance\globaldefs by #21
   \setevalue{@gd@\the\globaldefs}{\!!stringa}}

\def\dostopglobaldefs%
  {\doifdefinedelse{@gd@\the\globaldefs}
     {\globaldefs=\getvalue{@gd@\the\globaldefs}\relax}
     {\globaldefs=0\relax}}

\def\startlocal  {\dostartglobaldefs>-}
\def\stoplocal   {\dostopglobaldefs}
\def\startglobal {\dostartglobaldefs<+}
\def\stopglobal  {\dostopglobaldefs}

\ifx\stelpapierformaatin\undefined
  \let\stelpapierformaatin\relax
\fi

\def\dodefinieerpapierformaat[#1][#2]%
  {\ifsecondargument
     \getparameters
       [\??pp#1] % geen \c!schaal, scheelt hash ruimte
       [\c!breedte=\@@ppbreedte,\c!hoogte=\@@pphoogte,
        \c!offset=\@@ppoffset,#2]%
   \else
     \getparameters[\??pp][#1]%
     \stelpapierformaatin
   \fi}

\def\definieerpapierformaat%
  {\dodoubleempty\dodefinieerpapierformaat}

\definieerpapierformaat
  [\c!breedte=210mm,\c!hoogte=297mm,\c!offset=0pt]

\chardef\papermirror   =0
\chardef\printmirror   =0
\chardef\paperrotation =0
\chardef\paperreverse  =0
\chardef\printrotation =0
\chardef\printreverse  =0
\chardef\paperlandscape=0
\chardef\printlandscape=0

\def\papierschaal{1}

\newif\ifnegateprintbox

\def\dostelpapierrichtingin#1#2#3#4#5%
  {\global\chardef#2=0
   \global\chardef#5=0
   \gdef#3{0}%
   \gdef#4{0}%
   \global\negateprintboxfalse
   \processallactionsinset
     [#1]
     [   \v!liggend=>\global\chardef#2=1,
      \v!gespiegeld=>\global\chardef#5=1,
       \v!geroteerd=>\gdef#3{90}\gdef#4{270},
        \v!negatief=>\global\negateprintboxtrue,
                 90=>\gdef#3{90}\gdef#4{270},
                180=>\gdef#3{180}\gdef#4{0},
                270=>\gdef#3{270}\gdef#4{90}]}

\ifx\calculatepaperoffsets\undefined

  \def\calculatepaperoffsets#1%
    {\scratchdimen=\getvalue{\??pp#1\c!offset}%
     \global\advance\papierbreedte by -2\scratchdimen
     \global\advance\papierhoogte by -2\scratchdimen}

\fi

\def\dostelpapierformaatin[#1][#2]%
  {\doifinstringelse{=}{#1}
     {\getparameters[\??pp][#1]}
     {\doifinstringelse{=}{#2}
        {\getparameters[\??pp#1][#2]}
        {\dodostelpapierformaatin[#1][#2]}}}

\def\dodostelpapierformaatin[#1][#2]%
  {\ifsecondargument
     \xdef\herstelpapierformaat%
       {\noexpand\stelpapierformaatin[#1][#2]}%
     \dostelpapierrichtingin{#1}\paperlandscape\paperrotation\paperreverse\papermirror
     \dostelpapierrichtingin{#2}\printlandscape\printrotation\printreverse\printmirror
     \def\docommando##1%
       {\doifsomething{##1}{\doifdefined{\??pp##1\c!breedte}
          {\global\papierbreedte=\getvalue{\??pp##1\c!breedte}%
           \global\papierhoogte=\getvalue{\??pp##1\c!hoogte}%
           \calculatepaperoffsets{##1}%
           \xdef\papierformaat{##1}}}}%
     \processcommacommand[#1]\docommando
     \doifdefinedelse{\??pp#1\c!schaal}
       {\edef\papierschaal{\getvalue{\??pp#1\c!schaal}}}
       {\edef\papierschaal{1}}%
     \def\docommando##1%
       {\doifsomething{##1}{\doifdefined{\??pp##1\c!breedte}
          {\global\printpapierbreedte=\getvalue{\??pp##1\c!breedte}%
           \global\printpapierhoogte=\getvalue{\??pp##1\c!hoogte}%
           \xdef\printpapierformaat{##1}}}}%
     \processcommacommand[#2]\docommando
     \ifnum\paperlandscape>0
       \doglobal\swapdimens\papierbreedte\papierhoogte
     \fi
     \ifnum\printlandscape>0
       \doglobal\swapdimens\printpapierbreedte\printpapierhoogte
     \fi
     \ifdim\papierhoogte>\printpapierhoogte
       \global\printpapierhoogte=\papierhoogte
     \fi
     \ifdim\papierbreedte>\printpapierbreedte
       \global\printpapierbreedte=\papierbreedte
     \fi
     \calculatehsizes
     \calculatevsizes
     \global\newlogostrue
     \global\newbackgroundtrue
     \resetlayout
   \else\iffirstargument
     \stelpapierformaatin[#1][#2]%
   \else\ifx\papierformaat\undefined\else
     \herstelpapierformaat
   \fi\fi\fi}

\let\herstelpapierformaat\relax

\def\stelpapierformaatin%
  {\dodoubleempty\dostelpapierformaatin}

\def\checkforems[#1]%
  {\def\docommando##1%
     {\beforesplitstring##1\at em\to\asciia
      \doifnot{\asciia}{##1}
        {\aftersplitstring\asciia\at=\to\asciia
         \doifsomething{\asciia}
           {\showmessage{\m!systems}{10}{##1}}}}%
   \processcommalist[#1]\docommando}

\def\resetlayout%
  {\global\linkermargebreedte=\@@lylinkermarge
   \global\rechtermargebreedte=\@@lyrechtermarge
   \global\linkerrandbreedte=\@@lylinkerrand
   \global\rechterrandbreedte=\@@lyrechterrand
   \global\hoofdhoogte=\@@lyhoofd
   \global\voethoogte=\@@lyvoet
   \global\onderhoogte=\@@lyonder
   \global\bovenhoogte=\@@lyboven
   \global\rugwit=\@@lyrugwit
   \global\kopwit=\@@lykopwit
   \doifelse{\@@lygrid}{\v!ja}
     {\gridsnappingtrue}
     {\gridsnappingfalse}%
   \ifgridsnapping
     \widowpenalty=0 % is gewoon beter
     \clubpenalty =0 % zeker bij grids
   \else
     \widowpenalty=\defaultwidowpenalty
     \clubpenalty=\defaultclubpenalty
   \fi
   \stelwitruimtein
   \stelblankoin
   \doifelse{\@@lybreedte}{\v!midden}
     {\global\zetbreedte=\papierbreedte
      \global\advance\zetbreedte by -\rugwit
      \scratchdimen=\@@lysnijwit\relax
      \ifdim\scratchdimen=\!!zeropoint
        \scratchdimen=\rugwit
      \fi
      \global\advance\zetbreedte by -\scratchdimen}
     {\doifelse{\@@lybreedte}{\v!passend}
        {\global\zetbreedte=\papierbreedte
         \global\advance\zetbreedte by -\rugwit
         \scratchdimen=\rugwit
         \advance\scratchdimen by -\linkerrandbreedte
         \advance\scratchdimen by -\linkerrandafstand
%        \advance\scratchdimen by -\paginascheiding
         \advance\scratchdimen by -\linkermargebreedte
         \advance\scratchdimen by -\linkermargeafstand
         \ifdim\scratchdimen<\!!zeropoint
           \scratchdimen=\!!zeropoint
         \fi
         \global\advance\zetbreedte by -\rechtermargeafstand
         \global\advance\zetbreedte by -\rechtermargebreedte
%         \global\advance\zetbreedte by -\paginascheiding
         \global\advance\zetbreedte by -\rechterrandafstand
         \global\advance\zetbreedte by -\rechterrandbreedte
         \global\advance\zetbreedte by -\scratchdimen}
        {\global\zetbreedte=\@@lybreedte}}%
   \doifelse{\@@lyregels}{}
     {\doifelse{\@@lyhoogte}{\v!midden}
        {\global\zethoogte=\papierhoogte
         \global\advance\zethoogte by -\kopwit
         \scratchdimen=\@@lybodemwit\relax
         \ifdim\scratchdimen=\!!zeropoint
           \scratchdimen=\kopwit
         \fi
         \global\advance\zethoogte by -\scratchdimen}
        {\doifelse{\@@lyhoogte}{\v!passend}
           {\global\zethoogte=\papierhoogte
            \global\advance\zethoogte by -\kopwit
            \scratchdimen=\kopwit
            \advance\scratchdimen by -\bovenhoogte
            \advance\scratchdimen by -\bovenafstand
            \ifdim\scratchdimen<\!!zeropoint
              \scratchdimen=\!!zeropoint
            \fi
            \global\advance\zethoogte by -\onderafstand
            \global\advance\zethoogte by -\onderhoogte
            \global\advance\zethoogte by -\scratchdimen}
           {\global\zethoogte=\@@lyhoogte}}}
     {\global\zethoogte=\@@lyregels\lineheight
      \global\advance\zethoogte by \hoofdhoogte
      \global\advance\zethoogte by \voethoogte}%
   \rugoffset=\@@lyrugoffset
   \kopoffset=\@@lykopoffset
   \calculatehsizes
   \calculatevsizes
   \global\newlogostrue
   \global\newbackgroundtrue}

\def\checklayout%
  {\doifsomething{\@@lyregels}
     {\ifdim\zethoogte=\@@lyregels\lineheight \else \resetlayout \fi}}

\appendtoks \checklayout \to \everystarttext

\newif\ifdoublesidedprint

\def\presetcenterpagebox% in \stellayoutin !!!!!!!!!!!!!!!!
  {\doublesidedprintfalse
   \ExpandFirstAfter\processallactionsinset
     [\@@lyplaats]
     [      \v!midden=>{\stelpapierformaatin[\c!links=\hss,\c!rechts=\hss,\c!boven=\vss,\c!onder=\vss]},
             \v!links=>{\stelpapierformaatin[\c!links=,\c!rechts=\hss]},
            \v!rechts=>{\stelpapierformaatin[\c!links=\hss,\c!rechts=]},
             \v!onder=>{\stelpapierformaatin[\c!boven=\vss,\c!onder=]},
             \v!boven=>{\stelpapierformaatin[\c!boven=,\c!onder=\vss]},%
      \v!dubbelzijdig=>\doublesidedprinttrue,
       \v!enkelzijdig=>\doublesidedprintfalse]}

\def\complexstellayoutin[#1]%
  {\ConvertToConstant\doifnot{#1}{\v!reset}
     {\getparameters[\??ly][#1]%
      \checkforems[#1]}%
   \resetlayout
   \presetcenterpagebox}

\definecomplexorsimpleempty\stellayoutin

\let\@@zahoogte=\!!zeropoint

\def\dopushpagedimensions%
  {\xdef\oldteksthoogte{\the\teksthoogte}%
   \xdef\oldvoethoogte{\the\voethoogte}%
   \global\let\@@zahoogte=\@@zahoogte}

\def\dopoppagedimensions%
  {\global\teksthoogte=\oldteksthoogte
   \global\voethoogte=\oldvoethoogte
   \resetlayout
   \global\let\pushpagedimensions=\dopushpagedimensions
   \global\let\poppagedimensions=\relax}

\let\poppagedimensions=\relax
\let\pushpagedimensions=\dopushpagedimensions

% Elke \csname ... \endcsname wordt ook aangemaakt, dus ook
% in een test met \doifdefined. Bij veel bladzijden kan dit
% te veel macro's kosten. Vandaar de set \adaptedpages. Het
% kost tijd, maar scheelt macro's.

\def\adaptedpages{}

\def\adaptpagedimensions%
  {\rawdoifinsetelse{\realfolio}{\adaptedpages}
     {\getvalue{\??za\realfolio}%
      \letbeundefined{\??za\realfolio}}
     {}}

\def\checkpagedimensions%
  {\poppagedimensions
   \adaptpagedimensions}

\def\reportpagedimensions%
  {\ifx\poppagedimensions\relax
   \else
     \spatie\@@zahoogte\spatie-
   \fi
   \realfolio}

\def\dodopaslayoutaan[#1]%
  {\getparameters[\??za][\c!hoogte=,\c!regels=,#1]%
   \pushpagedimensions
   \doifelsenothing{\@@zaregels}
     {\showmessage{\m!layouts}{1}{\@@zahoogte,\realfolio}}
     {\showmessage{\m!layouts}{1}{\@@zaregels\space\v!regels,\realfolio}%
      \def\@@zahoogte{\@@zaregels\openlineheight}}%
   \doifelse{\@@zahoogte}{\v!max}
     {\balancedimensions{\teksthoogte}{\voethoogte}{\voethoogte}}
     {\balancedimensions{\teksthoogte}{\voethoogte}{\@@zahoogte}}%
   \ifdim\voethoogte<\!!zeropoint
     \global\advance\teksthoogte by \voethoogte
     \global\voethoogte=\!!zeropoint
     \global\xdef\@@zahoogte{\@@lyvoet\spatie(\v!max)}%
   \fi
   \setvsize
   \global\pagegoal=\vsize  % nog corrigeren voor insertions ?
   \global\newlogostrue
   \global\newbackgroundtrue
   \global\let\pushpagedimensions=\relax
   \global\let\poppagedimensions=\dopoppagedimensions}

\def\dopaslayoutaan[#1][#2]%
  {\doifelsenothing{#2}
     {\dodopaslayoutaan[#1]}
     {\def\docommando##1%
        {\addtocommalist{##1}\adaptedpages
         \setgvalue{\??za##1}{\dodopaslayoutaan[#2]}}%
      \processcommalist[#1]\docommando
      \adaptpagedimensions}}

\def\paslayoutaan%
  {\dodoubleempty\dopaslayoutaan}

\newif\ifmargeblokken

\def\dostelmargeblokkenin[#1]%
  {\getparameters[\??mb][#1]%
   \doifelse{\@@mbstatus}{\v!start}%
     {\showmessage{\m!layouts}{4}{}%
      \margeblokkentrue
      \let\somenextfloat=\dosomenextfloat
      \let\startmargeblok=\dostartmargeblok
      \let\stopmargeblok=\dostopmargeblok}%
     {\showmessage{\m!layouts}{5}{}%
      \margeblokkenfalse
      \def\somenextfloat[##1]%
        {\someelsefloat[##1,\v!hier]}%
      \let\startmargeblok=\dontstartmargeblok
      \let\stopmargeblok=\dontstopmargeblok}}

\def\stelmargeblokkenin%
  {\dosingleargument\dostelmargeblokkenin}

\newbox\marginbox

\def\dosomenextfloat[#1]%
  {\global\setbox\marginbox=\vbox
     {\hsize\@@mbbreedte
      \unvcopy\marginbox
      \ifvoid\marginbox\else\expandafter\@@mbtussen\fi
      \box\floatbox\filbreak}%
   \ifdim\ht\marginbox>\teksthoogte
     \dosavefloatinfo
   \else
     \doinsertfloatinfo
   \fi}

\newbox\preparedmarginbox

\def\reshapemargin%
  {\ifdim\ht\preparedmarginbox>\!!zeropoint
     \beginofshapebox
       \unvbox\preparedmarginbox
     \endofshapebox
     \reshapebox
       {\box\shapebox}%
     \setbox\preparedmarginbox=\vbox to \teksthoogte
       {\@@mbboven
        \flushshapebox
        \@@mbonder}%
   \fi}


\def\plaatsrechtermargeblok%
  {\hskip\rechtermargebreedte}

\def\plaatslinkermargeblok%
  {\hskip\linkermargebreedte}

\def\checkmargeblokken%
  {\setbox\preparedmarginbox=\vbox
     {\forgetall
      \splittopskip\topskip
      \ifvoid\marginbox\else
        \ifdim\ht\marginbox>\teksthoogte
          \vsplit\marginbox to \teksthoogte
        \else
          \unvbox\marginbox
        \fi
      \fi}%
   \reshapemargin
   \setbox\preparedmarginbox=\vbox
      {\@@mbvoor\box\preparedmarginbox\@@mbna}%
   \def\rightmarginbox%
     {\def\plaatsrechtermargeblok%
        {\setbox\preparedmarginbox=\hbox to \rechtermargebreedte
           {\@@mblinks\box\preparedmarginbox\@@mbrechts}%
         \vsmashbox\preparedmarginbox
         \box\preparedmarginbox}}%
   \def\leftmarginbox%
     {\def\plaatslinkermargeblok%
        {\setbox\preparedmarginbox=\hbox to \linkermargebreedte
           {\@@mbrechts\box\preparedmarginbox\@@mblinks}%
         \vsmashbox\preparedmarginbox
         \box\preparedmarginbox}}%
   \processaction
     [\@@mbplaats]
     [ \v!inmarge=>\doifbothsidesoverruled
                     \rightmarginbox
                   \orsideone
                     \rightmarginbox
                   \orsidetwo
                     \leftmarginbox
                   \od,
        \v!midden=>\doifbothsidesoverruled
                     \rightmarginbox
                   \orsideone
                     \leftmarginbox
                   \orsidetwo
                     \rightmarginbox
                   \od,
         \v!links=>\leftmarginbox,
        \v!rechts=>\rightmarginbox,
       \s!unknown=>\setbox\preparedmarginbox=\hbox{}]}

\def\dostartmargeblok%  % 2 maal \vbox ivm \unvbox elders
  {\global\setbox\marginbox=\vtop\bgroup\vbox\bgroup
     \hsize\@@mbbreedte
     \ifvoid\marginbox\else
       \unvbox\marginbox
       \@@mbtussen
     \fi
     \steluitlijnenin[\@@mbuitlijnen]%
     \dostartattributes\??mb\c!letter\c!kleur{}%
     \begstrut\ignorespaces}

\def\dostopmargeblok%
  {\unskip\endstrut
   \dostopattributes
   \egroup
   \egroup}

\def\dontstartmargeblok%
  {\@@mbvoor
   \bgroup
   \dostartattributes\??mb\c!letter\c!kleur{}}

\def\dontstopmargeblok%
  {\dostopattributes
   \egroup
   \@@mbna}

\newcounter\nofpostponedblocks

\newif\ifinuitstellen

\newevery\everytopofpage\relax

\appendtoks\the\everytopofpage\to\everystarttext
\appendtoks\global\everytopofpage{}\to\everystoptext

\def\douitstellen% PAS OP 2X MAIN-001
  {\the\everytopofpage
   \ifinuitstellen\else\ifcase\nofpostponedblocks\else % The \nof-test is
\bgroup
\black % else problems inside split verbatim 
\restoreglobalbodyfont % else problems inside split verbatim 
     \global\pagetotal\!!zeropoint % recently added 
     \global\inuitstellentrue                          % definitely needed
     \dorecurse{\nofpostponedblocks}                   % else we can loose
       {\haalbuffer[buf-\recurselevel]}                % or disorder floats
     \doflushfloats % new but potential dangerous      % and that is something
     \doglobal\newcounter\nofpostponedblocks           % we don't want, do we?
     \global\inuitstellenfalse                         % Anyhow, 'uitstellen'
\egroup
   \fi\fi}                                             % is still suboptimal.

\setvalue{\e!start\e!uitstellen}%
  {\doglobal\increment\nofpostponedblocks
   \showmessage{\m!layouts}{3}{\nofpostponedblocks}%
   \dostartbuffer[buf-\nofpostponedblocks]
     [\e!start\e!uitstellen][\e!stop\e!uitstellen]}

% \gotonextsubpage  : voor de pagebody
% \subpaginanummer  : alleen in de voet/kopregels
% \aantalsubpaginas : alleen in de voet/kopregels

% \firstsubpage     : eerste \realpageno, voor interne doeleinden
% \prevsubpage      : vorige \realpageno, voor interne doeleinden
% \nextsubpage      : volgende \realpageno, voor interne doeleinden
% \lastsubpage      : laatste \realpageno, voor interne doeleinden
% \nofsubpages      : laatste subpage (in berekeningen)
% \subpageno        : huidige subpage (in berekeningen)

\newif\ifsubpaging
\newif\ifshowingsubpage

\definieernummer
  [\s!subpage]

\stelnummerin
  [\s!subpage]
  [\c!wijze=\@@snwijze]

\def\resetsubpaginanummer%
  {\resetnummer[\s!subpage]%
   \global\subpageno=\ruwenummer[\s!subpage]}

\def\dostelsubpaginanummerin[#1]%
  {\doifelse{#1}{\v!reset}
     {\resetsubpaginanummer} % \resetnummer[\s!subpage]
     {\getparameters[\??sn][#1]%
      \processaction
        [\@@snstatus]
        [  \v!stop=>\ifsubpaging
                    \else
                      \subpagingfalse
                    \fi
                    \showingsubpagefalse,
          \v!start=>\subpagingtrue
                    \showingsubpagetrue,
           \v!geen=>\subpagingtrue
                    \showingsubpagefalse]}}

\def\aantalsubpaginas%
  {\ifshowingsubpage
     \nofsubpages
   \else
     0%
   \fi}

\def\subpaginanummer%
  {\ifshowingsubpage
     \the\subpageno
   \else
     0%
   \fi}

\def\stelsubpaginanummerin%
  {\dosingleargument\dostelsubpaginanummerin}

\def\newnofsubpages  {0}
\def\nofsubpages     {0}
\def\firstsubpage    {1}
\def\prevsubpage     {1}
\def\nextsubpage     {1}
\def\lastsubpage     {1}

\def\nextpage        {1}
\def\prevpage        {1}

\definetwopasslist{\s!subpage}

\def\savenofsubpages%
  {\ifsubpaging
     \showmessage{\m!layouts}{6}{\newnofsubpages,\the\subpageno}%
     \immediatewriteutilitycommand%
        {\twopassentry%
           {\s!subpage}%
           {\newnofsubpages}%
           {\the\subpageno}}%
   \fi}

\def\setsubpagenumbers%
  {\iftwopassdatafound
     \bgroup
     \xdef\nofsubpages{\twopassdata}%
     \xdef\firstsubpage{\realfolio}%
     \advance\realpageno by \nofsubpages
     \advance\realpageno by -1
     \xdef\lastsubpage{\realfolio}%
     \egroup
   \else
     \xdef\nofsubpages{0}%
   \fi}

\def\gotonextsubpage% overlapt behoorlijk met realpage macro
  {\global\let\checksubpages=\relax
   \ifsubpaging
     \edef\oldsubpage{\the\subpageno}%
     \verhoognummer[\s!subpage]%
     \global\subpageno=\ruwenummer[\s!subpage]\relax
     \ifnum\subpageno=1
       \gettwopassdata{\s!subpage}%
       \setsubpagenumbers
       \ifnum\oldsubpage>0
         \showmessage{\m!layouts}{6}{\newnofsubpages,\oldsubpage}%
         \edef\next%
           {\writeutilitycommand%
              {\twopassentry%
                 {\s!subpage}%
                 {\newnofsubpages}%
                 {\oldsubpage}}}%
         \next
       \fi
       \doglobal\increment\newnofsubpages\relax
     \fi
     \setglobalsystemreference\rt!page{\v!eerstesubpagina}\firstsubpage
     \setglobalsystemreference\rt!page{\v!laatstesubpagina}\lastsubpage
     \bgroup
     \ifnum\realpageno=\firstsubpage\relax
       \global\let\prevsubpage=\firstsubpage
       \setglobalsystemreference\rt!page{\v!sub\v!achteruit}\lastsubpage
     \else
       \xdef\prevsubpage{\realfolio}%
       \doglobal\decrement\prevsubpage
       \setglobalsystemreference\rt!page{\v!sub\v!achteruit}\prevsubpage
     \fi
     \setglobalsystemreference\rt!page{\v!vorigesubpagina}\prevsubpage
     \ifnum\realpageno=\lastsubpage\relax
       \global\let\nextsubpage=\lastsubpage
       \setglobalsystemreference\rt!page{\v!sub\v!vooruit}\firstsubpage
     \else
       \xdef\nextsubpage{\realfolio}%
       \doglobal\increment\nextsubpage
       \setglobalsystemreference\rt!page{\v!sub\v!vooruit}\nextsubpage
     \fi
     \setglobalsystemreference\rt!page{\v!volgendesubpagina}\nextsubpage
     \egroup
   \fi}

\def\checksubpages%
  {\getfromtwopassdata{\s!subpage}{1}%
   \setsubpagenumbers
   \global\let\checksubpages=\relax}

% Omdat \gotonextrealpage gebruik maakt van de hulpfile,
% moet het initialiseren van \realpageno plaatsvinden in
% een later stadium, namelijk zodra referenties worden
% gebruikt (anders gaat het mis op nog niet gedefinieerde
% lijstcommando's e.d.). De eerst aanroep vindt dan ook
% plaats vlak nadat de hulpfile voor de eerste maal is
% ingelezen.

\countdef\realpageno = 0   \realpageno = 1
\countdef\userpageno = 1   \userpageno = 1
\countdef\subpageno  = 2   \subpageno  = 0 % !!
\countdef\arrangeno  = 3   \arrangeno  = 0 % !!

% we don't want conflicts when \pageno is used by other
% packages, like CWEB, so we redefine \pageno

\newcount\pageno           \pageno     = 1

\def\setuserpageno#1%
  {\global\userpageno=#1\relax
   \global\pageno=\userpageno}

\def\realfolio   {\the\realpageno}
\def\folio       {\the\userpageno}
\def\firstpage   {1}
\def\lastpage    {1}
\def\currentpage {\the\realpageno}

\def\gotonextrealpage%
  {\global\advance\realpageno by 1
   \ifnum\realpageno>\lastpage
     \xdef\lastpage{\realfolio}%
   \fi
   \setglobalsystemreference\rt!page{\v!eerstepagina}\firstpage
   \setglobalsystemreference\rt!page{\v!laatstepagina}\lastpage
   \bgroup
   \ifnum\realpageno>1
     \advance\realpageno by -1
     \xdef\prevpage{\realfolio}%
     \setglobalsystemreference\rt!page{\v!achteruit}\prevpage
   \else
     \global\let\prevpage=\firstpage
     \setglobalsystemreference\rt!page{\v!achteruit}\lastpage
   \fi
   \setglobalsystemreference\rt!page{\v!vorigepagina}\prevpage
   \egroup
   \bgroup
   \ifnum\realpageno<\lastpage\relax
     \advance\realpageno by 1
     \xdef\nextpage{\realfolio}%
     \setglobalsystemreference\rt!page{\v!pagina}\nextpage
     \setglobalsystemreference\rt!page{\v!vooruit}\nextpage
     \bgroup
     \xdef\nextnextpage{\realfolio}%
     \ifodd\realpageno
       \setglobalsystemreference\rt!page{\v!volgendeonevenpagina}\nextnextpage
     \else
       \setglobalsystemreference\rt!page{\v!volgendeevenpagina}\nextnextpage
     \fi
     \advance\realpageno by 1
     \xdef\nextnextpage{\realfolio}%
     \ifnum\realpageno>\lastpage\relax
      %\ifodd\realpageno
      %  \setglobalsystemreference\rt!page{\v!volgendeonevenpagina}\lastpage
      %\else
      %  \setglobalsystemreference\rt!page{\v!volgendeevenpagina}\lastpage
      %\fi
     \else
       \ifodd\realpageno
         \setglobalsystemreference\rt!page{\v!volgendeonevenpagina}\nextnextpage
       \else
         \setglobalsystemreference\rt!page{\v!volgendeevenpagina}\nextnextpage
       \fi
     \fi
     \egroup
   \else
     \global\let\nextpage=\lastpage
     \setglobalsystemreference\rt!page{\v!pagina}\firstpage
     \setglobalsystemreference\rt!page{\v!vooruit}\firstpage
     \setglobalsystemreference\rt!page{\v!volgendeonevenpagina}\lastpage
     \setglobalsystemreference\rt!page{\v!volgendeevenpagina}\lastpage
   \fi
   \setglobalsystemreference\rt!page{\v!volgendepagina}\realfolio
   \egroup}

\def\checkrealpage%
  {\global\realpageno=0
   \gotonextrealpage
   \global\let\checkrealpage=\relax}

\def\savenofpages%
  {\advance\realpageno by -1
   \savecurrentvalue\lastpage{\realfolio}}%

\def\totaalaantalpaginas%
  {\lastpage}

\def\initializepaper%
  {\iflocation
     \dosetuppaper
       {\papierformaat}
       {\the\papierbreedte}
       {\the\papierhoogte}%
   \else
     \dosetuppaper
       {\printpapierformaat}
       {\the\printpapierbreedte}
       {\the\printpapierhoogte}%
   \fi}

\def\myshipout#1%
  {\voorpagina % voor de pagebody dus !  
   \dontshowcomposition
   \ifarrangingpages
     \actualarrange
       {\thisisrealpage{\realfolio}#1}%
   \else
     \actualshipout
       {\thisisrealpage{\realfolio}#1}%
   \fi
   \gotonextrealpage
   \napagina}

\newbox\postponedcontent

\def\flushatshipout%
  {\dowithnextbox
      {\global\setbox\postponedcontent=\hbox to \!!zeropoint
         {%\hskip-\maxdimen % niet hier, gaat mis in acrobat (clipt)
          \box\postponedcontent\box\nextbox}%
       \global\ht\postponedcontent=\!!zeropoint
       \global\dp\postponedcontent=\!!zeropoint
       \global\wd\postponedcontent=\!!zeropoint}%
   \hbox}

% \starttypen
% \def\pagestoshipout{1,3,5}
% \stoptypen

\newcounter\shippedoutpages
\let\pagestoshipout\empty      % {1,3,6}
\chardef\whichpagetoshipout=0 % 0=all 1=odd 2=even

\def\actualshipout#1%
  {\doglobal\increment\shippedoutpages
   \ifx\pagestoshipout\empty
     \ifcase\whichpagetoshipout\relax
       \donetrue
     \or % 1
       \ifodd\shippedoutpages\relax\donetrue\else\donefalse\fi
     \or % 2
       \ifodd\shippedoutpages\relax\donefalse\else\donetrue\fi
     \else
       \donetrue
     \fi
   \else
     \ExpandBothAfter\doifinsetelse{\shippedoutpages}{\pagestoshipout}
       \donetrue\donefalse
   \fi
   \ifdone
     \shipout\vbox
       {\forgetall
        \offinterlineskip
        \mindermeldingen
        \vskip-1in
        \hskip-1in
        \hbox
          {\setbox0=\hbox{#1}% just in case there are objects there
           \setbox\scratchbox=\hbox
             {\the\everyshipout
              \ifnum\realpageno=\lastpage\relax
                \the\everylastshipout
                \global\everylastshipout\emptytoks
              \fi}%
           \smashbox\scratchbox
           \box\scratchbox
           \box\postponedcontent % evt ver naar links !
           \box0}}%
   \else
     \message
       {[\ifarrangingpages arranged \fi page
         \ifarrangingpages\the\arrangeno\else\the\realpageno\fi\normalspace
         not flushed]}%
     \setbox0=\hbox{#1}%
     \deadcycles=0
   \fi}

\def\actualarrange#1%
  {\setbox0=\hbox{\thisisrealpage{\realfolio}#1}%
   \pusharrangedpage0
   \deadcycles=0 }

\def\goleftonpage%
  {\hskip-\linkermargeafstand
   \hskip-\linkermargebreedte
  %\hskip-\paginascheiding
   \hskip-\linkerrandafstand
   \hskip-\linkerrandbreedte}

\def\doswapmargins%
  {\let\doswapmargins=\relax % to prevent local swapping
   \swapmacros\@@lylinkermargeafstand\@@lyrechtermargeafstand
   \swapmacros\@@lylinkerrandafstand\@@lyrechterrandafstand
   \swapdimens\linkermargebreedte\rechtermargebreedte
   \swapdimens\linkerrandbreedte\rechterrandbreedte}

\def\doifmarginswapelse#1#2%
  {\doifbothsides#1\orsideone#1\orsidetwo#2\od}

\def\swapmargins%
  {\doifmarginswapelse{}{\doswapmargins}}

% Output routines
%
% \dopagecontents#1#2  : tekst, floats en footnotes
% \dopagebody#1#2      : hoofd, \pagecontents, voet
% \dooutput            : outputroutine
%
% \ifinpagebody

\def\doejectpage#1%
  {\bgroup                         % de \ifdim is nodig omdat
   \par                            % anders een eventuele
   \ifdim\pagetotal>\pagegoal\else % laatste regel boven de
     %\normalvfill                 % baseline te staan terwijl
     \normalvfil                   % baseline te staan terwijl
   \fi                             % de vorige bladzijden op
   #1%                             % de baseline staan
   \egroup}

% ^^ NOG NETTER MAKEN, TEGELIJK MET MULTI COLUMNS EN ACHTERGRONDEN!

\def\ejectpage%
  {\doejectpage\eject}

\def\superejectpage%
  {\doejectpage\supereject}

\def\ejectinsert%
  {\flushfootnotes
   \bgroup
   \noftopfloats=\!!thousand
   \nofbotfloats=0
   \doflushfloats
   \egroup}

% De volgende macro's worden gedefinieerd in de module
% colo-ini. Om resetten bij twee maal laden te voorkomen
% checken we wel even. Anders krijgen we een mark-build-up.

\newif\ifinpagebody

\doifundefined{pushcolor}      {\def\pushcolor{}}
\doifundefined{popcolor}       {\def\popcolor{}}
\doifundefined{startcolorpage} {\def\startcolorpage{}}
\doifundefined{stopcolorpage}  {\def\stopcolorpage{}}

% bewaren tvb documentatie
%
% \hbox to \hsize
%   {\en
%    \switchnaarkorps[5pt]%
%    \emergencystretch2em
%    \dimen0=\baselineskip
%    \baselineskip=\dimen0 plus 1pt
%    \hsize=.2\hsize
%    \vsize=2\hsize
%    \ruledvbox to \vsize{\input tufte \par}\hss
%    \ruledvbox to \vsize{\input tufte \par\kern-\prevdepth}\hss
%    \ruledvbox to \vsize{\input tufte \par\kern0pt}\hss
%    \ruledvbox to \vsize{\input tufte \par\vfill}\hss
%    \ruledvbox to \vsize{\input tufte \par\kern-\prevdepth\vfill}}
%
% \hbox to \hsize
%   {\en
%    \switchnaarkorps[5pt]%
%    \emergencystretch2em
%    \dimen0=\baselineskip
%    \baselineskip=\dimen0 plus 1pt
%    \hsize=.18\hsize
%    \vsize=2.5\hsize
%    \setbox0=\vbox{\input tufte\relax}%
%    \ruledvbox to \vsize{\unvcopy0}\hss
%    \ruledvbox to \vsize{\unvcopy0\kern-\dp0}\hss
%    \ruledvbox to \vsize{\unvcopy0\kern0pt}\hss
%    \ruledvbox to \vsize{\unvcopy0\vfill}\hss
%    \ruledvbox to \vsize{\unvcopy0\kern-\dp0\vfill}}

\def\dopagecontents#1#2% \box<n> \unvbox<n> 
  {\bgroup             % niet breedte zetten, kan fractie zijn! 
   \forgetall
   \boxmaxdepth=\maxdepth
   \setbox0=\vbox \ifbottomnotes to \teksthoogte \fi
     {\edef\currentpagedepth{\the\dp#2}% still to be derived from #1
      \dotopinsertions
      #1#2% \fuzzysnappedbox{#1}{#2}% goes wrong
      \pushcolor
      \ifgridsnapping
        \vskip-\currentpagedepth
        \vskip\openstrutdepth % \dp\strutbox
        \prevdepth\openstrutdepth % \dp\strutbox
        \dobotinsertions
        \vfil
      \else\ifr@ggedbottom
        \vskip-\currentpagedepth
        \vskip\openstrutdepth % \dp\strutbox
        \prevdepth\openstrutdepth % \dp\strutbox
        \dobotinsertions
        \vfil
      \else\ifb@selinebottom
        \kern-\currentpagedepth
        \kern\maxdepth
        \dobotinsertions
      \fi\fi\fi
      \ifdim\ht\footins>\!!zeropoint % beter dan \ifvoid\footins\else
        \kern\skip\footins
        \kern\ht\footins
      \fi}%
\ifbottomnotes
   \ifgridsnapping
     \getnoflines\teksthoogte
     \advance\noflines by -1
     \scratchdimen=\noflines\lineheight
     \advance\scratchdimen by \topskip
   \else
     \scratchdimen=\ht0
   \fi
\else
  \scratchdimen=\!!zeropoint
\fi
   \setbox2=\hbox
     {\ifvoid\savedfootins \else
        \setbox\footins=\box\savedfootins
      \fi
      \lower\scratchdimen\vbox{\placefootnotes}}%
   \smashbox2
\ifbottomnotes
    \ht0=\!!zeropoint
\fi
%% \setbox0= % todo, there must be a faster way to kill funny depths 
   \vbox to \teksthoogte
     {\box0\box2\ifbottomnotes\else\vfill\fi}%
%% \dp0=\!!zeropoint \box0 % new
   \egroup}

\def\dodummypageskip#1%
  {\getvalue{\s!dummy\c!commando#1}}

\setvalue{\s!dummy\c!commando\v!links}%
  {\hskip\linkerrandbreedte}

\setvalue{\s!dummy\c!commando\v!rechts}%
  {\hskip\rechterrandbreedte}

\setvalue{\s!dummy\c!commando\v!boven}%
  {\vskip\bovenhoogte} % \vbox to \bovenhoogte{\vss}}

\setvalue{\s!dummy\c!commando\v!onder}%
  {\vskip\onderhoogte} % \vbox to \onderhoogte{\vss}}

\def\plaatslinkerrandblok  {\dodummypageskip\v!links}
\def\plaatsrechterrandblok {\dodummypageskip\v!rechts}

\def\plaatsboventekstblok  {\dodummypageskip\v!boven}
\def\plaatsondertekstblok  {\dodummypageskip\v!onder}

% kan tzt nog eens als:
%
% \newtoks\everyboventekstblok
%
%\def\plaatsboventekstblok%
%  {\vbox to \bovenhoogte
%     {\the\everyboventekstblok}
%
% \def\doplaatsboventekstblok#1%
%   {\vbox to \bovenhoogte
%      {\@@tkboventekstvoor#1\@@tkboventekstna\kern\!!zeropoint}%
%    \vskip-\bovenhoogte}
%
% \appendtoks\interactiemenus[\v!boven]\to\everyboventekstblok
%
% kan vaker, is namelijk sneller als commalist

\newtoks\afterpage     \newtoks\aftereverypage
\newtoks\beforepage    \newtoks\beforeeverypage

\newif\ifshowgrid      \showgridfalse

\def\toongrid%
  {\tracegridsnappingtrue
   \showgridtrue}

\def\doplaatstekstblok#1#2%
  {\bgroup
   \setbox0=\hbox to \zetbreedte
     {\hss % new 
      \vbox to \teksthoogte                 % can be < \makeupwidth  
        {\offinterlineskip                  % so don't change this
         \tekstbreedte=\zetbreedte          % 
         \doifsomething{\@@lytekstbreedte}  %
           {\tekstbreedte=\@@lytekstbreedte}%
         \hsize=\tekstbreedte               % local variant of \sethsize
         \boxmaxdepth\maxdepth              %
         \noindent                          % the contents can be < \hsize
         \dopagecontents#1#2}%
      \hss}% new
\ifsomebackgroundfound\v!tekst % sneller 
  \setbox0=\hbox\localframed
    [\??ma\v!tekst]
    [\c!strut=\v!nee,\c!offset=\v!overlay,
     \c!breedte=\zetbreedte,\c!hoogte=\teksthoogte]
    {\dp0=\!!zeropoint\box0}%
\fi
   \ht0=\teksthoogte
   \wd0=\zetbreedte
   \dp0=\!!zeropoint % new, maybe a reason for small shifts 
   \ifshowgrid
     \setgridbox2\zetbreedte\teksthoogte
     \hbox{\color[red]{\box2}\hskip-\zetbreedte\box0}%
   \else
     \box0
   \fi
   \egroup}

\def\getmainbox#1#2%
  {\setbox0=\vbox
     {\offinterlineskip  % na \paginaletter !
      \calculatereducedvsizes
      \calculatehsizes
      \swapmargins
      \vskip\hoofdhoogte
      \vskip\hoofdafstand
      \hbox
        {\bgroup
           \swapmargins
           \goleftonpage
           \plaatslinkerrandblok
           \hskip\linkerrandafstand
          %\showpageseparation
           \plaatslinkermargeblok
           \hskip\linkermargeafstand
         \egroup
         \doplaatstekstblok#1#2%
         \bgroup
           \hskip\rechtermargeafstand
           \plaatstestinfo
           \plaatsrechtermargeblok
          %\showpageseparation
           \hskip\rechterrandafstand
           \plaatsrechterrandblok
         \egroup}%
      \vfill}
     \smashbox0
     \box0}

\def\centerpagebox#1%
  {\printpapierbreedte=\papierschaal\printpapierbreedte
   \printpapierhoogte =\papierschaal\printpapierhoogte
   \setbox#1=\vbox to \printpapierhoogte
     {\@@ppboven
      \hbox to \printpapierbreedte
        {\ifdoublesidedprint
           \doifbothsides
             \@@pplinks\box#1\@@pprechts
           \orsideone
             \@@pplinks\box#1\@@pprechts
           \orsidetwo
             \@@pprechts\box#1\@@pplinks
           \od
         \else
           \@@pplinks\box#1\@@pprechts
         \fi}%
      \par
      \@@pponder}}

\def\offsetprintbox#1%
  {\dimen0=\wd#1\dimen2=\ht#1\dimen4=\dp#1%
   \setbox#1=\vbox
     {\forgetall
      \offinterlineskip
      \vskip\kopoffset
      \doifbothsides
        \hskip\rugoffset
      \orsideone
        \hskip\rugoffset
      \orsidetwo
        \hskip-\rugoffset
      \od
      \box#1}%
   \wd#1=\dimen0\ht#1=\dimen2\dp#1=\dimen4}

\def\replicatebox#1#2#3%
  {\setbox#1=\vbox
     {\forgetall
      \offinterlineskip
      \dorecurse{#3}
        {\hbox{\dorecurse{#2}{\copy#1\hskip\@@lydx}\unskip}%
         \vskip\@@lydy}
      \unskip}}

\def\replicatepagebox#1%
  {\ifnum\@@lynx>0 \ifnum\@@lyny>0
     \replicatebox{#1}{\@@lynx}{\@@lyny}%
   \fi\fi}

\def\rotatepagebodybox#1#2#3%
  {\ifnum#2#3>0
     \setbox#1=\vbox
       {\edef\somerotation%
          {\ifdubbelzijdig\ifodd\realpageno#2\else#3\fi\else#2\fi}%
        \dorotatebox\somerotation\hbox{\box#1}}%
   \fi}

\def\rotatepaperbox#1%
  {\rotatepagebodybox{#1}\paperrotation\paperreverse}

\def\rotateprintbox#1%
  {\rotatepagebodybox{#1}\printrotation\printreverse}

\def\mirrorpagebodybox#1#2%
  {\ifcase#2\or
     \setbox#1=\vbox
       {\domirrorbox\vbox{\box#1}}%
   \fi}

\def\mirrorpaperbox#1%
  {\mirrorpagebodybox{#1}\papermirror}

\def\mirrorprintbox#1%
  {\mirrorpagebodybox{#1}\printmirror}

\def\scalepagebox#1%
  {\ifdim\@@lyschaal pt=1pt \else
     \setbox#1=\vbox
       {\schaal[\c!sx=\@@lyschaal,\c!sy=\@@lyschaal]{\box#1}}%
     \papierbreedte=\@@lyschaal\papierbreedte
     \papierhoogte =\@@lyschaal\papierhoogte
   \fi}

\def\negateprintbox#1%
  {\ifnegateprintbox
     \negatecolorbox{#1}%
   \fi}

\def\buildpagebox#1%
  {\setbox#1=\vbox to \papierhoogte
     {\hsize\papierbreedte
      \vskip\kopwit
      \doifbothsides
        \hskip\rugwit
      \orsideone
        \hskip\rugwit
      \orsidetwo
        \hskip\papierbreedte
        \hskip-\rugwit
        \hskip-\zetbreedte
      \od
      \box#1}%
   \dp#1=\!!zeropoint}

\def\pagecutmarksymbol%
  {\the\realpageno}%

\def\addpagecutmarks#1%
  {\doif{\@@lymarkering}{\v!aan}
     {\let\cutmarksymbol=\pagecutmarksymbol
      \makecutbox{#1}}}

\def\addpagecolormarks#1%
  {\doif{\@@lymarkering}{\v!kleur}
     {\let\cutmarksymbol=\pagecutmarksymbol
      \makecutbox{#1}%
      \ifnum\horizontalcutmarks>1 \chardef\colormarkoffset=4 \fi
      \ifnum\verticalcutmarks  >1 \chardef\colormarkoffset=4 \fi
      \colormarkbox{#1}}}

\newif\ifpagebodyornaments \pagebodyornamentstrue
\newif\ifarrangingpages    \arrangingpagesfalse

\let\poparrangedpages=\relax
\let\pusharrangedpage=\relax

\def\reportarrangedpage#1%
  {\showmessage
     {\m!systems}{23}
     {\the\realpageno.\the\pageno\ifnum\subpageno>0 .\the\subpageno\fi,#1}}

\def\buildpagebody#1#2%
  {\vbox
     {\beginrestorecatcodes
      \forgetall  % igv problemen, check: \boxmaxdepth\maxdimen
\ifnewbackground\enablemode[\systemmodeprefix\v!achtergrond]\fi
      \boxmaxdepth\maxdimen % new
      \mindermeldingen
      \setbox0=\vbox
        {\offinterlineskip
         \ifpagebodyornaments
           \getbackgroundbox
           \getlogobox
           \bgroup % else footnotes get inconsistent font/baseline
             \doconvertfont{\@@lyletter}{}%
             \offinterlineskip
             \gettextboxes
           \egroup
         \fi
         \getmainbox#1#2}% including footnotes 
      \buildpagebox0
      \ifpagebodyornaments
        \addpagebackground0
      \fi
      \ifarrangingpages \else
        \addpagecutmarks0
        \replicatepagebox0
        \scalepagebox0
        \mirrorpaperbox0
        \rotatepaperbox0
        \addpagecolormarks0
        \centerpagebox0
\addprintbackground0
        \mirrorprintbox0
        \rotateprintbox0
        \offsetprintbox0
        \negateprintbox0
      \fi
      \box0
      \endrestorecatcodes}}

\def\addprintbackground#1% 
  {\ifsomebackgroundfound\v!papier
     \setbox#1=\vbox\localframed
       [\??ma\v!papier]%
       [\c!offset=\v!overlay,\c!strut=\v!nee,
        \c!breedte=\printpapierbreedte,\c!hoogte=\printpapierhoogte]%
       {\noindent\box#1}%
   \fi}

\def\finishpagebox#1%
  {\ifarrangingpages
     \addpagecutmarks#1%
     \addpagecolormarks#1%
     \centerpagebox#1%
     \mirrorprintbox#1%
     \rotateprintbox#1%
     \offsetprintbox#1%
     \negateprintbox#1%
   \fi}

% TBV testdoeleinden:

\def\dotoonprint[#1][#2][#3]%
  {\framed
     [\c!offset=\v!overlay,
      \c!strut=\v!nee]
     {\forgetall
      \mindermeldingen
      \globaldefs=-1
      \dimen0=\pagegoal
      \definieerpapierformaat[X][\c!breedte=4em, \c!hoogte=6em]%
      \definieerpapierformaat[Y][\c!breedte=12em,\c!hoogte=14em]%
      \stelpapierformaatin[#1,X][#2,Y]%
      \stellayoutin[#3]%
      \setbox0=\vbox
        {\framed
          [\c!offset=\v!overlay,\c!strut=\v!nee,
           \c!breedte=\papierbreedte,\c!hoogte=\papierhoogte]
          {\ss ABC\par DEF}}%
      \dubbelzijdigfalse
      \def\cutmarklength{.5em}%
      \addpagecutmarks0%
      \replicatepagebox0%
      \scalepagebox0%
      \mirrorpaperbox0%
      \rotatepaperbox0%
      \centerpagebox0%
      \mirrorprintbox0%
      \rotateprintbox0%
      \offsetprintbox0%
      \pagegoal=\dimen0
      \box0}}

\def\toonprint%
  {\dotripleempty\dotoonprint}

% \switchnaarkorps[8pt]
%
% \startcombinatie[4*4]
%   {\toonprint}                                       {\strut}
%   {\toonprint[][][plaats=midden]}                    {\type{plaats=midden}}
%   {\toonprint[][][plaats=midden,markering=aan]}      {\type{markering=aan}\break
%                                                      \type{plaats=midden}}
%   {\toonprint[][][plaats=midden,markering=aan,nx=2]} {\type{markering=aan}\break
%                                                      \type{plaats=midden}\break
%                                                      \type{nx=2}}
%   {\toonprint[][][plaats=links]}                     {\type{plaats=links}}
%   {\toonprint[][][plaats=rechts]}                    {\type{plaats=rechts}}
%   {\toonprint[][][plaats={links,onder}]}             {\type{plaats={links,onder}}}
%   {\toonprint[][][plaats={rechts,onder}]}            {\type{plaats={rechts,onder}}}
%   {\toonprint[][][nx=2,ny=1]}                        {\type{nx=2,ny=1}}
%   {\toonprint[][][nx=1,ny=2]}                        {\type{nx=1,ny=2}}
%   {\toonprint[][][nx=2,ny=2]}                        {\type{nx=2,ny=2}}
%   {\toonprint[][][nx=2,ny=2,plaats=midden]}          {\type{nx=2,ny=2}\break
%                                                       \type{plaats=midden}}
%   {\toonprint[][][rugoffset=3pt]}                    {\type{rugoffset=.5cm}}
%   {\toonprint[][][kopoffset=3pt]}                    {\type{kopoffset=.5cm}}
%   {\toonprint[][][schaal=1.5]}                       {\type{schaal=1.5}}
%   {\toonprint[][][schaal=0.8]}                       {\type{schaal=0.8}}
% \stopcombinatie
%
% \startcombinatie[3*4]
%   {\toonprint[liggend][][plaats=midden]}              {\type{liggend}}
%   {\toonprint[][liggend][plaats=midden]}              {\strut\break\type{liggend}}
%   {\toonprint[liggend][liggend][plaats=midden]}       {\type{liggend}\break\type{liggend}}
%   {\toonprint[90][][plaats=midden]}                   {\type{90}}
%   {\toonprint[][90][plaats=midden]}                   {\strut\break\type{90}}
%   {\toonprint[90][90][plaats=midden]}                 {\type{90}\break\type{90}}
%   {\toonprint[180][][plaats=midden]}                  {\type{180}}
%   {\toonprint[][180][plaats=midden]}                  {\strut\break\type{180}}
%   {\toonprint[180][180][plaats=midden]}               {\type{180}\break\type{180}}
%   {\toonprint[gespiegeld][][plaats=midden]}           {\type{gespiegeld}}
%   {\toonprint[][gespiegeld][plaats=midden]}           {\strut\break\type{gespiegeld}}
%   {\toonprint[gespiegeld][gespiegeld][plaats=midden]} {\type{gespiegeld}\break\type{gespiegeld}}
% \stopcombinatie

\chardef\normalpagebox=255

\appendtoks \restoreglobalbodyfont \to \everypagebody
\appendtoks \restorecolumnsettings \to \everypagebody

\def\dopagebody#1#2%
  {\getallmarks
   \the\everypagebody
   \startcolorpage
   \gotonextsubpage  % nog eens: als in pagina (tbv standaard opmaak)
   \dontshowboxes    % dan hier blokkeren en verderop resetten
   \naastpagina
   \checkreferences
   \checkmargeblokken
   \dotoks\beforeeverypage
   \flushtoks\beforepage
   \inpagebodytrue\buildpagebody#1#2%
   \flushtoks\afterpage
   \dotoks\aftereverypage
   \resetpagina
   \updatelistreferences
   \resetlayoutregels % mischien in shipout
   \stopcolorpage}

\def\beforefinaloutput%
  {}

\def\afterfinaloutput%
  {\forgetall
   \vskip\!!zeropoint\relax
   \ifvoid\normalpagebox
   \else
     \unvbox\normalpagebox
     \penalty\outputpenalty
   \fi
   \ifnum\outputpenalty>-\@MM\relax
   \else
     \dosupereject
   \fi
   \inpagebodytrue  % needed for enabling \blanko !
   \dosetbothinserts
   \setvsize % this is needed for interacting components, like floats and multicolumns
   \adaptfuzzypagegoal} % watch this hack!

\def\setpagecounters%
  {\setuserpageno{\ruwenummer[\s!page]}%
   \doifelse{\@@snstatus}{\v!stop}
     {\global\subpageno=0}
     {\global\subpageno=\ruwenummer[\s!subpage]}}

\newtoks\pageboundsettings

\prependtoks \initializepaper \to \pageboundsettings

\def\dofinaloutput#1#2%
  {\beforefinaloutput
   \the\everybeforeshipout
   \ifspecialbasedsettings
     \myshipout{\hbox{\hbox to \!!zeropoint{\the\pageboundsettings}\hbox{\dopagebody#1#2\setpagecounters}}}%
   \else
     \the\pageboundsettings
     \myshipout{\hbox{\dopagebody#1#2\setpagecounters}}%
   \fi
   \the\everyaftershipout
   \afterfinaloutput
   \popcolor}  % ... and here ...

\def\donofinaloutput#1#2%
  {\beforefinaloutput
   \the\everybeforeshipout
   \setpagecounters
   \message{[-\the\realpageno]}%
   \setbox0=\hbox
     {\the\everyshipout
      \dopagebody#1#2}%
   \deadcycles=0
   \gotonextrealpage
   \the\everyaftershipout
   \afterfinaloutput
   \popcolor}  % ... and here

\let\checkpageversion=\relax

\def\finaloutput#1#2%
  {\checkpageversion
   \ifverwerken
     \ifgeselecteerd
       \dofinaloutput#1#2%
     \else
       \donofinaloutput#1#2%
     \fi
   \else
     \ifgeselecteerd
       \donofinaloutput#1#2%
     \else
       \dofinaloutput#1#2%
     \fi
   \fi
   \resetselectiepagina
   \verhoogpaginanummer
   \checkpagedimensions
   \ifnum\outputpenalty>-\@MM\relax
   \else
     \dosupereject
   \fi
   \douitstellen}

\def\dooutput%
  {\finaloutput\unvbox\normalpagebox}

\maxdeadcycles=1000

\output={\dooutput}

\newbox\leftlogos
\newbox\rightlogos

\newif\ifnewlogos

% \logostatus
%
% 0 = niet plaatsen                    > 0
% 1 = direkt plaatsen                  > 1
% 2 = berekenen en plaatsen            > 1
% 3 = een pagina berekenen en plaatsen > 2

\def\logostatus{0}

\def\gedefinieerdebeeldmerken{}
\def\teplaatsenbeeldmerken{}

\def\dodefinieerbeeldmerk[#1][#2][#3][#4]%
  {\addtocommalist{#1}\gedefinieerdebeeldmerken
   \setvalue{\??lo#2#3}{#1}%
   \getparameters[\??lo#2#3][#4]%
   \gdef\logostatus{2}}

\def\definieerbeeldmerk%
  {\doquadrupleargument\dodefinieerbeeldmerk}

\def\complexplaatsbeeldmerken[#1]%
  {\xdef\teplaatsenbeeldmerken{#1}%
   \gdef\logostatus{3}}

\def\simpleplaatsbeeldmerken%
  {\global\let\teplaatsenbeeldmerken=\gedefinieerdebeeldmerken
   \gdef\logostatus{3}}

\definecomplexorsimple\plaatsbeeldmerken

\def\doplaatsbeeldmerken#1#2%
  {\bgroup
   \setbox0=\vbox
     {\hbox
        {\ifnum\logostatus=3
           \def\docommando##1%
             {\ExpandBothAfter\doifinset{\getvalue{\??lo#1##1}}{\teplaatsenbeeldmerken}
                {#2{\hbox{\getvalue{\??lo#1##1\c!commando}}}}}%
         \else
           \def\docommando##1%
             {\doifvalue{\??lo#1##1\c!status}{\v!start}
                {#2{\hbox{\getvalue{\??lo#1##1\c!commando}}}}}%
         \fi
         \def\dodocommando##1##2##3##4##5##6%
           {\hskip\linkerrandafstand
           %\hskip\pageseparation
            \hbox to \linkermargebreedte{\docommando{##2}\hss}%
            \hskip\linkermargeafstand
            \hbox to \zetbreedte{\docommando{##3}\hss\docommando{##4}}%
            \hskip\rechtermargeafstand
            \hbox to \rechtermargebreedte{\hss\docommando{##5}}%
           %\hskip\pageseparation
            \hskip\rechterrandafstand
            \hbox to \rechterrandbreedte{\hss\docommando{##6}}}%
         \normalbaselines
         \hsmash
           {\hbox to \zetbreedte{\hss\docommando\c!midden\hss}}%
         \hsmash
           {\doifbothsides
              \hskip-\rugwit
            \orsideone
              \hskip-\rugwit
            \orsidetwo
              \hskip-\papierbreedte
              \hskip+\rugwit
              \hskip+\zetbreedte
            \od
            \hbox to \papierbreedte{\docommando\v!pagina\hss}}%
         \swapmargins
         \goleftonpage
         \doifbothsidesoverruled
           \dodocommando
             {\v!linkerrand}{\v!linkermarge}{\v!links}
             {\v!rechts}{\v!rechtermarge}{\v!rechterrand}%
         \orsideone
           \dodocommando
             {\v!linkerrand}{\v!linkermarge}{\v!links}
             {\v!rechts}{\v!rechtermarge}{\v!rechterrand}%
         \orsidetwo
           \dodocommando
             {\v!rechterrand}{\v!rechtermarge}{\v!rechts}
             {\v!links}{\v!linkermarge}{\v!linkerrand}%
         \od}}%
   \getboxheight\dimen0\of\box0\relax
   \vskip-\dimen0
   \box0
   \egroup}

\def\setlogobox#1#2%
  {\global\setbox#1=\vbox to \papierhoogte
     {\offinterlineskip
      \mindermeldingen
      \calculatereducedvsizes
      #2\relax
      \vskip-\kopwit
      \doplaatsbeeldmerken\v!boven\vsmash
      \vskip\kopwit
      \doplaatsbeeldmerken\v!hoofd\vsmash
      \vskip\hoofdhoogte
      \vskip\hoofdafstand
      \doplaatsbeeldmerken\v!tekst\vsmash  % evt \vbox
      \vskip\teksthoogte
      \vskip\voetafstand
      \vskip\voethoogte
      \doplaatsbeeldmerken\v!voet\vbox
      \vfilll
      \doplaatsbeeldmerken\v!onder\vbox%
      \vskip\kopwit}
  \smashbox#1}

\def\setlogoboxes%
  {\showmessage{\m!layouts}{7}{}%
   \setlogobox\leftlogos\relax
   \ifdubbelzijdig
     \setlogobox\rightlogos\doswapmargins
   \fi}

\def\getlogobox%
  {\ifnum\logostatus>0
     \ifnum\logostatus=3
       \setlogoboxes
       \gdef\logostatus{2}%
     \else\ifnum\logostatus=2
       \setlogoboxes
       \gdef\logostatus{1}%
     \else\ifnewlogos
       \gdef\logostatus{2}%
       \setlogoboxes
       \gdef\logostatus{1}%
       \global\newlogosfalse
     \fi\fi\fi
     \doifmarginswapelse
       {\copy\leftlogos}
       {\copy\rightlogos}%
   \fi}

% \frenchspacing leidt soms tot afbreken tussen -, vandaar
% de variant \newfrenchspacing.

\def\dofrenchspacing#1%
  {\sfcode`\.#1 \sfcode`\,#1\relax
   \sfcode`\?#1 \sfcode`\!#1\relax
   \sfcode`\:#1 \sfcode`\;#1\relax}

\def\frenchspacing%
  {\dofrenchspacing{1000}}   % \@m

\def\newfrenchspacing%
  {\dofrenchspacing{1050}}   % \@ml

\def\complexstelspatieringin[#1]%
  {\processaction
     [#1]
     [\v!opelkaar=>\newfrenchspacing,
          \v!ruim=>\nonfrenchspacing]%
   \updateraggedskips}

\def\simplestelspatieringin%
  {\updateraggedskips}

\definecomplexorsimple\stelspatieringin

\bgroup
\catcode`\~=\@@active       % eigenlijk is ~ al actief
\gdef\fixedspaces%          % in Plain \TeX, maar we weten
  {\catcode`\~=\@@active    % nooit wat er inmiddels is
   \def~{\fixedspace}}      % gebeurd, vandaar.
\egroup

\def\space      { }
\def\fixedspace {\hskip.5em\relax}
\def\nospace    {\unskip\ignorespaces}

\let\spatie     \space
\let\hardespatie\fixedspace
\let\geenspatie \nospace

\def\opelkaar%
  {\nointerlineskip}

\def\omlaag[#1]% nog eens mooier, relateren aan blanko
  {\nointerlineskip
   \vskip#1 }

\newskip\tussenwit
\tussenwit=\!!zeropoint

\def\blankokleinmaat%
  {\smallskipamount}

\def\blankomiddelmaat%
  {\medskipamount}

\def\blankogrootmaat%
  {\bigskipamount}

\def\currentwitruimte%
  {\!!zeropoint}

\def\stelwitruimteopnieuwin%
  {\expanded{\stelwitruimtein[\currentwitruimte]}}

\newif\ifwitruimteflexibel \witruimteflexibeltrue

%\def\dodostelwitruimtein[#1]%
%  {%\witruimteflexibeltrue
%   \processallactionsinset
%     [#1]
%     [\v!herstel=>,
%         \v!vast=>\witruimteflexibelfalse,
%     \v!flexibel=>\witruimteflexibeltrue,
%        \v!regel=>\tussenwit=\baselineskip,
%   \v!halveregel=>\tussenwit=.5\baselineskip,
%      \s!default=>\doifnot{\currentwitruimte}{\v!geen}
%                    {\stelwitruimteopnieuwin},
%      \s!unknown=>\@EA\assigndimension\@EA{\commalistelement} % \@EA is nodig
%                    {\tussenwit}
%                    {\blankokleinmaat}{\blankomiddelmaat}{\blankogrootmaat}]%   % te vangen
%   \edef\currentwitruimte%
%     {\ifdim\tussenwit=\!!zeropoint
%        \v!geen
%      \else
%        \ifgridsnapping\the\baselineskip\else\the\tussenwit\fi
%      \fi}%
%   \ifgridsnapping
%     \witruimteflexibelfalse
%     \tussenwit=1\tussenwit
%     \ifdim\tussenwit>\!!zeropoint
%       \tussenwit=\baselineskip
%     \fi
%   \else
%     \ifwitruimteflexibel \else \tussenwit=1\tussenwit \fi
%   \fi
%   \parskip=\tussenwit}
%
%\def\dostelwitruimtein[#1]%
%  {\expanded{\dodostelwitruimtein[#1]}}
%
%\def\stelwitruimtein%
%  {\dosingleempty\dostelwitruimtein}

\definecomplexorsimple\stelwitruimtein

\def\complexstelwitruimtein[#1]%
  {\expanded{\dostelwitruimtein[#1]}%
   \dodostelwitruimtein}

\def\dostelwitruimtein[#1]%
  {\processallactionsinset
     [#1]
     [\v!herstel=>,
         \v!vast=>\witruimteflexibelfalse,
     \v!flexibel=>\witruimteflexibeltrue,
        \v!regel=>\tussenwit=\baselineskip,
   \v!halveregel=>\tussenwit=.5\baselineskip,
      \s!default=>\doifnot{\currentwitruimte}{\v!geen}
                    {\stelwitruimteopnieuwin},
      \s!unknown=>\@EA\assigndimension\@EA{\commalistelement} % \@EA is nodig
                    {\tussenwit}
                    {\blankokleinmaat}{\blankomiddelmaat}{\blankogrootmaat}]}   % te vangen

\def\dodostelwitruimtein%
  {\edef\currentwitruimte%
     {\ifdim\tussenwit=\!!zeropoint
        \v!geen
      \else
        \ifgridsnapping\the\baselineskip\else\the\tussenwit\fi
      \fi}%
   \ifgridsnapping
     \witruimteflexibelfalse
     \tussenwit=1\tussenwit
     \ifdim\tussenwit>\!!zeropoint
       \tussenwit=\baselineskip
     \fi
   \else
     \ifwitruimteflexibel \else \tussenwit=1\tussenwit \fi
   \fi
   \parskip=\tussenwit}

\def\simplestelwitruimtein% == snelle \stelwitruimtein[\s!default]
  {\doifnot{\currentwitruimte}{\v!geen}
     {\stelwitruimteopnieuwin}%
   \dodostelwitruimtein}

\def\geenwitruimte%
  {\ifdim\parskip>\!!zeropoint\relax
     \ifdim\lastskip=-\parskip
     \else
       \vskip-\parskip
     \fi
   \fi}

\def\savecurrentwitruimte%
  {\edef\restorecurrentwitruimte%
     {\tussenwit=\the\tussenwit
      \parskip=\the\parskip
      \noexpand\def\noexpand\currentwitruimte{\currentwitruimte}%
      \ifwitruimteflexibel
        \noexpand\witruimteflexibeltrue
      \else
        \noexpand\witruimteflexibelfalse
      \fi}}

% deze variant is nodig binnen \startopelkaar
% steeds testen:
%
% \hoofdstuk{..}
% \plaatslijst[..]
% \hoofdstuk{..}
% \input tufte
%
% met/zonder witruimte

\def\witruimte%
  {\par
   \ifdim\parskip>\!!zeropoint\relax
    %\ifdim\lastskip>\parskip \else
     % \removelastskip interferes with blanko blokkeer en klein
       \vskip\parskip
    %\fi
   \fi}

\def\nonoblanko[#1]%
  {\par}

\def\noblanko%
  {\dosingleempty\nonoblanko}

% De onderstaande macro handelt ook de situatie dat er geen
% tekst tussen \start ... \stop is geplaatst. Daartoe wordt de
% laatste skip over de lege tekst heen gehaald. Dit komt goed
% van pas bij het plaatsen van (mogelijk lege) lijsten.

\newif\ifopelkaar

\def\noparskipsignal {0.00001pt}
\def\lastdoneparskip {0pt}

\def\startopelkaar%
  {\dosingleempty\dostartopelkaar}

\def\dostartopelkaar[#1]% nesting afvangen
  {\par
   \ifvmode
     \edef\lastdoneparskip{\the\lastskip}%
\edef\lastdoneprevdepth{\the\prevdepth}% zeer recent toegevoegd
     \ifdim\prevdepth=-1000pt   % toegevoegd omdat binnen
     \else                      % een vbox een extra skip
       \witruimte               % ongewenst is; dit kan
\baselinecorrection %% zie in \plaatsregister[n=1]
       \vskip\noparskipsignal   % waarschijnlijk ook in
     \fi                        % blanko blokkeer
     \bgroup
     \doifelse{#1}{\v!blanko}
       {\opelkaarfalse}
       {\opelkaartrue}%
     \blanko[\v!blokkeer]%
     \stelwitruimtein[\v!geen]
  \fi}

\def\stopopelkaar%
  {\par
\ifvmode
   \egroup
   \ifdim\lastskip=\noparskipsignal\relax
     \removelastskip
     \geenwitruimte
     \vskip-\lastdoneparskip
     \vskip+\lastdoneparskip
\prevdepth-\lastdoneprevdepth % zeer recent toegevoegd
   \fi
\fi}

\def\startvanelkaar%
  {\blanko
   \leavevmode
   \bgroup}

\def\stopvanelkaar%
  {\egroup
   \blanko}

% De onderstaande macro's moeten nog eens nader worden uitgewerkt.
% Ze spelen een rol bij de spatiering rond omkaderde teksten
% en/of boxen zonder diepte.

\def\toonregelcorrectie   {\showbaselinecorrection}
\def\regelcorrectie       {\baselinecorrection}

\definecomplexorsimpleempty\startregelcorrectie

% \prevdepth crosses pageboundaries!

\let\dorondomregelcorrectie=\relax

\def\complexstartregelcorrectie[#1]%
  {\bgroup
   \processaction
     [#1]
     [ \v!blanko=>\let\dorondomregelcorrectie=\blanko,
      \s!default=>\let\dorondomregelcorrectie=\relax,
      \s!unknown=>{\def\dorondomregelcorrectie{\blanko[#1]}}]%
   \dorondomregelcorrectie
   \startbaselinecorrection
   \offbaselinecorrection}

\def\stopregelcorrectie%
  {\stopbaselinecorrection
   \dorondomregelcorrectie
   \egroup}

\def\corrigeerwitruimte%
  {\dowithnextbox
     {\startbaselinecorrection
      \box\nextbox
      \stopbaselinecorrection}%
   \vbox}

%D There are two ways to influence the interline spacing. The
%D most general and often most consistent way is using
%D
%D \showsetup{\y!stelinterliniein}
%D
%D For instance
%D
%D \starttypen
%D \setupinterlinespace[line=2.8ex]
%D \stoptypen
%D
%D This setting adapts itself to the bodyfontsize, while for
%D instance saying
%D
%D \starttypen
%D \setupinterlinespace[line=12pt]
%D \stoptypen
%D
%D sets things fixed for all sizes, which is definitely not
%D what we want. Therefore one can also say:
%D
%D \starttypen
%D \definecorpsenvironment[9pt][interlinespace=11pt]
%D \stoptypen
%D
%D One can still use \type{\setupinterlinespace} (without
%D arguments) to set the interline space according to the
%D current font, e.g. a \type{\bfa}.

\newif\iflocalinterlinespace

% font-ini

\ifx\bodyfontinterlinespecs\undefined

  \let\bodyfontinterlinespecs\empty
  \let\bodyfontinterlinespace\empty

\fi

\def\presetnormallineheight%
  {\edef\normallineheight{\@@itregel}%
   \iflocalinterlinespace \else
     \doifdefined{\bodyfontinterlinespecs}
       {\doifsomething{\bodyfontinterlinespace}
          {\edef\normallineheight{\bodyfontinterlinespace}}}%
   \fi}

\def\complexstelinterliniein[#1]% \commalistelement ipv #1
  {\doifassignmentelse{#1}
     {\getparameters[\??it][#1]%
      \scratchdimen=0\@@ithoogte pt
      \advance\scratchdimen by 0\@@itdiepte pt
      \ifdim\scratchdimen>1pt
        \showmessage{\m!layouts}{10}{\@@ithoogte,\@@itdiepte}%
        \let\@@ithoogte=\strutheightfactor
        \let\@@itdiepte=\strutdepthfactor
      \else
        \let\strutheightfactor=\@@ithoogte
        \let\strutdepthfactor=\@@itdiepte
      \fi
      \let\minimallinedistance=\@@itafstand
      \let\normallineheight=\@@itregel % let ! ! ! ! ! ivm ex 
      \let\topskipfactor=\@@itboven
      \let\maxdepthfactor=\@@itonder
      \setfontparameters % redundant \setstrut
      \updateraggedskips} % yes indeed
     {\processallactionsinset % \regelwit = dummy !
        [#1]
        [     \v!aan=>\oninterlineskip,
              \v!uit=>\offinterlineskip,
            \v!reset=>\setfontparameters,
          \s!unknown=>\assignvalue{#1}{\regelwit}{1.00}{1.25}{1.50}%
                      \spacing{\regelwit}]}}

\def\simplestelinterliniein%
  {\localinterlinespacetrue
   \setfontparameters
   \updateraggedskips % funny one here
   \localinterlinespacefalse}

\definecomplexorsimple\stelinterliniein

% In earlier versions \type{\bigskipamount} was
% \type{\ht\strutbox} and the stretch was plus or minus
% \type{.4\dp\strutbox}. Don't ask me why. The most recent
% implementation is based on a user supplied distance, which
% is by default \type{.75\normalskipamount} where
% \type{\normalskipamount} equals the current baseline
% distance.

\newif\ifblankoreset        \blankoresetfalse
\newif\ifblankoblokkeer     \blankoblokkeerfalse
\newif\ifblankogeenwit      \blankogeenwitfalse
\newif\ifdoeblanko          \doeblankofalse
\newif\ifblankoflexibel     \blankoflexibeltrue
\newif\ifblankobuiten
\newif\ifblankoforceer

\newskip\blankoskip         \blankoskip=\bigskipamount
\newskip\blankoskipamount

\def\skipfactor       {.75}
\def\skipgluefactor   {.25}

\def\normalskipamount%
  {\openlineheight
     \ifgridsnapping \else \ifblankoflexibel
       \!!plus\skipgluefactor\openlineheight
       \!!minus\skipgluefactor\openlineheight
     \fi \fi
   \relax}

\def\regelafstand{\normalskipamount}

\def\deblankoskip{\skipfactor\regelafstand}

\def\laatsteblankoskip%
  {\blankoskip}

\def\geenblanko%
  {\removelastskip}

\def\dosingleblanko#1% ook nog \v!halveregel+fuzzysnap
  {\doifelse{#1}{\v!regel}
     {\blankoskipamount=\openlineheight}
     {\ifgridsnapping
        \assigndimension{#1}{\blankoskipamount}%
          {.25\openlineheight}{.5\openlineheight}{\openlineheight}%
      \else
        \assigndimension{#1}{\blankoskipamount}%
          {\smallskipamount}{\medskipamount}{\bigskipamount}%
      \fi}%
   \global\advance\blankoskip by \blankoskipamount}

\newif\iffuzzyvskip

% old
%
% \def\doblanko#1%
%   {\processallactionsinset
%      [#1]
%      [    \v!groot=>\dosingleblanko\v!groot, % happens often
%          \v!buiten=>\ifvmode\ifinner\blankobuitentrue\fi\fi,
%           \v!reset=>\global\blankoresettrue,
%        \v!flexibel=>\global\lokaalblankoflexibeltrue,
%            \v!vast=>\global\lokaalblankovasttrue,
%            \v!back=>\geenblanko,
%             \v!wit=>\global\advance\blankoskip by \parskip,
%         \v!formule=>\global\advance\blankoskip by \medskipamount,
%         \v!geenwit=>\global\blankogeenwittrue,
%            -\v!wit=>\global\advance\blankoskip by -\parskip,
%        \v!blokkeer=>\global\blankoblokkeertrue,
%         \v!forceer=>\global\blankoforceertrue,
%           \v!regel=>\global\advance\blankoskip by \lineheight,
%      \v!halveregel=>\global\fuzzyvskiptrue\global\advance\blankoskip by .5\lineheight,
%         \s!unknown=>{\herhaalmetcommando[#1]\dosingleblanko}]}
%
% new, see below

\def\doblanko#1%
  {\processallactionsinset
     [#1]
     [    \v!groot=>\dosingleblanko\v!groot, % happens often
         \v!buiten=>\ifvmode\ifinner\blankobuitentrue\fi\fi,
          \v!reset=>\global\blankoresettrue,
       \v!flexibel=>\global\lokaalblankoflexibeltrue,
           \v!vast=>\global\lokaalblankovasttrue,
           \v!back=>\geenblanko,
            \v!wit=>\global\advance\blankoskip by \parskip,
        \v!formule=>\global\advance\blankoskip by \medskipamount,
        \v!geenwit=>\global\blankogeenwittrue,
           -\v!wit=>\global\advance\blankoskip by -\parskip,
       \v!blokkeer=>\global\blankoblokkeertrue,
        \v!forceer=>\global\blankoforceertrue,
          \v!regel=>\global\advance\blankoskip by \lineheight,
     \v!halveregel=>\global\fuzzyvskiptrue\global\advance\blankoskip by .5\lineheight,
        \s!unknown=>\doindirectblanko{#1}]}

\def\oldprevdepth{\prevdepth}%
\def\newprevdepth{-1001pt}

\def\mindimen{0.00002pt} % beter 1sp 

\newif\iflokaalblankovast
\newif\iflokaalblankoflexibel

\def\docomplexdoblanko[#1]% pas op \relax's zijn nodig ivm volgende \if
  {\global\blankoresetfalse
   \global\blankoblokkeerfalse
   \global\blankogeenwitfalse
   \global\lokaalblankoflexibelfalse
   \global\lokaalblankovastfalse
   \global\blankoskip=\!!zeropoint
   \global\blankoforceerfalse
   \blankobuitenfalse
   \processcommalist[#1]\doblanko
\ifdim\blankoskip=\!!zeropoint\relax
  \iflokaalblankoflexibel \dosingleblanko\currentblanko \fi
  \iflokaalblankovast     \dosingleblanko\currentblanko \fi
\fi
   \ifblankobuiten
   \else
     \par
     \ifvmode          %in pos fonts gaat dit mis 
       \ifblankoforceer%\ifdim\prevdepth>\!!zeropoint\else
         % -1000pt signals top of page or column (\ejectcolumn)
         \vbox{\strut}\kern-\lineheight
       \fi%\fi
       \ifblankoblokkeer
         \global\doeblankofalse
         \edef\oldprevdepth{\the\prevdepth}%
         \prevdepth=\newprevdepth
       \else
         \global\doeblankotrue
       \fi
       \ifblankoreset
         \global\doeblankotrue
         \ifdim\prevdepth=\newprevdepth
           \prevdepth=\oldprevdepth
         \fi
       \fi
       \ifdoeblanko
         \ifdim\lastskip<\blankoskip\relax
           % else when \blanko[2*groot] + \blanko[3*groot] with parskip
           % equaling 1*groot, gives a groot=\parskip so adding a small
           % value makes it distinguishable; can also be done at parskip
           % setting time (better)
           \global\advance\blankoskip by \mindimen\relax % = skip
           % test this on 2* + 3* and parskip groot
           \ifblankogeenwit
             \global\advance\blankoskip by -\parskip
           \else
             \ifdim\lastskip=\parskip
             \else  % force this due to previous comment
               \ifdim\parskip>\!!zeropoint\relax
                 \ifdim\blankoskip<\parskip\relax
                   \global\blankoskip=\!!zeropoint
                 \else
                   \global\advance\blankoskip by -\parskip
                 \fi
               \fi
             \fi
           \fi
\ifblankoflexibel \else
  \blankoskip=1\blankoskip 
\fi
\iflokaalblankovast 
  \blankoskip=1\blankoskip 
\fi
\iflokaalblankoflexibel       
  \blankoskip=1\blankoskip 
    \!!plus\skipgluefactor\blankoskip
    \!!minus\skipgluefactor\blankoskip
\fi
           \ifdim\prevdepth=\newprevdepth
           \else
             \iffuzzyvskip
               \removelastfuzzyvskip
               \fuzzyvskip\blankoskip\relax
             \else
               \removelastskip
               \vskip\blankoskip\relax
             \fi
           \fi
         \else
           \iffuzzyvskip
             \removelastfuzzyvskip
             \fuzzyvskip\blankoskip\relax
           \fi
         \fi
       \fi
     \fi
   \fi
   \global\fuzzyvskipfalse
   \presetindentation}

\def\complexdodoblanko[#1]%
  {\flushfootnotes
   \ifopelkaar
     \ifinpagebody
       \expanded{\docomplexdoblanko[#1]}% \expanded=nieuw
     \else
       \par
     \fi
   \else
     \expanded{\docomplexdoblanko[#1]}% \expanded = nieuw
   \fi}

% old
%
% \def\doindirectblanko#1%
%   {\ifundefined{\??bo#1}% <-etex \expandafter\ifx\csname\??bo#1\endcsname\relax
%      \expanded{\complexdodoblanko[#1]}%
%    \else
%      \expandafter\complexdoblanko\expandafter[\csname\??bo#1\endcsname]%
%    \fi}
%
% \def\complexdoblanko[#1]% enables [force,8\bodyfontsize]
%   {\doifinstringelse{,}{#1}
%      {\expanded{\complexdodoblanko[#1]}}
%      {\doifnumberelse{#1}
%         {\expanded{\complexdodoblanko[#1]}}
%         {\doindirectblanko{#1}}}}
%
% new, more robust 
%
% \def\doindirectblanko#1%
%   {\edef\ascii{#1}\convertcommand\ascii\to\ascii
%    \ifundefined{\??bo\ascii}% <-etex \expandafter\ifx\csname\??bo#1\endcsname\rel
%      \herhaalmetcommando[#1]\dosingleblanko
%    \else
%      \expandafter\complexdoblanko\expandafter[\csname\??bo\ascii\endcsname]%
%    \fi}
% 
% even more robust

\def\doindirectblanko#1%
  {\edef\ascii{#1}\convertcommand\ascii\to\ascii
   \ifundefined{\??bo\ascii}% <-etex \expandafter\ifx\csname\??bo#1\endcsname
     \expanded{\herhaalmetcommando[#1]\noexpand\dosingleblanko}%
   \else
     \expandafter\complexdoblanko\expandafter[\csname\??bo\ascii\endcsname]%
   \fi}

\def\complexdoblanko[#1]% enables [force,8\bodyfontsize]
  {\expanded{\complexdodoblanko[#1]}}

\def\currentblanko%
  {\v!groot}

%D For a long time we had: 
%D
%D \starttypen 
%D \def\simpledoblanko%
%D   {\doifelse{\currentwitruimte}{\v!geen}
%D      {\blanko[\currentblanko]}
%D      {\blanko[\currentwitruimte]}}
%D \stoptypen
%D
%D But Berend de Boer wanted more control, so now we have:  

\def\simpledoblanko%
  {\doifelse{\currentwitruimte}{\v!geen}
     {\blanko[\currentblanko]}
     {\blanko[\s!default]}}

%D Another useful definition would be:
%D
%D \starttypen 
%D \definieerblanko
%D   [\s!default]
%D   [\v!groot]
%D \stoptypen

\def\blanko% % the \relax is definitely needed due to the many \if's
  {\relax\complexorsimple\doblanko}

%\def\dostelblankoin#1%
%  {\bgroup % rommelig 
%   \skip0=#1\relax
%   \xdef\globalblanko{\the\skip0}%
%   \egroup
%   \bigskipamount=\globalblanko
%   \smallskipamount=\globalblanko
%   \medskipamount=\globalblanko
%   \divide\medskipamount by 2\relax
%   \divide\smallskipamount by 4\relax}%

\def\dostelblankoin#1%
  {\bigskipamount=#1\relax
   \ifblankoflexibel \else
     \bigskipamount=1\bigskipamount
   \fi
   \smallskipamount=\bigskipamount
   \medskipamount=\bigskipamount
   \divide\medskipamount by 2
   \divide\smallskipamount by 4 }%

\def\complexstelblankoin[#1]%
  {\ifgridsnapping
     \blankoflexibelfalse
   \else
     \ExpandFirstAfter\processallactionsinset
       [#1]
       [ \v!flexibel=>\blankoflexibeltrue,
             \v!vast=>\blankoflexibelfalse]%
   \fi
   \ExpandFirstAfter\processallactionsinset
     [#1]
     [ \v!flexibel=>\dostelblankoin{\deblankoskip},
           \v!vast=>\dostelblankoin{\deblankoskip},
          \v!regel=>\edef\deblankoskip{\regelafstand}%
                    \dostelblankoin{\deblankoskip}%
                    \let\deblanko=\v!groot,
     \v!halveregel=>\scratchskip=.5\regelafstand
                    \edef\deblankoskip{\the\scratchskip}%
                    \dostelblankoin{\deblankoskip}%
                    \let\deblanko=\v!middel,
          \v!groot=>\ifgridsnapping
                      \edef\deblankoskip{\regelafstand}%
                      \dostelblankoin{\deblankoskip}%
                    \fi
                    \def\currentblanko{\v!groot}%
                    \let\deblanko=\v!groot,
         \v!middel=>\def\currentblanko{\v!middel}%
                    \let\deblanko=\v!middel,
          \v!klein=>\def\currentblanko{\v!klein}%
                    \let\deblanko=\v!klein,
        \v!normaal=>\dostelblankoin{\deblankoskip}%
                    \let\deblanko=\v!groot,
      \v!standaard=>\edef\deblankoskip{\skipfactor\regelafstand}%
                    \dostelblankoin{\deblankoskip}%
                    \let\deblanko=\v!groot,
        \s!default=>\dostelblankoin{\deblankoskip}%
                    \let\deblanko=\v!groot,
        \s!unknown=>\let\deblankoskip=\commalistelement
                    \dostelblankoin{\deblankoskip}%
                    \let\deblanko=\v!groot]%
   \stelwitruimtein}

% \definecomplexorsimpleempty\stelblankoin
%
% speed gain: 60 sec -> 30 sec

\definecomplexorsimple\stelblankoin

\def\simplestelblankoin% == snelle \stelblankoin[\s!default]
  {\ifgridsnapping
     \blankoflexibelfalse
   \fi
   \dostelblankoin{\deblankoskip}%
   \let\deblanko\v!groot
   \stelwitruimtein}

\def\dodefinieerblanko[#1][#2]%
  {\def\docommando##1{\setvalue{\??bo##1}{#2}}%
   \processcommalist[#1]\docommando}

\def\definieerblanko%
  {\dodoubleargument\dodefinieerblanko}

\def\savecurrentblanko%
  {\edef\restorecurrentblanko%
     {\bigskipamount=\the\bigskipamount
      \medskipamount=\the\medskipamount
      \smallskipamount=\the\smallskipamount
      \noexpand\def\noexpand\currentblanko{\currentblanko}%
      \ifblankoflexibel
        \noexpand\blankoflexibeltrue
      \else
        \noexpand\blankoflexibelfalse
      \fi}}

\def\inhibitblank% the fast, local way
  {\endgraf\ifvmode\prevdepth\newprevdepth\fi}

%D Now. 

\definieerblanko
  [\s!default]
  [\v!wit]

\let\currentvoorwit=\empty

\newdimen\voorwit

\newif\ifindentfirstparagraph % \indentfirstparagraphtrue

\def\presetindentation%
  {\doifoutervmode
     {\ifindentfirstparagraph\else\noindentation\fi}}

%\def\dostelinspringenin[#1]%
%  {\processallactionsinset
%     [#1]
%     [   \v!eerste=>\indentfirstparagraphtrue,
%       \v!volgende=>\indentfirstparagraphfalse,
%        \s!default=>\dodostelinspringenin,
%        \s!unknown=>\edef\currentvoorwit{\commalistelement}%
%                    \dodostelinspringenin]}
%
%\def\dodostelinspringenin%
%  {\assigndimension{\currentvoorwit}{\voorwit}{1em}{1.5em}{2em}%
%   \parindent=\voorwit\relax}
%
%\def\stelinspringenin%
%  {\dosingleempty\dostelinspringenin}

\definecomplexorsimple\stelinspringenin

\def\complexstelinspringenin[#1]%
  {\processallactionsinset
     [#1]
     [   \v!eerste=>\indentfirstparagraphtrue,
       \v!volgende=>\indentfirstparagraphfalse,
        \s!default=>\simplestelinspringenin,
        \s!unknown=>\edef\currentvoorwit{\commalistelement}%
                    \simplestelinspringenin]}

\def\simplestelinspringenin%
  {\assigndimension{\currentvoorwit}{\voorwit}{1em}{1.5em}{2em}%
   \parindent=\voorwit\relax}

\def\doinspringen[#1]%
  {\processallactionsinset
     [#1]
     [    \v!nee=>\parindent=\voorwit\relax\noindent,
         \v!niet=>\parindent=\voorwit\relax\noindent,
           \v!ja=>\parindent=\voorwit\relax,            % geen \indent !
       \v!eerste=>\indentfirstparagraphtrue,
     \v!volgende=>\indentfirstparagraphfalse,
       \v!altijd=>\parindent=\voorwit\relax,            % geen \indent !
        \v!nooit=>\parindent=\!!zeropoint\relax]}

\def\inspringen%
  {\dosingleargument\doinspringen}

\def\nietinspringen{\inspringen[\v!nee,\v!volgende]}
\def\welinspringen {\inspringen[\v!ja,\v!eerste]}

% Het gebruik van \skip's spaart \dimen's.

\newskip\xpositie
\newskip\ypositie

\newskip\xafmeting
\newskip\yafmeting

\newskip\xoffset
\newskip\yoffset

\newbox\positiebox

\def\startpositioneren%
  {\bgroup
   \xpositie=\!!zeropoint
   \ypositie=\!!zeropoint
   \xafmeting=\!!zeropoint
   \yafmeting=\!!zeropoint
   \xoffset=\!!zeropoint
   \yoffset=\!!zeropoint
   \hfuzz=30cm
   \vfuzz=30cm
   \setbox\positiebox=\hbox\bgroup}

\def\stoppositioneren%
  {\doifnot{\@@psoffset}{\v!ja}
     {\global\xoffset=\!!zeropoint
      \global\yoffset=\!!zeropoint}%
   \global\advance\xafmeting by \xoffset
   \global\advance\yafmeting by \yoffset
   \egroup
   \vbox to \yafmeting
     {\vskip\yoffset
      \hbox to \xafmeting
        {\hskip\xoffset
         \box\positiebox
         \hfill}%
      \vfill}%
   \egroup}

\def\resetpositioneren%
  {\getparameters[\??ps]
     [\c!status=\v!start,
      \c!eenheid=\s!cm,
      \c!factor=1,
      \c!xfactor=\@@psfactor,
      \c!yfactor=\@@psfactor,
      \c!schaal=1,
      \c!xschaal=\@@psschaal,
      \c!yschaal=\@@psschaal,
      \c!xstap=\v!absoluut,
      \c!ystap=\v!absoluut,
      \c!xoffset=\!!zeropoint,
      \c!yoffset=\!!zeropoint]}

\resetpositioneren

\def\stelpositionerenin%
  {\resetpositioneren
   \dodoubleargument\getparameters[\??ps]}%

% \def\positioneer(#1,#2)#3% \nextbox
%   {\setbox0=\hbox{#3}%
%    \def\berekenpositioneren##1##2##3##4##5##6##7##8##9%
%      {\skip0=##1\@@pseenheid\relax
%       \skip0=##8\skip0\relax
%       \skip0=##9\skip0\relax
%       \doifelse{##2}{\v!relatief}%
%         {\advance\skip0 by ##3\relax
%          \advance\skip0 by ##4\relax
%          \def##4{\!!zeropoint}}%
%         {\advance\skip0 by ##4\relax}%
%       ##3=\skip0\relax
%       \doifnot{\@@psstatus}{\v!overlay}
%         {\skip2=##5\relax
%          \advance\skip2 by ##3\relax
%          \ifdim##3<-##7\relax\global##7=-##3\relax\fi
%          \ifdim\skip2>##6\relax\global##6=\skip2\relax\fi}}%
%    \berekenpositioneren{#1}{\@@psxstap}{\xpositie}
%      {\@@psxoffset}{\wd0}{\xafmeting}{\xoffset}
%      {\@@psxschaal}{\@@psxfactor}%
%    \skip4=\ht0 \advance\skip4 by \dp0
%    \berekenpositioneren{#2}{\@@psystap}{\ypositie}
%      {\@@psyoffset}{\skip4}{\yafmeting}{\yoffset}
%      {\@@psyschaal}{\@@psyfactor}%
%    \vbox to \!!zeropoint % kan beter. 
%      {\vskip\ypositie
%       \hbox to \!!zeropoint
%         {\hskip\xpositie
%          \box0
%          \hskip-\xpositie}%
%       \vskip-\ypositie}%
%    \ignorespaces}

\def\berekenpositioneren#1#2#3#4#5#6#7#8#9%
  {\setdimensionwithunit\scratchskip{#1}\@@pseenheid % \scratchskip=#1\@@pseenheid
   \scratchskip=#8\scratchskip
   \scratchskip=#9\scratchskip
   \advance\scratchskip by #4\relax
   \doif{#2}{\v!relatief}%
     {\advance\scratchskip by #3%
      \let#4\!!zeropoint}%
   #3=\scratchskip\relax
   \doifnot{\@@psstatus}{\v!overlay}
     {\scratchskip=#5\relax
      \advance\scratchskip by #3\relax
      \ifdim#3<-#7\relax\global#7=-#3\relax\fi
      \ifdim\scratchskip>#6\relax\global#6=\scratchskip\relax\fi}}

\def\positioneer%
  {\dosingleempty\dopositioneer}

\def\dopositioneer[#1]#2(#3,#4)%
  {\dowithnextbox
     {\bgroup
      \getparameters[\??ps][#1]%
      \dontcomplain
      \berekenpositioneren{#3}{\@@psxstap}{\xpositie}
        {\@@psxoffset}{\wd\nextbox}{\xafmeting}{\xoffset}
        {\@@psxschaal}{\@@psxfactor}%
      \scratchdimen=\ht\nextbox \advance\scratchdimen by \dp\nextbox
      \berekenpositioneren{#4}{\@@psystap}{\ypositie}
        {\@@psyoffset}{\scratchdimen}{\yafmeting}{\yoffset}
        {\@@psyschaal}{\@@psyfactor}%
      \vbox to \!!zeropoint % kan beter. 
        {\vskip\ypositie
         \hbox to \!!zeropoint
           {\hskip\xpositie
            \box\nextbox
            \hskip-\xpositie}%
         \vskip-\ypositie}%
      \xdef\dopoppositioneer%
       {\xpositie=\the\xpositie
        \ypositie=\the\ypositie
        \noexpand\def\noexpand\@@psxoffset{\@@psxoffset}%
        \noexpand\def\noexpand\@@psyoffset{\@@psyoffset}}%
      \egroup
      \dopoppositioneer
      \ignorespaces}
   \hbox}

\newif\ifbinnenkolommen
\newif\if@@klbalanceren
\newif\if@@kluitlijnen

\binnenkolommenfalse

\def\stelkolommenin%
  {\dodoubleargument\dostelkolommenin}

\def\stelkolommenin[#1]%
  {\getparameters[\??kl][#1]%
   \nofcolumns=\@@kln\relax
   \processaction
     [\@@kllijn]
     [    \v!aan=>\let\betweencolumns=\linebetweencolumns,
          \v!uit=>\let\betweencolumns=\spacebetweencolumns,
      \s!default=>\let\betweencolumns=\spacebetweencolumns,
      \s!unknown=>\let\betweencolumns=\@@kllijn]}

\def\linebetweencolumns%
  {\bgroup
   \startcolorpage
   \ifdim\@@klafstand>\!!zeropoint
     \dimen0=\@@klafstand
   \else
     \dimen0=\linewidth
   \fi
   \advance\dimen0 by -\linewidth
   \hskip.5\dimen0
   \vrule
     \!!width\linewidth
     \ifb@selinebottom\!!depth\strutdepth\fi
   \hskip.5\dimen0\relax
   \stopcolorpage
   \egroup}

\def\spacebetweencolumns%
  {\hskip\@@klafstand}

\presetlocalframed[\??kl]

\def\backgroundfinishcolumnbox%
  {\doifinsetelse{\@@kloffset}{\v!geen,\v!overlay}
     {\let\@@kloffset\!!zeropoint}
     {\scratchdimen=\@@kloffset
      \advance\scratchdimen by -\@@kllijndikte
      \edef\@@kloffset{\the\scratchdimen}}%
   \localframed
     [\??kl]
     [\c!strut=\v!nee,
      \c!breedte=\v!passend,
      \c!hoogte=\v!passend,
      \c!uitlijnen=]}

\let\restorecolumnsettings\relax

\def\complexstartkolommen[#1]% %% \startkolommen
  {\bgroup
   \let\stopkolommen=\egroup
   \ifbinnenkolommen
   \else
     \stelkolommenin[#1]%
     \ifnum\@@kln>1\relax
       \witruimte
       \begingroup
       \doif{\@@kloptie}{\v!achtergrond}
         {\let\finishcolumnbox = \backgroundfinishcolumnbox
          \def\columntextoffset{\@@kloffset}}%
       \ifx\@@klcommando\empty\else
         \let\postprocesscolumnline\@@klcommando
       \fi
       \doifelsenothing{\@@klhoogte}
         {\heightencolumnsfalse}
         {\heightencolumnstrue}%
       \doifelse{\@@klrichting}{\v!rechts}
         {\reversecolumnsfalse}
         {\reversecolumnstrue}%
       \doifelse{\@@klbalanceren}{\v!ja}
         {\balancecolumnstrue}
         {\balancecolumnsfalse}%
       \processaction     % ook nog: laatsteuitlijnen
         [\@@kluitlijnen]
         [   \v!ja=>\stretchcolumnstrue
                    \inheritcolumnsfalse,
            \v!nee=>\stretchcolumnsfalse
                    \inheritcolumnsfalse,
          \v!tekst=>\stretchcolumnsfalse
                    \inheritcolumnstrue]%
       \nofcolumns=\@@kln
       %
       % probably more is needed, and how about nesting save's
       %
       \savecurrentblanko
       \savecurrentwitruimte
       \def\restorecolumnsettings%
         {\boxmaxdepth\maxdimen % done elsewhere
          \restorecurrentblanko
          \restorecurrentwitruimte}%
       %
       \edef\fixedcolumnheight{\@@klhoogte}%
\edef\minbalancetoplines{\@@klnboven}%
       \steltolerantiein[\@@kltolerantie]%     %% \startkolommen
       \stelblankoin[\@@klblanko]%
       \ifdim\tussenwit>\!!zeropoint
         \stelwitruimtein[\@@klblanko]%
       \fi
       \def\stopkolommen%
         {\endmulticolumns
          \global\binnenkolommenfalse
          \endgroup
          \egroup}%
       \global\binnenkolommentrue
       \beginmulticolumns
     \fi
   \fi}

\definecomplexorsimpleempty\startkolommen

%\def\kolom%
%  {\ifbinnenkolommen
%     \ejectcolumn
%   \fi}

\def\preferredejectcolumn%
  {\goodbreak}

% \def\forcedejectcolumn%
%   {\vfil
%    \penalty-200
%    \prevdepth=-1000pt % signals top of column to \blanko
%    \vfilneg}

\def\forcedejectcolumn%                               
  {\par                                                % todo: since 
   {\testrulewidth\!!zeropoint\ruledvskip\teksthoogte} % we misuse a 
   \penalty-200                                        % side effect
   \vskip-\teksthoogte
   \prevdepth=-1000pt} % signals top of column to \blanko

\def\kolom%
  {\dosingleempty\dokolom}

\def\dokolom[#1]%
  {\ifbinnenkolommen
     \iffirstargument
       \processaction
         [#1]
         [      \v!ja=>\forcedejectcolumn,
          \v!voorkeur=>\preferredejectcolumn]%
     \else
       \preferredejectcolumn
     \fi
   \fi}

%D Undocumented and still under development.

\def\startsimplecolumns%
  {\dosingleempty\dostartsimplecolumns}

\def\dostartsimplecolumns[#1]%
  {\bgroup
   \getparameters[\??kl][\c!breedte=\hsize,\c!afstand=1.5\korpsgrootte,\c!n=2,#1]%
   \setrigidcolumnhsize\@@klbreedte\@@klafstand\@@kln
   \setbox\scratchbox=\vbox\bgroup
   \forgetall} % \blanko[\v!blokkeer]

\def\stopsimplecolumns%
  {\removebottomthings 
   \egroup
   \rigidcolumnbalance\scratchbox
   \egroup}

\def\dotoonkader[#1][#2]%
  {\ifsecondargument
     \stelachtergrondenin
       [#1][#2]
       [\c!achtergrond=,
        \c!kader=\v!aan,
        \c!hoek=\v!recht,
        \c!kaderoffset=\!!zeropoint,
        \c!kaderdiepte=\!!zeropoint,
        \c!kaderkleur=]
   \else\iffirstargument
     \toonkader
       [\v!hoofd,\v!tekst,\v!voet]
       [#1]
   \else
     \toonkader
       [\v!hoofd,\v!tekst,\v!voet]
       [\v!linkerrand,\v!linkermarge,
        \v!tekst,
        \v!rechtermarge,\v!rechterrand]
   \fi\fi
   \stelachtergrondenin
     [\c!status=\v!herhaal]}

\def\toonkader{\dodoubleempty\dotoonkader}

\def\tooninstellingA#1#2%
  {#1&\PtToCm{\the#2}&\the#2&\tttf\string#2\cr}

\def\tooninstellingB#1#2#3%
  {#1&&#2#3&\tttf\string#3\cr}

\def\tooninstellingC#1#2% \relax is really needed here ! 
  {#1&\dimen0=#2\PtToCm{\the\dimen0}&\dimen0=#2\relax\the\dimen0&\tttf\string#2\cr}

%\startinterface english

\def\tooninstellingen% fallback 
  {\noindent
   \vbox
     {\forgetall
      \mindermeldingen
      \switchtobodyfont[\v!klein]
      \tabskip\!!zeropoint
      \halign
        {\strut##\quad\hss&##\quad\hss&##\quad\hss&##\hss\cr
         \tooninstellingA{paperheight}        \paperheight
         \tooninstellingA{paperwidth}         \paperwidth
         \tooninstellingA{printpaperheight}   \printpaperheight
         \tooninstellingA{printpaperwidth}    \printpaperwidth
         \noalign{\blanko}
         \tooninstellingA{topspace}           \topspace
         \tooninstellingA{backspace}          \backspace
         \tooninstellingA{height}             \makeupheight
         \tooninstellingA{width}              \makeupwidth
         \noalign{\blanko}
         \tooninstellingA{top}                \topheight
         \tooninstellingC{topdistance}        \topdistance
         \tooninstellingA{header}             \headerheight
         \tooninstellingC{headerdistance}     \headerdistance
         \tooninstellingA{textheight}         \textheight
         \tooninstellingC{footerdistance}     \footerdistance
         \tooninstellingA{footer}             \footerheight
         \tooninstellingC{bottomdistance}     \bottomdistance
         \tooninstellingA{bottom}             \bottomheight
         \noalign{\blanko}
         \tooninstellingA{leftedge}           \leftedgewidth
         \tooninstellingC{leftedgedistance}   \leftedgedistance
         \tooninstellingA{leftmargin}         \leftmarginwidth
         \tooninstellingC{leftmargindistance} \leftmargindistance
         \tooninstellingA{textwidth}          \textwidth
         \tooninstellingC{rightmargindistance}\rightmargindistance
         \tooninstellingA{rightmargin}        \rightmarginwidth
         \tooninstellingC{rightedgedistance}  \rightedgedistance
         \tooninstellingA{rightedge}          \rightedgewidth
         \noalign{\blanko}
         \tooninstellingB{bodyfontsize}       \the   \globalbodyfontsize
         \noalign{\blanko}
         \tooninstellingB{line}               \relax \normallineheight
         \tooninstellingB{height}             \relax \strutheightfactor
         \tooninstellingB{depth}              \relax \strutdepthfactor
         \tooninstellingB{topskip}            \relax \topskipfactor
         \tooninstellingB{maxdepth}           \relax \maxdepthfactor}}}

%\stopinterface

\startinterface dutch

\def\tooninstellingen%
  {\noindent
   \vbox
     {\forgetall
      \mindermeldingen
      \switchtobodyfont[\v!klein]
      \tabskip\!!zeropoint
      \halign
        {\strut##\quad\hss&##\quad\hss&##\quad\hss&##\hss\cr
         \tooninstellingA{papierhoogte}       \papierhoogte
         \tooninstellingA{papierbreedte}      \papierbreedte
         \tooninstellingA{printpapierhoogte}  \printpapierhoogte
         \tooninstellingA{printpapierbreedte} \printpapierbreedte
         \noalign{\blanko}
         \tooninstellingA{kopwit}             \kopwit
         \tooninstellingA{rugwit}             \rugwit
         \tooninstellingA{hoogte}             \zethoogte
         \tooninstellingA{breedte}            \zetbreedte
         \noalign{\blanko}
         \tooninstellingA{boven}              \bovenhoogte
         \tooninstellingC{bovenafstand}       \bovenafstand
         \tooninstellingA{hoofd}              \hoofdhoogte
         \tooninstellingC{hoofdafstand}       \hoofdafstand
         \tooninstellingA{teksthoogte}        \teksthoogte
         \tooninstellingC{voetafstand}        \voetafstand
         \tooninstellingA{voet}               \voethoogte
         \tooninstellingC{onderafstand}       \onderafstand
         \tooninstellingA{onder}              \onderhoogte
         \noalign{\blanko}
         \tooninstellingA{linkerrand}         \linkerrandbreedte
         \tooninstellingC{linkerrandafstand}  \linkerrandafstand
         \tooninstellingA{linkermarge}        \linkermargebreedte
         \tooninstellingC{linkermargeafstand} \linkermargeafstand
         \tooninstellingA{tekstbreedte}       \tekstbreedte
         \tooninstellingC{rechtermargeafstand}\rechtermargeafstand
         \tooninstellingA{rechtermarge}       \rechtermargebreedte
         \tooninstellingC{rechterrandafstand} \rechterrandafstand
         \tooninstellingA{rechterrand}        \rechterrandbreedte
         \noalign{\blanko}
         \tooninstellingB{korps}  \the   \globalbodyfontsize
         \noalign{\blanko}
         \tooninstellingB{regel}  \relax \normallineheight
         \tooninstellingB{hoogte} \relax \strutheightfactor
         \tooninstellingB{diepte} \relax \strutdepthfactor
         \tooninstellingB{boven}  \relax \topskipfactor
         \tooninstellingB{onder}  \relax \maxdepthfactor}}}

\stopinterface

\startinterface german

\def\tooninstellingen%
  {\noindent
   \vbox
     {\forgetall
      \mindermeldingen
      \switchtobodyfont[\v!klein]
      \tabskip\!!zeropoint
      \halign
        {\strut##\quad\hss&##\quad\hss&##\quad\hss&##\hss\cr
         \tooninstellingA{papierhoehe}          \papierhoehe
         \tooninstellingA{papierbreite}         \papierbreite
         \tooninstellingA{printpapierhoehe}     \printpapierhoehe
         \tooninstellingA{printpapierbreite}    \printpapierbreite
         \noalign{\blanko}
         \tooninstellingA{kopfweite}            \kopfweite
         \tooninstellingA{rumpfweite}           \rumpfweite
         \tooninstellingA{hoehe}                \satzhoehe
         \tooninstellingA{breite}               \satzbreite
         \noalign{\blanko}
         \tooninstellingA{oben}                 \hoeheoben
         \tooninstellingC{abstandoben}          \abstandoben
         \tooninstellingA{kopfzeile}            \kopfzeilenhoehe
         \tooninstellingC{kopfzeilenabstand}    \kopfzeilenabstand
         \tooninstellingA{texthoehe}            \texthoehe
         \tooninstellingC{fusszeileabstand}     \fusszeileabstand
         \tooninstellingA{fusszeilen}           \fusszeilenhoehe
         \tooninstellingC{abstandunten}         \abstandunten
         \tooninstellingA{hoeheunten}           \hoeheunten
         \noalign{\blanko}
         \tooninstellingA{linkerrand}           \breitelinkerrand
         \tooninstellingC{abstandlinkerrand}    \abstandlinkerrand
         \tooninstellingA{linkemarginal}        \linkemarginalbreite
         \tooninstellingC{linkemarginalafstand} \linkemarginalafstand
         \tooninstellingA{textbreite}           \textbreite
         \tooninstellingC{rechtemarginalafstand}\rechtemarginalafstand
         \tooninstellingA{rechtemarginal}       \rechtemarginalbreite
         \tooninstellingC{abstandrechterrand}   \abstandrechterrand
         \tooninstellingA{rechterrand}          \breiterechterrand
         \noalign{\blanko}
         \tooninstellingB{fliesstext}           \the   \globalbodyfontsize
         \noalign{\blanko}
         \tooninstellingB{linie}                \relax \normallineheight
         \tooninstellingB{hoehe}                \relax \strutheightfactor
         \tooninstellingB{tiefe}                \relax \strutdepthfactor
         \tooninstellingB{topskip}              \relax \topskipfactor
         \tooninstellingB{maxdepth}             \relax \maxdepthfactor}}}

\stopinterface

\startinterface czech

\def\tooninstellingen% 
  {\noindent
   \vbox
     {\forgetall
      \mindermeldingen
      \switchtobodyfont[\v!klein]
      \tabskip\!!zeropoint
      \halign
        {\strut##\quad\hss&##\quad\hss&##\quad\hss&##\hss\cr
         \tooninstellingA{vyskapapiru}          \vyskapapiru
         \tooninstellingA{sirkapapiru}          \sirkapapiru
         \tooninstellingA{vyskatiskpapiru}      \vyskatiskpapiru
         \tooninstellingA{sirkatiskpapiru}      \sirkatiskpapiru
         \tooninstellingA{hornimezera}          \hornimezera
         \tooninstellingA{spodnimezera}         \spodnimezera
         \tooninstellingA{vyska}                \vyskasazby
         \tooninstellingA{breite}               \sirkasazby
         \tooninstellingA{vyskatextu}           \vyskatextu
         \tooninstellingA{sirkatextu}           \sirkatextu
         \tooninstellingA{horejsek}             \vyskahorejsku
         \tooninstellingC{vzdalenosthorejsku}   \vzdalenosthorejsku
         \tooninstellingA{zahlavi}              \vyskazahlavi
         \tooninstellingC{vzdalenostzahlavi}    \vzdalenostzahlavi
         \tooninstellingC{fusszeileabstand}     \vzdalenostupati
         \tooninstellingA{upati}                \vyskaupati
         \tooninstellingC{vzdalenostspodku}     \vzdalenostspodku
         \tooninstellingA{spodek}               \vyakaspodku
         \tooninstellingA{levyokraj}            \sirkalevehookraje
         \tooninstellingC{vzdalenostlevehookraje}\vzdalenostlevehookraje
         \tooninstellingA{levamarginalie}       \sirkalevemarginalie
         \tooninstellingC{vzdalenostlevemarginalie}\vzdalenostlevemarginalie
         \tooninstellingC{vzdalenostpravemarginalie}\vzdalenostpravemarginalie
         \tooninstellingA{pravamarginalie}      \sirkapravemarginalie
         \tooninstellingC{vzdalenostpravehookraje}\vzdalenostpravehookraje
         \tooninstellingA{pravyokraj}           \sirkapravehookraje
         \noalign{\blanko}
         \tooninstellingB{zakladnivelikost}     \the   \globalbodyfontsize
         \noalign{\blanko}
         \tooninstellingB{linka}                \relax \normallineheight
         \tooninstellingB{vyska}                \relax \strutheightfactor
         \tooninstellingB{hloubka}              \relax \strutdepthfactor
         \tooninstellingB{topskip}              \relax \topskipfactor
         \tooninstellingB{maxdepth}             \relax \maxdepthfactor}}}

\stopinterface

\startinterface romanian

\def\tooninstellingen%
  {\noindent
   \vbox
     {\forgetall
      \mindermeldingen
      \switchtobodyfont[\v!klein]
      \tabskip\!!zeropoint
      \halign
        {\strut##\quad\hss&##\quad\hss&##\quad\hss&##\hss\cr
         \tooninstellingA{paperheight}        \paperheight
         \tooninstellingA{paperwidth}         \paperwidth
         \tooninstellingA{printpaperheight}   \printpaperheight
         \tooninstellingA{printpaperwidth}    \printpaperwidth
         \noalign{\blanko}
         \tooninstellingA{topspace}           \topspace
         \tooninstellingA{backspace}          \backspace
         \tooninstellingA{height}             \makeupheight
         \tooninstellingA{width}              \makeupwidth
         \noalign{\blanko}
         \tooninstellingA{top}                \topheight
         \tooninstellingC{topdistance}        \topdistance
         \tooninstellingA{header}             \headerheight
         \tooninstellingC{headerdistance}     \headerdistance
         \tooninstellingA{textheight}         \textheight
         \tooninstellingC{footerdistance}     \footerdistance
         \tooninstellingA{footer}             \footerheight
         \tooninstellingC{bottomdistance}     \bottomdistance
         \tooninstellingA{bottom}             \bottomheight
         \noalign{\blanko}
         \tooninstellingA{leftedge}           \leftedgewidth
         \tooninstellingC{leftedgedistance}   \leftedgedistance
         \tooninstellingA{leftmargin}         \leftmarginwidth
         \tooninstellingC{leftmargindistance} \leftmargindistance
         \tooninstellingA{textwidth}          \textwidth
         \tooninstellingC{rightmargindistance}\rightmargindistance
         \tooninstellingA{rightmargin}        \rightmarginwidth
         \tooninstellingC{rightedgedistance}  \rightedgedistance
         \tooninstellingA{rightedge}          \rightedgewidth
         \noalign{\blanko}
         \tooninstellingB{bodyfontsize}       \the   \globalbodyfontsize
         \noalign{\blanko}
         \tooninstellingB{line}               \relax \normallineheight
         \tooninstellingB{height}             \relax \strutheightfactor
         \tooninstellingB{depth}              \relax \strutdepthfactor
         \tooninstellingB{topskip}            \relax \topskipfactor
         \tooninstellingB{maxdepth}           \relax \maxdepthfactor}}}

\stopinterface

\def\toonlayout% interfereert lelijk met een \typefile er na
  {\bgroup
   \pagina
   \toonkader
   \stellayoutin[\c!markering=\v!aan]
   \dorecurse{4}{\tooninstellingen\pagina}
   \egroup}

\definetwopasslist{\s!paragraph}

\newcounter\nofraggedparagraphs

\def\doparagraphreference% looks very much like domarginreference
  {\doglobal\increment\nofraggedparagraphs\relax
   \edef\writeparref%
     {\writeutilitycommand%
        {\twopassentry%
           {\s!paragraph}%
           {\nofraggedparagraphs}%
           {\noexpand\realfolio}}}%
   \writeparref}

\def\setraggedparagraphmode#1#2%
  {\ifinner
     \ifdubbelzijdig
       \gettwopassdata{\s!paragraph}%
       \iftwopassdatafound
         \ifodd\twopassdata#1\else#2\fi
       \else
         \ifodd\realfolio#1\else#2\fi
       \fi
       \doparagraphreference
     \else
       #2\relax
     \fi
   \else
     #2\relax
   \fi}

% \let\doifrightpageelse\setraggedparagraphmode

% De onderstaande macro's zijn opgenomen in Plain TeX.
%
% \def\raggedright%
%   {\rightskip\z@ plus2em \spaceskip.3333em \xspaceskip.5em\relax}
%
% \def\ttraggedright%
%   {\tttf\rightskip\z@ plus2em\relax}
%
% \newif\ifr@ggedbottom
%
% \def\raggedbottom%
%   {\topskip 10\p@ plus60\p@ \r@ggedbottomtrue}
%
% \def\normalbottom%
%   {\topskip 10\p@ \r@ggedbottomfalse}
%
% en worden hieronder wat aangepast.

% the three boolean will become obsolete some day in favour 
% of \bottomraggedness

\chardef\bottomraggedness=0 % 0=ragged 1=normal/align 2=baseline

\def\bottomalignlimit{3\lineheight}

\newif\ifn@rmalbottom
\newif\ifr@ggedbottom
\newif\ifb@selinebottom

\def\raggedbottom%
  {\chardef\bottomraggedness=0
   \n@rmalbottomfalse
   \r@ggedbottomtrue
   \b@selinebottomfalse
   \settopskip}

\def\alignbottom%
  {\chardef\bottomraggedness=1
   \n@rmalbottomtrue
   \r@ggedbottomfalse
   \b@selinebottomfalse
   \settopskip}

\def\baselinebottom%
  {\chardef\bottomraggedness=2
   \n@rmalbottomfalse
   \r@ggedbottomfalse
   \b@selinebottomtrue
   \settopskip}

\let\normalbottom=\alignbottom % downward compatible

% so, the new one will be
% 
% \chardef\bottomraggedness=0 % 0=ragged 1=normal/align 2=baseline
% 
% \def\bottomalignlimit{3\lineheight} % will be settable 
% 
% \def\raggedbottom  {\chardef\bottomraggedness=0 \settopskip}
% \def\alignbottom   {\chardef\bottomraggedness=1 \settopskip}
% \def\baselinebottom{\chardef\bottomraggedness=2 \settopskip}
%
% \let\normalbottom  =\alignbottom

% \hyphenpenalty  = ( 2.5 * \hsize ) / \raggedness
% \tolerance     >= 1500 % was 200
% \raggedness     = 2 .. 6\korpsgrootte

\chardef\raggedstatus=0 % normal left center right 

\def\leftraggedness   {2\bodyfontsize}
\def\rightraggedness  {2\bodyfontsize}
\def\middleraggedness {6\bodyfontsize}

%D More hyphenation control, will be combined with align 
%D setup. 

\def\nohyphens%
  {\ifx\dohyphens\relax
     \edef\dohyphens%
       {\hyphenpenalty\the\hyphenpenalty
        \exhyphenpenalty\the\exhyphenpenalty\relax}%
   \fi
   \hyphenpenalty\@M
   \exhyphenpenalty\@M}

\let\dohyphens\relax

%D To prevent unwanted side effects, we also have to check 
%D for hyphens here: 

\def\setraggedness#1%
  {\ifnum\tolerance<1500\relax       % small values have
     \tolerance=1500\relax           % unwanted side effects
   \fi                               
   \spaceskip=2.5\hsize              % we misuse these registers
   \xspaceskip=#1\relax              % for temporary storage;
   \divide\spaceskip by \xspaceskip  % they are changed anyway
   \ifx\dohyphens\relax               
     \hyphenpenalty=\spaceskip       % \else no hyphens is active 
   \fi}                              

\let\updateraggedskips\relax

\def\setraggedskips#1#2#3#4#5#6#7% never change this name
  {\def\updateraggedskips%
     {\dosetraggedskips{#1}{#2}{#3}{#4}{#5}{#6}{#7}}%
   \updateraggedskips}

\def\dosetraggedskips#1#2#3#4#5#6#7%
  {\chardef\raggedstatus=#1\relax 
   \leftskip=1\leftskip\!!plus#2\relax   % zie: Tex By Topic 8.1.3
   \rightskip=1\rightskip\!!plus#3\relax % zie: Tex By Topic 8.1.3
   \spaceskip#4\relax
   \xspaceskip#5\relax
   \parfillskip\!!zeropoint\!!plus#6\relax
   \parindent#7\relax}

\def\notragged%
  {\setraggedskips{0}{0em}{0em}{0em}{0em}{1fil}{\parindent}}

\def\raggedleft%
  {\setraggedness\leftraggedness
   \setraggedskips{1}{\leftraggedness}{0em}{.3333em}{.5em}{0em}{0em}}

\def\raggedcenter%
  {\setraggedness\middleraggedness
   \setraggedskips{2}{\middleraggedness}{\middleraggedness}{.3333em}{.5em}{0em}{0em}}

%D We used to have:
%D
%D \starttypen
%D \def\raggedright%
%D   {\setraggedness\rightraggedness
%D    \setraggedskips{3}{0em}{\rightraggedness}{.3333em}{.5em}{0em}{\parindent}}
%D \stoptypen
%D
%D However, the next alternative, suggested by Taco, is better.

\def\raggedright%
  {\setraggedness\rightraggedness
   \setraggedskips{3}{0em}{\rightraggedness}{.3333em}{.5em}{1fil}{\parindent}}

\def\veryraggedleft%
  {\setraggedskips{1}{1fil}{0em}{.3333em}{.5em}{0em}{0em}}

%D When we want the last line to have a natural width:
%D
%D \starttypen
%D \def\veryraggedleft%
%D   {\setraggedskips{1}{1fil}{0em}{.3333em}{.5em}{0em}{-1fil}}
%D \stoptypen
%D
%D but this one is not accepted by the macros.

\def\veryraggedcenter%
  {\setraggedskips{2}{1fil}{1fil}{.3333em}{.5em}{0em}{0em}}

\def\veryraggedright%
  {\setraggedskips{3}{0em}{1fil}{.3333em}{.5em}{0em}{\parindent}}

\def\ttraggedright%
  {\tttf
   \setraggedskips{3}{0em}{\rightraggedness}{0em}{0em}{0em}{\parindent}} % {\voorwit}}

%D A bonus one:

\def\raggedwidecenter%
  {\setraggedness\middleraggedness
   \setraggedskips{2}{.5fil}{.5fil}{.3333em}{.5em}{0em}{0em}}

\newif\if@@asragged \@@asraggedtrue % old method 

\def\dodosteluitlijnenin[#1]%
  {\doifinsetelse{\v!ruim} {#1}{\!!doneatrue}{\!!doneafalse}%
   \doifinsetelse{\v!breed}{#1}{\!!donebtrue}{\!!donebfalse}%
   \ExpandFirstAfter\processallactionsinset
     [#1]
     [   \v!regel=>\baselinebottom,
         \v!onder=>\raggedbottom,
        \v!hoogte=>\normalbottom,
       \v!breedte=>\notragged,
       \v!normaal=>\notragged,
            \v!ja=>\notragged,
           \v!nee=>\raggedright,
\if@@asragged\v!binnen\else\v!buiten\fi
                 =>\setraggedparagraphmode\raggedleft\raggedright,
\if@@asragged\v!buiten\else\v!binnen\fi
                 =>\setraggedparagraphmode\raggedright\raggedleft,
\if@@asragged\v!links\else\v!rechts\fi
                 =>\if!!donea\veryraggedleft  \else\raggedleft  \fi,
\if@@asragged\v!rechts\else\v!links\fi
                 =>\if!!donea\veryraggedright \else\raggedright \fi,
        \v!midden=>\if!!doneb\raggedwidecenter\else
                   \if!!donea\veryraggedcenter\else\raggedcenter\fi\fi,
       \v!hangend=>\enableprotruding,
   \v!niethangend=>\disableprotruding,
    \v!afgebroken=>\dohyphens,
\v!nietafgebroken=>\nohyphens,
         \v!nieuw=>\@@asraggedfalse, % so new will give you consistency
         \v!reset=>\notragged\normalbottom]}

\def\dosteluitlijnenin[#1]%
  {\expanded{\dodosteluitlijnenin[#1]}}

\def\steluitlijnenin%
  {\dosingleargument\dosteluitlijnenin}

\def\startuitlijnen%
  {\bgroup
   \steluitlijnenin}

\def\stopuitlijnen
  {\par
   \egroup}

%\def\regellinks#1%
%  {\noindent\leftline{{\strut#1}}}
%
%\def\regelrechts#1%
%  {\noindent\rightline{{#1\strut}}}
%
%\def\regelmidden#1%
%  {\noindent\centerline{{\strut#1}}}

\def\doalignline#1#2%
  {\dowithnextbox
     {\noindent\hbox to \hsize
        {\strut#1\unhbox\nextbox#2}}
     \hbox}

% also supporting \\
%
% \def\doalignline#1#2%
%   {\dowithnextbox
%      {\noindent\hbox to \hsize
%         {\strut#1\unhbox\nextbox#2}}
%      \hbox\bgroup
%        \def\\{\egroup\par\doalignline#1#2\bgroup}\let\next=}

\def\doalignline#1#2%
  {\bgroup
   \def\\{\egroup\par\doalignline#1#2\bgroup}%
   \dowithnextbox
     {\noindent\hbox to \hsize
        {\strut#1\unhbox\nextbox#2}\egroup}
     \hbox}

% directe commando's

\def\regellinks {\doalignline \relax \hss  }
\def\regelrechts{\doalignline \hss   \relax}
\def\regelmidden{\doalignline \hss   \hss  }

\def\regelbegrensd#1{\limitatetext{#1}{\hsize}{\unknown}}

% indirecte commando's

\setvalue{regel\v!links }{\doalignline \relax \hss  }
\setvalue{regel\v!rechts}{\doalignline \hss   \relax}
\setvalue{regel\v!midden}{\doalignline \hss   \hss  }

\def\doregelplaats#1%
  {\getvalue{regel#1}}

\def\dosteltolerantiein[#1]%
  {\doifinsetelse{\v!vertikaal}{#1}%
     {\ExpandFirstAfter\processallactionsinset
        [#1]
        [\v!zeerstreng=>\def\bottomtolerance{},
             \v!streng=>\def\bottomtolerance{.050},
             \v!soepel=>\def\bottomtolerance{.075},
         \v!zeersoepel=>\def\bottomtolerance{.100}]}%
     {\ExpandFirstAfter\processallactionsinset
        [#1]
        [       \v!rek=>\emergencystretch=\bodyfontsize,
         \v!zeerstreng=>\tolerance=200,
             \v!streng=>\tolerance=1500,
             \v!soepel=>\tolerance=3000,
         \v!zeersoepel=>\tolerance=4500]}}

\def\steltolerantiein%
  {\dosingleargument\dosteltolerantiein}

\def\woordrechts%
  {\groupedcommand{\hfill\hbox}{\parfillskip\!!zeropoint}}

\newif\iflowinmargin

\def\stelinmargein%
  {\dodoubleempty\dostelinmargein}

\def\dostelinmargein[#1][#2]%
  {\ifsecondargument
     \doifundefinedelse{\??im#1\c!offset}
       {\presetlocalframed
          [\??im#1]%
        \getparameters
          [\??im#1]
          [\c!kader=\v!uit,
           \c!offset=\v!overlay,
           \c!regel=1,
           \c!scheider=,
           \c!breedte=\v!ruim,
           \c!afstand=\!!zeropoint,
           \c!letter=\@@imletter,
           \c!kleur=\@@imkleur,
           \c!plaats=\@@implaats,
           \c!uitlijnen=\@@imuitlijnen,
           \c!voor=\@@imvoor,
           \c!na=\@@imna,
           #2]}
       {\getparameters[\??im#1][#2]}%
   \else
     \getparameters[\??im][#1]%
   \fi}

\let\margetekstafstand  = \!!zeropoint
\def\margetekstregels     {1}
\def\margetekstnummer     {0}
\let\margetekstscheider = \empty

\def\margestrutheight{\ht\strutbox}

\def\maakmargetekstblok#1#2#3#4#5#6%
  {#4\relax
   \bgroup
   \forgetall % added, else problems with 'center' and nested itemize
   \mindermeldingen
   \hsize#1\relax
   \ifnum\margetekstnummer=0
     \def\margetekstnummer{#2}%
   \fi
\doifnumberelse{\margetekstnummer}
  {\ifnum\margetekstnummer>25 % to be translated 
     \writestatus{\m!systems}{potential margin stack overflow (\margetekstnummer)}%
   \fi}
  {}%
   \processaction
     [\getvalue{\??im\margetekstnummer\c!uitlijnen}]
     [     \v!ja=>\setvalue{\??im\margetekstnummer\c!uitlijnen}{#2},
          \v!nee=>\setvalue{\??im\margetekstnummer\c!uitlijnen}{\v!normaal},
       \v!binnen=>\setvalue{\??im\margetekstnummer\c!uitlijnen}{#2},
       \v!buiten=>\setvalue{\??im\margetekstnummer\c!uitlijnen}{#3},
        \v!links=>\setvalue{\??im\margetekstnummer\c!uitlijnen}{\v!links},
       \v!midden=>\setvalue{\??im\margetekstnummer\c!uitlijnen}{\v!midden},
       \v!rechts=>\setvalue{\??im\margetekstnummer\c!uitlijnen}{\v!rechts},
      \s!default=>\setvalue{\??im\margetekstnummer\c!uitlijnen}{#2}]%
   \setbox0=\vbox\localframed
     [\??im\margetekstnummer]
     [\c!strut=\v!nee]
     {\decrement\margetekstregels
      \@@imvoor
      \doattributes
        {\??im\margetekstnummer}\c!letter\c!kleur
        {\dorecurse{\margetekstregels}{\strut\\}%
         \xdef\margestrutheight{\the\ht\strutbox}%
         \begstrut#6\endstrut\endgraf}%
      \@@imna}%
   \ht0=\ht\strutbox
   \box0
   \egroup
   #5\relax}

\def\plaatsmargetekstscheider%
  {\ifnum\margincontent>0
     \bgroup
     \dimen0=\margetekstregels\lineheight
     \advance\dimen0 by -\lineheight
     \lower\dimen0\hbox{\margetekstscheider}%
     \egroup
   \fi}

\def\linkermargetekstblok#1%
  {\maakmargetekstblok
     {\linkermargebreedte}
     {\v!links}{\v!rechts}
     {\llap{\plaatsmargetekstscheider}}{\hskip\margetekstafstand}
     {#1}}

\def\rechtermargetekstblok#1%
  {\maakmargetekstblok
     {\rechtermargebreedte}
     {\v!rechts}{\v!links}
     {\hskip\margetekstafstand}{\rlap{\plaatsmargetekstscheider}}
     {#1}}

\def\doplacemargintext#1#2#3%
  {\strut
   \setbox0=\hbox{#1}%
   \dimen0=\ht0
   \advance\dimen0 by \dp0
   \ifdim\dimen0>\marginheight
     \global\marginheight=\dimen0
   \fi
   \setbox0=\hbox
     {#2{\hskip#3\strut
         \iflowinmargin\else
           \dimen0=\dp\strutbox
           \advance\dimen0 by \margestrutheight
           \advance\dimen0 by -\ht\strutbox
           \raise\dimen0
         \fi
         \box0}}%
   \ht0=\!!zeropoint
   \dp0=\!!zeropoint
   \gdef\margestrutheight{\the\ht\strutbox}%
  %\vadjust{\box0}} % fails in high math lines, let it be  
  %\hbox{\lower\dp\strutbox\box0}} % alas, wrong lapping, therefore useless 
   \dopositionmarginbox0}

%D This approach permits us to implement a better mechanism 
%D later.  

\ifx\dopositionmarginbox\undefined
  \def\dopositionmarginbox#1{\vadjust{\box#1}}
\fi

\def\doinlinker#1%
  {\doplacemargintext
     {\linkermargetekstblok{#1}\hskip\linkermargeafstand}
     \llap\!!zeropoint}

\def\doinrechter#1%
  {\doplacemargintext
     {\hskip\rechtermargeafstand\rechtermargetekstblok{#1}}
     \rlap\hsize}

\newcounter \nofmarginnotes
\newif      \iftrackingmarginnotes
\newif      \ifrightmargin            % documenteren

\definetwopasslist{\s!margin}

\def\domarginreference%
  {\doglobal\increment\nofmarginnotes\relax
   \edef\writemarref%
     {\writeutilitycommand%
        {\twopassentry%
           {\s!margin}%
           {\nofmarginnotes}%
           {\noexpand\realfolio}}}%
   \writemarref}

\def\dodoinmargenormal#1#2#3#4%
  {\iffirstsidefloatparagraph\geenwitruimte\fi % zo laat mogelijk
   \ifodd#1\relax
     \rightmargintrue
     #3{#4}%
   \else
     \rightmarginfalse
     #2{#4}%
   \fi}

\def\doinmargenormal#1#2#3%
  {\bgroup
   \iftrackingmarginnotes
     \gettwopassdata{\s!margin}%
     \iftwopassdatafound
       \dodoinmargenormal\twopassdata#1#2{#3}%
     \else
       \dodoinmargenormal\realfolio#1#2{#3}%
     \fi
     \domarginreference
   \else
     \dodoinmargenormal\realfolio#1#2{#3}%
   \fi
   \egroup}

\def\doinmargereverse#1#2#3%
  {\dodoinmargenormal\realfolio#2#1{#3}}

\def\doinmarge[#1][#2][#3][#4][#5]#6%
  {\doifcommonelse{+,-,\v!laag}{#4}
     {\dodoinmarge[#1][#2][#3][#4][#5]{#6}}
     {\dodoinmarge[#1][#2][#3][][#4]{#6}}%
   \ignorespaces}

\def\dodoinmarge[#1][#2][#3][#4][#5]#6%
  {\ignorespaces
   \doifinsetelse{\v!laag}{#4}
     {\lowinmargintrue}
     {\lowinmarginfalse}%
   \processaction
     [#1]
     [  \v!links=>#2{#6},
       \v!rechts=>#3{#6},
      \s!unknown=>\ifdubbelzijdig
                    \doifcommonelse{+,-}{#4}
                      {\doinmargereverse#2#3{#6}}
                      {\doinmargenormal#2#3{#6}}%
                  \else
                    #2{#6}%
                  \fi]%
   \rawpagereference{\s!mar}{#5}%
   \ignorespaces}

\def\inlinker%
  {\indentation\doquintupleempty\doinmarge
     [\v!links][\doinlinker][\doinrechter]}

\def\inrechter%
  {\indentation\doquintupleempty\doinmarge
     [\v!rechts][\doinlinker][\doinrechter]}

\def\inmarge%
  {\doquintupleempty\doinmarge
     [\@@implaats][\doinlinker][\doinrechter]}

\def\inanderemarge%
  {\doquintupleempty\doinmarge
     [\@@implaats][\doinrechter][\doinlinker]}

\newcounter\margincontent

\def\flushmargincontent% [#1][#2]#3% hier plaats 'globaal' (geen 1,2 enz)
  {\doinmarge[\@@implaats][\doinlinker][\doinrechter]} % [#1][#2]{#3}}

\newdimen\marginheight

\let\restoreinterlinepenalty=\relax

\def\flushmargincontents%
  {\restoreinterlinepenalty  % here?
   \ifnum\margincontent>0               % called quite often, so we
     \expandafter\doflushmargincontents % speed up the \fi scan by
   \fi}                                 % using a \do..

% \def\doflushmargincontents% % links + rechts
%   {\bgroup
%    \forgetall
%    \global\marginheight\!!zeropoint
%    \dorecurse{\margincontent}
%      {\bgroup
%       \edef\margetekstafstand {\getvalue{\??im\recurselevel\c!afstand}}%
%       \edef\margetekstregels  {\getvalue{\??im\recurselevel\c!regel}}%
%       \edef\margetekstscheider{\getvalue{\??im\recurselevel\c!scheider}}%
%       \let\margetekstnummer=\recurselevel
%       \getvalue{\??im\recurselevel}%
%       \global\setvalue{\??im\recurselevel}{}%
%       \egroup}%
%    \ifdim\marginheight>\lineheight % This is something real dirty!
%      \advance\marginheight by \pagetotal
%      \advance\marginheight by \lineheight  % a sort of bonus
%      \ifdim\marginheight>\pagegoal
%        \xdef\restoreinterlinepenalty%
%          {\global\let\restoreinterlinepenalty\relax
%           \global\interlinepenalty=\the\interlinepenalty}%
%        \global\interlinepenalty=10000
%      \fi
%    \else % We need the above because interlinepenalties overrule vadjusted \nobreaks.
%      %\vadjust
%      %  {\forgetall
%      %   \global\advance\marginheight by \lineheight
%      %   \global\divide\marginheight by \lineheight
%      %   \dorecurse{\number\marginheight}
%      %     {\nobreak\vskip\lineheight}%
%      %   \kern-\number\marginheight\lineheight}%
%      \vadjust{\nobreak}%
%    \fi
%    \doglobal\newcounter\margincontent
%    \egroup}

\def\doflushmargincontents% % links + rechts
  {\bgroup
   \forgetall
   \global\marginheight\!!zeropoint
   \dorecurse{\margincontent}
     {\bgroup
      \edef\margetekstafstand {\getvalue{\??im\recurselevel\c!afstand}}%
      \edef\margetekstregels  {\getvalue{\??im\recurselevel\c!regel}}%
      \edef\margetekstscheider{\getvalue{\??im\recurselevel\c!scheider}}%
      \let\margetekstnummer=\recurselevel
      \getvalue{\??im\recurselevel}%
      \global\setvalue{\??im\recurselevel}{}%
      \egroup}%
\ifbinnenkolommen 
  \donetrue  % how fuzzy 
\else\ifdim\marginheight>\lineheight\relax 
  \donetrue  % how dirty 
\else 
  \donefalse % how needed
\fi\fi
\ifdone
     \advance\marginheight by \pagetotal
     \advance\marginheight by \lineheight  % a sort of bonus
     \ifdim\marginheight>\pagegoal
       \xdef\restoreinterlinepenalty%
         {\global\let\restoreinterlinepenalty\relax
          \global\interlinepenalty=\the\interlinepenalty}%
       \global\interlinepenalty=10000
     \fi
   \else % We need the above because interlinepenalties overrule vadjusted \nobreaks.
     %\vadjust
     %  {\forgetall
     %   \global\advance\marginheight by \lineheight
     %   \global\divide\marginheight by \lineheight
     %   \dorecurse{\number\marginheight}
     %     {\nobreak\vskip\lineheight}%
     %   \kern-\number\marginheight\lineheight}%
     \vadjust{\nobreak}%
   \fi
   \doglobal\newcounter\margincontent
   \egroup}

% Some day: \definieermarkering[\v!margetitel]

\def\docomplexmargewoord#1#2#3%
  {\@EA\setgvalue\@EA{\@EA\??im\@EA\margincontent\@EA}\@EA
     {\@EA\stelinmargein\@EA[\margincontent][]%  see next macro
      \flushmargincontent[#1][#2]{#3}}}

\def\complexmargewoord[#1][#2]#3%
  {\doglobal\increment\margincontent
   \stelinmargein[\margincontent][]% see next macro
   \ifsecondargument
     \doifnumberelse{#1}
       {\docomplexmargewoord{#2}{#1}{#3}}
       {\docomplexmargewoord{#1}{#2}{#3}}%
   \else
     \doifnumberelse{#1}
       {\docomplexmargewoord{}{#1}{#3}}
       {\docomplexmargewoord{#1}{}{#3}}%
   \fi}

\def\margewoordpositie[#1]#2%
  {\ifnum#1>\margincontent
     \xdef\margincontent{#1}%
   \fi
   \stelinmargein[#1][]% when at outer level, saves local settings
   \setgvalue{\??im#1}%
     {\stelinmargein[#1][]% needed when par start outside group
      \flushmargincontent[][]{#2}}}

\def\margewoord%
  {\dodoubleempty\complexmargewoord}

\def\margetitel%
  {\margewoord}

\def\margetekst%
  {\margewoord}

\def\oplinker#1%
  {\strut
   \vadjust
     {\mindermeldingen
      \setbox0=\vtop{\forgetall\strut#1}%
      \getboxheight\dimen0\of\box0
      \vskip-\dimen0\
      \box0}}

%D \macros
%D   {inleftside,inleftmargin,inrightmargin,inrightside}
%D   {}
%D
%D The fast and clean way of putting things in the margin is
%D using \type{\rlap} or \type{\llap}. Unfortunately these
%D macro's don't handle indentation, left and right skips. We
%D therefore embed them in some macro's that (force and)
%D remove the indentation and restore it afterwards.

\def\inleftmargin#1%
  {\pushindentation
   \llap{#1\hskip\leftskip\hskip\linkermargeafstand}%
   \popindentation
   \ignorespaces}

\def\inrightmargin#1%
  {\pushindentation
   \rlap{\hskip\hsize\hskip-\rightskip\hskip\rechtermargeafstand#1}%
   \popindentation
   \ignorespaces}

\def\inleftside#1%
  {\inleftmargin
     {#1\relax
      \hskip\linkermargebreedte
     %\hskip\pageseparation
      \hskip\linkerrandafstand}}

\def\inrightside#1%
  {\inrightmargin
     {\hskip\rechtermargebreedte
      \hskip\rechterrandafstand
     %\hskip\pageseparation
      #1}}

%D We want to keep things efficient and therefore only handle
%D situations like:
%D
%D \startbuffer
%D                  \inleftside    {fine} some text \par
%D \strut           \inleftmargin  {fine} some text \par
%D \noindent        \inrightmargin {fine} some text \par
%D \noindent \strut \inrightside   {fine} some text \par
%D \stopbuffer
%D
%D \typebuffer
%D
%D which looks like:
%D
%D \bgroup
%D \haalbuffer
%D \parindent 30pt
%D \haalbuffer
%D \egroup

%D \macros
%D   {pushindentation,popindentation}
%D
%D The pushing and popping is done by:

\newbox\indentationboxA
\newbox\indentationboxB

\def\pushindentation%
  {\bgroup
   \ifhmode
     \unskip
     \setbox\indentationboxA=\lastbox       % get \strut if present
     \unskip
     \setbox\indentationboxB=\lastbox       % get \indent generated box
     \unskip
   \else
     \hskip\!!zeropoint                     % switch to horizontal mode
     \unskip
     \setbox\indentationboxA=\lastbox       % get \indent generated box
     \setbox\indentationboxB=\box\voidb@x
   \fi}

\def\popindentation%
  {\box\indentationboxB\box\indentationboxA % put back the boxes
   \egroup}

%D The only complication lays in \type{\strut}. In \PLAIN\
%D \TEX\ a \type{\strut} is defined as:
%D
%D \starttypen
%D \def\strut%
%D   {\relax\ifmmode\copy\strutbox\else\unhcopy\strutbox\fi}
%D \stoptypen
%D
%D But what is a \type{\strut}? Normally it's a rule of width
%D zero, but when made visual, it's a rule and a negative skip.
%D The mechanism for putting things in the margins described
%D here cannot handle this situation very well. One
%D characteristic of \type{\strut} is that the \type{\unhcopy}
%D results in entering horizontal mode, which in return leads
%D to some indentation.
%D
%D To serve our purpose a bit better, the macro \type{\strut}
%D can be redefined as:
%D
%D \starttypen
%D \def\strut%
%D   {\relax\ifmmode\else\hskip0pt\fi\copy\strutbox}
%D \stoptypen
%D
%D Or more compatible:
%D
%D \starttypen
%D \def\strut%
%D   {\relax\ifmmode
%D      \copy\strutbox
%D    \else
%D      \bgroup\setbox\strutbox=\normalhbox{\box\strutbox}\unhcopy\strutbox\egroup
%D    \fi}
%D \stoptypen
%D
%D In \CONTEXT\ however we save some processing time by putting
%D an extra \type{\hbox} around the \type{\strutbox}.

% dit zijn voorlopig lokale commando's

\def\woordinlinker {\inleftmargin}  % vervallen
\def\woordinrechter{\inrechtermarge} % vervallen

\def\woordinmarge%
  {\doquintupleempty\doinmarge
     [\@@implaats][\woordinlinker][\woordinrechter]}

% Standaard is \count0 in Plain TeX de paginateller. Omwille
% van de afhandeling van lokaal nummeren, definieren we
% echter een eigen nummer.

\definieernummer
  [\s!page]
  [\c!conversie=\@@nmconversie,
   \c!wijze=\@@nmwijze,
   \c!status=\@@nmstatus,
   \c!start=1]

% \@@pnstatus global, but \@@nmstatus local and only start/stop

\global\let\@@pnstatus\@@pnstatus

%\def\dostelpaginanummerin[#1]%
%  {\getparameters
%    [\??pn]
%     [%\c!status=\v!start,
%      \c!nummer=,
%      #1]%
%   \global\let\@@pnstatus\@@pnstatus
%   \doifsomething{\@@pnnummer}
%     {\setnummer[\s!page]{\@@pnnummer}%
%      \setuserpageno{\ruwenummer[\s!page]}}}

\def\dostelpaginanummerin[#1]%
  {\getparameters[\??pn][\c!nummer=,#1]%
   \global\let\@@pnstatus\@@pnstatus
   \doifsomething{\@@pnnummer}
     {\setnummer[\s!page]{\@@pnnummer}%
      \setuserpageno{\ruwenummer[\s!page]}}%
   % this makes starting at an even page possible
   \ifnum\realpageno=1 \ifodd\pageno \else
     \global\shiftedrealpagenotrue
   \fi \fi}

\def\stelpaginanummerin%
  {\dosingleargument\dostelpaginanummerin}

\def\verlaagpaginanummer%
  {\doifinset{\@@pnstatus}{\v!start,\v!leeg,\v!geen}
     {\verlaagnummer[\s!page]%
      \setuserpageno{\ruwenummer[\s!page]}}}

\def\verhoogpaginanummer%
  {\doifinset{\@@pnstatus}{\v!start,\v!leeg,\v!geen}
     {\verhoognummer[\s!page]%
      \setuserpageno{\ruwenummer[\s!page]}}%
   \doifinset{\@@pnstatus}{\v!handhaaf,\v!leeg}
     {\global\let\@@pnstatus\v!start}}

\def\checkpagecounter%
  {\checknummer{\s!page}}

\newif\ifpaginageblokkeerd
\paginageblokkeerdfalse

\def\testpagina[#1][#2]%
  {\ifpaginageblokkeerd \else
     \ifdim\pagetotal<\pagegoal
       \dimen0=\lineheight
       \multiply\dimen0 by #1\relax
       \advance\dimen0 by \pagetotal
       \ifdim\lastskip<\parskip
         \advance\dimen0 by \parskip
       \fi
       \advance\dimen0 by #2\relax
       \ifdim\dimen0>.99\pagegoal
         \penalty-\!!tenthousand\relax
       \fi
     \else
       \goodbreak
     \fi
   \fi}

\let\resetcurrentsectionmarks=\relax

% was: \resetsectionmarks[\firstsection], zie \handelpaginaaf

\def\insertdummypage%
  {\ejectinsert % beter
   \hardespatie
   \vfill
   \ejectpage}

\def\docomplexpagina[#1]%
  {\flushfootnotes
   \bgroup
   \processallactionsinset
     [#1]
     [    \v!reset=>\global\paginageblokkeerdfalse,
       \v!blokkeer=>\global\paginageblokkeerdtrue,
             \v!ja=>\ifpaginageblokkeerd\else
                      \ejectinsert
                      \ejectpage
                      \ifbinnenkolommen
                        \ejectpage  % anders soms geen overgang
                      \fi
                    \fi,
         \v!opmaak=>\ifpaginageblokkeerd\else
                      \eject
                    \fi,
         \v!blanko=>\pagebodyornamentsfalse,
            \v!nee=>\ifpaginageblokkeerd\else
                      \dosomebreak\nobreak
                    \fi,
       \v!voorkeur=>{\ifpaginageblokkeerd\else
                       \ifbinnenkolommen
                         \dosomebreak\goodbreak
                       \else
                         \testpagina[3][\!!zeropoint]%
                       \fi
                     \fi},
  \v!grotevoorkeur=>{\ifpaginageblokkeerd\else
                       \ifbinnenkolommen
                         \dosomebreak\goodbreak
                       \else
                         \testpagina[5][\!!zeropoint]%
                       \fi
                     \fi},
           \v!leeg=>{\ejectinsert
                     \ejectpage
                     \doifnotvalue{\??tk\v!hoofd\v!tekst\c!status}{\v!stop}
                       {\stelhoofdin[\c!status=\v!leeg]}%
                     \doifnotvalue{\??tk\v!voet\v!tekst\c!status}{\v!stop}
                       {\stelvoetin[\c!status=\v!leeg]}%
                     \insertdummypage},
          \v!links=>{\ejectinsert
                     \superejectpage
                     \doifbothsidesoverruled
                     \orsideone
                       \resetcurrentsectionmarks
                       \insertdummypage
                     \orsidetwo
                     \od},
         \v!rechts=>{\ejectinsert
                     \superejectpage
                     \doifbothsidesoverruled
                     \orsideone
                     \orsidetwo
                       \resetcurrentsectionmarks
                       \insertdummypage
                     \od},
           \v!even=>\pagina
                    \doifonevenpaginaelse
                      {\resetcurrentsectionmarks\insertdummypage}{},
         \v!oneven=>\pagina
                    \doifonevenpaginaelse
                      {}{\resetcurrentsectionmarks\insertdummypage},
        \v!viertal=>{\ifdubbelzijdig
                       \!!counta=\realpageno
                       \!!countb=\realpageno
                       \divide\!!counta by 4
                       \divide\!!countb by 2
                       \ifnum\!!counta=\!!countb
                       \else
                         \pagina
                         \pagina[\v!leeg]%
                         \pagina[\v!leeg]%
                       \fi
                     \fi},
        \v!laatste=>{\ejectinsert
                     \superejectpage
                     \doifbothsidesoverruled
                       \naastpagina
                     \orsideone
                     \orsidetwo
                       %\ifodd\realpageno \else % kan weer weg
                         \geenhoofdenvoetregels
                         \insertdummypage
                       %\fi
                     \od
                     \filluparrangedpages},
        \s!unknown=>\doifinstringelse{+}{#1}
                      {\ejectinsert\ejectpage
                       \dorecurse{#1}{\insertdummypage}}
                      {\doifnumberelse{#1}
                         {\ejectinsert\ejectpage
                          \doloop
                            {\ifnum\userpageno<#1\relax
                               \insertdummypage
                             \else
                               \exitloop
                             \fi}}
                         {}}]%
   \egroup}

\def\complexpagina[#1]%
  {\expanded{\docomplexpagina[#1]}}

\def\simplepagina%
  {\docomplexpagina[\v!ja]}

\definecomplexorsimple\pagina

\def\resetpagina%
  {\global\paginageblokkeerdfalse}

% \getpagestatus
% \ifrightpage als odd/enkelzijdig

\newif\ifrightpage \rightpagetrue

\newcounter \nofpagesets

\definetwopasslist{\s!page}

\def\dopagesetreference%
  {\doglobal\increment\nofpagesets\relax
   \edef\writepagref%
     {\writeutilitycommand
        {\twopassentry
           {\s!page}%
           {\nofpagesets}%
           {\noexpand\realfolio}}}%
   \writepagref}

\def\getpagestatus% hierboven gebruiken
  {\ifdubbelzijdig
     \gettwopassdata{\s!page}%
     \iftwopassdatafound \else
       \let\twopassdata=\realpageno
     \fi
     \ifodd\twopassdata
       \global\rightpagetrue
     \else
       \global\rightpagefalse
     \fi
     \dopagesetreference
   \else
     \global\rightpagetrue
   \fi}


% De onderstaande macro's lijken op het eerste gezicht vrij
% ingewikkeld en omslachtig. Dit is het gevolg van een
% dubbel optioneel zijn van argumenten: zowel het eerste als
% de twee laatste argumenten zijn optioneel. Dit is mede het
% gevolg van een uitbreiding naar marges en randen, waarbij
% upward-compatibiliteit zwaar heeft gewogen.

\def\dostellayouttekstin[#1][#2][#3]%
  {\ifthirdargument
     \getparameters[\??tk#1#2][#3]%
   \else
     \getparameters[\??tk#1\v!tekst][#2]%
   \fi
   \calculatevsizes}

\def\stelbovenin {\dotripleempty\dostellayouttekstin[\v!boven]}
\def\stelhoofdin {\dotripleempty\dostellayouttekstin[\v!hoofd]}
\def\steltekstin {\dotripleempty\dostellayouttekstin[\v!tekst]}
\def\stelvoetin  {\dotripleempty\dostellayouttekstin[\v!voet]}
\def\stelonderin {\dotripleempty\dostellayouttekstin[\v!onder]}

\letvalue{\??tk\v!boven\v!tekst\c!status}=\v!normaal
\letvalue{\??tk\v!hoofd\v!tekst\c!status}=\v!normaal
\letvalue{\??tk\v!tekst\v!tekst\c!status}=\v!normaal
\letvalue{\??tk\v!voet \v!tekst\c!status}=\v!normaal
\letvalue{\??tk\v!onder\v!tekst\c!status}=\v!normaal

\def\geenhoofdenvoetregels%
  {\stelhoofdin[\c!status=\v!leeg]%
   \stelvoetin[\c!status=\v!leeg]}

\def\geenbovenenonderregels%
  {\stelbovenin[\c!status=\v!leeg]%
   \stelonderin[\c!status=\v!leeg]}

% THIS: !!!

\def\dolimitateteksten#1#2%
  {\doifelsevaluenothing{#1}{#2}{\limitatetext{#2}{\getvalue{#1}}{...}}}

\def\doteksten#1#2#3#4#5#6%
  {\bgroup
  %\showcomposition % I need to test first
   \convertargument#6\to\ascii
   \doifsomething{\ascii}
     {\doattributes{#1#2}#3#4%
        {\doifvalue{#1\v!tekst\c!strut}{\v!ja}{\setstrut\strut}% here!
         \doifdefinedelse{\??mk\ascii\c!koppeling} % brrr
           {\dolimitateteksten{#1#2#5}{\haalmarkering[\ascii][\v!eerste]}}
           {\ConvertConstantAfter\doifelse{\v!paginanummer}{#6}
              {\@@plaatspaginanummer}
              {\ConvertConstantAfter\doifelse{\v!datum}{#6}
                 {\currentdate}   % #6{}{}{} -> {} needed for lookahead macros, like \uniqueMPgraphic
                 {\opeenregel\dolimitateteksten{#1#2#5}{#6{}{}{}}}}}}}%
  \egroup}

\def\dodoteksten#1#2#3#4#5#6%
  {\doifonevenpaginaelse
     {\doteksten{#1}{#2}#3{#4}}  % #3 => provides three arguments
     {\doteksten{#1}{#2}#5{#6}}} % #5 => provides three arguments

\def\dodododoteksten[#1][#2][#3][#4][#5][#6]%
  {\ifsixthargument
     \setvalue{\??tk#1#2\c!linkertekst}%
       {\dodoteksten{\??tk#1}{#2}
          {\c!linkerletter \c!linkerkleur \c!linkerbreedte }{#3}
          {\c!rechterletter\c!rechterkleur\c!rechterbreedte}{#6}}%
     \setvalue{\??tk#1#2\c!rechtertekst}%
       {\dodoteksten{\??tk#1}{#2}
          {\c!rechterletter\c!rechterkleur\c!rechterbreedte}{#4}
          {\c!linkerletter \c!linkerkleur \c!linkerbreedte }{#5}}%
   \else\iffifthargument
     \setvalue{\??tk#1\v!tekst\c!linkertekst}%
       {\dodoteksten{\??tk#1}{\v!tekst}
          {\c!linkerletter \c!linkerkleur \c!linkerbreedte }{#2}
          {\c!rechterletter\c!rechterkleur\c!rechterbreedte}{#5}}%
     \setvalue{\??tk#1\v!tekst\c!rechtertekst}%
       {\dodoteksten{\??tk#1}{\v!tekst}
          {\c!rechterletter\c!rechterkleur\c!rechterbreedte}{#3}
          {\c!linkerletter \c!linkerkleur \c!linkerbreedte }{#4}}%
   \else\iffourthargument
     \setvalue{\??tk#1#2\c!linkertekst}%
       {\dodoteksten{\??tk#1}{#2}
          {\c!linkerletter\c!linkerkleur\c!linkerbreedte}{#3}
          {\c!linkerletter\c!linkerkleur\c!linkerbreedte}{#3}}%
     \setvalue{\??tk#1#2\c!rechtertekst}%
       {\dodoteksten{\??tk#1}{#2}
          {\c!rechterletter\c!rechterkleur\c!rechterbreedte}{#4}
          {\c!rechterletter\c!rechterkleur\c!rechterbreedte}{#4}}%
   \else\ifthirdargument
     \setvalue{\??tk#1\v!tekst\c!linkertekst}%
       {\dodoteksten{\??tk#1}{\v!tekst}
          {\c!linkerletter\c!linkerkleur\c!linkerbreedte}{#2}
          {\c!linkerletter\c!linkerkleur\c!linkerbreedte}{#2}}%
     \setvalue{\??tk#1\v!tekst\c!rechtertekst}%
       {\dodoteksten{\??tk#1}{\v!tekst}
          {\c!rechterletter\c!rechterkleur\c!rechterbreedte}{#3}
          {\c!rechterletter\c!rechterkleur\c!rechterbreedte}{#3}}%
   \else\ifsecondargument % new
     \setvalue{\??tk#1\v!tekst\c!linkertekst}{}%
     \setvalue{\??tk#1\v!tekst\c!rechtertekst}{}%
     \setvalue{\??tk#1\v!tekst\c!middentekst}%
       {\doteksten{\??tk#1}{\v!tekst}\c!letter\c!kleur\c!breedte{#2}}%
   \else
     \dosixtupleempty\dodododoteksten[#1][\v!tekst][][][][]%
     \dosixtupleempty\dodododoteksten[#1][\v!marge][][][][]%
     \dosixtupleempty\dodododoteksten[#1][\v!rand] [][][][]%
   \fi\fi\fi\fi\fi}

\def\stelboventekstenin {\dosixtupleempty\dodododoteksten[\v!boven]}
\def\stelhoofdtekstenin {\dosixtupleempty\dodododoteksten[\v!hoofd]}
\def\stelteksttekstenin {\dosixtupleempty\dodododoteksten[\v!tekst]}
\def\stelvoettekstenin  {\dosixtupleempty\dodododoteksten[\v!voet]}
\def\stelondertekstenin {\dosixtupleempty\dodododoteksten[\v!onder]}

\def\@@nmpre#1{\setbox0=\hbox{#1}\ifdim\wd0=\!!zeropoint\else\unhbox0\tfskip\fi}
\def\@@nmpos#1{\setbox0=\hbox{#1}\ifdim\wd0=\!!zeropoint\else\tfskip\unhbox0\fi}

\def\dodoplaatsteksten#1#2#3#4#5#6% \hsize toegevoegd
  {\hbox                          % \hss's niet meer wijzigen
     {\hbox to \linkerrandbreedte 
        {\hsize\linkerrandbreedte
         \hss\getvalue{\??tk#1\v!rand#2}}%
      \hskip\linkerrandafstand
     %\hskip\pageseparation
      \hbox to \linkermargebreedte
        {\hsize\linkermargebreedte
         \hsmash{\hbox to \linkermargebreedte
           {\hss\getvalue{\??tk#1\v!marge#2}}}%
         \hsmash{\hbox to \linkermargebreedte
           {\hss#5{\??tk#1\v!marge\c!margetekst}}}%
         \hss}% let op: \smashed
      \hskip\linkermargeafstand
      \hbox to \zetbreedte
        {\hsize\zetbreedte
         \hsmash{\hbox to \zetbreedte
           {\@@nmpre{#5{\??tk#1\v!tekst\c!kantlijntekst}}%
            \getvalue{\??tk#1\v!tekst#2}\hss}}%
         \hsmash{\hbox to \zetbreedte
           {\hss\getvalue{\??tk#1\v!tekst#3}\hss}}%
         \hsmash{\hbox to \zetbreedte
           {\hss\getvalue{\??tk#1\v!tekst#4}%
            \@@nmpos{#6{\??tk#1\v!tekst\c!kantlijntekst}}}}%
         \hss}%
      \hskip\rechtermargeafstand
      \hbox to \rechtermargebreedte
        {\hsize\rechtermargebreedte
         \hsmash{\hbox to \rechtermargebreedte
           {\getvalue{\??tk#1\v!marge#4}\hss}}%
         \hsmash{\hbox to \rechtermargebreedte
           {#6{\??tk#1\v!marge\c!margetekst}\hss}}%
         \hss}% let op: \smashed
     %\hskip\pageseparation
      \hskip\rechterrandafstand
      \hbox to \rechterrandbreedte
        {\hsize\rechterrandbreedte
         \getvalue{\??tk#1\v!rand#4}\hss}}}

\def\doplaatslayoutregel#1#2%
  {\ifdim#2>\!!zeropoint\relax  % prevents pagenumbers when zero height
     \goleftonpage
     \hbox
       {\setbox0=\vbox to #2
          {\forgetall
           \vsize#2
           \normalbaselines
           \def\\{ \ignorespaces}%
           \def\crlf{ \ignorespaces}%
           \getvalue{\??tk#1\v!tekst\c!voor}%
           \doifbothsidesoverruled
             \dodoplaatsteksten#1\c!linkertekst\c!middentekst\c!rechtertekst
               \gobbleoneargument\getvalue
           \orsideone
             \dodoplaatsteksten#1\c!linkertekst\c!middentekst\c!rechtertekst
               \gobbleoneargument\getvalue
           \orsidetwo
             \dodoplaatsteksten#1\c!rechtertekst\c!middentekst\c!linkertekst
               \getvalue\gobbleoneargument
           \od
           \getvalue{\??tk#1\v!tekst\c!na}%
           \kern\!!zeropoint}% keep the \dp, beware of \vtops, never change this!
        \dp0=\!!zeropoint
        \box0}%
   \fi}

% \stelhoofdin[status=normaal] \titel{NORMAAL} \dorecurse{8}{\input tufte} \pagina
% \stelhoofdin[status=hoog]    \titel{HOOG}    \dorecurse{8}{\input tufte} \pagina
% \stelhoofdin[status=leeg]    \titel{LEEG}    \dorecurse{8}{\input tufte} \pagina
% \stelhoofdin[status=geen]    \titel{GEEN}    \dorecurse{8}{\input tufte} \pagina
% \stelhoofdin[status=stop]    \titel{STOP}    \dorecurse{8}{\input tufte} \pagina

\def\definieertekst%
  {\dosixtupleempty\dodefinieertekst}

\def\dodefinieertekst[#1][#2][#3][#4][#5][#6]%
  {\ifsixthargument
     \setvalue{\??tk#2#1}{\dosixtupleempty\dodododoteksten[#2][#3][#4][#5][#6]}%
   \else\iffourthargument
     \setvalue{\??tk#2#1}{\dosixtupleempty\dodododoteksten[#2][#3][#4]}%
   \else
     \setvalue{\??tk#2#1}{\dosixtupleempty\dodododoteksten[#2][#3]}%
   \fi\fi}

% \definieertekst[hoofdstuk][voet][paginanummer]
% \stelkopin[hoofdstuk][hoofd=hoog,voet=hoofdstuk]
% \stelhoofdtekstenin[paginanummer]
% \stelvoettekstenin[links][rechts]
% \hoofdstuk{eerste} \dorecurse{20}{\input tufte \relax}
% \hoofdstuk{tweede} \dorecurse{20}{\input tufte \relax}

% todo: commalist aflopen {empty,next} {first,empty}

\def\plaatslayoutregel#1#2%  % handelt o.b.v. tekst
  {\ExpandFirstAfter\processaction
     [\getvalue{\??tk#1\v!tekst\c!status}]
     [        \v!geen=>,
              \v!hoog=>, % is reset later on
             \v!start=>\setgvalue{\??tk#1\v!tekst\c!status}{\v!normaal}%
                       \doplaatslayoutregel{#1}{#2},
              \v!stop=>\vskip#2\relax,
              \v!leeg=>\setgvalue{\??tk#1\v!tekst\c!status}{\v!normaal}%
                       \vskip#2\relax,
     \v!geenmarkering=>\bgroup
                       \setgvalue{\??tk#1\v!tekst\c!status}{\v!normaal}%
                       \let\dohaalmarkering=\nohaalmarkering
                       \doplaatslayoutregel{#1}{#2}%
                       \egroup,
           \v!normaal=>\doplaatslayoutregel{#1}{#2},
           \s!default=>\doplaatslayoutregel{#1}{#2},
           \s!unknown=>\bgroup % new
                       \setgvalue{\??tk#1\v!tekst\c!status}{\v!normaal}%
                       \getvalue{\??tk#1\commalistelement}%
                       \doplaatslayoutregel{#1}{#2}%
                       \egroup]}

\def\resetlayoutregel#1%
  {\doifvalue{\??tk#1\v!tekst\c!status}{\v!hoog}
     {\setgvalue{\??tk#1\v!tekst\c!status}{\v!normaal}% ! global
      \doglobal\calculatevsizes
      \global\newlogostrue
      \global\newbackgroundtrue}}

\def\resetlayoutregels%
  {\resetlayoutregel\v!boven
   \resetlayoutregel\v!hoofd
   \resetlayoutregel\v!tekst
   \resetlayoutregel\v!voet
   \resetlayoutregel\v!onder}

\def\plaatsbovenregel {\plaatslayoutregel\v!boven\bovenhoogte}
\def\plaatshoofdregel {\plaatslayoutregel\v!hoofd\hoofdhoogte}
\def\plaatstekstregel {\plaatslayoutregel\v!tekst\teksthoogte}
\def\plaatsvoetregel  {\plaatslayoutregel\v!voet\voethoogte}
\def\plaatsonderregel {\plaatslayoutregel\v!onder\onderhoogte}

\def\gettextboxes%  elders weghalen
  {\bgroup
   \setbox0=\vbox
     {\mindermeldingen
      \calculatereducedvsizes
      \swapmargins
      \forgetall
      \offinterlineskip
      \vskip-\bovenhoogte
      \vskip-\bovenafstand
      \plaatsbovenregel
      \vskip-\bovenhoogte
      \plaatsboventekstblok
      \vskip\bovenafstand
      \plaatshoofdregel
      \vskip\hoofdafstand
      \placepositionanchors
      \vskip-\teksthoogte
      \plaatstekstregel
      \vskip\voetafstand
      \plaatsvoetregel
      \vskip\onderafstand
      \plaatsonderregel
      \vskip-\onderhoogte
      \plaatsondertekstblok
      \plaatsversieaanduiding
      \vfilll}%
  \smashbox0
  \box0
  \egroup}

\ifx\undefined\placepositionanchors
  \def\placepositionanchors{\vskip\teksthoogte}
\fi

\def\@@nmin     {} % kan vervallen  (upward compatibility)
\def\@@nmplaats {} % mag {plaats, in} zijn

\newcounter\@@pagenumberlocation

\def\do@@plaatspaginanummer#1%
  {\ifnum#1=\@@pagenumberlocation\@@plaatspaginanummer\fi}

\def\dodosetpagenumberlocation#1% tricky because of ...texts
  {\increment\@@pagenumberlocation
   \ifx\@@nmplaats\empty\else
     \def\dododosetpagenumberlocation##1%
       {\donetrue
        \setevalue{\??tk#1##1}%
          {\noexpand\do@@plaatspaginanummer{\@@pagenumberlocation}}}%
     \donefalse
     \ExpandFirstAfter\processallactionsinset
       [\@@nmplaats]
       [    \v!midden=>\dododosetpagenumberlocation{\v!tekst\c!middentekst},
             \v!links=>\dododosetpagenumberlocation{\v!tekst\c!linkertekst},
            \v!rechts=>\dododosetpagenumberlocation{\v!tekst\c!rechtertekst},
          \v!inlinker=>\dododosetpagenumberlocation{\v!marge\c!linkertekst},
         \v!inrechter=>\dododosetpagenumberlocation{\v!marge\c!rechtertekst},
           \v!inmarge=>\dododosetpagenumberlocation{\v!marge\ifdubbelzijdig
                         \c!margetekst\else\c!rechtertekst\fi},
             \v!marge=>\dododosetpagenumberlocation{\v!marge\ifdubbelzijdig
                         \c!margetekst\else\c!rechtertekst\fi},
           \v!opmarge=>\dododosetpagenumberlocation{\v!tekst\c!kantlijntekst},
          \v!kantlijn=>\dododosetpagenumberlocation{\v!tekst\c!kantlijntekst}]%
     \ifdone \else
       \dododosetpagenumberlocation{\v!tekst\c!middentekst}% default
     \fi
   \fi}

\def\dosetpagenumberlocation%
  {\ExpandBothAfter\doifinsetelse{\v!hoofd}{\@@nmplaats,\@@nmin}
     {\dodosetpagenumberlocation\v!hoofd}
     {\dodosetpagenumberlocation\v!voet}}

\def\dostelnummeringin[#1]%
  {\getparameters[\??nm][#1]%
   \preparepaginaprefix{\??nm}%
   \enkelzijdigfalse
   \dubbelzijdigfalse
   \ExpandFirstAfter\processallactionsinset
     [\@@nmvariant]
     [ \v!enkelzijdig=>\enkelzijdigtrue,
      \v!dubbelzijdig=>\dubbelzijdigtrue]%
   \ifdubbelzijdig
     \trackingmarginnotestrue
   \else
     \trackingmarginnotesfalse
   \fi
   \dosetpagenumberlocation
   \global\newbackgroundtrue
   \global\newlogostrue}

\def\stelnummeringin%
  {\dosingleempty\dostelnummeringin}

% listig: hangt af van \@@kolijst

% erg fout
%
% \def\preparepaginaprefix#1%
%   {\def\dopreparepaginaprefix##1%
%      {\doifvalue{#1##1\c!nummer}{\v!ja}
%         {\setvalue{#1\getvalue{\??by##1}\c!nummer}{\v!ja}}}%
%    \processcommacommand[\@@kolijst]\dopreparepaginaprefix}
%
% nog fouter
%
% \def\preparepaginaprefix#1%
%   {\def\dopreparepaginaprefix##1%
%      {\doifelsevalue{#1##1\v!nummer}{\v!ja}                    % v
%         {\setvalue{#1\getvalue{\??by##1}\v!nummer}{\v!ja}}     % v
%         {\setvalue{#1\getvalue{\??by##1}\v!nummer}{\v!nee}}}%  % v
%    \processcommacommand[\@@kolijst]\dopreparepaginaprefix}
%
% best, beware, chapter (yes) can be followed by title (no)

\def\preparepaginaprefix#1%
  {\def\dopreparepaginaprefix##1%
     {\setvalue{#1\getvalue{\??by##1}\v!nummer}{\v!nee}}%  %v
   \processcommacommand[\@@kolijst]\dopreparepaginaprefix
   \def\dopreparepaginaprefix##1%
     {\doifvalue{#1##1\v!nummer}{\v!ja}                    %v
        {\setvalue{#1\getvalue{\??by##1}\v!nummer}{\v!ja}}}%
   \processcommacommand[\@@kolijst]\dopreparepaginaprefix}

\def\dopaginaprefix#1#2%
  {\let\normaluchar\uchar\let\uchar\relax % ugly but needed
   \doifelsevalue{#1#2\v!nummer}{\v!ja} % \v! and no \c!
     {\@EA\beforesplitstring\@EA{\postprefix}\at:\to\preprefix
      \@EA\aftersplitstring\@EA{\postprefix}\at:\to\postprefix
      \let\uchar\normaluchar % ugly but needed
      \doifsomething{\preprefix}
        {\doifnot{\preprefix}{0}{\preprefix\@@nmnummerscheider}}}%
     {\@EA\aftersplitstring\@EA{\postprefix}\at:\to\postprefix
      \let\uchar\normaluchar}} % ugly but needed

\def\paginaprefix#1[#2::#3::#4]% kan wat sneller ####1:0:
  {\bgroup
   \edef\postprefix{#3}%
   \def\donexttrackcommando##1%
     {\dopaginaprefix{#1}{##1}%
      \donexttracklevel{##1}}%
   \donexttrackcommando\firstsection
   \egroup}

\unexpanded\def\@@plaatspaginanummer% called in empty tests
 %{\doif{\@@pnstatus}{\v!start}
  {\doif{\@@nmstatus\@@pnstatus}{\v!start\v!start}
     {{\@@nmcommando{\doattributes\??nm\c!letter\c!kleur{\completepagenumber}}}}}

\def\@@plaatspaginascheider% still used ? 
 %{\doif{\@@pnstatus}{\v!start}%
  {\doif{\@@nmstatus\@@pnstatus}{\v!start\v!start}
     {\@@nmtekstscheider}}

\def\userfolio%  naast realfolio
  {\nummer[\s!page]}

\def\pagenumber%
  {\userfolio}

\def\pageprefixes%
  {\def\donexttrackcommando##1%
      {\doifvalue{\??nm##1\v!nummer}{\v!ja}  % v
         {\ifnum\countervalue{\??se##1}>0\relax
            \getvalue{##1\c!nummer}\@@nmnummerscheider
          \fi}%
       \doifsomething{\@@nmtekst}
         {\@@nmtekst\@@nmnummerscheider}%
       \donexttracklevel{##1}}%
    \donexttrackcommando{\firstsection}}

\unexpanded\def\completepagenumber% 
  {\doif{\@@nmstatus\@@pnstatus}{\v!start\v!start} 
     {\@@nmlinks\labeltexts{\v!paginanummer}{\pageprefixes\pagenumber}\@@nmrechts}}

\let\volledigepaginanummer\completepagenumber

\unexpanded\def\plaatspaginanummer%
  {\doif{\@@nmstatus\@@pnstatus}{\v!start\v!start}
     {\labeltexts{\v!paginanummer}{\pagenumber}}}

\def\translatednumber[#1::#2::#3]%
  {#3}

% hier nog uti blokkeren

% don't change this / test case:
%
% \setupbackgrounds[state=repeat]
% \setupbackgrounds[text][text][background=whatever]
% \couplepage[chapter][before={\defineoverlay[whatever][ON]}]
% \setuphead[chapter][before={\pagetype[chapter]}] 
% \chapter{First} \page test \chapter{second} \page test 

\newif\ifgeselecteerd
\geselecteerdtrue

\newif\ifselecteren
\selecterenfalse

\newif\ifverwerken
\verwerkentrue

\def\selectie{}
\def\paginasoort{}

\let\naastpagina=\relax
\let\napagina=\relax
\let\voorpagina=\relax

\def\dovoorpagina%
  {\doifsomething{\paginasoort}
     {\def\dododopagina##1%
        {\global\let\voorpagina=\relax
         \getvalue{\??pg##1\c!voor}}%
      \processcommacommand[\paginasoort]\dododopagina}}

\def\dododonapagina#1%
  {\global\let\napagina=\relax
   \gdef\paginasoort{}%
   \getvalue{\??pg#1\c!na}}

\def\donapagina%
  {\doifsomething{\paginasoort}
     {\def\dodopagina##1%
        {\doifelsevalue{\??pg##1\c!optie}{\v!dubbelzijdig}
           {\doifbothsidesoverruled
              \dododonapagina{##1}%
            \orsideone
              \dododonapagina{##1}%
            \orsidetwo
            \od}%
           {\dododonapagina{##1}}}%
      \processcommacommand[\paginasoort]\dodopagina}}

\def\dosoortpagina[#1]%
  {\doglobal\addtocommalist{#1}\paginasoort
   \ifselecteren
     \ExpandBothAfter\doifcommon{#1}{\selectie}
       {\global\geselecteerdtrue}%
   \fi
   \gdef\voorpagina{\dovoorpagina}%
   \gdef\napagina{\donapagina}}

\def\soortpagina%
  {\dosingleargument\dosoortpagina}

\def\dokoppelpagina[#1][#2]%
  {\getparameters
     [\??pg]
     [\c!voor=,
      \c!na=,
      \c!optie=,
      #2]%
   \def\docommando##1%
     {\getparameters
        [\??pg##1]
        [\c!voor=\@@pgvoor,
         \c!na=\@@pgna,
         \c!optie=\@@pgoptie]}%
   \processcommalist[#1]\docommando}%

\def\koppelpagina%
  {\dodoubleargument\dokoppelpagina}

\def\doverwerkpagina[#1][#2]%
  {\processaction
     [#2]
     [ \v!ja=>\global\verwerkentrue,
      \v!nee=>\global\verwerkenfalse]%
   \gdef\selectie{#1}%
   \global\selecterentrue
   \global\geselecteerdfalse}

\def\verwerkpagina%
  {\dodoubleargument\doverwerkpagina}

\def\resetselectiepagina%
  {\ifselecteren
     \doifbothsidesoverruled
       \global\geselecteerdfalse
     \orsideone
     \orsidetwo
       \global\geselecteerdfalse
     \od
   \fi}

\newif\iflocation

\def\ifinteractief{\iflocation}

\def\previoussectionformat{}
\def\currentsectionformat{}

\let\updatelistreferences=\relax
\def\updatedlistreferences{}

\def\setsectionlistreference#1#2%
  {\ifnum\countervalue{\??se\previoussection{#1}}>0\relax
     \xdef\previoussectionformat{\@@longformatnumber{\previoussection{#1}}}%
   \else
     \xdef\previoussectionformat{}%
   \fi
   \xdef\currentsectionformat{\@@longformatnumber{#1}}}

\def\startlistreferences#1%
  {\thisissomeinternal{\s!lst}{#1\currentsectionformat}%
   \setxvalue{\s!lst:#1}{\realfolio}% to be sure
   \setxvalue{\s!lst:#1\currentsectionformat}{\realfolio}%
   \setxvalue{\e!vorigelokale#1}{\s!lst:#1\previoussectionformat}%
   \setxvalue{\e!huidigelokale#1}{\s!lst:#1\currentsectionformat}%
   \doifelse{\currentsectionformat}{}
     {\setglobalcrossreference
        {\e!vorige#1}{}{\realfolio}{}}
%
     {\setglobalsystemreference\rt!list
        {\e!vorige#1}{\getvalue{\e!vorigelokale#1}}}%
%
%         {\definereference[\e!vorige#1][\getvalue{\e!vorigelokale#1}]%
%
   \def\stoplistreferences{\dostoplistreferences}}

\def\dostoplistreferences#1%
  {\iflijstgeplaatst
     \addtocommalist{#1}\updatedlistreferences                % nog global (\doglobal)
     \global\let\updatedlistreferences=\updatedlistreferences % een noodverbandje
     \gdef\updatelistreferences%
       {\def\docommando####1%
%
          {\setglobalsystemreference\rt!list
             {\e!vorige####1}{\getvalue{\e!huidigelokale####1}}}%
%
%         {\definereference[\e!vorige####1][\getvalue{\e!huidigelokale####1}]%
%
        \processcommacommand[\updatedlistreferences]\docommando
        \global\let\updatelistreferences=\relax
        \global\let\updatedlistreferences=\empty}%
   \fi}

\def\stoplistreferences%
  {\gobbleoneargument}

% de rest

\newcount\prefixteller

\def\referenceprefix{}

% \def\showlocation            #1{#1}
% \def\showcontrastlocation#1#2#3{#3}
% \def\showcoloredlocation   #1#2{#2}

\unexpanded\def\referencepagenumber[#1]%
  {\paginaprefix\??rf[#1]\translatednumber[#1]}

\newif\ifinregels
\newif\ifregelnummersinmarge

\def\stelregelsin%
  {\dodoubleargument\getparameters[\??rg]}

\def\startregels%
  {\@@rgvoor
   \witruimte
  %\pagina[\v!voorkeur]} gaat mis na koppen, nieuw: later \nobreak
   \begingroup
   \def\@@rgstepyes{\parindent\!!zeropoint}%
   \def\@@rgstepno{\parindent\!!zeropoint}%
   \edef\@@rgparindent{\the\parindent}%
   \gdef\@@rglinesteptoggle{1}%
   \processaction
     [\@@rginspringen]
     [    \v!ja=>\def\@@rgstepyes{\parindent\@@rgparindent}%
                 \def\@@rgstepno {\parindent\@@rgparindent},
      \v!oneven=>\def\@@rgstepyes{\parindent\!!zeropoint  }%
                 \def\@@rgstepno {\parindent\@@rgparindent},
        \v!even=>\def\@@rgstepno {\parindent\!!zeropoint  }%
                 \def\@@rgstepyes{\parindent\@@rgparindent}]%
   \inregelstrue
   \stelwitruimtein[\v!geen]%
   \obeylines
   \let\checkindentation=\relax
   \@@rgstepno
   \ignorespaces
   \gdef\afterfirstobeyedline% tzt two pass, net als opsomming
     {\gdef\afterfirstobeyedline%
        {\nobreak
         \global\let\afterfirstobeyedline\relax}}%
   \def\obeyedline%
     {\par
      \afterfirstobeyedline
      \ifdim\lastskip>\!!zeropoint
        \gdef\@@rglinesteptoggle{0}%
      \else
        \doglobal\increment\@@rglinesteptoggle
      \fi
      \ifodd\@@rglinesteptoggle\relax
        \@@rgstepyes
      \else
        \@@rgstepno
      \fi
      \futurelet\next\dobetweenthelines}%
   \GotoPar}

% \def\dobetweenthelines%
%   {\convertcommand \next      \to\!!stringa % very ugly and fuzzy
%    \convertargument\obeyedline\to\!!stringb % but needed anyway
%    \ifx\!!stringa\!!stringb                 % but alas, it fails 
%      \@@rgtussen                            % hopelessly in non 
%    \fi}                                     % etex

\def\dobetweenthelines%
  {\doifmeaningelse{\next}{\obeyedline}{\@@rgtussen}{}}

\def\stopregels%
  {\endgroup
   \@@rgna}

\newcount\linenumber
\newcount\linestepper
\newif\ifinregelnummeren

% het gebruik van \setlocalreference scheelt een hash entry

\def\dodoshowlinenumber% for use elsewhere, to be extended 
  {\doschrijfregelnummer
   \global\advance\linenumber by 1\relax}%

\def\regelweergave%
  {\convertnumber\@@rnconversie\linenumber}%

\def\dostelregelnummerenin[#1]%
  {\getparameters
     [\??rn]
     [\c!start=1,
      \c!stap=1,
      #1]%
   \global\linenumber=1\relax}

\def\stelregelnummerenin%
  {\dosingleargument\dostelregelnummerenin}

\def\dostartnummerenLINE%                % !! \everypar !!
  {\EveryPar{\schrijfregelnummer}}

\def\dostopnummerenLINE%
  {\egroup}

\def\dodoschrijfregelnummer%
  {\setbox0=\hbox{\regelweergave}%
   \vsmashbox0
   \ifregelnummersinmarge
     \llap{\hbox{\box0\hskip\linkermargeafstand}}%
   \else
     \llap{\hbox to \@@rnbreedte{\box0\hss}}%
   \fi}

\def\complexstartregelnummeren[#1]%
  {\doifnotinset{\v!verder}{#1}
     {\global\linenumber=1\relax}%
   \doifinsetelse{\@@rnplaats}{\v!inmarge,\v!marge}
     {\regelnummersinmargetrue}
     {\regelnummersinmargefalse}%
   \ifregelnummersinmarge\else
     \advance\leftskip by \@@rnbreedte\relax
   \fi
   \ifinregels
     \let\dostartnummeren=\dostartnummerenLINE
     \let\stopregelnummeren=\dostopnummerenLINE
     \def\schrijfregelnummer%
       {\doschrijfregelnummer
        \global\advance\linenumber by 1\relax}%
   \else
     \let\dostartnummeren=\dostartnummerenPAR
     \let\stopregelnummeren=\dostopnummerenPAR
     \def\schrijfregelnummer%
       {\global\advance\linenumber by -1\relax
        \doschrijfregelnummer}%
   \fi
   \dostartnummeren}

\def\startregelnummeren%
  {\bgroup
   \inregelnummerentrue
   \complexorsimpleempty\startregelnummeren}

\def\doschrijfregelnummer%
  {\ifnum\linenumber<\@@rnstart\relax
   \else
     \!!counta=\linenumber
     \divide\!!counta by \@@rnstap\relax
     \multiply\!!counta by \@@rnstap\relax
     \ifnum\!!counta=\linenumber
       \doattributes\??rn\c!letter\c!kleur{\dodoschrijfregelnummer}%
     \fi
   \fi}

\def\eenregel[#1]%
  {\regelreferentie0[#1]\ignorespaces}

\def\startregel[#1]%
  {\regelreferentie1[#1]\ignorespaces}

\def\stopregel[#1]%
  {\unskip\regelreferentie2[#1]}

% \def\inregellabel#1%
%   {\doifinstringelse{--}{#1}
%      {\labeltext{\v!regels}}
%      {\labeltext{\v!regel}}}
% 
% \def\inregel#1[#2]%
%   {\doifelsenothing{#1}
%      {\in{\inregellabel{\currenttextreference}}[\@@rnprefix#2]}
%      {\in{#1}[\@@rnprefix#2]}}
%
% double labels: 

\def\inregel#1[#2]%
  {\doifelsenothing{#1}
     {\doifinstringelse{--}{\currenttextreference}
        {\in{\leftlabeltext\v!regels}{\rightlabeltext\v!regels}[\@@rnprefix#2]}
        {\in{\leftlabeltext\v!regel }{\rightlabeltext\v!regel }[\@@rnprefix#2]}}
     {\in{#1}[\@@rnprefix#2]}}

\def\dostartnummerenPAR%
  {\beginofshapebox
   \doglobal\newcounter\linereference}

% localcrossref heroverwegen

\def\setlinereference#1#2#3#4%
  {\setxvalue{lrf:#1}{\noexpand\dogetlinereference{#2}{#3}{#4}}}

\def\getlinereference#1%
  {\getvalue{lrf:#1}}

\def\dogetlinereference#1#2#3%
  {\edef\linereferencename{#1}%
   \edef\linereferenceline{#2}%
   \edef\linereferenceplus{#3}}

% 1 xxx xxx xxx xxx xxx xxx xxx
% 2 xxx yyy yyy yyy yyy yyy yyy <= start y
% 3 yyy yyy yyy yyy yyy yyy yyy
% 4 yyy yyy yyy yyy yyy xxx xxx <= stop y
% 5 xxx xxx xxx xxx xxx xxx xxx

%\def\regelreferentie#1[#2]%
%  {\bgroup
%   \dimen0=\dp\strutbox
%   \doif{\@@rnrefereren}{\v!aan}
%     {\doglobal\increment\linereference
%      % start 1=>(n=y,l=0,p=1)
%      % stop  2=>(n=y,l=0,p=2)
%      \setxvalue{lrf:n:\linereference}{\@@rnprefix#2}%
%      \setxvalue{lrf:l:\linereference}{0}%
%      \setxvalue{lrf:p:\linereference}{#1}%
%      \advance\dimen0 by \linereference sp}%
%   \prewordbreak
%   \vrule \!!width \!!zeropoint \!!depth \dimen0 \!!height \!!zeropoint
%   \prewordbreak
%   \egroup}

\def\regelreferentie#1[#2]%
  {\bgroup
   \dimen0=\dp\strutbox
   \doif{\@@rnrefereren}{\v!aan}
     {\doglobal\increment\linereference
      % start 1=>(n=y,l=0,p=1)
      % stop  2=>(n=y,l=0,p=2)
      \setlinereference{\linereference}{\@@rnprefix#2}{0}{#1}%
      \advance\dimen0 by \linereference sp}%
   \prewordbreak
   \vrule \!!width \!!zeropoint \!!depth \dimen0 \!!height \!!zeropoint
   \prewordbreak
   \egroup}

\def\dostopnummerenPAR% dp's -> openstrutdepth
  {\endofshapebox
   \checkreferences
   \linestepper=0
   \reshapebox{\global\advance\linestepper by 1\relax}%
   \global\advance\linenumber by \linestepper
   \doifelse{\@@rnrefereren}{\v!aan}
     {\reshapebox % We are going back!
        {\global\advance\linenumber by -1
         \dimen0=\dp\shapebox
         \advance\dimen0 by -\dp\strutbox
         \ifdim\dimen0>\!!zeropoint\relax
           % 1=>4 | 2=>4 1=>2
           % start 1=>(n=y,l=2,p=1)
           % stop  2=>(n=y,l=4,p=2)
           \dostepwiserecurse{1}{\number\dimen0}{1}
             {\getlinereference\recurselevel
              \setlinereference\recurselevel
                {\linereferencename}{\the\linenumber}{\linereferenceplus}}%
         \fi}%
      \global\advance\linenumber by \linestepper
      \ifnum\linereference>0 % anders vreemde loop in paragraphs+recurse
        \dorecurse{\linereference}
          {\getlinereference\recurselevel
           \ifnum\linereferenceplus=2 % stop
             % ref y: text = 4 / Kan dit buiten referentie mechanisme om?
             \expanded{\setlocalcrossreference
               {\referenceprefix\linereferencename}{}{}{\linereferenceline}}%
           \fi}%
        \dorecurse{\linereference}
          {\getlinereference\recurselevel
           \ifnum\linereferenceplus<2 % start / lone
             \ifnum\linereferenceplus=1 % start
               \getreferenceelements{\linereferencename}% text = 4
               \ifnum\linereferenceline<0\currenttextreference\relax % 0 prevents error
                 \edef\linereferenceline{\linereferenceline--\currenttextreference}%
               \fi
             \fi
             \expanded{\setlocalcrossreference
               {\referenceprefix\linereferencename}{}{}{\linereferenceline}}%
           \fi}%
        \global\let\scratchline=\linenumber  % We are going back!
        \reshapebox
          {\doglobal\decrement\scratchline
           \hbox
             {\dorecurse{\linereference}
                {\getlinereference\recurselevel
                 \getreferenceelements{\linereferencename}%
                 \beforesplitstring\currenttextreference--\at--\to\firstline
                 \ifnum\firstline=\scratchline\relax
                   % beter een rawtextreference
                   \textreference[\linereferencename]{\currenttextreference}%
                   \expanded{\setlocalcrossreference
                     {\referenceprefix\linereferencename}{}{}{0}}% ==done
                 \fi}%
              \dimen0=\dp\shapebox
              \advance\dimen0 by -\dp\strutbox
              \ifdim\dimen0>\!!zeropoint\relax
                \dp\shapebox=\dp\strutbox
              \fi
              \schrijfregelnummer\box\shapebox}}% no \strut !
      \else
        \reshapebox{\hbox{\schrijfregelnummer\box\shapebox}}% no \strut !
      \fi}
     {\reshapebox{\global\advance\linenumber by -1}%
      \global\advance\linenumber by \linestepper
      \reshapebox{\hbox{\schrijfregelnummer\box\shapebox}}}% no \strut !
   \global\advance\linenumber by \linestepper
   \flushshapebox
   \egroup}

\def\crlf%
  {\ifhmode\unskip\else\strut\fi\ifcase\raggedstatus\hfil\fi\break}

\def\opeenregel%
  {\def\crlf{\ifhmode\unskip\fi\space}\let\\\crlf}

\newcount\internalparagraphnumber

\def\stelparagraafnummerenin%
  {\dosingleempty\dostelparagraafnummerenin}

\def\dostelparagraafnummerenin[#1]%
  {\getparameters
     [\??ph][#1]%
   \processaction
     [\@@phstatus]
     [\v!start=>\let\showparagraphnumber\doshowparagraphnumberA,
       \v!stop=>\let\showparagraphnumber\relax,
      \v!regel=>\let\showparagraphnumber\doshowparagraphnumberB,
      \v!reset=>\global\internalparagraphnumber=0  
                \let\showparagraphnumber\doshowparagraphnumberA]}

\def\dodoshowparagraphnumber%
  {\global\advance\internalparagraphnumber 1
   \inleftmargin % \tf normalizes em 
     {\tf{\doattributes\??ph\c!letter\c!kleur{\the\internalparagraphnumber}}%
      \kern\@@phafstand}}

\def\doshowparagraphnumberA%
  {\ifprocessingverbatim
     \iflinepar\dodoshowparagraphnumber\fi
   \else
     \dodoshowparagraphnumber
   \fi}

\def\doshowparagraphnumberB%
  {\ifinregelnummeren
     \doshowparagraphnumberA
   \fi}

\newbox\opmaak

\def\setopmaaklayout[#1]%
  {\stelvoetin [\c!status=\getvalue{\??do#1\c!voetstatus}]%
   \stelhoofdin[\c!status=\getvalue{\??do#1\c!hoofdstatus}]%
   \steltekstin[\c!status=\getvalue{\??do#1\c!tekststatus}]%
   \stelonderin[\c!status=\getvalue{\??do#1\c!onderstatus}]%
   \stelbovenin[\c!status=\getvalue{\??do#1\c!bovenstatus}]}

\def\dododostartopmaak[#1]%
  {\doifvaluesomething{\??do#1\c!pagina}
     {\ExpandFirstAfter\pagina[\getvalue{\??do#1\c!pagina}]}%
   \soortpagina[#1]%
   \setopmaaklayout[#1]%
   \getvalue{\??do#1\c!commandos}%
   \global\setbox\opmaak=\vbox to \getvalue{\??do#1\c!hoogte}%
   \bgroup
   \forgetall
   \hsize=\getvalue{\??do#1\c!breedte}%
   \getvalue{\??do#1\c!boven}}

\def\dododostopopmaak[#1]%
  {\getvalue{\??do#1\c!onder}%
   \egroup}

\def\doshipoutopmaak[#1]%
  {\bgroup
   \getvalue{\??do#1\c!voor}%
   \box\opmaak
   \setopmaaklayout[#1]%
   \pagina
   \getvalue{\??do#1\c!na}%
   \ifdubbelzijdig
     \ifodd\realpageno\else
       \processaction
         [\getvalue{\??do#1\c!dubbelzijdig}]
         [  \v!ja=>\null\pagina\verlaagpaginanummer,
          \v!leeg=>\pagebodyornamentsfalse
                   \null\pagina\verlaagpaginanummer]%
     \fi
   \fi
   \verlaagpaginanummer
   \egroup}

\def\doflushopmaak[#1]%
  {\ifverwerken
     \ifgeselecteerd
       \doshipoutopmaak[#1]%
     \fi
   \else
     \ifgeselecteerd
     \else
       \doshipoutopmaak[#1]%
     \fi
   \fi
   \ifselecteren
     \global\geselecteerdfalse
   \fi}

\def\dodostartopmaak[#1][#2]%
  {\begingroup
   \stelopmaakin[#1][#2]%
   \dododostartopmaak[#1]}

%\def\dodostopopmaak[#1]%
%  {\dododostopopmaak[#1]%
%   \doflushopmaak[#1]%
%   \endgroup}

\def\dodostopopmaak[#1]%
  {\dododostopopmaak[#1]%
   \doflushopmaak[#1]%
   \endgroup
   \calculatehsizes
   \calculatevsizes}

\def\dostartopmaak[#1][#2]%
  {\iffirstargument
     \dodostartopmaak[#1][#2]%
     \def\stopopmaak%
       {\dodostopopmaak[#1]}%
   \else
     \pagina
     \stelhoofdin[\c!status=\v!leeg]
     \stelvoetin[\c!status=\v!leeg]
     \vbox to \teksthoogte % nog een topskip optie
       \bgroup
       \def\stopopmaak%
         {\egroup
          \eject}%
   \fi}

\def\startopmaak%
  {\dodoubleempty\dostartopmaak}

\def\dodefinieeropmaak[#1][#2]%
  {\getparameters
     [\??do#1]%
     [\c!breedte=\zetbreedte,
      \c!hoogte=\teksthoogte,
      \c!voffset=\!!zeropoint,
      \c!hoffset=\!!zeropoint,
      \c!commandos=,
      \c!pagina=\v!rechts,
      \c!dubbelzijdig=\v!leeg,
      \c!voor=,
      \c!boven=\vss,
      \c!onder=\vss,
      \c!na=,
      \c!onderstatus=\v!normaal,
      \c!bovenstatus=\v!normaal,
      \c!tekststatus=\v!normaal,
      \c!hoofdstatus=\v!stop,
      \c!voetstatus=\v!stop,
      #2]%
   \setvalue{\e!start#1\e!opmaak}%
     {\dodoubleempty\dodostartopmaak[#1]}%
   \setvalue{\e!stop#1\e!opmaak}%
     {\dodostopopmaak[#1]}}

\def\definieeropmaak%
  {\dodoubleargument\dodefinieeropmaak}

\def\dostelopmaakin[#1]%
  {\getparameters[\??do#1]}

\def\stelopmaakin%
  {\dodoubleargument\dostelopmaakin}

\newskip\linkssmaller
\newskip\rechtssmaller
\newskip\middensmaller

\def\dosinglesmaller#1%
  {\processaction
     [#1]
     [    \v!links=>\global\advance\linkssmaller  by \@@sllinks,
         \v!midden=>\global\advance\middensmaller by \@@slmidden,
         \v!rechts=>\global\advance\rechtssmaller by \@@slrechts,
        \s!unknown=>\global\advance\middensmaller by \commalistelement]}

\def\dosmaller[#1]%
  {\processaction
     [#1]
     [    \v!links=>\global\advance\linkssmaller  by \@@sllinks,
         \v!midden=>\global\advance\middensmaller by \@@slmidden,
         \v!rechts=>\global\advance\rechtssmaller by \@@slrechts,
        \s!unknown=>{\herhaalmetcommando[#1]\dosinglesmaller}]}

\def\complexstartsmaller[#1]%
  {\par
   \bgroup
   \global\linkssmaller=\!!zeropoint
   \global\rechtssmaller=\!!zeropoint
   \global\middensmaller=\!!zeropoint
   \processcommalistwithparameters[#1]\dosmaller
   \advance\leftskip  by \linkssmaller
   \advance\rightskip by \rechtssmaller
   \advance\leftskip  by \middensmaller
   \advance\rightskip by \middensmaller}

\def\simplestartsmaller%
  {\startsmaller[\v!midden]}

\definecomplexorsimple\startsmaller

\def\stopsmaller%
  {\par % else skips forgotten
   \egroup}

\def\stelsmallerin%
  {\dodoubleargument\getparameters[\??sl]}

\def\dodefinieerhbox[#1][#2]%
  {\setvalue{hbox#1}##1%
     {\hbox to #2{\begstrut##1\endstrut\hss}}}

\def\definieerhbox%
  {\dodoubleargument\dodefinieerhbox}

\def\lrcbox#1#2#%
  {\vbox#2\bgroup
   \let\\=\endgraf
   \forgetall#1\let\next=}

\def\lbox%
  {\lrcbox\raggedleft}

\def\rbox%
  {\lrcbox\raggedright}

\def\cbox%
  {\lrcbox\raggedcenter}

\def\dosetraggedvbox#1%
  {\processaction
     [#1]
     [  \v!links=>\def\raggedbox{\lbox},
       \v!rechts=>\def\raggedbox{\rbox},
       \v!midden=>\def\raggedbox{\cbox},
          \v!nee=>\def\raggedbox{\vbox\bgroup\raggedright\let\next=},
      \s!default=>\def\raggedbox{\vbox},
      \s!unknown=>\def\raggedbox{\vbox}]}

\def\dosetraggedhbox#1%
  {\processaction
     [#1]
     [  \v!links=>\let\raggedbox\regellinks,
       \v!rechts=>\let\raggedbox\regelrechts,
       \v!midden=>\let\raggedbox\regelmidden,
      \v!normaal=>\let\raggedbox\hbox,
      \s!default=>\let\raggedbox\hbox,
      \s!unknown=>\let\raggedbox\hbox]}

% \def\dosetraggedcommand#1% ook ruim,rechts en zo
%   {\processaction
%      [#1]
%      [  \v!links=>\def\raggedcommand{\raggedleft},
%        \v!rechts=>\def\raggedcommand{\raggedright},
%        \v!midden=>\def\raggedcommand{\raggedcenter},
%           \v!nee=>\def\raggedcommand{\raggedright},
%       \v!normaal=>\let\raggedcommand\relax,
%       \s!default=>\def\raggedcommand{\raggedcenter},
%       \s!unknown=>\let\raggedcommand\relax]}

\def\dosetraggedcommand#1% 
  {\expanded{\dodosetraggedcommand{#1}}}
 
\def\dodosetraggedcommand#1% 
  {\doifinsetelse{\v!ruim} {#1}{\!!doneatrue}{\!!doneafalse}%
   \doifinsetelse{\v!breed}{#1}{\!!donebtrue}{\!!donebfalse}%
   \let\raggedcommand\relax
   \let\raggedtopcommand\empty
   \let\raggedbottomcommand\empty
   \!!donectrue
   \ExpandFirstAfter\processallactionsinset
     [#1]
     [  \v!links=>\if!!donea      \def\raggedcommand{\veryraggedleft}%
                  \else           \def\raggedcommand{\raggedleft}%
                  \fi
                  \!!donecfalse,                    % {\v!links,\v!midden}
       \v!rechts=>\if!!donea      \def\raggedcommand{\veryraggedright}%
                  \else           \def\raggedcommand{\raggedright}%
                  \fi
                  \!!donecfalse,                    % {\v!rechts,\v!midden}
       \v!midden=>\if!!donec
                    \if!!doneb      \def\raggedcommand{\raggedwidecenter}%
                    \else\if!!donea \def\raggedcommand{\veryraggedcenter}%
                    \else           \def\raggedcommand{\raggedcenter}%
                    \fi\fi
                    \!!donecfalse                   % {\v!midden,\v!midden}
                  \else
                    \let\raggedbottomcommand\vfilll % bonus, pretty strong 
                    \let\raggedtopcommand   \vfilll % used with \framed for 
                  \fi,                              % instance in tables
         \v!hoog=>\let\raggedbottomcommand\vfilll,  % and since we lack a 
         \v!laag=>\let\raggedtopcommand   \vfilll,  % proper keyword, but
         \v!laho=>\let\raggedbottomcommand\vfilll   % we do support the 
                  \let\raggedtopcommand   \vfilll,  % ugly laho (lohi)
          \v!nee=>\def\raggedcommand{\raggedright},
           \v!ja=>\let\raggedcommand\relax,
      \v!normaal=>\let\raggedcommand\relax]}

\def\stelplaatsblokkenin%
  {\dodoubleargument\getparameters[\??bk]}

\def\stelblokkopjesin%
  {\dodoubleargument\getparameters[\??kj]}%

\def\dostelplaatsblokin[#1][#2]%
  {\getparameters[\??fl#1][#2]}

\def\stelplaatsblokin%
  {\dodoubleargument\dostelplaatsblokin}

\def\dostelblokkopjein[#1][#2]%
  {\getparameters[\??kj#1][#2]}

\def\stelblokkopjein%
  {\dodoubleargument\dostelblokkopjein}

\def\doleegblok#1%
  {\localframed
      [\??fl#1][\c!kader=\v!aan]%
      {\getmessage{\m!floatblocks}{12}}}

\def\docomplexplaatsblok[#1][#2][#3]#4%
  {\flushfootnotes
   \ifmargeblokken
     \doifinset{\v!marge}{#2}
       {\bgroup
        \everypar{\egroup\the\everypar}%
        \hsize=\@@mbbreedte}%
   \fi
   \global\insidefloattrue
   \dowithnextboxcontent
     {\postponefootnotes} % new 
     {\docompletefloat
        {#1}{#3}{#1}{#2}{#1}{#4}
        {\box\nextbox}}%
   \vbox}

\def\docomplexstarttekstblok[#1][#2][#3]%
  {\flushfootnotes
   \flushsidefloats % hoort eigenlijk niet hier
   \docomplexplaatsblok[#1][\v!tekst,#2,\v!links][#3]}

\def\docomplexreserveerblok[#1][#2][#3][#4]#5%
  {\getvalue{\e!plaats#1}[#3][#4]{#5}{\localframed[\??fl#1][#2]{#1}}}

\def\docomplexstartreserveertekstblok[#1][#2][#3][#4]%
  {\flushsidefloats % hoort eigenlijk niet hier
   \docomplexreserveerblok[#1][#2][\v!tekst,#3,\v!links][#4]}

\def\dodefinieerplaatsblok[#1][#2]%        #1=naam  #2=meervoud
  {\presetlocalframed[\??fl#1]%
   \stelplaatsblokin
     [#1]
     [\c!breedte=15\korpsgrootte,
      \c!hoogte=10\korpsgrootte,
      \c!kader=\@@bkkader,
      \c!straal=\@@bkstraal,
      \c!hoek=\@@bkhoek,
      \c!plaats=\@@bkplaats,
      \c!achtergrond=\@@bkachtergrond,
      \c!achtergrondraster=\@@bkachtergrondraster,
      \c!achtergrondkleur=\@@bkachtergrondkleur,
      \c!achtergrondoffset=\@@bkachtergrondoffset,
      \c!bovenkader=\@@bkbovenkader,
      \c!onderkader=\@@bkonderkader,
      \c!linkerkader=\@@bklinkerkader,
      \c!rechterkader=\@@bkrechterkader,
      \c!kaderoffset=\@@bkkaderoffset,
      \c!paginaovergangen=]%
   \stelblokkopjein
     [#1]
     [\c!plaats=\@@kjplaats,
      %\c!voor=\@@kjvoor,
      \c!tussen=\@@kjtussen,
      %\c!na=\@@kjna,
      \c!breedte=\@@kjbreedte,
      \c!kopletter=\@@kjkopletter,
      \c!letter=\@@kjletter,
      \c!kleur=\@@kjkleur,
      \c!uitlijnen=\@@kjuitlijnen,
      \c!nummer=\@@kjnummer,
      \c!wijze=\@@kjwijze,
      \c!blokwijze=\@@kjblokwijze,
      \c!sectienummer=\@@kjsectienummer,
      \c!conversie=\@@kjconversie]%
   \doorlabelen
     [#1]
     [\c!tekst=#1,
      \c!plaats=\v!intekst,
      \c!wijze=\getvalue{\??kj#1\c!wijze},
      \c!blokwijze=\getvalue{\??kj#1\c!blokwijze},
      \c!sectienummer=\getvalue{\??kj#1\c!sectienummer},
      \c!conversie=\getvalue{\??kj#1\c!conversie}]%
   \presetlabeltext[#1=\Word{#1}~]%
   \presetheadtext[#2=\Word{#2}]%
   \definieerlijst[#1]%
   \setvalue{\e!plaats\e!lijstmet#2}%
     {\dodoubleempty\doplaatslijst[#1]}%
   \setvalue{\e!volledige\e!lijstmet#2}%
     {\dotripleempty\dodovolledigelijst[#1][#2]}%
   \setvalue{\e!plaats#1}%
     {\dotripleempty\docomplexplaatsblok[#1]}%
   \setvalue{\e!reserveer#1}%
     {\doquadrupleempty\docomplexreserveerblok[#1]}%
   \setvalue{\e!start#1\e!tekst}%
     {\dotripleempty\docomplexstarttekstblok[#1]}%
   \setvalue{\e!stop#1\e!tekst}%
     {\dostoptextfloat}%
   \setvalue{\e!start\e!reserveer#1\e!tekst}%
     {\doquadrupleempty\docomplexstartreserveertekstblok[#1]}%
   \setvalue{\e!stop\e!reserveer#1\e!tekst}%
     {\dostoptextfloat}%
   \setvalue{\e!lege#1}%
     {\doleegblok{#1}}%
   \setvalue{\e!leeg#1}%
     {\doleegblok{#1}}}

\def\definieerplaatsblok%
  {\dodoubleargument\dodefinieerplaatsblok}

% De onderstaande  macro's ondersteunen het zetten van tekst
% rond figuren. De macro's zijn ontworpen door Daniel Comenetz
% en gepubliceerd in TUGBoat Volume 14 (1993), No. 1: Anchored
% Figures at Either Margin. De macro's zijn slechts op enkele
% punten door mij aangepast.

% afhankelijke variabelen
%
% \overgap    vervangen door   \floatsidetopskip
% \sidegap    vervangen door   \floatsideskip
% \undergap   vervangen door   \floatsidebottomskip
%
% \prskp      vervangen door   \tussenwit

% toegevoegde macro's/aanroepen
%
% \logsidefloat       : loginformatie
% \flushsidefloats    : nodig voor koppen

% recente wijzigingen:
%
% namen aangepast: \float... enz. i.p.v. \pic

% Pas op: \EveryPar{\EveryPar{}\margetitel{whatever}}
% \plaatsfiguur[links]{}{} moet goed gaan. In dat geval
% begint de tekst terecht wat lager.

\newdimen\sidefloatheight      % includes the topskip
\newdimen\sidefloatwidth
\newdimen\sidefloathsize
\newdimen\sidefloatvsize       \def\nofloatvsize{-1pt }

\newbox\floatbottom

\newif\ifrightfloat
\newif\ifmarginfloat
\newif\ifroomforfloat
\newif\iffloatshort
\newif\iffloatflag
\newif\iffloatrighteqo
\newif\iffloatlefteqo

\let\floatrighteqo=\eqno
\let\floatleftleqo=\leqno

% Watch it even more! In inner, gaat't mis omdat daar
% pagetotal enz niet zijn aangepast. Inner kan overigens niet
% betrouwbaar worden getest!

% NOT TOEVOEGEN: \the\everytrace

\everypar    ={\flushfootnotes
               \ifinner\else\checksidefloat\fi
               \checkindentation
               \showparagraphnumber  % here ? 
               \flushmargincontents
               \flushcomments}
\neverypar   ={}
\everydisplay={\flushfootnotes
               \adjustsidefloatdisplaylines}

\def\flushsidefloats%
  {\par
   \!!heighta=\sidefloatvsize
   \advance\!!heighta by -\pagetotal
   \ifdim\!!heighta>\!!zeropoint
     \witruimte % nog checken op interferentie
     \kern\!!heighta
   \fi
   \global\sidefloatvsize=\nofloatvsize
   \global\floatflagfalse}

\def\forgetsidefloats%
  {\global\sidefloatvsize=\nofloatvsize
   \global\floatflagfalse}

\def\flushsidefloatsafterpar%
  {\xdef\oldpagetotal{\the\pagetotal}%
   \gdef\checksidefloat%
     {\dochecksidefloat
      \ifdim\oldpagetotal=\pagetotal \else
        \xdef\checksidefloat{\dochecksidefloat}%
        \flushsidefloats
      \fi}}

\let\logsidefloat=\relax

\def\pushpenalties%
  {\widowpenalty=1
   \clubpenalty=2
   \brokenpenalty=1
   \let\pushpenalties=\relax
   \edef\poppenalties%
     {\widowpenalty=\the\widowpenalty
      \clubpenalty=\the\clubpenalty
      \brokenpenalty=\the\brokenpenalty}}

\let\poppenalties=\relax

\def\restorepenalties%
  {\ifnum\outputpenalty=\!!tenthousand\relax
   \else
     \penalty\outputpenalty
   \fi}

\def\sidefloatoutput%
  {\iffloatshort
     \unvbox\normalpagebox
     \setbox\floatbottom=\lastbox
     \ifdim\wd\floatbottom>\sidefloathsize
       \penalty-201
       \box\floatbottom
     \else
       \ifvoid\floatbottom
       \else
         \restoreleftindent
         \ifdim\wd\floatbottom<\sidefloathsize
           \parskip=\!!zeropoint
           %\noindent
           \vadjust{\penalty-1}%
           \iffloatlefteqo
             \global\floatlefteqofalse
           \else
             \global\advance\sidefloathsize by -\wd\floatbottom
             \iffloatrighteqo
               \global\floatrighteqofalse
             \else
               \global\divide\sidefloathsize by 2
             \fi
             \hskip\sidefloathsize
           \fi
         \fi
         \box\floatbottom
         \restorepenalties
       \fi
     \fi
     \global\holdinginserts=0
     \global\floatshortfalse
   \else
     \finaloutput\unvbox\normalpagebox
     \global\sidefloatvsize=\nofloatvsize
     \poppenalties
   \fi}

\def\restoreleftindent%
  {\ifrightfloat
   \else
     \parskip=\!!zeropoint
     \noindent
     \vadjust{\penalty-1}%
     \hskip\sidefloatwidth
   \fi}

\def\eqno%
  {\iffloatshort
     \global\floatrighteqotrue
   \fi
   \floatrighteqo}

\def\leftmarginfloat#1%
  {\global\rightfloatfalse\marginfloattrue\putsidefloat{#1}}

\def\rightmarginfloat#1%
  {\global\rightfloattrue\marginfloattrue\putsidefloat{#1}}

\def\leftfloat#1%
  {\global\rightfloatfalse\marginfloatfalse\putsidefloat{#1}}

\def\rightfloat#1%
  {\global\rightfloattrue\marginfloatfalse\putsidefloat{#1}}

\def\putsidefloat#1%
  {\par
   \witruimte
   \previoussidefloat
   \stallsidefloat
%   \setbox\floatbox=\hbox{\vbox % pretty ugly, will be rewritten
%     {\vskip\ifmarginfloat-\sidefloattopskip\else\sidefloattopoffset\fi#1}}
   \setbox\floatbox=\hbox
     {\vbox{\vskip\ifmarginfloat-\fi\sidefloattopoffset#1}}%
   \measuresidefloat
   \ifroomforfloat
     \setsidefloat
   \else
     \tosssidefloat
     \measuresidefloat
     \stallsidefloat
     \setsidefloat
   \fi}

\def\progresssidefloat%
  {\!!heighta=\sidefloatvsize
   \iffloatflag
     \advance\!!heighta by -\dimen3
     \global\floatflagfalse
   \else
     \advance\!!heighta by -\pagetotal
   \fi}

\def\tosssidefloat%
  {\vfill\eject}

\def\measuresidefloat%
  {\global\floatflagtrue
   \dimen3=\pagetotal
   \ifmarginfloat
     \global\sidefloatwidth=\!!zeropoint
   \else
     \global\sidefloatwidth=\wd\floatbox
     \global\advance\sidefloatwidth by \floatsideskip
   \fi
   \global\sidefloathsize=\hsize
   \global\advance\sidefloathsize by -\sidefloatwidth
   \global\sidefloatheight=\ht\floatbox
\global\advance\sidefloatheight by \dp\floatbox
   \global\advance\sidefloatheight by \sidefloattopskip
   \global\sidefloatvsize=\sidefloatheight
   \global\advance\sidefloatvsize by \dimen3
   \dimen0=\sidefloatvsize
%   \advance\dimen0 by -\baselineskip
%\ifgridsnapping
%  \advance\dimen0 by .5\openlineheight % \vsize slightly too large
%\fi
   \ifdim\dimen0>.99\pagegoal          \relax
     \roomforfloatfalse
   \else
     \dimen0=\pagegoal
     \advance\dimen0 by -\sidefloatvsize
     \ifdim\dimen0<\sidefloatbottomskip
       \global\advance\sidefloatvsize by \dimen0
       \global\floatshorttrue
       \pushpenalties
       \global\holdinginserts=1
     \else
       \global\advance\sidefloatvsize by \sidefloatbottomskip
       \global\floatshortfalse
     \fi
     \roomforfloattrue
   \fi}

\def\setsidefloat%
  {\vbox{\strut}\vskip-\lineheight
   \kern\sidefloattopskip
   \edef\presidefloatdepth{\the\prevdepth}%
   \nointerlineskip
   \bgroup
   \everypar={}%
   \parskip=\!!zeropoint
   \logsidefloat
   \ifrightfloat
     \hfill
     \ifmarginfloat
       \rlap{\hskip\rechtermargeafstand\hskip\rightskip\unhbox\floatbox}%
     \else
       \unhbox\floatbox
     \fi
   \else
     \noindent
     \ifmarginfloat
       \llap{\unhbox\floatbox\hskip\leftskip\hskip\linkermargeafstand}%
     \else
       \unhbox\floatbox
     \fi
     \hfill
   \fi
   \egroup
   \par
   \kern-\sidefloatheight
   \penalty10001
   \normalbaselines
   \prevdepth=\presidefloatdepth
   %\noindent
   \resetsidefloatparagraph
   \ignorespaces}

\newcount\sidefloatparagraph

\def\iffirstsidefloatparagraph%
  {\ifnum\sidefloatparagraph=1\relax}

\def\setsidefloatparagraph%
  {\global\advance\sidefloatparagraph by 1\relax}

\def\resetsidefloatparagraph%
  {\global\sidefloatparagraph=0\relax}

\def\dochecksidefloat%
  {\progresssidefloat
   \ifdim\!!heighta>\!!zeropoint
     \advance\!!heighta by \sidefloatbottomskip
     \!!counta=\!!heighta
     \divide\!!counta by \baselineskip
     \ifnum\!!counta>0
       \ifrightfloat
         \hangindent=-\sidefloatwidth
       \else
         \hangindent=\sidefloatwidth
       \fi
       \hangafter=-\!!counta
     \fi
     \setsidefloatparagraph
   \else
     \resetsidefloatparagraph
   \fi
   \parskip=\tussenwit}

\def\checksidefloat%
  {\dochecksidefloat}

\def\doadjustsidefloatdisplaylines%
  {\par
   \vskip-\parskip
   \noindent
   \ignorespaces}

\def\adjustsidefloatdisplaylines%
  {\aftergroup\doadjustsidefloatdisplaylines}

\def\previoussidefloat%
  {\progresssidefloat
   \ifdim\!!heighta>\!!zeropoint \relax
     \iffloatshort
       \global\floatshortfalse
       \tosssidefloat
     \else
       \kern\!!heighta
     \fi
   \fi}

\def\stallsidefloat%
  {\!!counta=\pageshrink
   \divide\!!counta by \baselineskip
   \advance\!!counta by 1
   \parskip=\!!zeropoint
   \dorecurse{\!!counta}{\line{}}
   \kern-\!!counta\baselineskip
   \penalty0\relax}

% De onderstaande macro's zijn verantwoordelijk voor het plaatsen
% van floats. De macro's moeten nog worden aangepast en
% uitgebreid:
%
% -  nofloatpermitted : top, bot en mid counters en geen topins
%    als reeds midfloat of botfloat
%
% -  links, rechts, midden als niet hangend

\newif\ifsomefloatwaiting     \somefloatwaitingfalse
\newif\ifroomforfloat         \roomforfloattrue
\newif\ifnofloatpermitted     \nofloatpermittedfalse
\newif\iffloatsonpage         \floatsonpagefalse

\newcount\totalnoffloats      \totalnoffloats=0
\newcount\savednoffloats      \savednoffloats=0
\newcount\noffloatinserts     \noffloatinserts=0

\newbox\floatlist

\newinsert\botins

\skip\botins=\!!zeropoint
\count\botins=\!!thousand
\dimen\botins=\maxdimen

\newdimen\topinserted
\topinserted=\!!zeropoint

\newdimen\botinserted
\botinserted=\!!zeropoint

\newif\ifflushingfloats
\flushingfloatsfalse

\newbox\floattext

\newdimen\floattextwidth
\newdimen\floattextheight

\newbox\floatbox

\newdimen\floatwidth
\newdimen\floatheight

% Er wordt bij \v!altijd als dat nodig is hernummerd.
% Daarbij wordt gebruik gemaakt van de opgeslagen nummers en
% volgorde.

\definetwopasslist{\s!float}

\def\dofloatreference%
  {\doglobal\increment\numberedfloat
   \edef\dodofloatreference%
     {\writeutilitycommand%
        {\twopassentry%
           {\s!float}%
           {\numberedfloat}%
           {\hetnummer}}}%
   \dodofloatreference}

\def\redofloatorder#1%
  {\doglobal\increment\nofplacedfloats\relax
   \gettwopassdata{\s!float}%
   \iftwopassdatafound
     \doifnot{\hetnummer}{\twopassdata}
       {\edef\oldhetnummer{\hetnummer}%
        \xdef\hetnummer{\twopassdata}%
        \showmessage
          {\m!floatblocks}{1}
          {\nofplacedfloats,#1 \oldhetnummer,\hetnummer}}%
   \fi}

% In \dofloatinfomessage wordt {{ }} gebruikt omdat anders
% binnen \startuitstellen...\stopuitstellen geen goede
% melding in de marge volgt: \ifinner is dan namelijk true.

\def\dofloatinfomessage#1#2#3%
  {\bgroup
   \showmessage{\m!floatblocks}{#2}{#3}%
   \@EA\floatinfo\@EA#1\@EA{\currentmessagetext}%
   \egroup}

\def\dosavefloatinfo%
  {\dofloatinfomessage{>}{2}{\the\totalnoffloats}}

\def\dofloatflushedinfo%
  {\bgroup
   \!!counta=\totalnoffloats
   \advance\!!counta by -\savednoffloats
   \dofloatinfomessage{<}{3}{\the\!!counta}%
   \egroup}

\def\doinsertfloatinfo%
  {\dofloatinfomessage{<}{4}{\the\totalnoffloats}}

% ook voetnoten saven

\def\dosavefloat%
  {\global\setbox\floatlist=\vbox
     {\nointerlineskip
      \box\floatbox
      \unvbox\floatlist}%
   \global\advance\savednoffloats by 1
   \global\somefloatwaitingtrue
   \dosavefloatinfo
   \nonoindentation}

\def\doresavefloat%
  {\global\setbox\floatlist=\vbox
     {\nointerlineskip
      \unvbox\floatlist
      \box\floatbox}%
   \global\advance\savednoffloats by 1
   \global\somefloatwaitingtrue}

\def\doreversesavefloat%
  {\global\setbox\floatlist=\vbox
     {\nointerlineskip
      \unvbox\floatlist
      \box\floatbox}%
   \global\advance\savednoffloats by 1
   \global\somefloatwaitingtrue
   \dosavefloatinfo}

\def\checkwaitingfloats#1%
  {\ifsomefloatwaiting
     \doifinsetelse{\v!altijd}{#1}
       {\showmessage{\m!floatblocks}{5}{}}
       {\doflushfloats}%
   \fi}

\def\doflushfloats%
  {\global\floatsonpagefalse
   \global\flushingfloatstrue
   \ifsomefloatwaiting
     \par
     \ifvmode\prevdepth=\maxdimen\fi % prevents whitespace
     \dodoflushfloats
   \fi
   \global\savednoffloats=0
   \global\somefloatwaitingfalse
   \global\flushingfloatsfalse}

\def\dodoflushfloats% moet nog beter: als precies passend, niet onder baseline
  {\ifsomefloatwaiting
     \bgroup % \box\floatbox can be in use!
     \dogetfloat
     %\forgetall % NJET!
     \witruimte
     \blanko[\@@bkvoorwit]
     \flushfloatbox
     %\ifnum\savednoffloats>1 % REMOVED
     %\else
       \blanko[\@@bknawit]
     %\fi
     \egroup
     \dofloatflushedinfo
     \expandafter\dodoflushfloats
   \fi}

\newbox\globalscratchbox

\def\dogetfloat%
  {\ifsomefloatwaiting
     \global\setbox\floatlist=\vbox
       {\unvbox\floatlist
        \global\setbox\globalscratchbox=\lastbox}%
     \setbox\floatbox=\box\globalscratchbox % local !
     \global\advance\savednoffloats by -1\relax
     \ifnum\savednoffloats=0
       \global\somefloatwaitingfalse
     \fi
   \else
     \global\savednoffloats=0
     \global\setbox\floatbox=\box\voidb@x
   \fi}

\def\dotopfloat%
  {\ifdim\topinserted=\!!zeropoint\relax
     \topofinserttrue
   \else
     \topofinsertfalse
   \fi
   \global\advance\topinserted by \ht\floatbox
   \global\advance\topinserted by \dp\floatbox
   \global\advance\topinserted by \floatbottomskip
   \insert\topins
     {\forgetall
      \iftopofinsert
        \kern-\lineskip\par\prevdepth=\maxdimen
      \else
        %\blanko[-\@@bknawit,\@@bkvoorwit]% inserts can't look back
        \betweenfloatblanko
      \fi
      \flushfloatbox
      \blanko[\@@bknawit]}%
   \doinsertfloatinfo}

% The number of topinserts also influences the float order,
% in this respect that when a moved float does not fit, but a
% next one does, it is indeed placed. Take for instance a
% sequence of 20 floats, large and small, where a large one
% migrates and the next smaller one is inserted.

\def\dodosettopinserts%
  {\ifnum\noffloatinserts<\noftopfloats
     \dogetfloat
     \ifdim\topinserted=\!!zeropoint\relax
       \topofinserttrue
     \else
       \topofinsertfalse
     \fi
     \global\advance\topinserted by \ht\floatbox
     \global\advance\topinserted by \dp\floatbox
     \global\advance\topinserted by \floatbottomskip\relax
     \ifdim\topinserted<\teksthoogte\relax
       \xdef\totaltopinserted{\the\topinserted}%
       \insert\topins
         {\forgetall
          \iftopofinsert
            \kern-\lineskip\par\prevdepth=\maxdimen
          \else
            %\blanko[-\@@bknawit,\@@bkvoorwit]% inserts can't look back
            \betweenfloatblanko
          \fi
          \flushfloatbox
          \blanko[\@@bknawit]}%
       \ifsomefloatwaiting
         \advance\noffloatinserts by 1
       \else
         \noffloatinserts=\noftopfloats\relax
       \fi
       \dofloatflushedinfo
     \else
       \doresavefloat
       \noffloatinserts=\noftopfloats\relax
     \fi
   \else
     \ifsomefloatwaiting
       \showmessage{\m!floatblocks}{6}{\the\noftopfloats}%
     \fi
     \let\dodosettopinserts=\relax
   \fi
   \dodosettopinserts}

\def\dosettopinserts%
  {\bgroup
   \ifsomefloatwaiting
     \noffloatinserts=0
     \let\totaltopinserted=\!!zeropoint
     \dodosettopinserts
     \ifnum\@@bknonder=0
       \ifnum\@@bknregels>0
         \ifdim\totaltopinserted>\!!zeropoint\relax
           \dimen0=\lineheight
           \dimen0=\@@bknregels\dimen0
           \advance\dimen0 by \totaltopinserted\relax
           \ifdim\dimen0>\teksthoogte
             \showmessage{\m!floatblocks}{8}{\@@bknregels}%
             \vfilll\eject
           \fi
         \fi
       \fi
     \fi
   \fi
   \egroup}

\def\dodosetbotinserts%
  {\ifnum\noffloatinserts<\nofbotfloats\relax
     \dogetfloat
     \global\advance\botinserted by \ht\floatbox\relax
     \global\advance\botinserted by \dp\floatbox\relax
     \global\advance\botinserted by \floattopskip\relax
     \ifdim\botinserted<\pagegoal\relax
       \insert\botins
         {\forgetall
          \blanko[\@@bkvoorwit]%
          \flushfloatbox}%
       \ifsomefloatwaiting
         \advance\noffloatinserts by 1
       \else
         \noffloatinserts=\nofbotfloats
       \fi
       \dofloatflushedinfo
     \else
       \doresavefloat
       \noffloatinserts=\nofbotfloats\relax
     \fi
     \global\nofloatpermittedtrue % vgl topfloats s!
   \else
     \ifsomefloatwaiting
       \showmessage{\m!floatblocks}{7}{\the\nofbotfloats}%
     \fi
     \let\dodosetbotinserts=\relax
   \fi
   \dodosetbotinserts}

\def\dosetbotinserts%
  {\bgroup
   \ifsomefloatwaiting
     \noffloatinserts=0
     \dodosetbotinserts
   \fi
   \egroup}

\def\dobotfloat%
  {\global\advance\botinserted by \ht\floatbox
   \global\advance\botinserted by \dp\floatbox
   \global\advance\botinserted by \floattopskip
   \insert\botins
     {\forgetall
      \blanko[\@@bkvoorwit]%
      \flushfloatbox}%
   %\global\nofloatpermittedtrue
   \doinsertfloatinfo}

\def\dosetbothinserts%
  {\ifflushingfloats
     \global\topinserted=\!!zeropoint\relax
     \global\botinserted=\!!zeropoint\relax
   \else
     \global\topinserted=\!!zeropoint\relax
     \dosettopinserts
     \global\botinserted=\topinserted\relax
     \dosetbotinserts
   \fi}

\def\dotopinsertions%
  {\ifvoid\topins\else
     \ifgridsnapping
       %\topsnaptogrid{\box\topins}
       \box\topins % already snapped
     \else
       \unvbox\topins
     \fi
   \fi
   \global\topinserted=\!!zeropoint\relax}

\def\dobotinsertions%
  {\ifvoid\botins\else
     \ifgridsnapping
       \snaptogrid\hbox{\box\botins}
     \else
       \unvbox\botins
     \fi
   \fi
   \global\botinserted=\!!zeropoint\relax
   \global\nofloatpermittedfalse}

\newif\iftopofinsert
\newif\iftestfloatbox %\testfloatboxtrue

%\def\flushfloatbox% nog verder doorvoeren en meer info in marge
%  {\iftestfloatbox
%     \ruledhbox{\box\floatbox}%
%   \else
%     \box\floatbox
%   \fi}

% \testfloatboxtrue
%
% testfloatbox gaat mis, niet in midden, dus elders

\def\flushfloatbox% nog verder doorvoeren en meer info in marge
  {\snaptogrid\hbox{\iftestfloatbox\ruledhbox\fi{\copy\floatbox}}}

% beter de laatste skip buiten de \insert uitvoeren,
% bovendien bij volle flush onder baseline.

\def\betweenfloatblanko% assumes that \@@bknawit is present
  {\bgroup
   \setbox0=\vbox{\strut\blanko[\@@bkvoorwit]\strut}%
   \setbox2=\vbox{\strut\blanko[\@@bknawit]\strut}%
   \ifdim\ht0>\ht2
     \blanko[-\@@bknawit,\@@bkvoorwit]
   \fi
   \egroup}

\def\doroomfloat%
  {\ifnofloatpermitted
     \global\roomforfloatfalse
   \else
     \dimen0=\pagetotal
     \advance\dimen0 by \ht\floatbox
     \advance\dimen0 by \dp\floatbox
     \advance\dimen0 by \floattopskip
     \advance\dimen0 by -\pageshrink  % toegevoegd
%\ifgridsnapping
%  \advance\dimen0 by .5\openlineheight % \vsize slightly too large
%\fi
     \ifdim\dimen0>\pagegoal
       \global\roomforfloatfalse
     \else
       \global\roomforfloattrue
     \fi
   \fi}

\def\doexecfloat% spacing between two successive must be better
  {\baselinecorrection
   \witruimte
   \blanko[\@@bkvoorwit]%
   \flushfloatbox
   \blanko[\@@bknawit]%
   \doinsertfloatinfo
   \doif{\@@bkspringvolgendein}{\v!nee}{\noindentation}} % new 

\def\somefixdfloat[#1]%
  {\doroomfloat
   \ifroomforfloat\else
     \goodbreak
   \fi
   \showmessage{\m!floatblocks}{9}{}%
   \doexecfloat}

\def\somesidefloat[#1]%  links, rechts     NOG TESTEN EN AANPASSEN
  {\ifbinnenkolommen
     \someelsefloat[\v!hier]%
   \else
     \checkwaitingfloats{#1}%
     \def\logsidefloat%
       {\doinsertfloatinfo}%
     \setbox\floatbox=\vbox{\box\floatbox}%
     \wd\floatbox=\floatwidth
     \processfirstactioninset
       [#1]
       [     \v!links=>\leftfloat{\box\floatbox},
            \v!rechts=>\rightfloat{\box\floatbox},
          \v!inlinker=>\leftmarginfloat{\box\floatbox},
         \v!inrechter=>\rightmarginfloat{\box\floatbox},
           \v!inmarge=>{\doinmargenormal\leftmarginfloat
                        \rightmarginfloat{\box\floatbox}}]%
     \doifinset{\v!lang}{#1}
       {\flushsidefloatsafterpar}%
   \fi}

\def\sometextfloat[#1]%  lang, links, rechts, hoog, midden, laag, offset
  {\checkwaitingfloats{#1}%
   \def\dostoptextfloat{\dodostoptextfloat[#1]}%
   \global\floattextwidth=\hsize
   \global\floatwidth=\wd\floatbox
   \global\floatheight=\ht\floatbox % forget about the depth
   \global\advance\floattextwidth by -\floatwidth
   \global\advance\floattextwidth by -\@@bkmarge\relax % was \tfskipsize
   \doifinsetelse{\v!lang}{#1}
     {\floattextheight=\pagegoal
      \advance\floattextheight by -\pagetotal
      \advance\floattextheight by -\bigskipamount     % lelijk
      \ifdim\floattextheight>\teksthoogte
        \floattextheight=\teksthoogte
      \fi
      \boxmaxdepth=\!!zeropoint\relax            % toegevoegd
      \ifdim\floattextheight<\floatheight
        \floattextheight=\floatheight
      \fi
      \setbox\floattext=\vbox to \floattextheight}
     {\setbox\floattext=\vbox}%
   \bgroup
\forgetall\stelblankoin\stelwitruimtein % new, also needed for footnotes
   \blanko[\v!blokkeer]
   \hsize\floattextwidth
   \ignorespaces}

\def\dodostoptextfloat[#1]%
  {\egroup
   \doifnotinset{\v!lang}{#1}%
     {\ifdim\ht\floattext<\floatheight
        \floattextheight=\floatheight
      \else
        \floattextheight=\ht\floattext
      \fi}%
   \setbox\floatbox=\vbox to \floattextheight
     {\hsize\floatwidth
      \doifinsetelse{\v!beide}{#1}%
        {\doifinsetelse{\v!laag}{#1}
           {\vfill\box\floatbox}
           {\doifinsetelse{\v!midden}{#1}
              {\vfill\box\floatbox\vfill}
              {\box\floatbox\vfill}}}
        {\box\floatbox\vfill}}%
    \setbox\floattext=\vbox to \floattextheight
     {\hsize\floattextwidth
      \doifinsetelse{\v!laag}{#1}
        {\vfill
         \box\floattext
         \doifinset{\c!offset}{#1}{\witruimte\blanko}}
        {\doifinsetelse{\v!midden}{#1}
           {\vfill
            \box\floattext
            \vfill}
           {\doifinset{\v!offset}{#1}{\witruimte\blanko}%
            \box\floattext
            \vfill}}}%
   \doifinsetelse{\v!rechts}{#1}%
     {\setbox\floatbox=\hbox to \hsize
        {\box\floattext
         \hfill
         \box\floatbox}}
     {\setbox\floatbox=\hbox to \hsize
        {\box\floatbox
         \hfill
         \box\floattext}}%
   \baselinecorrection
   \witruimte
   \blanko[\@@bkvoorwit]%
   \doifnotinset{\v!lang}{#1}%
     {\dp\floatbox=\openstrutdepth}% dp\strutbox}%      % toegevoegd
   \box\floatbox
   \blanko[\@@bknawit]%
   \doinsertfloatinfo}

\def\somefacefloat[#1]%  links, rechts, midden, hoog, midden, laag
  {\checkwaitingfloats{#1}%
   \startnaast\box\floatbox\stopnaast
   \doinsertfloatinfo}

\def\somepagefloat[#1]%  links, rechts, midden, hoog, midden, laag
  {\checkwaitingfloats{#1}%
   \vbox to \teksthoogte
     {\doifnotinset{\v!hoog}{#1}{\vfill}%
      \box\floatbox
      \doifnotinset{\v!laag}{#1}{\vfill}}%
   \doinsertfloatinfo
   \pagina}                      % toegevoegd

\def\someelsefloat[#1]%
  {\doifinsetelse{\v!hier}{#1}
     {\doifinsetelse{\v!altijd}{#1}
        {\pagina[\v!voorkeur]%
         \doroomfloat
         \ifroomforfloat
           \doexecfloat
         \else
           \showmessage{\m!floatblocks}{9}{}%
           \doreversesavefloat
         \fi}
        {\ifsomefloatwaiting
           \dosavefloat
         \else
           \pagina[\v!voorkeur]%
           \doroomfloat
           \ifroomforfloat
             \doexecfloat
           \else
             \dosavefloat
           \fi
         \fi}}
     {\doifinsetelse{\v!altijd}{#1}
        {\doroomfloat
         \ifroomforfloat
           \processallactionsinset
             [#1]
             [   \v!boven=>\dotopfloat,
                 \v!onder=>\dobotfloat,
               \s!default=>\doexecfloat]%
         \else
           \showmessage{\m!floatblocks}{9}{}%
           \doreversesavefloat
         \fi}
        {\doroomfloat
         \ifroomforfloat
           \processallactionsinset
             [#1]
             [  \v!boven=>\dotopfloat,
                \v!onder=>\dobotfloat,
              \s!default=>\doexecfloat]%
         \else
           \dosavefloat
         \fi}}}

% De onderstaande macro wordt gebruikt bij de macros
% voor het plaatsen van tabellen en figuren (klopt niet
% meer).
%
% \dofloat         {plaats} {label1} {label2} {kader}
%
% \docompletefloat {nummer} {referentie} {lijst}
%                  {plaats} {label1} {label2} {inhoud}
%
% \box\floatbox    inhoud+referentie
%
% \do???float#1    #1 = boxnummer
%
% \ifinsidefloat   wordt \true gezet voor \docompletefloat en \false
%                  na float plaatsen; kan worden gebruikt om in
%                  andere commando's witruimte te onderdrukken

\newdimen\floattopskip          \floattopskip=12pt
\newdimen\floatbottomskip       \floatbottomskip=12pt
\newdimen\floatsideskip         \floatsideskip=12pt

\newdimen\sidefloattopskip      \sidefloattopskip=\floattopskip
\newdimen\sidefloatbottomskip   \sidefloatbottomskip=\floatbottomskip
\def\sidefloattopoffset         {\openstrutdepth} % {\dp\strutbox}

\newcount\noftopfloats          \noftopfloats=2
\newcount\nofbotfloats          \nofbotfloats=0

\def\calculatefloatskips%
  {{\def\calculatefloatskips##1##2%
      {\doifelsenothing{##2}
         {\global##1=\!!zeropoint}
         {\doifelse{##2}{\v!geen}
            {\global##1=\!!zeropoint}
            {\setbox0=\vbox{\witruimte\@EA\blanko\@EA[##2]}%
             \global##1=\ht0}}}%
    \calculatefloatskips\floattopskip\@@bkvoorwit
    \calculatefloatskips\floatbottomskip\@@bknawit
    \calculatefloatskips\sidefloattopskip\@@bkzijvoorwit
    \calculatefloatskips\sidefloatbottomskip\@@bkzijnawit
    \def\sidefloattopoffset{\openstrutdepth}% {\dp\strutbox}%
    \global\floatsideskip=\@@bkmarge\relax
    \global\noftopfloats=\@@bknboven\relax
    \global\nofbotfloats=\@@bknonder\relax}}

\newif\ifinsidefloat

\def\floatcaptionsuffix{} % an optional suffix
\def\floatcaptionnumber{} % a logical counter

\def\dosetfloatcaption#1#2#3%
  {\def\dofloattekst%
     {{\doattributes{\??kj#1}\c!letter\c!kleur{#3}}}%
   \doifelsevalue{\??kj#1\c!nummer}{\v!ja}
     {\def\dofloatnummer%
        {{\xdef\floatcaptionnumber{#1}%
          \hbox{\doattributes{\??kj#1}\c!kopletter\c!kopkleur
             {\strut#2\floatcaptionsuffix}}}%
          \ConvertToConstant\doifnot{#3}{}
            {\tfskip
             \emergencystretch=.5em}}}
     {\let\dofloatnummer=\empty}}

\def\putborderedfloat#1\in#2\\%
  {\setbox#2=\vbox
     {\localframed
        [\??fl#1]
        [\c!breedte=\@@bkbreedte,
         \c!hoogte=\@@bkhoogte,
         \c!plaats=\v!normaal,
         \c!offset=\@@bkoffset]%
        {\box\floatbox}}}

\newbox\captionbox

\def\putcompletecaption#1#2#3#4%
  {\noindent
   \xdef\floatcaptionnumber{#1}%
   \doattributes{\??kj#1}\c!letter\c!kleur
     {\doifvalue{\??kj#1\c!nummer}{\v!ja}
        {\hbox{\doattributes{\??kj#1}\c!kopletter\c!kopkleur{\strut#2\floatcaptionsuffix}}%
         \ConvertToConstant\doifnot{#3}{}
           {\ifcase#4\relax
              \tfskip\emergencystretch=.5em
            \else
              \ifx\@@kjkjtussen\empty\else\unskip\@@kjkjtussen\fi
            \fi}}%
      \begstrut#3\endstrut\endgraf}}

% \def\dosetpagfloat#1#2#3#4% \copy wegwerken 
%   {\bgroup
%    \forgetall
%    \postponefootnotes
%    \mindermeldingen
%   %\showcomposition
%    \putborderedfloat#4\in4\\%
%    \def\locatefloat%
%      {\doregelplaats\@@flflplaats}%
%    \ConvertToConstant\doifelse{#3}{\v!geen}
%      {\global\setbox\floatbox=\vbox % pas op als wd groter dan hsize
%         {\ifbinnenkolommen\ifdim\wd4>\hsize  
%            \let\locatefloat\relax            
%          \fi\fi                              
%          \locatefloat{\copy4}}}  % we need \wd4 later 
%      {\setbox2=\hbox  
%         {\footnotesenabledfalse\putcompletecaption{#4}{#2}{#3}{0}}%
%       \doifinsetelse{\@@kjkjplaats}{\v!hoog,\v!midden,\v!laag}
%         {\dimen0=\hsize
%          \advance\dimen0 by -\wd4\relax
%          \advance\dimen0 by -\@@bkmarge\relax % \was tfskipsize\relax
%          \ifdim\wd2>\dimen0\relax
%            \dimen2=1.3\dimen0\relax
%            \ifdim\wd2<\dimen2\relax
%              \dimen0=0.8\dimen0\relax
%            \fi
%          \fi
%          \setbox2=\vbox
%            {\forgetall
%             \hsize=\dimen0\relax
%             \raggedright
%             \putcompletecaption{#4}{#2}{#3}{1}}}
%         {\doifelse{\@@kjkjbreedte}{\v!max}
%            {\dosetraggedvbox{\@@kjkjuitlijnen}%
%             \setbox2=\raggedbox
%               {\hsize\wd4\relax
%                \putcompletecaption{#4}{#2}{#3}{0}}}
%            {\ifdim\wd2>\wd4\relax
%               \doifelse{\@@kjkjbreedte}{\v!passend}
%                 {\ifdim\wd4<15\korpsgrootte\relax
%                    \dimen0=15\korpsgrootte\relax
%                  \else
%                    \dimen0=\wd4\relax
%                  \fi
%                  \ifdim\wd4>\hsize
%                    \setbox0=\vbox
%                      {\forgetall
%                       \hsize=1.0\wd4
%                       \footnotesenabledfalse
%                       \putcompletecaption{#4}{#2}{#3}{0}}%
%                    \ifdim\ht0>\lineheight\relax
%                      \setbox2=\vbox
%                        {\forgetall
%                         \hsize=0.9\wd4
%                         \putcompletecaption{#4}{#2}{#3}{0}}%
%                    \else
%                     %\setbox2=\vbox % was 0, bug
%                     %  {\forgetall
%                     %   \hsize=1.0\wd4
%                     %   \putcompletecaption{#4}{#2}{#3}{0}}%
%                      \setbox2=\hbox{\putcompletecaption{#4}{#2}{#3}{0}}% 
%                    \fi
%                  \else
%                    \setbox0=\vbox
%                      {\forgetall
%                       \dimen2=1.5\dimen0\relax
%                       \ifdim\dimen2<\hsize
%                         \hsize=\dimen2\relax
%                       \fi
%                       \footnotesenabledfalse
%                       \putcompletecaption{#4}{#2}{#3}{0}}%
%                    \ifdim\ht0>\lineheight\relax
%                      \setbox2=\vbox
%                        {\forgetall
%                         \dimen2=1.2\dimen0\relax
%                         \ifdim\dimen2<\hsize
%                           \hsize=\dimen2\relax
%                         \fi
%                         \putcompletecaption{#4}{#2}{#3}{0}}%
%                    \else
%                     %\setbox2=\vbox % was 0, bug
%                     %  {\forgetall
%                     %   \dimen2=1.5\dimen0\relax
%                     %   \ifdim\dimen2<\hsize
%                     %     \hsize=\dimen2\relax
%                     %   \fi
%                     %   \putcompletecaption{#4}{#2}{#3}{0}}%
%                      \setbox2=\hbox{\putcompletecaption{#4}{#2}{#3}{0}}% 
%                    \fi
%                  \fi}
%                 {\dosetraggedvbox{\@@kjkjuitlijnen}%
%                  \setbox2=\raggedbox
%                    {\hsize\@@kjkjbreedte
%                     \putcompletecaption{#4}{#2}{#3}{0}}}%
%             \else
% %              \setbox2=\hbox % we want footnotes ! 
% %                {\putcompletecaption{#4}{#2}{#3}{0}}%
% \raggedcenter % default 
%                \dosetraggedvbox{\@@kjkjuitlijnen}%
%                \setbox2=\raggedbox
%                  {\hsize\wd4
%                   \putcompletecaption{#4}{#2}{#3}{0}}%
%             \fi}}%
%       \global\setbox\floatbox=\vbox
%         {\forgetall
%          \processaction
%            [\@@kjkjplaats]
%            [ \v!boven=>\locatefloat{\copy2}%
%                        \endgraf\@@kjkjtussen
%                        \locatefloat{\copy4},
%              \v!onder=>\locatefloat{\copy4}%
%                        \endgraf\@@kjkjtussen
%                        \locatefloat{\copy2},
%               \v!hoog=>\locatefloat
%                          {\doifelse{\@@flflplaats}{\v!links}
%                             {\copy4
%                              \tfskip
%                              \vbox to\ht4{\@@kjkjtussen\copy2\vfill}}
%                             {\vbox to\ht4{\@@kjkjtussen\copy2\vfill}%
%                              \tfskip
%                              \copy4}},
%               \v!laag=>\locatefloat
%                          {\doifelse{\@@flflplaats}{\v!links}
%                             {\copy4
%                              \tfskip
%                              \vbox to\ht4
%                                {\vfill\copy2\@@kjkjtussen}}
%                             {\vbox to\ht4
%                                {\vfill\copy2\@@kjkjtussen}%
%                              \tfskip
%                              \copy4}},
%             \v!midden=>\locatefloat
%                          {\doifelse{\@@flflplaats}{\v!links}
%                             {\copy4
%                              \tfskip
%                              \vbox to\ht4{\vfill\copy2\vfill}}
%                             {\vbox to\ht4{\vfill\copy2\vfill}%
%                              \tfskip
%                              \copy4}},
%               \v!geen=>\locatefloat{\copy4}]}}%
%   % extended: 
%    \postcenterfloatbox{\wd4}%
%   %\ifdim\wd4>\hsize
%   %  \global\setbox\floatbox=
%   %    \hbox to \ifbinnenkolommen\wd4\else\hsize\fi
%   %      {\hss\box\floatbox\hss}%
%   %\fi
%    \egroup}

% new

\newbox\tempfloatbox
\newbox\tempcaptionbox

%\stelblokkopjesin[\c!breedte=5cm]
%\stelblokkopjesin[\c!uitlijnen=\v!links]
%\stelblokkopjesin[\c!uitlijnen=\v!rechts]

\def\dosetpagfloat#1#2#3#4% \copy wegwerken 
  {\bgroup
   \forgetall
   \postponefootnotes
   \mindermeldingen
   \putborderedfloat#4\in\tempfloatbox\\%
   \def\locatefloat%
     {\doregelplaats\@@flflplaats}%
   \ConvertToConstant\doifelse{#3}{\v!geen}
     {\dopreparenocaption{#1}{#2}{#3}{#4}
      \edef\width{\the\wd\floatbox}}
     {\setbox\tempcaptionbox=\hbox  
        {\footnotesenabledfalse\putcompletecaption{#4}{#2}{#3}{0}}%
      \doifinsetelse{\@@kjkjplaats}{\v!hoog,\v!midden,\v!laag}
        {\dopreparesidecaption{#1}{#2}{#3}{#4}}
        {\doifelse{\@@kjkjbreedte}{\v!max}
           {\dopreparestackcaptionmax{#1}{#2}{#3}{#4}}
           {\ifdim\wd\tempcaptionbox>\wd\tempfloatbox % wider caption
              \doifelse{\@@kjkjbreedte}{\v!passend}
                {\dopreparestackcaptionaut{#1}{#2}{#3}{#4}}
                {\dopreparestackcaptionwid{#1}{#2}{#3}{#4}}%
            \else
              \dopreparestackcaptionmin{#1}{#2}{#3}{#4}%
            \fi}}
      \edef\width{\the\wd\tempfloatbox}%
      \buildfloatbox}%
   \postcenterfloatbox\width
   \egroup}

\def\dopreparenocaption#1#2#3#4%
  {\global\setbox\floatbox=\vbox % pas op als wd groter dan hsize
     {\ifbinnenkolommen\ifdim\wd\tempfloatbox>\hsize  
        \let\locatefloat\relax            
      \fi\fi                              
      \locatefloat{\copy\tempfloatbox}}} 

\def\dopreparestackcaptionmax#1#2#3#4%
  {\dosetraggedvbox{\@@kjkjuitlijnen}%
   \setbox\tempcaptionbox=\raggedbox
     {\hsize\wd\tempfloatbox
      \putcompletecaption{#4}{#2}{#3}{0}}}

\def\captionminwidth  {15\korpsgrootte}
\def\captionovershoot {2em}

\def\dopreparestackcaptionaut#1#2#3#4%
  {\doifsomething{\@@kjkjuitlijnen}
     {\ExpandBothAfter\doifnotinset{\v!midden}{\@@kjkjuitlijnen}
        {\let\captionovershoot\!!zeropoint}}% 
   \ifdim\wd\tempfloatbox>\hsize
     % float is wider than \hsize 
     \dosetraggedvbox\@@kjkjuitlijnen
     \setbox\scratchbox=\raggedbox % trial run 
       {\hsize=\wd\tempfloatbox
        \footnotesenabledfalse
        \putcompletecaption{#4}{#2}{#3}{0}}%
     \ifdim\ht\scratchbox>\lineheight % more lines 
       \dosetraggedvbox\@@kjkjuitlijnen
       \setbox\tempcaptionbox=\raggedbox
         {\hsize=\wd\tempfloatbox
          \advance\hsize -\captionovershoot\relax
          \ifdim\hsize<\captionminwidth\relax
            \hsize=\wd\tempfloatbox
          \fi
          \putcompletecaption{#4}{#2}{#3}{0}}%
     \else
       \setbox\tempcaptionbox=\raggedbox
         {\hsize=\wd\tempfloatbox
          \putcompletecaption{#4}{#2}{#3}{0}}%
     \fi
   \else
     % float is smaller of equal to \hsize 
     \ifdim\wd\tempfloatbox<\captionminwidth\relax
       \scratchdimen\captionminwidth % float smaller than min width 
     \else 
       \scratchdimen\wd\tempfloatbox % float width 
     \fi 
     \setbox\scratchbox=\vbox     % test with overshoot 
       {\advance\scratchdimen \captionovershoot
        \advance\scratchdimen 3em % an average word length 
        \ifdim\scratchdimen<\hsize \hsize=\scratchdimen \fi
        \footnotesenabledfalse
        \putcompletecaption{#4}{#2}{#3}{0}}%
     \ifdim\ht\scratchbox>\lineheight 
       % at least an average word longer than a line 
       \dosetraggedvbox\@@kjkjuitlijnen
       \setbox\tempcaptionbox\raggedbox
         {\advance\scratchdimen \captionovershoot
          \ifdim\scratchdimen<\hsize \hsize=\scratchdimen \fi
          \putcompletecaption{#4}{#2}{#3}{0}}%
     \else
       % just over a line, don't use an overshoot
       \dosetraggedvbox\@@kjkjuitlijnen
       \setbox\tempcaptionbox=\raggedbox
         {\hsize\scratchdimen
          \putcompletecaption{#4}{#2}{#3}{0}}%
     \fi
   \fi}

\def\dopreparestackcaptionwid#1#2#3#4%
  {\dosetraggedvbox\@@kjkjuitlijnen
   \setbox\tempcaptionbox=\raggedbox
     {\hsize\@@kjkjbreedte
      \putcompletecaption{#4}{#2}{#3}{0}}}

\def\dopreparestackcaptionmin#1#2#3#4%
  {\raggedcenter                     % the default 
   \dosetraggedvbox\@@kjkjuitlijnen  % when given 
   \setbox\tempcaptionbox=\raggedbox % vbox, keeps footnotes 
     {\hsize\wd\tempfloatbox
      \putcompletecaption{#4}{#2}{#3}{0}}}

\def\dopreparesidecaption#1#2#3#4%
  {\dimen0=\hsize
   \advance\dimen0 by -\wd\tempfloatbox
   \advance\dimen0 by -\@@bkmarge\relax % was \tfskipsize\relax
   \ifdim\wd\tempcaptionbox>\dimen0
     \dimen2=1.3\dimen0
     \ifdim\wd\tempcaptionbox<\dimen2
       \dimen0=0.8\dimen0
     \fi
   \fi
   \setbox\tempcaptionbox=\vbox
     {\hsize=\dimen0
      \raggedright
      \putcompletecaption{#4}{#2}{#3}{1}}}

\def\buildfloatbox%
  {\global\setbox\floatbox=\vbox
     {\forgetall
      \processaction
        [\@@kjkjplaats]
        [ \v!boven=>\locatefloat{\box\tempcaptionbox}%
                    \endgraf\@@kjkjtussen
                    \locatefloat{\box\tempfloatbox},
          \v!onder=>\locatefloat{\box\tempfloatbox}%
                    \endgraf\@@kjkjtussen
                    \locatefloat{\box\tempcaptionbox},
           \v!hoog=>\locatefloat
                      {\doifelse{\@@flflplaats}{\v!links}
                         {\box\tempfloatbox
                          \tfskip
                          \vbox to\ht\tempfloatbox{\@@kjkjtussen\box\tempcaptionbox\vfill}}
                         {\vbox to\ht\tempfloatbox{\@@kjkjtussen\box\tempcaptionbox\vfill}%
                          \tfskip
                          \box\tempfloatbox}},
           \v!laag=>\locatefloat
                      {\doifelse{\@@flflplaats}{\v!links}
                         {\box\tempfloatbox
                          \tfskip
                          \vbox to\ht\tempfloatbox
                            {\vfill\box\tempcaptionbox\@@kjkjtussen}}
                         {\vbox to\ht\tempfloatbox
                            {\vfill\box\tempcaptionbox\@@kjkjtussen}%
                          \tfskip
                          \box\tempfloatbox}},
         \v!midden=>\locatefloat
                      {\doifelse{\@@flflplaats}{\v!links}
                         {\box\tempfloatbox
                          \tfskip
                          \vbox to\ht\tempfloatbox{\vfill\box\tempcaptionbox\vfill}}
                         {\vbox to\ht\tempfloatbox{\vfill\box\tempcaptionbox\vfill}%
                          \tfskip
                          \box\tempfloatbox}},
           \v!geen=>\locatefloat{\box\tempfloatbox}]}}

\newif\ifpostponecolumnfloats \postponecolumnfloatsfalse % don't change 

\def\postcenterfloatbox#1%
  {\ifbinnenkolommen
     \ifpostponecolumnfloats
       \scratchdimen=\makeupwidth
     \else
       \scratchdimen=#1\relax
     \fi
   \else\ifdim#1>\hsize
     \scratchdimen=\hsize
   \else
     \scratchdimen=\wd\floatbox
   \fi\fi
   \global\setbox\floatbox=\hbox to \scratchdimen
     {\hfill\box\floatbox\hfill}} % geen \hss, gaat mis in kolommen ! 

\def\dosetparfloat#1#2#3#4%
  {\bgroup
   \forgetall
   \postponefootnotes
   \mindermeldingen
   %\showcomposition
   \putborderedfloat#4\in4\\
   \ConvertToConstant\doifelse{#3}{\v!geen}
     {\global\setbox\floatbox=\vbox{\box4}}
     {\setbox2=\hbox
        {\forgetall\putcompletecaption{#4}{#2}{#3}{0}}%
      \doifelse{\@@kjkjbreedte}{\v!max}
        {\dosetraggedvbox{\@@kjkjuitlijnen}%
         \setbox2=\raggedbox
           {\hsize\wd4\putcompletecaption{#4}{#2}{#3}{0}}}%
        {\doifelse{\@@kjkjbreedte}{\v!passend}
           {\ifdim\wd2>\wd4\relax
              \setbox2=\vbox
                {\forgetall\hsize\wd4\putcompletecaption{#4}{#2}{#3}{0}}%
            \else
              \setbox2=\hbox to \wd4
                {\hss\box2\hss}%
            \fi}
           {\dosetraggedvbox{\@@kjkjuitlijnen}%
            \setbox2=\raggedbox
              {\hsize\wd4\putcompletecaption{#4}{#2}{#3}{0}}}}%
      \global\setbox\floatbox=\vbox
        {\processaction
           [\@@kjkjplaats]
           [ \v!boven=>\box2\endgraf\@@kjkjtussen\box4,
             \v!onder=>\box4\endgraf\@@kjkjtussen\box2,
              \v!geen=>\box4,
           \s!unknown=>\box4\endgraf\@@kjkjtussen\box2]}}%
   \egroup}

\newif\ifparfloat

\long\def\dosetfloatbox#1#2#3#4%
  {\ifvisible
     \par
     \doifcommonelse
        {#1}{\v!links,\v!rechts,\v!inlinker,\v!inrechter,\v!inmarge}
        {\global\parfloattrue}
        {\global\parfloatfalse}%
     \ifbinnenkolommen
       \global\parfloatfalse
     \fi
     \edef\@@kjkjbreedte  {\getvalue{\??kj#4\c!breedte}}%
     \def \@@kjkjtussen   {\getvalue{\??kj#4\c!tussen}}%  geen \edef
     \edef\@@kjkjplaats   {\getvalue{\??kj#4\c!plaats}}%
     \edef\@@kjkjuitlijnen{\getvalue{\??kj#4\c!uitlijnen}}%
     \edef\@@flflplaats   {\getvalue{\??fl#4\c!plaats}}%
     \ifparfloat
       \dosetparfloat{#1}{#2}{#3}{#4}%
     \else
       \dosetpagfloat{#1}{#2}{#3}{#4}%
     \fi
     \setbox\floatbox=\hbox{\black\box\floatbox}%
     \global\floatheight=\ht\floatbox
     \global\advance\floatheight by \dp\floatbox
     \global\floatwidth=\wd\floatbox
     \global\advance\totalnoffloats by 1
     \doifnotinset{\v!marge}{#1} % gaat namelijk nog fout
       {\setbox\floatbox=\vbox
          {\parindent\!!zeropoint
           \ifvoorlopig
             \inleftmargin{\framed{\infofont\the\totalnoffloats}}%
           \fi
           \box\floatbox}}%
     \wd\floatbox=\floatwidth
     \dimen0=\floatheight
     \advance\dimen0 by \lineheight
     \ifdim\dimen0<\teksthoogte
     \else
       \global\floatheight=\teksthoogte
       \global\advance\floatheight by -\lineheight
       \ht\floatbox=\floatheight
       \dp\floatbox=\!!zeropoint
       \showmessage{\m!floatblocks}{10}{\the\totalnoffloats}%
     \fi
   \fi}

\def\dogetfloatbox#1%
  {\ifvisible
     \let\next\relax % ivm eetex
     \processfirstactioninset
       [#1]
       [    \v!hier=>\def\next{\global\floatsonpagetrue\someelsefloat[#1]},
         \v!forceer=>\def\next{\global\floatsonpagetrue\somefixdfloat[#1]},
           \v!links=>\def\next{\global\floatsonpagetrue\somesidefloat[#1]\presetindentation},
          \v!rechts=>\def\next{\global\floatsonpagetrue\somesidefloat[#1]},
           \v!tekst=>\def\next{\global\floatsonpagetrue\sometextfloat[#1]},
           \v!boven=>\def\next{\someelsefloat[#1]\nonoindentation}, % !
           \v!onder=>\def\next{\global\floatsonpagetrue\someelsefloat[#1]},
           \v!marge=>\def\next{\somenextfloat[#1]\nonoindentation}, % !
          \v!pagina=>\def\next{\global\floatsonpagetrue\somepagefloat[#1]},
           \v!naast=>\def\next{\global\floatsonpagetrue\somefacefloat[#1]},
         \v!inmarge=>\def\next{\global\floatsonpagetrue\somesidefloat[#1]},
        \v!inlinker=>\def\next{\global\floatsonpagetrue\somesidefloat[#1]},
       \v!inrechter=>\def\next{\global\floatsonpagetrue\somesidefloat[#1]},
         \s!default=>\def\next{\global\floatsonpagetrue\someelsefloat[\v!hier,#1]},
         \s!unknown=>\def\next{\global\floatsonpagetrue\someelsefloat[\v!hier,#1]}]%
     \next
   \fi}

\long\def\dofloat#1#2#3#4%
  {\dosetfloatbox{#1}{#2}{#3}{#4}%
   \dogetfloatbox{#1}}%

\long\def\docompletefloat#1#2#3#4#5#6#7%
  {\flushsidefloats
   \calculatefloatskips
   \bgroup
   \global\setbox\floatbox=\vbox{#7}%
   \dimen0=\ht\floatbox
   \advance\dimen0 by \dp\floatbox
   \ifdim\dimen0=\!!zeropoint\relax
     \showmessage{\m!floatblocks}{11}{}%
     \global\setbox\floatbox=\vbox{\getvalue{\e!lege#3}}%
   \fi
   \ConvertToConstant\doifelse{#6}{\v!geen}
     {\global\setbox\floatbox=\vbox
        {\unvbox\floatbox
         \vss % gets rid of the depth
         \rawpagereference{\s!flt}{#2}}%
      \egroup\dofloat{#4}{}{#6}{#1}}
     {\doglobal\convertargument#6\to\asciititle % \asciititle is global 
      \doifelsevalue{\??kj#1\c!nummer}{\v!ja}
        {\verhoognummer[#1]%
         \maakhetnummer[#1]%
         \global\setbox\floatbox=\vbox
            {\unvbox\floatbox % no \vss, keep the depth
             \dofloatreference
             \redofloatorder{#1}%
             \rawreference{\s!flt}{#2}{{\hetnummer}{\asciititle}}%
             \doschrijfnaarlijst{#3}{\hetnummer}{#6}{#3}}%
         \egroup\dofloat{#4}{\labeltexts{#5}{\hetnummer}}{#6}{#1}}
        {\global\setbox\floatbox=\vbox
           {\unvbox\floatbox % no \vss, keep the depth
            \rawreference{\s!flt}{#2}{{}{\asciititle}}}%
         \egroup\dofloat{#4}{}{#6}{#1}}}%
   \global\insidefloatfalse}

\def\plaatsvolledig#1#2#3#4%   kop, ref, tit, do
  {#1[#2]{#3}%
   #4%
   \pagina[\v!ja]}

\definieernummer
  [\??si]
  [\c!wijze=\v!per\v!tekst,
   \c!conversie=\@@siconversie]

\def\stelplaatsbloksplitsenin%
  {\dodoubleargument\getparameters[\??si]}

% ook (continued)

\def\dosplitsplaatsblok[#1]#2% nog dubbele refs
  {\ifbinnenkolommen         % tzt ook nog figuren splitten
     % not yet supported
   \else
     \bgroup
     \insidefloattrue
     \getparameters[\??si][#1]%
     \resetnummer[\??si]%
     \def\floatcaptionsuffix{\nummer[\??si]}%
     \TABLEcaptionheight=\@@siregels\lineheight
\def\docomplexpagina[##1]{\goodbreak}%
     \dowithnextbox
       {\forgetall
        \mindermeldingen
        \doloop
          {\setbox2\vsplit\nextbox to \lineheight
           \setbox2=\vbox{\unvbox2}
           \ifdim\ht2>\lineheight
             \verhoognummer[\??si]%
             \ifnum\ruwenummer[\??si]=1 \ifdim\ht\nextbox=\!!zeropoint
               \let\floatcaptionsuffix=\empty
             \fi\fi
             \bgroup
             #2{\unvbox2}
             \egroup
             \ifdim\ht\nextbox>\!!zeropoint
               \pagina
               \verlaagnummer[\floatcaptionnumber]%
             \fi
           \fi
           \ifdim\ht\nextbox>\!!zeropoint\else
             \expandafter\exitloop
           \fi}%
        \egroup}
     \vbox
   \fi}

\def\splitsplaatsblok%
  {\dosingleempty\dosplitsplaatsblok}

\newbox\facingbox
\newbox\facingpage

\newif\iffacingpages \facingpagesfalse

\def\shipoutfacingpage%
  {\iffacingpages
     \ifnum\realpageno>1
       \bgroup
       \pagebodyornamentsfalse
       \setbox\facingpage=\vbox to \zethoogte
         {\unvbox\facingpage\vfil}%
       \myshipout{\buildpagebody\box\facingpage}%
       \egroup
     \else
       \global\setbox\facingpage=\box\voidb@x
     \fi
   \fi}

\def\naastpagina%
  {\shipoutfacingpage}

\def\facefloat%               redefined
  {\startnaast\box\floatbox\stopnaast}

\def\startnaast% beter: \dowithnextbox
  {\iffacingpages
     \global\setbox\facingbox=\vbox
       \bgroup
       \hsize=\zetbreedte
   \else
     \def\next{\gobbleuntil\stopnaast}%
     \expandafter\next
   \fi}

\def\stopnaast%
  {\egroup
   \global\setbox\facingpage=\vbox
     {\ifvoid\facingpage
        \vskip\openstrutdepth % \dp\strutbox
      \else
        \unvbox\facingpage
      \fi
      \box\facingbox
      \blanko}}

\def\dostelnaastplaatsenin[#1]%
  {\getparameters[\??np][#1]%
   \doifelse{\@@npstatus}{\v!start}
     {\global\facingpagestrue}
     {\global\facingpagesfalse}}

\def\stelnaastplaatsenin%
  {\dosingleargument\dostelnaastplaatsenin}

% Don't use \@@mawhatevercommand directly, use \getvalue instead.

\newif\ifnewbackground
\newif\ifsomebackground

\newbox\leftbackground
\newbox\rightbackground

\def\ifsomebackgroundfound#1%
  {\edef\!!stringe{\??ma#1}%
   \doifelsevaluenothing{\!!stringe\c!achtergrond   }
  {\doifelsevaluenothing{\!!stringe\c!voorgrondkleur}
         {\doifelsevalue{\!!stringe\c!kader       }\v!aan\!!doneatrue
         {\doifelsevalue{\!!stringe\c!linkerkader }\v!aan\!!doneatrue
         {\doifelsevalue{\!!stringe\c!rechterkader}\v!aan\!!doneatrue
         {\doifelsevalue{\!!stringe\c!bovenkader  }\v!aan\!!doneatrue
         {\doifelsevalue{\!!stringe\c!onderkader  }\v!aan\!!doneatrue
                                                         \!!doneafalse}}}}}
                                                         \!!doneatrue}
                                                         \!!doneatrue
   \if!!donea}

\def\doaddpagebackground#1#2% 
  {\ifsomebackgroundfound{#1}%
     \edef\setpagebackgrounddepth%
       {\dp#2=\the\dp#2}%
     \setbox#2=\vbox\localframed
       [\??ma#1]
       [\c!strut=\v!nee,\c!offset=\v!overlay,
        \c!breedte=\papierbreedte,\c!hoogte=\papierhoogte]
       {\dp#2=\!!zeropoint\box#2}%
     \setpagebackgrounddepth
   \fi}

\def\addpagebackground#1%
  {\doifbothsidesoverruled
     \doaddpagebackground{\v!rechterpagina}{#1}%
   \orsideone
     \doaddpagebackground{\v!rechterpagina}{#1}%
   \orsidetwo
     \doaddpagebackground{\v!linkerpagina}{#1}%
   \od
   \doaddpagebackground{\v!pagina}{#1}}

\let\pagebackgroundhoffset = \!!zeropoint
\let\pagebackgroundvoffset = \!!zeropoint
\let\pagebackgrounddepth   = \!!zeropoint

% #1 = breedte
% #2 = hoogte
% #3 = pos
% #4 = pos

%D Better (no zero dimension mp figs) and faster:  

\def\dododopagebodybackground#1#2#3#4%
  {\ifsomebackgroundfound{#3#4}%
     \ifdim#1>\!!zeropoint\relax
       \ifdim#2>\!!zeropoint\relax
         \localframed
           [\??ma#3#4]
           [\c!breedte=#1,\c!hoogte=#2,\c!offset=\v!overlay]
           {\getvalue{\??ma#3#4\c!commando}}% {\hsize=#1\vsize=#2....}
       \fi
     \fi
   \else
     \hskip#1%
   \fi}

\def\dodopagebodybackground#1#2%
  {\setbox0=\vbox to #2
     \bgroup\hbox\bgroup
       \swapmargins
       \goleftonpage
       \dododopagebodybackground\linkerrandbreedte#2#1\v!linkerrand
       \hskip\linkerrandafstand
      %\hskip\pageseparation
       \dododopagebodybackground\linkermargebreedte#2#1\v!linkermarge
       \hskip\linkermargeafstand
       \dododopagebodybackground\zetbreedte#2#1\v!tekst
       \hskip\rechtermargeafstand
       \dododopagebodybackground\rechtermargebreedte#2#1\v!rechtermarge
      %\hskip\pageseparation
       \hskip\rechterrandafstand
       \dododopagebodybackground\rechterrandbreedte#2#1\v!rechterrand
     \egroup\egroup
   \wd0=\!!zeropoint\relax
   \box0\relax}

\def\setbackgroundbox#1#2%
  {\global\setbox#1=\vbox
     {\offinterlineskip
      \mindermeldingen
      \calculatereducedvsizes
      #2\relax
      \vskip-\bovenhoogte
      \vskip-\bovenafstand
      \dodopagebodybackground\v!boven\bovenhoogte
      \vskip\bovenafstand
      \dodopagebodybackground\v!hoofd\hoofdhoogte
      \vskip\hoofdafstand
      \dodopagebodybackground\v!tekst\teksthoogte
      \vskip\voetafstand
      \dodopagebodybackground\v!voet\voethoogte
      \vskip\onderafstand
      \dodopagebodybackground\v!onder\onderhoogte
      \vfilll}%
  \smashbox#1}

\def\setbackgroundboxes%
  {\ifsomebackground\ifnewbackground
     \showmessage{\m!layouts}{8}{}%
     \setbackgroundbox\leftbackground\relax
     \ifdubbelzijdig
       \setbackgroundbox\rightbackground\doswapmargins
     \fi
    %\global\newbackgroundfalse
     \doifnot{\@@mastatus}{\v!herhaal}{\global\newbackgroundfalse}%
     \doifelsevaluenothing{\??ma\v!tekst\v!tekst\c!achtergrond}
       {\global\let\pagebackgroundhoffset=\!!zeropoint
        \global\let\pagebackgroundvoffset=\!!zeropoint
        \global\let\pagebackgrounddepth  =\!!zeropoint}
       {\bgroup
        \dimen0=\getvalue{\??ma\v!pagina\c!offset}%
        \doifnothing
          {\getvalue{\??ma\v!boven\v!tekst\c!achtergrond}%
           \getvalue{\??ma\v!onder\v!tekst\c!achtergrond}}
          {\xdef\pagebackgroundhoffset{\the\dimen0}}%
        \doifnothing
          {\getvalue{\??ma\v!tekst\v!rechterrand\c!achtergrond}%
           \getvalue{\??ma\v!tekst\v!linkerrand \c!achtergrond}}
          {\xdef\pagebackgroundvoffset{\the\dimen0}%
           \dimen0=\getvalue{\??ma\v!pagina\c!diepte}%
           \xdef\pagebackgrounddepth{\the\dimen0}}%
        \egroup}%
   \fi\fi}

\def\getbackgroundbox%
  {\ifsomebackground
     \setbackgroundboxes
     \startinteractie
     \doifmarginswapelse
       {\copy\leftbackground}
       {\copy\rightbackground}%
     \stopinteractie
   \fi}

% saves us hundreds of unused hash entries if not needed

%\def\docheckbackgrounddefinitions% allocates about 1000 hash-entries
%  {\doifdefined{\??ma\v!pagina\c!achtergrond}% skip first pass
%     {\def\dodocommando##1##2%
%        {\copylocalframed[\??ma##1##2][\??ma\v!pagina]%
%         \getparameters[\??ma##1##2]
%           [\c!achtergrond=,\c!kader=,\c!kleur=,\c!raster=,
%            \c!onderkader=,\c!bovenkader=,\c!linkerkader=,\c!rechterkader=]%
%         \copyparameters
%           [\??ma##1##2\c!kader][\??ma##1##2]
%           [\c!kleur,\c!raster]%
%         \copyparameters
%           [\??ma##1##2\c!achtergrond][\??ma##1##2]
%           [\c!kleur,\c!raster]}%
%      \def\docommando##1%
%        {\dodocommando##1\v!linkerrand   \dodocommando##1\v!linkermarge
%         \dodocommando##1\v!tekst
%         \dodocommando##1\v!rechtermarge \dodocommando##1\v!rechterrand}%
%      \docommando\v!boven \docommando\v!hoofd
%      \docommando\v!tekst
%      \docommando\v!voet  \docommando\v!onder
%      \def\docheckbackgrounddefinitions%
%        {\global\somebackgroundtrue}}}

\def\dostelachtergrondenin[#1][#2][#3]%
  {\ifthirdargument
     \global\somebackgroundtrue
     \def\docommando##1%
       {\doifinsetelse{##1}{\v!papier,\v!pagina,\v!linkerpagina,\v!rechterpagina}
          {\getparameters[\??ma##1][#3]%
           }%\dosetpageseparation}
          {\def\dodocommando####1%
             {\getparameters[\??ma##1####1][#3]}%
           \processcommalist[#2]\dodocommando}}%
     \processcommalist[#1]\docommando
   \else\ifsecondargument
     \global\somebackgroundtrue
     \doifcommonelse{#1}{\v!tekst,\v!papier,\v!pagina,\v!linkerpagina,\v!rechterpagina}
       {\def\docommando##1%
          {\getparameters[\??ma##1][#2]}%
        \processcommalist[#1]\docommando
        }%\dosetpageseparation}
       {\stelachtergrondenin
          [#1]%
          [\v!linkerrand,\v!linkermarge,\v!tekst,\v!rechtermarge,\v!rechterrand]%
          [#2]}%
   \else\iffirstargument
     \getparameters[\??ma][#1]%
   \fi\fi\fi
   \doifelse{\@@mastatus}{\v!stop}
     {\global\newbackgroundfalse}
     {\global\newbackgroundtrue}}
 
\def\stelachtergrondenin%
  {\dotripleempty\dostelachtergrondenin}

% a lot of setups, including short ones

\presetlocalframed [\??ma\v!papier] 
\presetlocalframed [\??ma\v!pagina]
\presetlocalframed [\??ma\v!linkerpagina]
\presetlocalframed [\??ma\v!rechterpagina]

\copyparameters
  [\??ma\v!papier\c!kader][\??ma\v!pagina]
  [\c!offset,\c!diepte,\c!straal,\c!hoek,\c!kleur,\c!raster]

\copyparameters
  [\??ma\v!papier\c!achtergrond][\??ma\v!pagina]
  [\c!offset,\c!diepte,\c!straal,\c!hoek,\c!kleur,\c!raster]

\copyparameters
  [\??ma\v!pagina\c!kader][\??ma\v!pagina]
  [\c!offset,\c!diepte,\c!straal,\c!hoek,\c!kleur,\c!raster]

\copyparameters
  [\??ma\v!pagina\c!achtergrond][\??ma\v!pagina]
  [\c!offset,\c!diepte,\c!straal,\c!hoek,\c!kleur,\c!raster]

\copyparameters
  [\??ma\v!linkerpagina\c!kader][\??ma\v!linkerpagina]
  [\c!offset,\c!diepte,\c!straal,\c!hoek,\c!kleur,\c!raster]

\copyparameters
  [\??ma\v!linkerpagina\c!achtergrond][\??ma\v!linkerpagina]
  [\c!offset,\c!diepte,\c!straal,\c!hoek,\c!kleur,\c!raster]

\copyparameters
  [\??ma\v!rechterpagina\c!kader][\??ma\v!rechterpagina]
  [\c!offset,\c!diepte,\c!straal,\c!hoek,\c!kleur,\c!raster]

\copyparameters
  [\??ma\v!rechterpagina\c!achtergrond][\??ma\v!rechterpagina]
  [\c!offset,\c!diepte,\c!straal,\c!hoek,\c!kleur,\c!raster]

\def\dodocommando#1#2%
  {\copylocalframed
     [\??ma#1#2][\??ma\v!pagina]%
   \getparameters
     [\??ma#1#2]
     [\c!achtergrond=,\c!kader=,\c!kleur=,\c!raster=\@@rsraster,
      \c!onderkader=,\c!bovenkader=,\c!linkerkader=,\c!rechterkader=]%
   \setvalue{\??ma#1#2\c!kleur}{\getvalue{\??ma\v!pagina\c!kleur }}%
   \setvalue{\??ma#1#2\c!raster}{\getvalue{\??ma\v!pagina\c!raster}}%
   \setvalue{\??ma#1#2\c!kaderkleur}{\getvalue{\??ma#1#2\c!kleur}}%
   \setvalue{\??ma#1#2\c!achtergrondkleur}{\getvalue{\??ma#1#2\c!kleur}}%
   \setvalue{\??ma#1#2\c!achtergrondraster}{\getvalue{\??ma#1#2\c!raster}}}

\dodocommando\v!tekst\empty

\def\docommando#1%
  {\dodocommando#1\v!linkerrand   
   \dodocommando#1\v!linkermarge
   \dodocommando#1\v!tekst
   \dodocommando#1\v!rechtermarge 
   \dodocommando#1\v!rechterrand}

\docommando\v!boven 
\docommando\v!hoofd
\docommando\v!tekst
\docommando\v!voet  
\docommando\v!onder

\let\dodocommando\relax \let\docommando\relax

\def\currentproject     {}
\def\currentproduct     {}
\def\currentenvironment {}
\def\currentcomponent   {}

\def\loadedfiles        {}
\def\processedfiles     {}

\let\geenfilesmeer=\relax

\newcounter\textlevel
\newcounter\fileprocesslevel

\setvalue{\c!file::0}{\jobname}

\def\processedfile% is used in styles, don't change ! 
  {\getvalue{\c!file::\fileprocesslevel}}

%\def\processfile#1%
%  {\doglobal\increment\fileprocesslevel
%   \setxvalue{\c!file::\fileprocesslevel}{#1}%
%   \@EA\doglobal\@EA\addtocommalist\@EA{#1}\processedfiles
%   \readlocfile{#1}{}{}
%   \doglobal\decrement\fileprocesslevel}

\def\processlocalfile#1#2%
  {\doglobal\increment\fileprocesslevel
   \setxvalue{\c!file::\fileprocesslevel}{#2}%
   \@EA\doglobal\@EA\addtocommalist\@EA{#2}\processedfiles
   #1{#2}{}{}% #1=\readlocfile|\readsetfile{dir} #2=filename
   \doglobal\decrement\fileprocesslevel}

\def\processfile#1%
  {\relax
   \ifx\allinputpaths\empty
     \def\next{\processlocalfile\readlocfile}%
   \else
     \let\filepath\empty
     \def\docommando##1%
       {\doiffileelse{\pathplusfile{##1}{#1}}
          {\donetrue\def\filepath{##1}}
          {\donefalse}%
        \ifdone\expandafter\quitcommalist\fi}%
     \doifparentfileelse{#1} % new 
       {\processcommacommand  [\allinputpaths]\docommando}
       {\processcommacommand[.,\allinputpaths]\docommando}%
     \ifx\filepath\empty
       \def\next{\processlocalfile\readlocfile}% fall back ../../..
     \else
       \def\next{\processlocalfile{\readsetfile\filepath}}% file found
     \fi
   \fi
   \next{#1}}

\let\allinputpaths\empty

\def\usepath[#1]%
  {\def\docommando##1%
     {\doifelse{##1}{\v!reset}
        {\let\allinputpaths\empty}
        {\addtocommalist{##1}\allinputpaths}}%
   \processcommalist[#1]\docommando}

\def\registreerfileinfo[#1#2]#3% geen \showmessage ? 
  {\writestatus{\m!systems}{#1#2 file #3 at line \the\inputlineno}%
   \immediatewriteutility{f #1 {#3}}}

\doifundefined{preloadfonts}    {\let\preloadfonts=\relax}
\doifundefined{preloadspecials} {\let\preloadspecials=\relax}

\def\loadsystemfiles%
  {\readsysfile{\f!newfilename}
     {\showmessage{\m!systems}{2}{\f!newfilename}}{}%
   \readsysfile{\f!oldfilename}
     {\showmessage{\m!systems}{2}{\f!oldfilename}}{}%
   \readsysfile{\f!filfilename}
     {\showmessage{\m!systems}{2}{\f!filfilename}}{}%
   \readsysfile{\f!sysfilename}
     {\showmessage{\m!systems}{2}{\f!sysfilename}}{}}

% test \@@svgebied

\def\loadallsystemfiles#1%
  {\ifx\@@svgebied\empty
     \readsysfile{#1}{\showmessage{\m!systems}{2}{#1}}{}%
   \else
     \def\doloadsystemfile##1%
       {\readsetfile{##1}{#1}{\showmessage{\m!systems}{2}{#1}}{}}%
     \processcommacommand[\@@svgebied]\doloadsystemfile
   \fi}

\def\loadsystemfiles%
  {\readsysfile{\f!newfilename}
     {\showmessage{\m!systems}{2}{\f!newfilename}}{}%
   \readsysfile{\f!oldfilename}
     {\showmessage{\m!systems}{2}{\f!oldfilename}}{}%
   \loadallsystemfiles\f!filfilename
   \loadallsystemfiles\f!sysfilename}

%D Loading of \type {cont-usr.tex} (edited by the user) 
%D and \type {cont-fmt.tex} (generated by texexec). 

\def\loaduserspecifications
  {\readsysfile{\f!usrfilename}
     {\showmessage{\m!systems}{2}{\f!usrfilename}}{}%
   \readjobfile{\f!fmtfilename}
     {\showmessage{\m!systems}{2}{\f!fmtfilename}}{}}

%D We don't want multiple jobfiles to interfere. 

\bgroup
\catcode`\%=\@@other
\xdef\texcommentsymbol{%}
\egroup

\def\loadoptionfile% 
  {\readjobfile{\jobname.\f!optionextension}
     {\showmessage{\m!systems}{2}{\jobname.\f!optionextension}}%
     {}}

% \newevery \everyjob \EveryJob
% \appendtoks ... \to \everyjob

\prependtoks \showcontextbanner   \to \everyjob

\appendtoks \loadsystemfiles      \to \everyjob
\appendtoks \preloadfonts         \to \everyjob
\appendtoks \settopskip           \to \everyjob
\appendtoks \preloadlanguages     \to \everyjob
\appendtoks \preloadspecials      \to \everyjob
\appendtoks \openspecialfile      \to \everyjob
%appendtoks \checkutilityfile     \to \everyjob % obsolete 
\appendtoks \openutilities        \to \everyjob
\appendtoks \loadoptionfile       \to \everyjob
\appendtoks \loadtwopassdata      \to \everyjob
\appendtoks \setupfootnotes       \to \everyjob % depends on bodyfont
\appendtoks \initializeMPgraphics \to \everyjob % after loading system files

\appendtoks \pagina[\v!laatste] \pagina           \to \everybye
\appendtoks \ifarrangingpages\poparrangedpages\fi \to \everybye
\appendtoks \registreerfileinfo[end]{\jobname}    \to \everybye

\appendtoks \savenofpages    \to \everybye
\appendtoks \savenofsubpages \to \everybye

\appendtoks \closeutilities    \to \everygoodbye
\appendtoks \stopcopyingblocks \to \everygoodbye
\appendtoks \closespecialfile  \to \everygoodbye

\appendtoks \checkreferences \to \everystarttext % nieuw 4-12-1999 

\def\doateverystarttext%
  {\the\everystarttext
   \global\let\doateverystarttext\relax}

\def\starttekst%
  {\doateverystarttext
   \ifnum\textlevel=0\relax
    \registreerfileinfo[begin]{\jobname}%
    \startcopyingblocks
   \fi
   \doglobal\increment\textlevel\relax}

\def\stoptekst%
  {\doglobal\decrement\textlevel\relax
   \ifnum\textlevel>0 \else
     \the\everystoptext
    %\the\everybye            % 
    %\the\everygoodbye        % == \end (new)
    %\expandafter\normalend   %
     \expandafter\end
   \fi}

\let\normalend=\end

\def\end%
  {\ifnum\textlevel>0 \else
     \the\everybye
     \the\everygoodbye
     \global\everygoodbye\emptytoks % rather unneeded
     \global\everybye\emptytoks     % but for sure 
     \expandafter\normalend
   \fi}

\def\doexecutefileonce#1%
  {\beforesplitstring#1\at.\to\currentfile
   \ExpandBothAfter\doifnotinset{\currentfile}{\loadedfiles}%
     {\ExpandFirstAfter\addtocommalist{\currentfile}\loadedfiles
      \doexecutefile{#1}}}

\def\doexecutefile#1%
  {\registreerfileinfo[begin]{#1}
   \processfile{#1}%
   \registreerfileinfo[end]{#1}}

\def\donotexecutefile#1%
  {}

\def\verwerkfile#1 %
  {\doexecutefile{#1}}

\def\omgeving #1 % at outermost level only
  {\def\startomgeving ##1 {}%
   \let\stopomgeving=\relax
   \startreadingfile
   \processfile{#1}% \readlocfile{#1}{}{}%
   \stopreadingfile}

\newcounter\filelevel

\def\!!donextlevel#1#2#3#4#5#6\\%
  {\beforesplitstring#6\at.\to#1
   \ifnum\filelevel=0\relax
     \starttekst
     \def\projekt   ##1 {#2{##1}}%
     \def\omgeving  ##1 {#3{##1}}%
     \def\produkt   ##1 {#4{##1}}%
     \def\onderdeel ##1 {#5{##1}}%
   \fi
   \increment\filelevel\relax
   \ExpandFirstAfter\addtocommalist{#1}\loadedfiles}

\def\doprevlevel%
  {\ifnum\filelevel=1
     \expandafter\stoptekst
   \else
     \decrement\filelevel\relax
     \expandafter\endinput
   \fi}

\def\startprojekt #1 %
  {\!!donextlevel\currentproject
     \donotexecutefile\doexecutefileonce
     \doexecutefileonce\doexecutefile#1\\}

\def\stopprojekt%
  {\doprevlevel}

\def\startprodukt #1 %
  {\doateverystarttext
   \!!donextlevel\currentproduct
     \doexecutefileonce\doexecutefileonce
     \donotexecutefile\doexecutefile#1\\}

\def\stopprodukt%
  {\doprevlevel}

\def\startonderdeel #1 %
  {\doateverystarttext
   \!!donextlevel\currentcomponent
     \doexecutefileonce\doexecutefileonce
     \donotexecutefile\doexecutefile#1\\}

\def\stoponderdeel%
  {\doprevlevel}

\def\startomgeving #1 %
  {\!!donextlevel\currentenvironment
     \donotexecutefile\doexecutefileonce
     \donotexecutefile\donotexecutefile#1\\}

\def\stopomgeving%
  {\doprevlevel}

\long\def\skipdeelomgeving#1\stopdeelomgeving%
  {}

\def\startdeelomgeving[#1]%
  {\def\partialenvironments{}%
   \def\docommando##1%
     {\beforesplitstring##1\at.\to\someevironment
      \ExpandFirstAfter\addtocommalist{\someevironment}\partialenvironments}%
   \processcommalist[#1]\docommando
   \ExpandBothAfter\doifcommonelse
       {\currentproject,\currentproduct,
        \currentcomponent,\currentenvironment}
       {\partialenvironments}
     {\let\stopdeelomgeving=\relax
      \let\next=\relax}
     {\let\next=\skipdeelomgeving}%
   \next}

\def\startproduct{\startprodukt}
\def\stopproduct {\stopprodukt}
\def\startproject{\startprojekt}
\def\stopproject {\stopprojekt}

\def\project{\projekt}
\def\product{\produkt}

\def\deelomgeving #1 %
  {\doexecutefileonce{#1}}

\expanded
  {\long\noexpand\def\csname\e!start\e!instellingen\endcsname##1 ##2\csname\e!stop\e!instellingen\endcsname%
     {\noexpand\long\noexpand\setvalue{\??su##1}{##2}}}

\long\def\startsetups#1 #2\stopsetups% for international purposes
  {\long\setvalue{\??su#1}{#2}}

\def\dodosetups#1%
  {\getvalue{\??su#1}}

\def\dosetups[#1]%
  {\iffirstargument
     \dodosetups{#1}%
   \else
     \expandafter\dodosetups
   \fi}

\def\setups%
  {\dosingleargument\dosetups}

\newif\ifvoorlopig
\voorlopigfalse

\newif\ifconcept
\conceptfalse

\def\infofont%
  {\getvalue{7pttttf}}

\edef\utilityversion{1998.07.07} % was: 1996.03.15  % status variables
\edef\utilityversion{1998.12.20} % was: 1998.07.07  % index attributes

\def\doplaatsversieaanduiding#1#2%
  {\doifsomething{#2}
     {\@EA\convertargument#2\to\ascii
      \ #1: \ascii\
      \!!doneatrue}}

\def\plaatsversieaanduiding% nog engels maken
  {\ifvoorlopig
     \vskip\!!sixpoint
     \hbox to \zetbreedte
       {\infofont
        \getmessage\m!systems{27}: \currentdate\
        \doplaatsversieaanduiding{Project}\currentproject
        \doplaatsversieaanduiding{Produkt}\currentproduct
        \doplaatsversieaanduiding{Onderdeel}\currentcomponent
        \if!!donea\else\ File: \jobname\fi
        \hss\reportpagedimensions}%
   \fi
   \ifconcept
     \vskip\!!sixpoint
     \hbox to \zetbreedte
       {\infofont
        Concept: \currentdate
        \hss\reportpagedimensions}%
   \fi}

% tot hier

\def\doversie[#1]%
  {\voorlopigfalse
   \conceptfalse
   \overfullrule=\!!zeropoint
   \processaction
     [#1]
     [\v!voorlopig=>\voorlopigtrue
                    \overfullrule=5pt,
        \v!concept=>\concepttrue]}

\def\versie%
  {\dosingleargument\doversie}

% this will be inserts some day

\newbox\referentieinfobox
\newbox\registerinfobox
\newbox\floatinfobox

\def\dotestinfo#1#2#3%
  {\ifvoorlopig\ifinpagebody\else
     \begingroup
       \convertargument#3\to\ascii
       \xdef\extratestinfo%
         {#2 \ascii}%
       \gdef\totaltestinfo%
         {\global\setbox#1=\vbox
            {\unvbox#1\relax
             \hbox
               {\infofont
                \strut
                \expandafter\doboundtext\expandafter
                   {\extratestinfo}
                   {12em}
                   {..}%
                \quad}}}%
     \endgroup
     \ifinner
       \aftergroup\totaltestinfo
     \else
       \totaltestinfo
     \fi
   \fi\fi}

\def\referentieinfo%
 {\dotestinfo\referentieinfobox}

\def\registerinfo%
 {\dotestinfo\registerinfobox}

\def\floatinfo%
 {\dotestinfo\floatinfobox}

\def\plaatstestinfo%
  {\setbox0=\vbox to \teksthoogte
     {\forgetall
      \infofont
      \hsize10em
      \ifvoid\floatinfobox\else
        \strut \getmessage\m!systems{24}
        \vskip\!!sixpoint
        \unvbox\floatinfobox
        \vskip\!!twelvepoint
      \fi
      \ifvoid\referentieinfobox\else
        \strut \getmessage\m!systems{25}
        \vskip\!!sixpoint
        \unvbox\referentieinfobox
        \vskip\!!twelvepoint
      \fi
      \ifvoid\registerinfobox\else
        \strut \getmessage\m!systems{26}
        \vskip\!!sixpoint
        \unvbox\registerinfobox
      \fi
      \vss}%
   \wd0=\!!zeropoint
   \box0\relax}

% \docommando kan niet worden gebruikt omdat deze macro
%  soms lokaal wordt gebruikt

% te zijner tijd:
%
% \definevariable {pc}  % ProtectedCommand
%
% \def\executeprotected#1%
%   {\csname\??pc\string#1\endcsname}
%
% \def\defineprotected#1#2%
%   {\expandafter\def\csname\??pc\string#2\endcsname}
%
% \def\defineunprotected#1%
%   {\def#1}
%
% \def\doprotected%
%   {\ifx\next\define
%      \let\next=\defineprotected
%    \else
%      \let\next=\executeprotected
%    \fi
%    \next}
%
% \def\unexpanded%
%   {\futurelet\next\doprotected}
%
% \unexpanded\define\ziezo{ziezo}
%
% \unexpanded\ziezo

\def\complexdefinieer[#1]#2#3%
  {\ifx#2\undefined
   \else
     \showmessage{\m!systems}{4}{\string#2}%
   \fi
   \ifcase0#1\def#2{#3}%
   \or\def#2##1{#3}%
   \or\def#2##1##2{#3}%
   \or\def#2##1##2##3{#3}%
   \or\def#2##1##2##3##4{#3}%
   \or\def#2##1##2##3##4##5{#3}%
   \or\def#2##1##2##3##4##5##6{#3}%
   \or\def#2##1##2##3##4##5##6##7{#3}%
   \or\def#2##1##2##3##4##5##6##7##8{#3}%
   \or\def#2##1##2##3##4##5##6##7##8##9{#3}%
   \else\def#2{#3}%
   \fi}

\definecomplexorsimpleempty\definieer

\unexpanded\def\naam#1%
  {\getvalue{#1}}

\def\gebruikcommandos#1%
  {\bgroup
   \def\docommando##1%
     {\setbox0=\hbox{\getvalue{\string##1}##1}}%
   \processcommalist[#1]\docommando
   \egroup}

\def\complexstart[#1]{\bgroup\getvalue{\e!start#1}}
\def\complexstop [#1]{\getvalue{\e!stop #1}\egroup}

\def\simplestart{\bgroup}
\def\simplestop {\egroup}

\definecomplexorsimple\start
\definecomplexorsimple\stop

\def\dodefinieerstartstop[#1][#2]%
  {\getparameters
     [\??be#1]
     [\c!voor=,
      \c!na=,
      \c!commandos=,
      \c!letter=,
      #2]%
%   \setvalue{\e!stel#1\e!in}[##1]%
%     {\dodoubleargument\getparameters[\??be##1]}%
   \unexpanded\setvalue{#1}%
     {\groupedcommand
        {\getvalue{\??be#1\c!commandos}%
         \dostartattributes{\??be#1}\c!letter\c!kleur}
        {\dostopattributes}}%
   \setvalue{\e!start#1}%
     {\getvalue{\??be#1\c!voor}%
      \bgroup
      \getvalue{\??be#1\c!commandos}%
      \dostartattributes{\??be#1}\c!letter\c!kleur{}}%
   \setvalue{\e!stop#1}%
     {\dostopattributes
      \egroup
      \getvalue{\??be#1\c!na}}}

\def\definieerstartstop%
  {\dodoubleargument\dodefinieerstartstop}

\def\stelstartstopin[#1]%
  {\dodoubleargument\getparameters[\??be#1]}

% gejat van Knuth (zie \copyright, p356)

\def\omcirkeld#1%
  {{\ooalign{\hfil\raise0.07ex\hbox{{\tfx#1}}\hfil\crcr\mathhexbox20D}}}

\def\copyright
  {\omcirkeld{c}}

\def\dosetupsystem[#1]%
  {\getparameters[\??sv][#1]%
   \setuprandomize[\@@svwillekeur]%
   \beforesplitstring\@@svresolutie\at dpi\to\@@svresolutie
   \let\outputresolution=\@@svresolutie}

\def\setupsystem%
  {\dosingleargument\dosetupsystem}

\def\setuprandomize[#1]%
  {\doifsomething{#1}
     {\bgroup
      \setrandomseed{-1}%
      \processaction
        [#1]
        [  \v!klein=>\divide\time  900, % 15   taco vragen hoe
          \v!middel=>\divide\time 1800, % 30   time werkt; nodig voor 
           \v!groot=>\divide\time 3600, % 60   random pos deadlock
         \v!normaal=>,
         \s!default=>,
         \s!unknown=>\time=#1]%    
      \nextrandom   
      \egroup}}

\def\outputfilename{\@@svfile} 

% Default-instellingen (verborgen)

\resetutilities

% Uitgestelde instellingen

\def\dooutput{\sidefloatoutput}           % redefinition of \dooutput

% Default-instellingen (zichtbaar)

\setupsystem
  [\c!gebied=,
   \c!resolutie=600dpi,
   \c!willekeur=,
   \c!file=\jobname,
   \c!korps=\normalizedlocalbodyfontsize] % of iets anders

% Pas op:
%
% Omdat er geen fonts geladen zijn kunnen we bij de maten geen
% em's gebruiken. Bij afstanden is dit geen probleem, omdat
% deze pas een rol spelen als er al een font geladen is.

\stellayoutin
  [             \c!kopwit=.08417508418\papierhoogte,  % .08333  2.5cm
                 \c!boven=\!!zeropoint,
          \c!bovenafstand=\!!zeropoint,
                 \c!hoofd=.06734006734\papierhoogte,  % .06667  2.0cm
          \c!hoofdafstand=\!!zeropoint,
                \c!hoogte=.84175084175\papierhoogte,  % .83333 25.0cm
           \c!voetafstand=\@@lyhoofdafstand,
                  \c!voet=.06734006734\papierhoogte,  % .06667  2.0cm
          \c!onderafstand=\@@lybovenafstand,
                 \c!onder=\!!zeropoint,
                \c!rugwit=.11904761905\papierbreedte, %         2.5cm
                  \c!rand=\!!zeropoint,
           \c!randafstand=\@@lymargeafstand,
                 \c!marge=.12649983170\papierbreedte, % snijwit-2*afstand
          \c!margeafstand=.02008341748\papierbreedte, %        12.0pt
            \c!linkerrand=\@@lyrand,
     \c!linkerrandafstand=\@@lyrandafstand,
           \c!linkermarge=\@@lymarge,
    \c!linkermargeafstand=\@@lymargeafstand,
               \c!breedte=.71428571429\papierbreedte, %        15.0cm
   \c!rechtermargeafstand=\@@lymargeafstand,
          \c!rechtermarge=\@@lymarge,
    \c!rechterrandafstand=\@@lyrandafstand,
           \c!rechterrand=\@@lyrand,
             \c!kopoffset=\!!zeropoint,
             \c!rugoffset=\!!zeropoint,
          \c!tekstbreedte=, % dangerous here \tekstbreedte
                \c!letter=,
             \c!markering=\v!uit,
                \c!plaats=, % \v!enkelzijdig, but empty is signal 
                \c!schaal=1,
                    \c!nx=1,
                    \c!ny=1,
                    \c!dx=\!!zeropoint,
                    \c!dy=\!!zeropoint,
                  \c!grid=\v!nee,
                \c!regels=,
               \c!snijwit=\!!zeropoint,
              \c!bodemwit=\!!zeropoint]

% instellingen hierop terugvallen, bijvoorbeeld de volgende:

\definieerpapierformaat [A0] [\c!breedte=841mm, \c!hoogte=1189mm]
\definieerpapierformaat [A1] [\c!breedte=594mm, \c!hoogte=841mm]
\definieerpapierformaat [A2] [\c!breedte=420mm, \c!hoogte=594mm]
\definieerpapierformaat [A3] [\c!breedte=297mm, \c!hoogte=420mm]
\definieerpapierformaat [A4] [\c!breedte=210mm, \c!hoogte=297mm]
\definieerpapierformaat [A5] [\c!breedte=148mm, \c!hoogte=210mm]
\definieerpapierformaat [A6] [\c!breedte=105mm, \c!hoogte=148mm]
\definieerpapierformaat [A7] [\c!breedte=74mm,  \c!hoogte=105mm]
\definieerpapierformaat [A8] [\c!breedte=52mm,  \c!hoogte=74mm]
\definieerpapierformaat [A9] [\c!breedte=37mm,  \c!hoogte=52mm]

\definieerpapierformaat [B0] [\c!breedte=1000mm,\c!hoogte=1414mm]
\definieerpapierformaat [B1] [\c!breedte=707mm, \c!hoogte=1000mm]
\definieerpapierformaat [B2] [\c!breedte=500mm, \c!hoogte=707mm]
\definieerpapierformaat [B3] [\c!breedte=354mm, \c!hoogte=500mm]
\definieerpapierformaat [B4] [\c!breedte=250mm, \c!hoogte=354mm]
\definieerpapierformaat [B5] [\c!breedte=177mm, \c!hoogte=250mm]
\definieerpapierformaat [B6] [\c!breedte=125mm, \c!hoogte=177mm]
\definieerpapierformaat [B7] [\c!breedte=88mm,  \c!hoogte=125mm]
\definieerpapierformaat [B8] [\c!breedte=63mm,  \c!hoogte=88mm]
\definieerpapierformaat [B9] [\c!breedte=44mm,  \c!hoogte=63mm]

\definieerpapierformaat [C0] [\c!breedte=917mm, \c!hoogte=1297mm]
\definieerpapierformaat [C1] [\c!breedte=649mm, \c!hoogte=917mm]
\definieerpapierformaat [C2] [\c!breedte=459mm, \c!hoogte=649mm]
\definieerpapierformaat [C3] [\c!breedte=324mm, \c!hoogte=459mm]
\definieerpapierformaat [C4] [\c!breedte=229mm, \c!hoogte=324mm]
\definieerpapierformaat [C5] [\c!breedte=162mm, \c!hoogte=229mm]
\definieerpapierformaat [C6] [\c!breedte=115mm, \c!hoogte=162mm]
\definieerpapierformaat [C7] [\c!breedte=81mm,  \c!hoogte=115mm]
\definieerpapierformaat [C8] [\c!breedte=57mm,  \c!hoogte=81mm]
\definieerpapierformaat [C9] [\c!breedte=40mm,  \c!hoogte=57mm]

\definieerpapierformaat [S3] [\c!breedte=300pt, \c!hoogte=225pt]
\definieerpapierformaat [S4] [\c!breedte=400pt, \c!hoogte=300pt]
\definieerpapierformaat [S5] [\c!breedte=500pt, \c!hoogte=375pt]
\definieerpapierformaat [S6] [\c!breedte=600pt, \c!hoogte=450pt]

\definieerpapierformaat [CD] [\c!breedte=120mm, \c!hoogte=120mm]

\definieerpapierformaat [letter]    [\c!breedte=8.5in,  \c!hoogte=11in]
\definieerpapierformaat [2*letter]  [\c!breedte=11in,   \c!hoogte=17in]
\definieerpapierformaat [legal]     [\c!breedte=8.5in,  \c!hoogte=14in]
\definieerpapierformaat [folio]     [\c!breedte=8.5in,  \c!hoogte=13in]
\definieerpapierformaat [executive] [\c!breedte=7.25in, \c!hoogte=10.5in]

\definieerpapierformaat [envelope 9]  [\c!breedte=8.88in, \c!hoogte=3.88in]
\definieerpapierformaat [envelope 10] [\c!breedte=9.5in,  \c!hoogte=4.13in]
\definieerpapierformaat [envelope 11] [\c!breedte=10.38in,\c!hoogte=4.5in]
\definieerpapierformaat [envelope 12] [\c!breedte=11.0in, \c!hoogte=4.75in]
\definieerpapierformaat [envelope 14] [\c!breedte=11.5in, \c!hoogte=5.0in]
\definieerpapierformaat [monarch]     [\c!breedte=7.5in,  \c!hoogte=3.88in]
\definieerpapierformaat [check]       [\c!breedte=8.58in, \c!hoogte=3.88in]
\definieerpapierformaat [DL]          [\c!breedte=220mm,  \c!hoogte=110mm]

% Let op: na \stellayoutin (omdat dit wordt aangeroepen).

\stelpapierformaatin
  [A4][A4]

\stelpapierformaatin
  [\c!boven=,
   \c!onder=\vss,
   \c!links=,
   \c!rechts=\hss]

\stelinterliniein
  [\c!hoogte=.72,
   \c!diepte=.28,
   \c!boven=1.0,
   \c!onder=0.4,
   \c!afstand=1pt,
   \c!regel=2.8ex]

\stelkolommenin
  [\c!n=2,
   \c!nboven=1,
   \c!commando=,
   \c!richting=\v!rechts,
   \c!lijn=\v!uit,
   \c!tolerantie=\v!soepel,
   \c!afstand=1.5\korpsgrootte, % influenced by switching
   \c!hoogte=,
   \c!balanceren=\v!ja,
   \c!uitlijnen=\v!tekst,
   \c!blanko={\v!regel,\v!vast},
   \c!optie=,
   \c!lijndikte=\linewidth,
   \c!offset=.5\korpsgrootte]

\stelhoofdtekstenin [\v!tekst] [] []
\stelhoofdtekstenin [\v!marge] [] []
\stelhoofdtekstenin [\v!rand]  [] []

\stelvoettekstenin  [\v!tekst] [] []
\stelvoettekstenin  [\v!marge] [] []
\stelvoettekstenin  [\v!rand]  [] []

\stelteksttekstenin [\v!tekst] [] []
\stelteksttekstenin [\v!marge] [] []
\stelteksttekstenin [\v!rand]  [] []

\stelondertekstenin [\v!tekst] [] []
\stelondertekstenin [\v!marge] [] []
\stelondertekstenin [\v!rand]  [] []

\stelboventekstenin [\v!tekst] [] []
\stelboventekstenin [\v!marge] [] []
\stelboventekstenin [\v!rand]  [] []

\stelhoofdin [\c!status=\v!normaal,\c!voor=,\c!na=]
\stelvoetin  [\c!status=\v!normaal,\c!voor=,\c!na=]
\steltekstin [\c!status=\v!normaal,\c!voor=,\c!na=]
\stelbovenin [\c!status=\v!normaal,\c!voor=,\c!na=]
\stelonderin [\c!status=\v!normaal,\c!voor=,\c!na=]

\stelhoofdin              [\c!na=\vss]
\steltekstin [\c!voor=\vss,\c!na=\vss]
\stelvoetin  [\c!voor=\vss]

\stelbovenin [\c!voor=\vss,\c!na=\vss]
\stelonderin [\c!voor=\vss,\c!na=\vss]

\stelhoofdin  % \get??tk#1#2#3 would save quite some 3K in fmt size  
  [\v!tekst]
  [\c!strut=\v!ja,
   \c!letter=,
   \c!kleur=,
   \c!linkertekst=,
   \c!middentekst=,
   \c!rechtertekst=,
   \c!kantlijntekst=,
   \c!margetekst=,
   \c!linkerletter=\getvalue{\??tk\v!hoofd\v!tekst\c!letter},
   \c!rechterletter=\getvalue{\??tk\v!hoofd\v!tekst\c!letter},
   \c!linkerkleur=\getvalue{\??tk\v!hoofd\v!tekst\c!kleur},
   \c!rechterkleur=\getvalue{\??tk\v!hoofd\v!tekst\c!kleur},
   \c!breedte=,
   \c!linkerbreedte=\getvalue{\??tk\v!hoofd\v!tekst\c!breedte},
   \c!rechterbreedte=\getvalue{\??tk\v!hoofd\v!tekst\c!breedte}]

\stelhoofdin
  [\v!marge]
  [\c!letter=,
   \c!kleur=,
   \c!linkertekst=,
   \c!middentekst=,
   \c!rechtertekst=,
   \c!kantlijntekst=,
   \c!margetekst=,
   \c!linkerletter=\getvalue{\??tk\v!hoofd\v!marge\c!letter},
   \c!rechterletter=\getvalue{\??tk\v!hoofd\v!marge\c!letter},
   \c!linkerkleur=\getvalue{\??tk\v!hoofd\v!marge\c!kleur},
   \c!rechterkleur=\getvalue{\??tk\v!hoofd\v!marge\c!kleur},
   \c!breedte=,
   \c!linkerbreedte=\getvalue{\??tk\v!hoofd\v!marge\c!breedte},
   \c!rechterbreedte=\getvalue{\??tk\v!hoofd\v!marge\c!breedte}]

\stelhoofdin
  [\v!rand]
  [\c!letter=,
   \c!kleur=,
   \c!linkertekst=,
   \c!middentekst=,
   \c!rechtertekst=,
   \c!kantlijntekst=,
   \c!margetekst=,
   \c!linkerletter=\getvalue{\??tk\v!hoofd\v!rand\c!letter},
   \c!rechterletter=\getvalue{\??tk\v!hoofd\v!rand\c!letter},
   \c!linkerkleur=\getvalue{\??tk\v!hoofd\v!rand\c!kleur},
   \c!rechterkleur=\getvalue{\??tk\v!hoofd\v!rand\c!kleur},
   \c!breedte=,
   \c!linkerbreedte=\getvalue{\??tk\v!hoofd\v!rand\c!breedte},
   \c!rechterbreedte=\getvalue{\??tk\v!hoofd\v!rand\c!breedte}]

\stelvoetin
  [\v!tekst]
  [\c!strut=\v!ja,
   \c!letter=,
   \c!kleur=,
   \c!linkertekst=,
   \c!middentekst=,
   \c!rechtertekst=,
   \c!kantlijntekst=,
   \c!margetekst=,
   \c!linkerletter=\getvalue{\??tk\v!voet\v!tekst\c!letter},
   \c!rechterletter=\getvalue{\??tk\v!voet\v!tekst\c!letter},
   \c!linkerkleur=\getvalue{\??tk\v!voet\v!tekst\c!kleur},
   \c!rechterkleur=\getvalue{\??tk\v!voet\v!tekst\c!kleur},
   \c!breedte=,
   \c!linkerbreedte=\getvalue{\??tk\v!voet\v!tekst\c!breedte},
   \c!rechterbreedte=\getvalue{\??tk\v!voet\v!tekst\c!breedte}]

\stelvoetin
  [\v!marge]
  [\c!letter=,
   \c!kleur=,
   \c!linkertekst=,
   \c!middentekst=,
   \c!rechtertekst=,
   \c!kantlijntekst=,
   \c!margetekst=,
   \c!linkerletter=\getvalue{\??tk\v!voet\v!marge\c!letter},
   \c!rechterletter=\getvalue{\??tk\v!voet\v!marge\c!letter},
   \c!linkerkleur=\getvalue{\??tk\v!voet\v!marge\c!kleur},
   \c!rechterkleur=\getvalue{\??tk\v!voet\v!marge\c!kleur},
   \c!breedte=,
   \c!linkerbreedte=\getvalue{\??tk\v!voet\v!marge\c!breedte},
   \c!rechterbreedte=\getvalue{\??tk\v!voet\v!marge\c!breedte}]

\stelvoetin
  [\v!rand]
  [\c!letter=,
   \c!kleur=,
   \c!linkertekst=,
   \c!middentekst=,
   \c!rechtertekst=,
   \c!kantlijntekst=,
   \c!margetekst=,
   \c!linkerletter=\getvalue{\??tk\v!voet\v!rand\c!letter},
   \c!rechterletter=\getvalue{\??tk\v!voet\v!rand\c!letter},
   \c!linkerkleur=\getvalue{\??tk\v!voet\v!rand\c!kleur},
   \c!rechterkleur=\getvalue{\??tk\v!voet\v!rand\c!kleur},
   \c!breedte=,
   \c!linkerbreedte=\getvalue{\??tk\v!voet\v!rand\c!breedte},
   \c!rechterbreedte=\getvalue{\??tk\v!voet\v!rand\c!breedte}]

\stelbovenin
  [\v!tekst]
  [\c!letter=,
   \c!kleur=,
   \c!linkertekst=,
   \c!middentekst=,
   \c!rechtertekst=,
   \c!kantlijntekst=,
   \c!margetekst=,
   \c!linkerletter=\getvalue{\??tk\v!boven\v!tekst\c!letter},
   \c!rechterletter=\getvalue{\??tk\v!boven\v!tekst\c!letter},
   \c!linkerkleur=\getvalue{\??tk\v!boven\v!tekst\c!kleur},
   \c!rechterkleur=\getvalue{\??tk\v!boven\v!tekst\c!kleur},
   \c!breedte=,
   \c!linkerbreedte=\getvalue{\??tk\v!boven\v!tekst\c!breedte},
   \c!rechterbreedte=\getvalue{\??tk\v!boven\v!tekst\c!breedte}]

\stelbovenin
  [\v!marge]
  [\c!letter=,
   \c!kleur=,
   \c!linkertekst=,
   \c!middentekst=,
   \c!rechtertekst=,
   \c!kantlijntekst=,
   \c!margetekst=,
   \c!linkerletter=\getvalue{\??tk\v!boven\v!marge\c!letter},
   \c!rechterletter=\getvalue{\??tk\v!boven\v!marge\c!letter},
   \c!linkerkleur=\getvalue{\??tk\v!boven\v!marge\c!kleur},
   \c!rechterkleur=\getvalue{\??tk\v!boven\v!marge\c!kleur},
   \c!breedte=,
   \c!linkerbreedte=\getvalue{\??tk\v!boven\v!marge\c!breedte},
   \c!rechterbreedte=\getvalue{\??tk\v!boven\v!marge\c!breedte}]

\stelbovenin
  [\v!rand]
  [\c!letter=,
   \c!kleur=,
   \c!linkertekst=,
   \c!middentekst=,
   \c!rechtertekst=,
   \c!kantlijntekst=,
   \c!margetekst=,
   \c!linkerletter=\getvalue{\??tk\v!boven\v!rand\c!letter},
   \c!rechterletter=\getvalue{\??tk\v!boven\v!rand\c!letter},
   \c!linkerkleur=\getvalue{\??tk\v!boven\v!rand\c!kleur},
   \c!rechterkleur=\getvalue{\??tk\v!boven\v!rand\c!kleur},
   \c!breedte=,
   \c!linkerbreedte=\getvalue{\??tk\v!boven\v!rand\c!breedte},
   \c!rechterbreedte=\getvalue{\??tk\v!boven\v!rand\c!breedte}]

\stelonderin
  [\v!tekst]
  [\c!letter=,
   \c!kleur=,
   \c!linkertekst=,
   \c!middentekst=,
   \c!rechtertekst=,
   \c!kantlijntekst=,
   \c!margetekst=,
   \c!linkerletter=\getvalue{\??tk\v!onder\v!rand\c!letter},
   \c!rechterletter=\getvalue{\??tk\v!onder\v!rand\c!letter},
   \c!linkerkleur=\getvalue{\??tk\v!onder\v!rand\c!kleur},
   \c!rechterkleur=\getvalue{\??tk\v!onder\v!rand\c!kleur},
   \c!breedte=,
   \c!linkerbreedte=\getvalue{\??tk\v!onder\v!rand\c!breedte},
   \c!rechterbreedte=\getvalue{\??tk\v!onder\v!rand\c!breedte}]

\stelonderin
  [\v!marge]
  [\c!letter=,
   \c!kleur=,
   \c!linkertekst=,
   \c!middentekst=,
   \c!rechtertekst=,
   \c!kantlijntekst=,
   \c!margetekst=,
   \c!linkerletter=\getvalue{\??tk\v!onder\v!marge\c!letter},
   \c!rechterletter=\getvalue{\??tk\v!onder\v!marge\c!letter},
   \c!linkerkleur=\getvalue{\??tk\v!onder\v!marge\c!kleur},
   \c!rechterkleur=\getvalue{\??tk\v!onder\v!marge\c!kleur},
   \c!breedte=,
   \c!linkerbreedte=\getvalue{\??tk\v!onder\v!marge\c!breedte},
   \c!rechterbreedte=\getvalue{\??tk\v!onder\v!marge\c!breedte}]

\stelonderin
  [\v!rand]
  [\c!letter=,
   \c!kleur=,
   \c!linkertekst=,
   \c!middentekst=,
   \c!rechtertekst=,
   \c!kantlijntekst=,
   \c!margetekst=,
   \c!linkerletter=\getvalue{\??tk\v!onder\v!rand\c!letter},
   \c!rechterletter=\getvalue{\??tk\v!onder\v!rand\c!letter},
   \c!linkerkleur=\getvalue{\??tk\v!onder\v!rand\c!kleur},
   \c!rechterkleur=\getvalue{\??tk\v!onder\v!rand\c!kleur},
   \c!breedte=,
   \c!linkerbreedte=\getvalue{\??tk\v!onder\v!rand\c!breedte},
   \c!rechterbreedte=\getvalue{\??tk\v!onder\v!rand\c!breedte}]

\steltekstin
  [\v!tekst]
  [\c!letter=,
   \c!kleur=,
   \c!linkertekst=,
   \c!middentekst=,
   \c!rechtertekst=,
   \c!kantlijntekst=,
   \c!margetekst=,
   \c!linkerletter=\getvalue{\??tk\v!tekst\v!tekst\c!letter},
   \c!rechterletter=\getvalue{\??tk\v!tekst\v!tekst\c!letter},
   \c!linkerkleur=\getvalue{\??tk\v!tekst\v!tekst\c!kleur},
   \c!rechterkleur=\getvalue{\??tk\v!tekst\v!tekst\c!kleur},
   \c!breedte=,
   \c!linkerbreedte=\getvalue{\??tk\v!tekst\v!tekst\c!breedte},
   \c!rechterbreedte=\getvalue{\??tk\v!tekst\v!tekst\c!breedte}]

\steltekstin
  [\v!marge]
  [\c!letter=,
   \c!kleur=,
   \c!linkertekst=,
   \c!middentekst=,
   \c!rechtertekst=,
   \c!kantlijntekst=,
   \c!margetekst=,
   \c!linkerletter=\getvalue{\??tk\v!tekst\v!marge\c!letter},
   \c!rechterletter=\getvalue{\??tk\v!tekst\v!marge\c!letter},
   \c!linkerkleur=\getvalue{\??tk\v!tekst\v!marge\c!kleur},
   \c!rechterkleur=\getvalue{\??tk\v!tekst\v!marge\c!kleur},
   \c!breedte=,
   \c!linkerbreedte=\getvalue{\??tk\v!tekst\v!marge\c!breedte},
   \c!rechterbreedte=\getvalue{\??tk\v!tekst\v!marge\c!breedte}]

\steltekstin
  [\v!rand]
  [\c!letter=,
   \c!kleur=,
   \c!linkertekst=,
   \c!middentekst=,
   \c!rechtertekst=,
   \c!kantlijntekst=,
   \c!margetekst=,
   \c!linkerletter=\getvalue{\??tk\v!tekst\v!rand\c!letter},
   \c!rechterletter=\getvalue{\??tk\v!tekst\v!rand\c!letter},
   \c!linkerkleur=\getvalue{\??tk\v!tekst\v!rand\c!kleur},
   \c!rechterkleur=\getvalue{\??tk\v!tekst\v!rand\c!kleur},
   \c!breedte=,
   \c!linkerbreedte=\getvalue{\??tk\v!tekst\v!rand\c!breedte},
   \c!rechterbreedte=\getvalue{\??tk\v!tekst\v!rand\c!breedte}]

\stelblankoin
  [\v!standaard,
   \v!groot]

\definieerblanko[\v!default] [\currentblanko]
\definieerblanko[\v!voor]    [\v!default]
\definieerblanko[\v!tussen]  [\v!default]
\definieerblanko[\v!na]      [\v!voor]

% doen?

\def\@@blankovoor  {\blanko[\v!voor]}   %
\def\@@blankotussen{\blanko[\v!tussen]} %  scheelt 5 tokens == >20 bytes
\def\@@blankona    {\blanko[\v!na]}     %

\stelblokkopjesin
  [\c!plaats=\v!onder,
   \c!voor=\blanko,
   \c!tussen={\blanko[\v!middel]},
   \c!na=\blanko,
   \c!breedte=\v!passend,
   \c!kopletter=\v!vet,
   \c!letter=\v!normaal,
   \c!kleur=,
   \c!uitlijnen=,
   \c!nummer=\v!ja,
   \c!wijze=\@@nrwijze,
   \c!blokwijze=\@@nrblokwijze,
   \c!sectienummer=\@@nrsectienummer,
   \c!conversie=\v!cijfers]

\stelplaatsblokkenin
  [\c!plaats=\v!midden,
   \c!breedte=\v!passend,
   \c!hoogte=\v!passend,
   \c!offset=\v!overlay,
   \c!kader=\v!uit,
   \c!straal=.5\korpsgrootte,
   \c!hoek=\v!recht,
   \c!achtergrond=,
   \c!achtergrondraster=\@@rsraster,
   \c!achtergrondkleur=,
   \c!achtergrondoffset=\!!zeropoint,
   \c!bovenkader=,
   \c!onderkader=,
   \c!linkerkader=,
   \c!rechterkader=,
   \c!kaderoffset=\!!zeropoint,
   \c!voor=,
   \c!na=,
   \c!voorwit=\v!groot,
   \c!nawit=\v!groot,
   \c!zijvoorwit=\@@bkvoorwit,
   \c!zijnawit=\@@bknawit,
   \c!springvolgendein=\v!nee,
   \c!marge=1em,
   \c!nboven=2,
   \c!nonder=0,
   \c!nregels=4]

\stelplaatsbloksplitsenin
  [\c!conversie=\v!letter, % \v!romeins
   \c!regels=3]

\stelwitruimtein
  [\v!geen]

\inspringen
  [\v!nooit]

\stelinspringenin
  [\v!geen]

\stelreferentielijstin
  [\c!letter=\v!normaal]

\stelinmargein
  [\c!letter=\v!vet,
   \c!kleur=,
   \c!plaats=\v!beide,
   \c!uitlijnen=\v!binnen,
   \c!voor=,
   \c!na=]

\stelinmargein
  [\v!links]
  [\c!plaats=\v!links]
%  \c!uitlijnen=\v!links] % njet

\stelinmargein
  [\v!rechts]
  [\c!plaats=\v!rechts]
%  \c!uitlijnen=\v!rechts] % njet

\versie
  [\v!definitief]

\stelpaginanummerin
  [\c!status=\v!start,
   \c!nummer=1]

\stelsubpaginanummerin
  [\c!wijze=\v!per\v!deel,
   \c!status=\v!stop]

\stelsmallerin
  [\c!links=1.5em,
   \c!rechts=1.5em,
   \c!midden=1.5em]

\stelregelnummerenin
  [\c!conversie=\v!cijfers,
   \c!start=1,
   \c!stap=1,
   \c!plaats=\v!inmarge,
   \c!letter=,
   \c!kleur=,
   \c!breedte=2em,
   \c!prefix=,
   \c!refereren=\v!aan]

\stelparagraafnummerenin
  [\c!status=\v!stop,
   \c!letter=,
   \c!kleur=,
   \c!afstand=\ifregelnummersinmarge2em\else\!!zeropoint\fi] 

\definieeropmaak
  [\v!standaard]
  [\c!breedte=\zetbreedte,
   \c!hoogte=\teksthoogte,
   \c!voffset=\!!zeropoint,
   \c!hoffset=\!!zeropoint,
   \c!pagina=\v!rechts,
   \c!dubbelzijdig=\v!leeg]

\stelpositionerenin
  [\c!eenheid=\s!cm,
   \c!factor=1,
   \c!schaal=1,
   \c!xstap=\v!absoluut,
   \c!ystap=\v!absoluut,
   \c!offset=\v!ja,
   \c!xoffset=\!!zeropoint,
   \c!yoffset=\!!zeropoint]

\stelregelsin
  [\c!voor=\blanko,
   \c!na=\blanko,
   \c!tussen=\blanko,
   \c!inspringen=\v!nee]

\stelkoppeltekenin
  [\c!teken=\compoundhyphen]

\stelnaastplaatsenin
  [\c!status=\v!stop]

\steltolerantiein
  [\v!horizontaal,\v!zeerstreng]

\steltolerantiein
  [\v!vertikaal,\v!streng]

\steluitlijnenin
  [\v!onder,
   \v!breedte]

\stelspatieringin
  [\v!opelkaar]

\definieerplaatsblok
  [\v!figuur]
  [\v!figuren]

\definieerplaatsblok
  [\v!tabel]
  [\v!tabellen]

\stelplaatsblokin
  [\v!tabel]
  [\c!kader=\v!uit]

\definieerplaatsblok
  [\v!intermezzo]
  [\v!intermezzos]

\definieerplaatsblok
  [\v!grafiek]
  [\v!grafieken]

\stelmargeblokkenin
  [\c!status=\v!start,
   \c!plaats=\v!inmarge,
   \c!breedte=\rechtermargebreedte,
   \c!letter=,
   \c!kleur=,
   \c!uitlijnen=,
   \c!links=,
   \c!rechts=,
   \c!boven=,
   \c!tussen=\blanko,
   \c!onder=\vfill,
   \c!voor=,
   \c!na=]

\stelachtergrondenin
  [\c!status=\c!start]

\stelachtergrondenin
  [\v!papier,\v!pagina,\v!linkerpagina,\v!rechterpagina]
  [\c!kader=\v!uit,
   \c!straal=.5\korpsgrootte,
   \c!hoek=\v!recht,
   \c!achtergrond=,
   \c!raster=\@@rsraster,
   \c!kleur=,
   \c!kaderoffset=\getvalue{\??ma\v!pagina\c!offset}, 
   \c!achtergrondoffset=\getvalue{\??ma\v!pagina\c!offset},
   \c!offset=\!!zeropoint, % later set to \v!overlay, watch out ! 
   %\c!scheider=\v!nee,
   \c!diepte=\!!zeropoint]

\global\somebackgroundfalse

\def\documentstyle%
  {\showmessage{\m!systems}{3}{}
   \stoptekst}

\let\documentclass\documentstyle

\protect \endinput
