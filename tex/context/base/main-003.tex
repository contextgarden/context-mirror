%D \module
%D   [       file=main-003,
%D        version=1997.03.31,
%D          title=\CONTEXT\ Core Macros,
%D       subtitle=1C (to be split),
%D         author=Hans Hagen,
%D           date=\currentdate,
%D      copyright={PRAGMA / Hans Hagen \& Ton Otten}]
%C
%C This module is part of the \CONTEXT\ macro||package and is
%C therefore copyrighted by \PRAGMA. See mreadme.pdf for
%C details.

%D This module is still to be split and documented. 

\writestatus{loading}{Context Core Macros (3)}

\unprotect

\startmessages  dutch  library: systems
     41: externe file -- in groep -- bestaat niet
\stopmessages

\startmessages  english  library: systems
     41: external file -- in group -- does not exist
\stopmessages

\startmessages  german  library: systems
     41: Externe Datei -- in Gruppe -- existiert nicht
\stopmessages

\startmessages  czech  library: systems
     41: externi soubor -- ve skupine -- neexistuje
\stopmessages

\startmessages  italian  library: systems
     41: il file esterno -- del gruppo -- non esiste
\stopmessages

\startmessages  norwegian  library: systems
     41: ekstern fil -- i gruppe -- eksisterer ikke
\stopmessages

\definetabulate
  [\e!legenda]
  [|emj1|i1|mR|]

\setuptabulate
  [\e!legenda]
  [\c!eenheid=.75em,\c!binnen=\setquicktabulate\leg,EQ={=}]

\definetabulate
  [\e!legenda][\v!twee]
  [|emj1|emk1|i1|mR|]

\definetabulate
  [\e!gegeven]
  [|R|ecmj1|i1mR|]

\setuptabulate
  [\e!gegeven]
  [\c!eenheid=.75em,\c!binnen=\setquicktabulate\geg,EQ={=}]

\unexpanded\def\xbox%
  {\bgroup\aftergroup\egroup\hbox\bgroup\tx\let\next=}

\unexpanded\def\xxbox%
  {\bgroup\aftergroup\egroup\hbox\bgroup\txx\let\next=}

% \def\mrm#1%
%   {$\rm#1$}

%D \macros 
%D   {definepairedbox, setuppairedbox, placepairedbox}
%D
%D Paired boxes, formally called legends, but from now on a
%D legend is just an instance, are primarily meant for
%D typesetting some text alongside an illustration. Although
%D there is quite some variation possible, the functionality is
%D kept simple, if only because in most cases such pairs are
%D typeset sober. 
%D 
%D The location specification accepts a pair, where the first
%D keyword specifies the arrangement, and the second one the
%D alignment. The first key of the location pair is one of 
%D \type {left}, \type {right}, \type {top} or \type {bottom}, 
%D while the second key can also be \type {middle}. 
%D
%D The first box is just collected in an horizontal box, but 
%D the second one is a vertical box that gets passed the 
%D bodyfont and alignment settings. 

%  \startbuffer[test]
%  \test left   \test left,top    \test left,bottom  \test left,middle
%  \test right  \test right,top   \test right,bottom \test right,middle
%  \test top    \test top,left    \test top,right    \test top,middle
%  \test bottom \test bottom,left \test bottom,right \test bottom,middle
%  \stopbuffer
%  
%  \def\showtest#1% 
%    {\pagina
%     \typebuffer[demo]
%     \def\test##1
%       {\startlinecorrection[blank]
%        \getbuffer[demo]%
%        \ruledhbox\placelegend
%          [bodyfont=6pt,location={##1}]
%          {\framed[width=.25\textwidth]{\tttf##1}}
%          {#1}
%        \stoplinecorrection}
%     \getbuffer[test]}
%  
%  \startbuffer[demo]
%  \setuplegend
%    [width=\hsize,maxwidth=\makeupwidth,
%     height=\vsize,maxheight=\makeupheight]
%  \stopbuffer
%  
%  \showtest{These examples demonstrate the default settings.}
%  
%  \startbuffer[demo]
%  \setuplegend
%    [width=\textwidth,
%     maxwidth=\textwidth]
%  \stopbuffer
%  
%  \showtest{\input tufte }
%  
%  \startbuffer[demo]
%  \setuplegend
%    [width=.65\textwidth]
%  \stopbuffer
%  
%  \showtest{\input knuth }
%  
%  \startbuffer[demo]
%  \setuplegend
%    [height=2cm]
%  \stopbuffer
%  
%  \showtest{These examples demonstrate some other settings.}
%  
%  \startbuffer[demo]
%  \setuplegend
%    [width=.65\textwidth,
%     height=2cm]
%  \stopbuffer
%  
%  \showtest{These examples demonstrate some other settings.}
%  
%  \startbuffer[demo]
%  \setuplegend
%    [n=2,align=right,width=.5\textwidth]
%  \stopbuffer
%  
%  \showtest{\input zapf }

%D \macros 
%D   {setuplegend, placelegend}
%D
%D It makes sense to typeset a legend to a figure in \TEX\ 
%D and not in a drawing package. The macro \type {\placelegend}
%D combines a figure (or something else) and its legend. This
%D command is just a paired box.
%D
%D The legend is placed according to \type {location}, being 
%D \type {bottom} or \type {right}. The macro macro is used as
%D follows. 
%D
%D \starttypen 
%D \placefigure
%D   {whow}
%D   {\placelegend
%D      {\externalfigure[cow]}
%D      {\starttabulation
%D       \NC 1 \NC head \NC \NR
%D       \NC 2 \NC legs \NC \NR
%D       \NC 3 \NC tail \NC \NR
%D       \stoptabulation}}
%D 
%D \placefigure
%D   {whow}
%D   {\placelegend
%D      {\externalfigure[cow]}
%D      {\starttabulation[|l|l|l|l|]
%D       \NC 1 \NC head \NC 3 \NC tail \NC \NR
%D       \NC 2 \NC legs \NC   \NC      \NC \NR
%D       \stoptabulation}}
%D 
%D \placefigure
%D   {whow}
%D   {\placelegend[n=2]
%D      {\externalfigure[cow]}
%D      {\starttabulation
%D       \NC 1 \NC head \NC \NR
%D       \NC 2 \NC legs \NC \NR
%D       \NC 3 \NC tail \NC \NR
%D       \stoptabulation}}
%D 
%D \placefigure
%D   {whow}
%D   {\placelegend[n=2]
%D      {\externalfigure[cow]}
%D      {head \par legs \par tail}}
%D 
%D \placefigure
%D   {whow}
%D   {\placelegend[n=2]
%D      {\externalfigure[cow]}
%D      {\startitemize[packed]
%D       \item head \item legs \item  tail \item belly \item horns 
%D       \stopitemize}}
%D 
%D \placefigure
%D   {whow}
%D   {\placelegend[n=2,width=.8\hsize]
%D      {\externalfigure[cow]}
%D      {\startitemize[packed]
%D       \item head \item legs \item  tail \item belly \item horns 
%D       \stopitemize}}
%D \stoptypen 

% \def\setuplegend%
%   {\dodoubleargument\getparameters[\??ld]}
% 
% \setuplegend
%   [\c!n=1,
%    \c!afstand=1em,
%    \c!tussen={\blanko[\v!middel]},
%    \c!breedte=\hsize,
%    \c!hoogte=\vsize,
%    \c!korps=,
%    \c!plaats=\v!onder]
% 
% \def\placelegend%
%   {\bgroup
%    \dosingleempty\doplacelegend}
% 
% \def\doplacelegend[#1]% watch the hsize/vsize tricks
%   {\setuplegend[#1]%  % and don't change them 
%    \dowithnextbox
%      {\switchtobodyfont[\@@ldkorps]% split under same regime  
%       \scratchdimen=\wd\nextbox
%       \doifelse{\@@ldplaats}{\v!rechts}
%         {\vsize=\ht\nextbox
%          \vsize=\@@ldhoogte
%          \hsize=\zetbreedte
%          \advance\hsize by -\scratchdimen
%          \advance\hsize by -\@@ldafstand
%          \plaatsnaastelkaar{\box\nextbox}\bgroup}
%         {\hsize\scratchdimen
%          \plaatsonderelkaar{\box\nextbox}\bgroup}%
%        \hsize\@@ldbreedte
%        \doif{\@@ldplaats}{\v!rechts}{\hskip\@@ldafstand}%
%        \ifnum\@@ldn>1
%          \setrigidcolumnhsize\hsize\@@ldafstand\@@ldn
%        \fi
%        \dowithnextbox
%          {\doifelse{\@@ldplaats}{\v!rechts}
%             {\vbox to \vsize
%                {\ifnum\@@ldn>1
%                   \rigidcolumnbalance\nextbox
%                 \else
%                   \box\nextbox
%                 \fi
%                 \vfill}}
%             {\vbox
%                {\@@ldtussen
%                 \ifnum\@@ldn>1
%                   \rigidcolumnbalance\nextbox
%                 \else
%                   \box\nextbox
%                 \fi}}%
%           \egroup\egroup}%
%          \vbox
%            \bgroup
%            \forgetall
%            \tolerantTABLEbreaktrue % hm.
%            \blanko[\v!blokkeer]%
%            \everypar{\begstrut}%
%            \let\next=}
%    \hbox}

\newbox\firstpairedbox 
\newbox\secondpairedbox 

\def\definepairedbox%
  {\dodoubleempty\dodefinepairedbox}

\def\dodefinepairedbox[#1][#2]%
  {\getparameters
     [\??ld#1]
     [\c!n=1,
      \c!afstand=\bodyfontsize,
      \c!tussen={\blanko[\v!middel]},
      \c!breedte=\hsize,
      \c!hoogte=\vsize,
      \c!maxbreedte=\zetbreedte,
      \c!maxhoogte=\zethoogte,
      \c!korps=,
      \c!uitlijnen=,
      \c!plaats=\v!onder,
      #2]%
   \setvalue{\e!stel#1\e!in}{\setuppairedbox[#1]}%
   \setvalue{\e!plaats#1}{\placepairedbox[#1]}}

\def\setuppairedbox%
  {\dodoubleempty\dosetuppairedbox}

\def\dosetuppairedbox[#1]%
  {\getparameters[\??ld#1]}

\def\placepairedbox% 
  {\bgroup\dodoubleempty\doplacepairedbox}

\def\doplacepairedbox[#1][#2]% watch the hsize/vsize tricks
  {\setuppairedbox[#1][#2]%     % and don't change them 
   \copyparameters 
     [\??ld][\??ld#1]
     [\c!n,\c!afstand,\c!tussen,
      \c!breedte,\c!hoogte,\c!maxbreedte,\c!maxhoogte,
      \c!korps,\c!uitlijnen,\c!plaats]%
   \beforefirstpairedbox
   \dowithnextbox
     {\betweenbothpairedboxes
      \dowithnextbox
        {\afterbothpairedboxes
         \egroup}
      \vbox\bgroup
        \insidesecondpairedbox
        \let\next=}
   \hbox}

\def\beforefirstpairedbox%
  {\chardef\pairedlocationa=1 % left 
   \chardef\pairedlocationb=4 % middle
   \getfromcommacommand[\@@ldplaats][1]%
   \processaction 
     [\commalistelement]
     [ \v!links=>\chardef\pairedlocationa=0,
      \v!rechts=>\chardef\pairedlocationa=1,
       \v!boven=>\chardef\pairedlocationa=2,
       \v!onder=>\chardef\pairedlocationa=3]%
   \getfromcommacommand[\@@ldplaats][2]%
   \processaction 
     [\commalistelement]
     [ \v!links=>\chardef\pairedlocationb=0,
      \v!rechts=>\chardef\pairedlocationb=1,
        \v!hoog=>\chardef\pairedlocationb=2,
       \v!boven=>\chardef\pairedlocationb=2,
        \v!laag=>\chardef\pairedlocationb=3,
       \v!onder=>\chardef\pairedlocationb=3,
      \v!midden=>\chardef\pairedlocationb=4]}

\def\betweenbothpairedboxes%
  {\switchtobodyfont[\@@ldkorps]% split under same regime  
   \setbox\firstpairedbox=\box\nextbox
   \ifnum\pairedlocationa<2
     \hsize\wd\firstpairedbox % trick 
     \hsize=\@@ldbreedte
     \scratchdimen=\wd\firstpairedbox
     \advance\scratchdimen by \@@ldafstand
     \bgroup\advance\scratchdimen by \hsize
     \ifdim\scratchdimen>\@@ldmaxbreedte\relax
       \egroup
       \hsize=\@@ldmaxbreedte
       \advance\hsize by -\scratchdimen
     \else
       \egroup
     \fi
   \else
     \hsize\wd\firstpairedbox 
     \hsize\@@ldbreedte % can be \hsize
     \ifdim\hsize>\@@ldmaxbreedte\relax \hsize=\@@ldmaxbreedte \fi % can be \hsize
   \fi
   \ifnum\@@ldn>1
     \setrigidcolumnhsize\hsize\@@ldafstand\@@ldn
   \fi}

\def\afterbothpairedboxes%
  {\setbox\secondpairedbox=\vbox
     {\ifnum\@@ldn>1 \rigidcolumnbalance\nextbox \else \box\nextbox \fi}%
   \ifnum\pairedlocationa<2\hbox\else\vbox\fi\bgroup % hide vsize 
   \forgetall
   \ifnum\pairedlocationa<2 
     \scratchdimen=\maxoftwoboxdimens\ht\firstpairedbox\secondpairedbox
     \vsize=\scratchdimen 
     \ifdim\scratchdimen<\@@ldhoogte\relax % can be \vsize
       \scratchdimen=\@@ldhoogte 
     \fi
     \ifdim\scratchdimen>\@@ldmaxhoogte\relax
       \scratchdimen=\@@ldmaxhoogte 
     \fi
     \valignpairedbox\firstpairedbox \scratchdimen
     \valignpairedbox\secondpairedbox\scratchdimen
   \else
     \scratchdimen=\maxoftwoboxdimens\wd\firstpairedbox\secondpairedbox
     \halignpairedbox\firstpairedbox \scratchdimen
     \halignpairedbox\secondpairedbox\scratchdimen
     \scratchdimen=\ht\secondpairedbox
     \vsize=\scratchdimen
     \ifdim\ht\secondpairedbox<\@@ldhoogte\relax % can be \vsize
       \scratchdimen=\@@ldhoogte\relax % \relax needed 
     \fi
     \ifdim\scratchdimen>\@@ldmaxhoogte\relax % todo: totale hoogte
       \scratchdimen=\@@ldmaxhoogte\relax % \relax needed 
     \fi
     \ifdim\scratchdimen>\ht\secondpairedbox
       \setbox\secondpairedbox\vbox to \scratchdimen
         {\ifnum\pairedlocationa=3 \vss\fi
          \box\secondpairedbox
          \ifnum\pairedlocationa=2 \vss\fi}%
     \fi
   \fi
   \ifcase\pairedlocationa
     \box\secondpairedbox\hskip\@@ldafstand\box\firstpairedbox \or
     \box\firstpairedbox \hskip\@@ldafstand\box\secondpairedbox\or
     \box\secondpairedbox\par  \@@ldtussen \box\firstpairedbox \or
     \box\firstpairedbox \par  \@@ldtussen \box\secondpairedbox\else
   \fi
   \egroup}

\def\insidesecondpairedbox%
  {\forgetall
   \steluitlijnenin[\@@lduitlijnen]%
   \tolerantTABLEbreaktrue % hm.
   \blanko[\v!blokkeer]%
   \everypar{\begstrut}}

\def\maxoftwoboxdimens#1#2#3%
  {#1\ifdim#1#2>#1#3 #2\else#3\fi}

\def\valignpairedbox#1#2%
  {\setbox#1=\vbox to #2
     {\ifcase\pairedlocationb\or\or\or\vss\or\vss\fi
      \box#1\relax
      \ifcase\pairedlocationb\or\or\vss\or\or\vss\fi}}

\def\halignpairedbox#1#2%
  {\setbox#1=\hbox to #2
     {\ifcase\pairedlocationb\or\hss\or\or\or\hss\fi
      \box#1\relax
      \ifcase\pairedlocationb\hss\or\or\or\or\hss\fi}}

\definepairedbox[\e!legenda]

\newcount\horcombinatie  % counter
\newcount\totcombinatie

\def\stelcombinatiesin%
  {\dodoubleargument\getparameters[\??co]}

\long\def\dodostartcombinatie[#1*#2*#3]%
  {\stelfractiesin
     [\c!n=\v!passend,
      \c!afstand=\@@coafstand]%
   \global\horcombinatie=#1\relax
   \global\totcombinatie=#2\relax
   \xdef\maxhorcombinatie{\the\horcombinatie}%
   \multiply\totcombinatie by \horcombinatie
   \tabskip=\!!zeropoint
   \doifelse{\@@cobreedte}{\v!passend}
     {\halign}
     {\halign to \@@cobreedte}%
   \bgroup&\hfil##\hfil&\tabskip\!!zeropoint \!!plus 1fill##\cr
   \docombinatie}

% \def\docombinatie%
%   {\dowithnextbox
%      {\setbox0=\box\nextbox
%       \dowithnextbox
%         {\setbox2=\box\nextbox
%          \dodocombinatie}
%      \hbox}
%   \hbox}

\def\docombinatie% we want to add struts but still ignore an empty box
  {\dowithnextbox%
     {\setbox0=\box\nextbox
      \dowithnextbox
        {\setbox2=\box\nextbox
         \dodocombinatie}
      \vtop\bgroup
        \def\next%
          {\futurelet\nexttoken\nextnext}
        \def\nextnext%
          {\ifx\nexttoken\egroup \else % the next box is empty  
             \hsize\wd0
             \steluitlijnenin[\@@couitlijnen]
             \bgroup
             \aftergroup\endstrut
             \aftergroup\egroup
             \begstrut
           \fi}
        \afterassignment\next\let\nexttoken=}
  \hbox}

\def\dodocombinatie%
  {\vbox
     {\forgetall % \stelwitruimtein[\v!geen]%
      \vbox
        {\copy0}%
      \ifdim\ht2>\!!zeropoint\relax % beter dan \wd2, nu \strut mogelijk
        \@@cotussen
       %\vtop
       %  {\nointerlineskip  % recently added
       %   \hsize\wd0
       %   \steluitlijnenin[\@@couitlijnen]%  % \raggedcenter
       %   \begstrut\unhbox2\endstrut}%
        \box2
      \fi}%
   \ifnum\totcombinatie>1
     \global\advance\totcombinatie by -1
     \global\advance\horcombinatie by -1
     \ifnum\horcombinatie=0
       \def\next%
         {\cr\noalign
            {\forgetall %\stelwitruimtein[\v!geen]%
             \nointerlineskip
             \@@cona
             \@@covoor
             \vss
             \nointerlineskip}%
          \global\horcombinatie=\maxhorcombinatie\relax
          \docombinatie}%
     \else
       \def\next%
         {&&&\hskip\@@coafstand&\docombinatie}%
     \fi
   \else
     \def\next%
       {\cr\egroup}%
   \fi
   \next}

\def\complexdostartcombinatie[#1]%
  {\dodostartcombinatie[#1*1*]}

\def\simpledostartcombinatie%
  {\complexdostartcombinatie[2]}

\def\startcombinatie%
  {\bgroup
   \forgetall
   \doifelse{\@@cohoogte}{\v!passend}
     {\vbox}
     {\vbox to \@@cohoogte}%
   \bgroup
   \complexorsimple\dostartcombinatie}

\def\stopcombinatie%
  {\egroup
   \egroup}

\def\plaatsondernaastelkaar#1#2%
  {\bgroup
   \def\doplaatsondernaastelkaar%
     {#2\cr\omit\bgroup#2%
      \aftergroup#2%
      \aftergroup\cr
      \aftergroup\egroup
      \aftergroup\egroup
      \let\next=}%
   #1\bgroup##\cr
   \omit\bgroup#2%
   \aftergroup\doplaatsondernaastelkaar
   \let\next=}

\def\plaatsonderelkaar%
  {\plaatsondernaastelkaar\halign\hss}

\def\plaatsnaastelkaar%
  {\plaatsondernaastelkaar\valign\vss}

\def\dogebruikexternefiles[#1][#2]%
  {\getparameters
     [\??fi#1]
     [\c!file=,
      \c!korps=,
      \c!optie=,
      #2]}

\def\gebruikexternefiles%
  {\dodoubleargument\dogebruikexternefiles}

\def\dostelexternefilesin[#1][#2]%
  {\doifundefinedelse{\??fi#1\c!file}
     {\gebruikexternefiles[#1][#2]}
     {\getparameters[\??fi#1][#2]}}

\def\stelexternefilesin%
  {\dodoubleargument\dostelexternefilesin}

\def\verwerkexternefile#1#2#3%
  {\bgroup
   \getparameters[\??fi#1][\c!file=,#3]%
   \doinputonce{\getvalue{\??fi#1\c!file}}%
   \ExpandFirstAfter\switchtobodyfont[\getvalue{\??fi#1\c!korps}]%
   \readsysfile{#2}  % beter: loc of fix gebied
     {}
     {\showmessage{\m!systems}{41}{#2,#1}}%
   \egroup}

\def\dogebruikexternefile[#1][#2][#3][#4]%
  {\stelexternefilesin[#1][]%
   \doinputonce{\getvalue{\??fi#1\c!file}}%
   \doifelsenothing{#2}
     {\setvalue{#3}{\verwerkexternefile{#1}{#3}{#4}}}
     {\setvalue{#2}{\verwerkexternefile{#1}{#3}{#4}}}}

\def\gebruikexternefile%
  {\doquadrupleargument\dogebruikexternefile}


\presetlocalframed[\??ro]

\def\stelroterenin%
  {\dodoubleargument\getparameters[\??ro]}

% \ht, \vfillvoor, \vfillna, \wd, \hfillvoor, \hfillna

\def\dodostoproteer#1#2#3#4#5#6%
  {\dontshowcomposition
   \vbox to #1\nextbox
     {#2\relax
      \hbox to #4\nextbox
        {#5\relax % \number removes leading spaces too 
         \edef\@@rorotatie{\number\@@rorotatie}%
         \doifelsenothing{\@@rorotatie}
           {\dostartrotation{90}}
           {\dostartrotation{\@@rorotatie}}% 
         \wd\nextbox=\!!zeropoint
         \ht\nextbox=\!!zeropoint
         \box\nextbox
         \dostoprotation
         #6}
      #3}%
   \egroup}


\def\dostoproteer%
  {\!!counta=\@@rorotatie
   \divide\!!counta by 90
   \ifcase\!!counta
     \dodostoproteer\ht\relax\vfill\wd\relax\hfill
   \or
     \dodostoproteer\wd\vfill\relax\ht\relax\hfill
   \or
     \dodostoproteer\ht\vfill\relax\wd\hfill\relax
   \or
     \dodostoproteer\wd\relax\vfill\ht\hfill\relax
   \or
     \dodostoproteer\ht\relax\vfill\wd\relax\hfill
   \else
     \def\@@rotatie{90}%
     \dodostoproteer\ht\relax\vfill\wd\relax\hfill
   \fi}

\def\dorotatebox#1% {angle} \hbox/\vbox/\vtop
  {\bgroup
   \hbox\bgroup % compatibility hack
   \dowithnextbox
     {\edef\@@rorotatie{#1}%
      \setbox\nextbox=\vbox{\box\nextbox}%
      \dostoproteer
      \egroup}}

\def\complexroteer[#1]%
  {\dowithnextbox
     {\getparameters[\??ro][#1]%
      \dostoproteer}%
   \vbox\localframed[\??ro][#1]}

\def\roteer%
  {\bgroup     % \roteer kan argument zijn
   \complexorsimpleempty\roteer}

% schaal

\def\doscalelikeafigure%
  {\doifsomething{\@@xyfactor\@@xyschaal\@@xyhfactor\@@xybreedte\@@xyhoogte}
     {\let \@@efschaal \@@xyschaal
      \let \@@effactor \@@xyfactor
      \let \@@efbfactor\@@xybfactor 
      \let \@@efhfactor\@@xyhfactor
      \let \@@efbreedte\@@xybreedte
      \let \@@efhoogte \@@xyhoogte
      \let \@@epx      \!!zeropoint
      \let \@@epy      \!!zeropoint   
      \edef\@@epw     {\the\wd\nextbox}%
      \edef\@@eph     {\the\ht\nextbox}%
      \setfactorfiguresize
      \setscalefiguresize
      \setdimensionfiguresize
      \convertfigureinsertscale\@@epx\figx\figxsca\scax
      \convertfigureinsertscale\@@epy\figy\figysca\scay
      \scratchdimen=\scax pt \divide\scratchdimen by 100 
      \edef\@@xysx{\withoutpt\the\scratchdimen}%
      \scratchdimen=\scay pt \divide\scratchdimen by 100 
      \edef\@@xysy{\withoutpt\the\scratchdimen}}}   

\def\doschaal[#1]%
  {\bgroup
   \forgetall
   \getparameters
     [\??xy]
     [\c!schaal=,\c!breedte=,\c!hoogte=,
      \c!factor=,\c!hfactor=,\c!bfactor=,
      \c!sx=1,\c!sy=1,#1]%
   \dowithnextbox
     {\dontshowcomposition
      \doscalelikeafigure
      \dimen0=\@@xysy\ht\nextbox
      \dimen2=\@@xysy\dp\nextbox
      \dimen4=\@@xysx\wd\nextbox
      \dimen6=\dimen0\advance\dimen6 by \dimen2 
      \setbox\nextbox=\vbox to \dimen6
        {\ht\nextbox=\!!zeropoint
         \dp\nextbox=\!!zeropoint
         \vfill % erbij
         \dostartscaling\@@xysx\@@xysy\box\nextbox\dostopscaling}%
      \ht\nextbox=\dimen0
      \dp\nextbox=\dimen2
      \wd\nextbox=\dimen4
      \box\nextbox
      \egroup}
   \hbox}

\def\schaal%
  {\dosingleempty\doschaal}

% mirror

\def\domirrorbox% \hbox/\vbox/\vtop
  {\bgroup
   \dowithnextbox
     {\dontshowcomposition
      \dimen0=\wd\nextbox
      \setbox\nextbox=\vbox
        {\dostartmirroring\hskip-\wd\nextbox\box\nextbox\dostopmirroring}%
      \wd\nextbox=\dimen0
      \box\nextbox
      \egroup}}

\def\spiegel%
  {\domirrorbox\hbox}

%\setbox0=\hbox{gans}
%
%\ruledhbox{\copy0 \schaal[sx=2,sy=2]{\copy0}}
%
%\spiegel{\ruledhbox{\copy0 \schaal{\box0}}}

% verdelen \hsize in fracties, wordt nog wat algemener, 
% beetje vaag nu 
%
% \fractie[n/m,elementen,afstand]
%
% \fractie[2/5,3,1em]
% \fractie[2/5,3,1em]
% \fractie[1/5,3,1em]
%
% \stelfractiesin[afstand=,aantal=]  (passend,passend)

\def\??fr{@@fr}

\def\stelfractiesin%
  {\dodoubleargument\getparameters[\??fr]}

\def\dodofractie[#1/#2,#3,#4,#5]%
  {\doifelsenothing{#3}
     {\doifelse{\@@frn}{\v!passend}
        {\!!counta=#2\relax}
        {\!!counta=\@@frn\relax}}
     {\!!counta=#3\relax}%
   \doifelsenothing{#4}
     {\doifelse{\@@frafstand}{\v!passend}
        {\!!widtha=\!!zeropoint}
        {\!!widtha=\@@frafstand}}
     {\!!widtha=#4}%
   \advance\!!counta by -1\relax
   \multiply\!!widtha by \!!counta
   \advance\hsize by -\!!widtha
   \divide\hsize by #2\relax
   \hsize=#1\hsize}

\def\dofractie[#1]%
  {\dodofractie[#1,,,,,,]}

\def\fractie%
  {\dosingleargument\dofractie}

\stelfractiesin
  [\c!afstand=\tfskipsize,
   \c!n=\v!passend]

% Standaardinstellingen

\stelroterenin
  [\c!rotatie=90,
   \c!breedte=\v!passend,
   \c!hoogte=\v!passend,
   \c!offset=\v!overlay,
   \c!kader=\v!uit]

\stelcombinatiesin
  [\c!breedte=\v!passend,
   \c!hoogte=\v!passend,
   \c!afstand=1em,
   \c!voor=\blanko,
   \c!tussen={\blanko[\v!middel]},
   \c!na=,
   \c!uitlijnen=\v!midden]

\gebruikexternefiles
  [pictex]
  [\c!korps=\v!klein,
   \c!file=pictex]

\gebruikexternefiles
  [table]
  [\c!file=table]

\protect

\endinput
