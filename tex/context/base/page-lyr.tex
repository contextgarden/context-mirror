%D \module
%D   [       file=page-lyr,
%D        version=2000.10.20,
%D          title=\CONTEXT\ Page Macros,
%D       subtitle=Layers,
%D         author=Hans Hagen,
%D           date=\currentdate,
%D      copyright={PRAGMA / Hans Hagen \& Ton Otten}]
%C
%C This module is part of the \CONTEXT\ macro||package and is
%C therefore copyrighted by \PRAGMA. See mreadme.pdf for
%C details.

\writestatus{loading}{Context Page Macros / Layers}

\unprotect

% When being backgrounds layers get the background offset
% displacement. Should be an option, on by default
% (compatibility).

% positie=forceer == ja maar dan ook in status=herhaal

%D The layering mechanism implemented here is independent of
%D the output routine, but future extensions may depend on a
%D more close cooperation.

%D First we overload a macro from \type {core-rul}. From now on
%D we accept a (optional) argument: the specific layer it
%D will go in. This means that we can move an overlay from one
%D background to the other using the dimensions of the parent.

\ifx\undefined\defineoverlay \message{loaded to early} \wait \fi

\def\defineoverlay
  {\dotripleempty\dodefineoverlay}

\def\dodefineoverlay[#1][#2][#3]% overlay [layer] content
  {\ifthirdargument
     \writestatus{BEWARE}{This (overlay definition) has changed!}% temp
     \def\docommando##1%
       {\setvalue{\??ov##1}####1####2####3####4####5####6####7%
          {\setlayer[#2]{\executedefinedoverlay
             ##1\\#3\\{####1}{####2}{####3}{####4}{####5}{####6}{####7}}}}%
   \else
     \def\docommando##1%
       {\setvalue{\??ov##1}{\executedefinedoverlay
          ##1\\#2\\}}%
   \fi
   \processcommalist[#1]\docommando}

%D When tracing is turned on, a couple of boxes will
%D show up as well as the reference point.

\newif\iftracelayers % \tracelayerstrue

%D This handy constant saved some string memory.

\def\@@layerbox{@@layerbox}

%D \macros
%D   {definelayer}
%D
%D Each layer gets its own (global) box. This also means that
%D the data that goes into a layer, is typeset immediately.
%D Each layer automatically gets an associated overlay,
%D which can be used in any background assignment.

% todo : links/rechts

\def\definelayer
  {\dodoubleargument\dodefinelayer}

\def\dodefinelayer[#1][#2]%
  {\setuplayer
     [#1]
     [\c!dubbelzijdig=,\c!preset=,
      \c!status=\v!start,\c!richting=\v!normaal,\c!optie=,
      \c!x=\!!zeropoint,\c!y=\!!zeropoint,\c!positie=\v!nee,
      \c!regel=0,\c!kolom=0,
      \c!breedte=\nextboxwd,\c!hoogte=\nextboxht,
      \c!offset=\!!zeropoint,\c!rotatie=, % geen 0 !
      \c!hoffset=\!!zeropoint,\c!voffset=\!!zeropoint,
      \c!plaats=rb,\c!positie=\v!nee,\c!pagina=,
      \c!methode=\v!overlay,
      \c!sx=1,\c!sy=1,\c!hoek=,#2]%
   \doifvalue{\??ll#1\c!dubbelzijdig}\v!ja
     {\dopresetlayerbox{\v!links #1}%
      \dopresetlayerbox{\v!rechts#1}}%
   \dopresetlayerbox{#1}%
   \defineoverlay[#1][\composedlayer{#1}]}

\def\dopresetlayerbox#1%
  {\ifundefined{\@@layerbox#1}%
     \expandafter\newbox\csname\@@layerbox#1\endcsname
   \else
     \resetlayer[#1]%
   \fi}

%D \macros
%D   {setuplayer}
%D
%D After a layer is defined, you can change its
%D characteristics.

\def\setuplayer
  {\dodoubleargument\dosetuplayer}

\def\dosetuplayer[#1][#2]%
  {\def\docommando##1{\getparameters[\??ll##1][#2]}%
   \processcommalist[#1]\docommando}

%D \macros
%D   {setlayer}
%D
%D Data is moved into a layer with the following macro. When
%D \type {position} is set, relative positioning is used, with
%D the current point as reference point. Otherwise the topleft
%D corner is used as reference point.
%D
%D \starttypen
%D \setlayer [identifier] [optional parameters] {data}
%D \stoptypen

\def\currentlayerdata{0}

\def\setlayer
  {\dotripleempty\dosetlayer}

\def\dosetlayer[#1][#2][#3]% #4 == box do \fi is ok
  {\doifelsevalue{\??ll#1\c!status}\v!stop
     {\dowithnextbox\donothing\hbox}
     {\ifthirdargument
        \dodosetlayer[#1][#2][#3]%
      \else
        \dodosetlayer[#1][][#2]%
      \fi}}

\def\dodosetlayer[#1][#2][#3]% #2 = links/rechts
  {\bgroup
   \recalculatebackgrounds
   \recalculatelogos
   \doglobal\increment\currentlayerdata
   \forgetall
   \dontcomplain
   \doifvalue{\??ll#1\c!optie}\v!test\tracelayerstrue
   \iftracelayers\traceboxplacementtrue\fi
   \dowithnextbox % sneller als aparte macro
     {\ifundefined{\@@layerbox#1}%
        \writestatus{layer}{unknown layer #1}%
      \else
        \doifelse{#2}\v!even
          {\ifodd\realpageno
          %  discard nextbox
           \else
             \dododosetlayer[#1][\v!links][#3]%
           \fi}%
          {\doifelse{#2}\v!oneven
             {\ifodd\realpageno
                \dododosetlayer[#1][\v!rechts][#3]%
             %\else
             %  discard nextbox
              \fi}%
             {\dododosetlayer[#1][#2][#3]}}%
      \fi
      \egroup}
     \hbox}

\newbox\layerbox

\newdimen\@@layerxsiz \newdimen\@@layerxoff \newdimen\@@layerxpos
\newdimen\@@layerysiz \newdimen\@@layeryoff \newdimen\@@layerypos

\let\lastlayerxpos\!!zeropoint
\let\lastlayerypos\!!zeropoint
\let\lastlayerwd  \!!zeropoint
\let\lastlayerht  \!!zeropoint
\let\lastlayerdp  \!!zeropoint

% todo left/right

\def\setlastlayerpos#1%
  {\edef\layerpage{\MPp{lyr:\currentlayerdata}}%
   \scratchdimen\MPx{lyr:#1:\layerpage}%
   \scratchdimen-\scratchdimen
   \advance\scratchdimen\MPx{lyr:\currentlayerdata}%
   \xdef\lastlayerxpos{\the\scratchdimen}%
   \scratchdimen\MPy{lyr:#1:\layerpage}%
   \advance\scratchdimen-\MPy{lyr:\currentlayerdata}%
   \xdef\lastlayerypos{\the\scratchdimen}}

\def\definelayerpreset
 {\dodoubleargument\dodefinelayerpreset}

\def\dodefinelayerpreset[#1][#2]%
  {\setvalue{\??ll\??ll#1}{\dopresetlayer{#2}}}

\def\dopresetlayer#1#2#3% #1=list #2=tag #3=list
  {\getparameters[\??ll#2][#1,#3]}

\letempty\currentlayer

\def\layerparameter#1{\csname\??ll\currentlayer#1\endcsname}

\newdimen\layerwidth
\newdimen\layerheight

\chardef\@@lacome=1 % LAyerCOnstructionMEthod / temp, will be default

\def\dododosetlayer[#1][#2][#3]% will be sped up
  {% we use the global width, never change this
   \def\currentlayer{#1}%
   \@@layerxsiz\layerparameter\c!breedte
   \@@layerysiz\layerparameter\c!hoogte
   \layerwidth \@@layerxsiz
   \layerheight\@@layerysiz
   % preroll
   \getparameters[\??ll#1][#3]%
   % presets and real roll
   \executeifdefined{\??ll\??ll\layerparameter\c!preset}\gobbletwoarguments{#1}{#3}%
   % that was real slow
   \doif{\layerparameter\c!positie}\v!overlay  % slow
     {\getparameters[\??ll\currentlayer][\c!breedte=\zeropoint,\c!hoogte=\zeropoint,\c!positie=\v!ja]}%
   \doifsomething{\layerparameter\c!rotatie}
     {\setbox\nextbox\hbox
        {\rotate % to be checked with new rotation
           [\c!plaats=\v!hoog,\c!rotatie=\layerparameter\c!rotatie]
           {\flushnextbox}}}%
   % no, not local
   % \@@layerxsiz\layerparameter\c!breedte
   % \@@layerysiz\layerparameter\c!hoogte
   % never change that
   \@@layerxpos\layerparameter\c!x
   \@@layerypos\layerparameter\c!y
   \doifelse{\layerparameter\c!hoffset}\v!max
     {\@@layerxoff\@@layerxsiz}{\@@layerxoff\layerparameter\c!hoffset}%
   \doifelse{\layerparameter\c!voffset}\v!max
     {\@@layeryoff\@@layerysiz}{\@@layeryoff\layerparameter\c!voffset}%
   \advance\@@layerxoff\layerparameter\c!offset
   \advance\@@layeryoff\layerparameter\c!offset
   \@@layerxpos\layerparameter\c!sx\@@layerxpos
   \@@layerypos\layerparameter\c!sy\@@layerypos
   \@@layerxoff\layerparameter\c!sx\@@layerxoff
   \@@layeryoff\layerparameter\c!sy\@@layeryoff
   \doifelse{\layerparameter\c!positie}\v!ja   % combine ^
     {\setlastlayerpos{#2#1}% todo l/r %%%%%%%%%%%%
      \@@layerxpos\lastlayerxpos
      \@@layerypos\lastlayerypos
      \letgvalue{\??ll#1\layerpage\c!positie}\v!ja
      \letgvalue{\??ll#1\c!status}\v!start % needed ?
      \setbox\layerbox\vbox to \@@layerysiz
        {\hbox to \@@layerxsiz{\xypos{lyr:\currentlayerdata}\hss}\vss}}
     {\setbox\layerbox\emptybox
      \globallet\lastlayerxpos\!!zeropoint
      \globallet\lastlayerypos\!!zeropoint
      \ExpandBothAfter\doifinset\v!onder{\layerparameter\c!hoek}
        {\ifnum\layerparameter\c!regel=\zerocount\else % can be < 0
           \scratchcounter\layerparameter\c!regel
           \scratchcounter-\scratchcounter
           \advance\scratchcounter\layoutlines
           \advance\scratchcounter\plusone
           \setevalue{\??ll#1\c!regel}{\the\scratchcounter}%
         \fi
         \ifdim\@@layerysiz>\zeropoint
           \advance\@@layerypos-\@@layerysiz
           \@@layerypos-\@@layerypos
           \@@layeryoff-\@@layeryoff
         \fi}%
      \ExpandBothAfter\doifinset\v!rechts{\layerparameter\c!hoek}
        {\ifnum\layerparameter\c!kolom=\zerocount\else % can be < 0
           \scratchcounter\layerparameter\c!kolom
           \scratchcounter-\scratchcounter
           \advance\scratchcounter \layoutcolumns
           \advance\scratchcounter \plusone
           \setevalue{\??ll#1\c!kolom}{\the\scratchcounter}%
         \fi
         \ifdim\@@layerxsiz>\zeropoint
           \advance\@@layerxpos-\@@layerxsiz
           \@@layerxpos-\@@layerxpos
           \@@layerxoff-\@@layerxoff
         \fi}%
      \ExpandBothAfter\doif\v!midden{\layerparameter\c!hoek}
        {\ifdim\@@layerxsiz>\zeropoint
           \advance\@@layerxpos.5\@@layerxsiz
         \fi
         \ifdim\@@layerysiz>\zeropoint
           \advance\@@layerypos.5\@@layerysiz
         \fi}%
      \edef\layerpage{\layerparameter\c!pagina}}%
   \doifsomething\layerpage
     {\edef\layerpage{:\layerpage}%
      \doifundefined{\@@layerbox#2#1\layerpage}
        {\global\expandafter\newbox\csname\@@layerbox#2#1\layerpage\endcsname}}%
   \dontcomplain % more comfortable
   \chardef\layerpagebox\csname\@@layerbox#2#1\layerpage\endcsname
   \ifvoid\layerpagebox
     \gsetboxllx\layerpagebox\zeropoint
     \gsetboxlly\layerpagebox\zeropoint
   \fi
   \global\setbox\layerpagebox\vbox %to \layerparameter\c!hoogte % new, otherwise no negative y possible
     {\offinterlineskip
     %postpone, to after nextboxwd correction % \hsize\layerparameter\c!breedte % new, keep box small
     %\ifvoid\csname\@@layerbox#1\layerpage\endcsname\else % why not #2#1
      \ifvoid\layerpagebox
        \let\lastlayerwidth \zeropoint
        \let\lastlayerheight\zeropoint
      \else
        \edef\lastlayerwidth {\the\wd\layerpagebox}%
        \edef\lastlayerheight{\the\ht\layerpagebox}%
        \ht\layerpagebox\zeropoint
        \dp\layerpagebox\zeropoint
        \wd\layerpagebox\zeropoint
        \doifnotvalue{\layerparameter\c!richting}\v!omgekeerd{\box\layerpagebox}%
      \fi
      % don't move
      \xdef\lastlayerwd{\wd\nextboxwd}%
      \xdef\lastlayerht{\ht\nextboxht}% % not entirely ok when grid !
      \xdef\lastlayerdp{\dp\nextboxdp}% % not entirely ok when grid !
      % this code
      \doifelse{\layerparameter\c!plaats}\v!grid\donetrue\donefalse
      \ifdone
        \nextboxht\strutheight
        \nextboxdp\strutdepth
      \else
        \setbox\nextbox\hbox{\alignedbox[\layerparameter\c!plaats]\vbox{\flushnextbox}}%
      \fi
      \ifnum\layerparameter\c!regel=\zerocount\else % no \ifcase, can be negative
        \advance\@@layerypos \layerparameter\c!regel\lineheight
        \advance\@@layerypos \topskip
        \advance\@@layerypos-\lineheight
        \advance\@@layerypos-\nextboxht
      \fi
      \ifnum\layerparameter\c!kolom=\zerocount\else % no \ifcase, can be negative
        \advance\@@layerxpos \layoutcolumnoffset{\layerparameter\c!kolom}%
      \fi
      \ifdone
        \setbox\nextbox\hbox{\alignedbox[rb]\vbox{\flushnextbox}}%
      \fi
      % ll registration
      \scratchdimen\@@layerxpos
      \advance\scratchdimen\@@layerxoff
      \ifdim\scratchdimen<\getboxllx\layerpagebox
        \gsetboxllx\layerpagebox\scratchdimen
      \fi
\ifcase\@@lacome\or % this test will become obsolete
      \advance\scratchdimen\nextboxwd
      \nextboxwd\ifdim\scratchdimen>\lastlayerwidth \scratchdimen \else \lastlayerwidth \fi
\fi
      \scratchdimen\@@layerypos
      \advance\scratchdimen\@@layeryoff
      \ifdim\scratchdimen<\getboxlly\layerpagebox
        \gsetboxlly\layerpagebox\scratchdimen
      \fi
      % ll compensation
\ifcase\@@lacome\or % this test will become obsolete
      \advance\scratchdimen\nextboxht
      \advance\scratchdimen\nextboxdp
      \nextboxht\ifdim\scratchdimen>\lastlayerheight \scratchdimen \else \lastlayerheight \fi
      \nextboxdp\zeropoint
\fi
      % placement
      \hsize\layerparameter\c!breedte % new, keep box small
      \vbox to \layerparameter\c!hoogte \bgroup
        \smashbox\nextbox
        \vskip\@@layerypos
        \vskip\@@layeryoff
        \hskip\@@layerxpos
        \hskip\@@layerxoff
        \flushnextbox
        \ifvoid\layerpagebox
          % already flushed
        \else
          % the reverse case % check !
          \vskip-\@@layerypos
          \vskip-\@@layeryoff
          \box\layerpagebox
        \fi
      \egroup}%
   % when position is true, the layerbox holds the compensation and needs
   % to be placed; never change this !
   \ifvoid\layerbox\else\box\layerbox\fi}

%D Given the task to be accomplished, the previous macro is
%D not even that complicated. It mainly comes down to skipping
%D to the right place and placing a box on top of or below the
%D existing content. In the case of position tracking, another
%D reference point is chosen.

%D \macros
%D  {doifelselayerdata}
%D

\def\doifelselayerdata#1%
  {\ifundefined{\@@layerbox#1}%
     \@EA\secondoftwoarguments
   \else\ifvoid\csname\@@layerbox#1\endcsname
     \@EAEAEA\secondoftwoarguments
   \else
     \@EAEAEA\firstoftwoarguments
   \fi\fi}

%D \macros
%D  {flushlayer}
%D
%D When we flush a layer, we flush both the main one and the
%D page dependent one (when defined). This feature is more
%D efficient in \ETEX\ since there testing for an undefined
%D macro does not takes hash space.

\unexpanded\def\flushlayer[#1]%
  {\doifelsevalue{\??ll#1\c!status}\v!volgende
      {\global\letvalue{\??ll#1\c!status}\v!start} % dangerous, stack-built-up
      {\doifelsevalue{\??ll#1\c!dubbelzijdig}\v!ja
         {\doifundefinedelse{\@@layerbox#1}%
            {\dodoflushlayerA[#1]}
            {\doifbothsidesoverruled
               \dodoflushlayerB\v!links [#1]%  left
             \orsideone
               \dodoflushlayerB\v!rechts[#1]% right
             \orsidetwo
               \dodoflushlayerB\v!links [#1]%  left
             \od}}
         {\dodoflushlayerA[#1]}}}

\def\dodoflushlayerA[#1]%
  {\doifnotvalue{\??ll#1\c!status}\v!stop
     {\startoverlay
        {\dodoflushlayer1{#1}{#1}}
        {\dodoflushlayer0{#1}{#1:\realfolio}}
      \stopoverlay}}

\def\dodoflushlayerB#1[#2]%
  {\doifnotvalue{\??ll#2\c!status}\v!stop
     {\startoverlay
        {\dodoflushlayer1{#2}{#2}}
        {\dodoflushlayer0{#2}{#2:\realfolio}}
        {\dodoflushlayer1{#2}{#1#2}}
        {\dodoflushlayer0{#2}{#1#2:\realfolio}}
      \stopoverlay}}

\def\dodoflushlayer#1#2#3%
  {\ifundefined{\@@layerbox#3}%
     \ifcase#1\else\writestatus{layer}{unknown layer #3}\fi
   \else
     \bgroup
     \forgetall
     \offinterlineskip
     \doifvalue{\??ll#2\c!optie}\v!test\tracelayerstrue
     \iftracelayers\traceboxplacementtrue\fi
     \!!doneafalse
     \!!donebfalse
     \doifvalue{\??ll#2\c!methode}\v!overlay\!!doneatrue
     \doifvalue{\??ll#2\c!methode}\v!passend\!!donebtrue
     \!!donectrue
     \ifcase#1\else
       \doifnotvalue{\??ll#2\c!positie}\v!ja
         {\doifvalue{\??ll#2\c!herhaal}\v!ja\!!donecfalse
          \doifvalue{\??ll#2\c!status}\v!herhaal\!!donecfalse}% old method
     \fi
     \chardef\layerbox\csname\@@layerbox#3\endcsname
     % we need to copy in order to retain the negative offsets for a next
     % stage of additions, i.e. llx/lly accumulate in repeat mode and the
     % compensation is may differ each flush depending on added content
     \setbox\nextbox \if!!doneb
       \vbox
         {\scratchdimen\getboxlly\layerbox
          \vskip-\scratchdimen
          \scratchdimen\getboxllx\layerbox
          \hskip-\scratchdimen
          \advance\scratchdimen-\wd\layerbox
          \hsize-\scratchdimen
          \if!!donec\box\else\copy\fi\layerbox}%
     \else
        \if!!donec\box\else\copy\fi\layerbox % sorry for the delay due to copying
     \fi
     \iftracelayers \ruledvbox \else \vbox \fi \if!!donea to \overlayheight \fi
       {\hbox \if!!donea to \overlaywidth \fi
          {% klopt dit? #3 en niet #2 ?
           \doifvalue{\??ll#3\realfolio\c!positie}\v!ja
             {\xypos{lyr:#3:\realfolio}}%
           \doifoverlayelse{#3}
             {\box\nextbox}
             {\startlayoutcomponent{l:#3}{layer #3}\box\nextbox\stoplayoutcomponent}%
           \hss}%
        \vss}%
     \if!!donec
       \gsetboxllx\layerbox\zeropoint
       \gsetboxlly\layerbox\zeropoint
     \fi
     \egroup
   \fi}

% \definelayer[test][method=fit] \setupcolors[state=start] \tracelayerstrue
%
% \framed[framecolor=red,offset=overlay]{\setlayer[test]{aa}\setlayer[test][x=10pt]{g}\flushlayer[test]}
% \framed[framecolor=red,offset=overlay]{\setlayer[test]{aa}\setlayer[test][x=-10pt]{bb}\flushlayer[test]}
% \framed[framecolor=red,offset=overlay]{\setlayer[test][x=-20pt]{cccccc}\flushlayer[test]}
% \framed[framecolor=red,offset=overlay]{\setlayer[test]{dd}\setlayer[test][x=-20pt,y=-3pt]{eeeeee}\flushlayer[test]}

%D \macros
%D  {composedlayer,placelayer,tightlayer}
%D
%D This is a handy shortcut, which saves a couple of braces
%D when we use it as parameter. This name also suits better
%D to other layering commands.

\def\composedlayer#1{\flushlayer[#1]}

\let\placelayer\flushlayer

\def\tightlayer[#1]%
  {\hbox
     {\def\currentlayer{#1}% todo: left/right
      \hsize\layerparameter\c!breedte
      \vsize\layerparameter\c!hoogte
      \composedlayer{#1}}}

%D \macros
%D  {resetlayer}
%D
%D This macro hardly needs an explanation (and is seldom
%D needed as well).

\def\doresetlayer#1%
  {\ifundefined{\@@layerbox#1}\else
     \global\setbox\csname\@@layerbox#1\endcsname\emptybox
   \fi}

\def\resetlayer[#1]%
  {\doresetlayer{#1}%
   \doifvalue{\??ll#1\c!dubbelzijdig}\v!ja % kind of redundant test
     {\doresetlayer{\v!links #1}%
      \doresetlayer{\v!rechts#1}}%
   \doresetlayer{#1:\realfolio}}

%D \macros
%D  {setMPlayer}
%D
%D The following layer macro uses the positions that are
%D registered by \METAPOST.
%D
%D \starttypen
%D \definelayer[test]
%D
%D \setMPlayer [test] [somepos-1] {Whatever we want here!}
%D \setMPlayer [test] [somepos-2] {Whatever we need there!}
%D \setMPlayer [test] [somepos-3] {\externalfigure[cow.mps][width=2cm]}
%D
%D \startuseMPgraphic{oeps}
%D   draw fullcircle scaled 10cm withcolor red ;
%D   register ("somepos-1",2cm,3cm,center currentpicture) ;
%D   register ("somepos-2",8cm,5cm,(-1cm,-2cm)) ;
%D   register ("somepos-3",0cm,0cm,(-2cm,2cm)) ;
%D \stopuseMPgraphic
%D
%D \getMPlayer[test]{\useMPgraphic{oeps}}
%D \stoptypen
%D
%D The last line is equivalent to
%D
%D \starttypen
%D \framed
%D   [background={foreground,test},offset=overlay]
%D   {\useMPgraphic{oeps}}
%D \stoptypen

\def\setMPlayer
  {\dotripleempty\dosetMPlayer}

\def\MPlayerwidth {\hsize}
\def\MPlayerheight{\vsize}

\def\dosetMPlayer[#1][#2][#3]%
  {\edef\MPlayerwidth {\MPw{#2}}%
   \edef\MPlayerheight{\MPh{#2}}%
   \setlayer[#1][\c!x=\MPx{#2},\c!y=\MPy{#2},\c!positie=\v!nee,#3]}

\def\getMPlayer
  {\dodoubleempty\dogetMPlayer}

\def\dogetMPlayer[#1][#2]%
  {\framed
     [\c!achtergrond={\v!voorgrond,#1},
      \c!kader=\v!uit,
      \c!offset=\v!overlay,#2]}

% Some day this (old) mechanism will be combined/integrated
% in overlays

\newskip\xposition  \newskip\yposition
\newskip\xdimension \newskip\ydimension
\newskip\xoffset    \newskip\yoffset

\newbox\positionbox

\def\startpositioning
  {\bgroup
   \xposition \zeropoint     \yposition \zeropoint
   \xdimension\zeropoint     \ydimension\zeropoint
   \xoffset   \zeropoint     \yoffset   \zeropoint
   \hfuzz     \papierbreedte \vfuzz     \papierhoogte
   \setbox\positionbox\hbox\bgroup}

\def\stoppositioning
  {\doifnot\@@psoffset\v!ja
     {\global\xoffset\zeropoint
      \global\yoffset\zeropoint}%
   \global\advance\xdimension \xoffset
   \global\advance\ydimension \yoffset
   \egroup
   \vbox to \ydimension
     {\vskip\yoffset
      \hbox to \xdimension
        {\hskip\xoffset
         \box\positionbox
         \hfill}
      \vfill}%
   \egroup}

\def\resetpositioning
  {\getparameters[\??ps]
     [\c!status=\v!start,%
      \c!eenheid=\s!cm,%
      \c!factor=1,%
      \c!schaal=1,%
      \c!xfactor=\@@psfactor,%
      \c!yfactor=\@@psfactor,%
      \c!xschaal=\@@psschaal,%
      \c!yschaal=\@@psschaal,%
      \c!xstap=\v!absoluut,%
      \c!ystap=\v!absoluut,%
      \c!xoffset=\!!zeropoint,%
      \c!yoffset=\!!zeropoint]}

\def\setuppositioning
  {\resetpositioning
   \dodoubleargument\getparameters[\??ps]}

\def\calculateposition#1#2#3#4#5#6#7#8#9%
  {\setdimensionwithunit\scratchskip{#1}\@@pseenheid
   \scratchskip#8\scratchskip
   \scratchskip#9\scratchskip
   \advance\scratchskip #4\relax
   \doif{#2}\v!relatief
     {\advance\scratchskip #3%
      \let#4\!!zeropoint}%
   #3\scratchskip\relax
   \doifnot\@@psstatus\v!overlay
     {\scratchskip#5\relax
      \advance\scratchskip #3\relax
      \ifdim#3<-#7\relax          \global#7-#3\relax          \fi
      \ifdim\scratchskip>#6\relax \global#6\scratchskip\relax \fi}}

\def\position
  {\dosingleempty\doposition}

\def\doposition[#1]#2(#3,#4)%
  {\dowithnextbox
     {\bgroup
      \getparameters[\??ps][#1]%
      \dontcomplain
      \calculateposition{#3}\@@psxstap\xposition
        \@@psxoffset{\nextboxwd}\xdimension\xoffset
        \@@psxschaal\@@psxfactor
      \scratchdimen\nextboxht \advance\scratchdimen \nextboxdp
      \calculateposition{#4}\@@psystap\yposition
        \@@psyoffset\scratchdimen\ydimension\yoffset
        \@@psyschaal\@@psyfactor
      \vbox to \zeropoint % kan beter.
        {\vskip\yposition
         \hbox to \zeropoint
           {\hskip\xposition
            \flushnextbox
            \hss}
         \vss}%
      \xdef\dopoppositioning
        {\xposition\the\xposition
         \yposition\the\yposition
         \noexpand\def\noexpand\@@psxoffset{\@@psxoffset}%
         \noexpand\def\noexpand\@@psyoffset{\@@psyoffset}}%
      \egroup
      \dopoppositioning
      \ignorespaces}
   \hbox}

\resetpositioning

\setuppositioning
  [\c!eenheid=\s!cm,
   \c!factor=1,
   \c!schaal=1,
   \c!xstap=\v!absoluut,
   \c!ystap=\v!absoluut,
   \c!offset=\v!ja,
   \c!xoffset=\!!zeropoint,
   \c!yoffset=\!!zeropoint]

\protect \endinput