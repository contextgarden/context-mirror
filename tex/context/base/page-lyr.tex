%D \module
%D   [       file=page-lyr,
%D        version=2000.10.20,
%D          title=\CONTEXT\ Page Macros,
%D       subtitle=Layers, 
%D         author=Hans Hagen,
%D           date=\currentdate,
%D      copyright={PRAGMA / Hans Hagen \& Ton Otten}]
%C
%C This module is part of the \CONTEXT\ macro||package and is
%C therefore copyrighted by \PRAGMA. See mreadme.pdf for
%C details.

\writestatus{loading}{Context Page Macros / Layers} 

\unprotect

%D The layering mechanism implemented here is independent of
%D the output routine, but future extensions may depend on a
%D more close cooperation. 

%D First we overload a macro from \type {core-rul}. From now on
%D we accept a third (optional) argument: the specific layer it
%D will go in. This means that we can move an overlay from one
%D background to the other using the dimensions of the parent.

\ifx\undefined\defineoverlay \message{loaded to early} \wait \fi 

\def\defineoverlay%
  {\dotripleempty\dodefineoverlay}

\def\dodefineoverlay[#1][#2][#3]%
  {\ifthirdargument
     \def\docommando##1%
       {\setvalue{\??ov##1}####1####2####3####4####5####6####7%
          {\setlayer[#3]{\executedefinedoverlay
             ##1\\#2\\{####1}{####2}{####3}{####4}{####5}{####6}{####7}}}}%
   \else
     \def\docommando##1%
       {\setvalue{\??ov##1}{\executedefinedoverlay
          ##1\\#2\\}}%
   \fi
   \processcommalist[#1]\docommando}

%D When tracing is turned on, a couple of boxes will 
%D show up as well as the reference point. 

\newif\iftracelayers % \tracelayerstrue

%D This handy constant saved some string memory. 

\def\@@layerbox{@@layerbox}

%D \macros 
%D   {definelayer}
%D
%D Each layer gets its own (global) box. This also means that 
%D the data that goes into a layer, is typeset immediately. 
%D Each layer automatically gets an associated overlay, 
%D which can be used in any background assignment.

\def\definelayer%
  {\dodoubleargument\dodefinelayer}

\def\dodefinelayer[#1][#2]%
  {\ifundefined{\@@layerbox#1}%
     \expandafter\newbox\csname\@@layerbox#1\endcsname
   \else
     \setbox\csname\@@layerbox#1\endcsname=\voidb@x
   \fi
   \defineoverlay
     [#1][\composedlayer{#1}]%
   \setuplayer
     [#1]
     [\c!status=\v!start,\c!richting=\v!normaal,\c!optie=,
      \c!x=\!!zeropoint,\c!y=\!!zeropoint,\c!positie=\v!nee,
      \c!breedte=\wd\nextbox,\c!hoogte=\ht\nextbox,
      \c!hoffset=\!!zeropoint,\c!voffset=\!!zeropoint,
      \c!plaats=rb,\c!positie=\v!nee,\c!pagina=,#2]}

%D \macros 
%D   {setuplayer}
%D
%D After a layer is defined, you can change its 
%D characteristics. 

\def\setuplayer%
  {\dodoubleargument\dosetuplayer}

\def\dosetuplayer[#1]%
  {\getparameters[\??ll#1]}

%D \macros 
%D   {setlayer}
%D 
%D Data is moved intoa layer with the following macro. When
%D \type {position} is set, relative positioning is used, with
%D the current point as reference point. Otherwise the topleft
%D corner is used as reference point. 
%D 
%D \starttypen 
%D \setlayer [identifier] [optional parameters] {data} 
%D \stoptypen 

\def\setlayer%
  {\dodoubleargument\dosetlayer}

\def\dosetlayer[#1][#2]%
  {\doifnotvalue{\??ll#1\c!status}{\v!stop}{\dodosetlayer[#1][#2]}}

\def\dodosetlayer[#1][#2]%
  {\bgroup
   \recalculatebackgrounds
   \recalculatelogos
   \doglobal\increment\currentlayerdata
   \forgetall
   \dontcomplain
   \doifvalue{\??ll#1\c!optie}{\v!test}{\tracelayerstrue}%
   \dowithnextbox % sneller als aparte macro
     {\ifundefined{\@@layerbox#1}%
        \writestatus{layer}{unknown layer #1}%
      \else
        \dododosetlayer[#1][#2]%
      \fi
      \egroup}
     \hbox}

\newbox\layerbox

\def\dododosetlayer[#1][#2]% 
  {\getparameters[\??ll#1][#2]%
   \doifvalue{\??ll#1\c!positie}{\v!overlay}
     {\getparameters[\??ll#1]
        [\c!positie=\v!ja,
         \c!breedte=\!!zeropoint,
         \c!hoogte=\!!zeropoint]}%
   \doifelsevalue{\??ll#1\c!positie}{\v!ja}
     {\edef\layerpage{\MPp{lyr:\currentlayerdata}}%
      \scratchdimen\MPx{lyr:#1:\layerpage}\scratchdimen-\scratchdimen
      \advance\scratchdimen\MPx{lyr:\currentlayerdata}%
      \setevalue{\??ll#1\c!x}{\the\scratchdimen}%
      \scratchdimen\MPy{lyr:#1:\layerpage}%
      \advance\scratchdimen-\MPy{lyr:\currentlayerdata}%
      \setevalue{\??ll#1\c!y}{\the\scratchdimen}%
      \setxvalue{\??ll#1\layerpage\c!positie}{\v!ja}%
      \setxvalue{\??ll#1\c!status}{\v!start}% needed ?
      \setbox\layerbox\vbox to \getvalue{\??ll#1\c!hoogte} % \ht\nextbox
        {\hbox to \getvalue{\??ll#1\c!breedte} % \wd\nextbox
           {\xypos{lyr:\currentlayerdata}\hss}%
         \vss}}
     {\setbox\layerbox=\box\voidb@x
      \edef\layerpage{\getvalue{\??ll#1\c!pagina}}}%
   \doifsomething{\layerpage}
     {\edef\layerpage{:\layerpage}%
      \doifundefined{\@@layerbox#1\layerpage}
        {\global\expandafter\newbox\csname\@@layerbox#1\layerpage\endcsname}}%
   \global\setbox\csname\@@layerbox#1\layerpage\endcsname=\vbox
     {\offinterlineskip
      \ifvoid\csname\@@layerbox#1\layerpage\endcsname\else
        \ht \csname\@@layerbox#1\layerpage\endcsname\!!zeropoint
        \dp \csname\@@layerbox#1\layerpage\endcsname\!!zeropoint
        \wd \csname\@@layerbox#1\layerpage\endcsname\!!zeropoint
        \doifnotvalue{\??ll#1\c!richting}{\v!omgekeerd}
          {\box\csname\@@layerbox#1\layerpage\endcsname}%
      \fi
      \setbox\nextbox=\hbox
        {\alignedbox[\getvalue{\??ll#1\c!plaats}]\vbox{\box\nextbox}}%
      \smashbox\nextbox
      \scratchdimen=\getvalue{\??ll#1\c!y}%
      \advance\scratchdimen\getvalue{\??ll#1\c!voffset}%
      \vskip\scratchdimen 
      \hskip\getvalue{\??ll#1\c!x}%
      \hskip\getvalue{\??ll#1\c!hoffset}%
      \box\nextbox % \hfill
      \ifvoid\csname\@@layerbox#1\layerpage\endcsname
        % already flushed 
       \else
        % the reverse case
        \vskip-\scratchdimen
        \box\csname\@@layerbox#1\layerpage\endcsname
      \fi}%
   \ifvoid\layerbox\else\box\layerbox\fi}

%D Given the task to be accomplished, the previous macro is
%D not even that complicated. It mainly comes down to skipping
%D to the right place and placing a box on top of or below the
%D existing content. In the case of position tracking, another
%D reference point is chosen. 

%D \macros
%D  {flushlayer} 
%D
%D When we flush a layer, we flush both the main one and the
%D page dependent one (when defined). This feature is more 
%D efficient in \ETEX\ since there testing for an undefined 
%D macro does not takes hash space. 

\def\flushlayer[#1]%
  {\doifvalue{\??ll#1\c!status}{\v!start}
     {\startoverlay
        {\doflushlayer1{#1}}
        {\doflushlayer0{#1:\realfolio}}
      \stopoverlay}}

\def\doflushlayer#1#2%
  {\ifundefined{\@@layerbox#2}%
     \ifcase#1\else\writestatus{layer}{unknown layer #2}\fi
   \else
     \doifvalue{\??ll#1\c!optie}{\v!test}\tracelayerstrue
     \iftracelayers \ruledvbox \else \vbox \fi to \overlayheight
       {\forgetall
        \hbox to \overlaywidth
          {\doifvalue{\??ll#2\realfolio\c!positie}{\v!ja}
             {\xypos{lyr:#2:\realfolio}}%
           \box\csname\@@layerbox#2\endcsname\hss}%
        \vss}%
   \fi}

%D \macros
%D  {composedlayer} 
%D
%D This is a handy shortcut, which saves a couple of braces 
%D when we use it as parameter. This name also suits better 
%D to other layering commands. 

\def\composedlayer#1{\flushlayer[#1]}

%D \macros
%D  {resetlayer} 
%D
%D This macro hardly needs an explanation (and is seldom 
%D needed as well). 

\def\doresetlayer#1%
  {\ifundefined{\@@layerbox#1}\else
     \global\setbox\csname\@@layerbox#1\endcsname\voidb@x
   \fi}

\def\resetlayer[#1]%
  {\doresetlayer{#1}\doresetlayer{#1:\realfolio}}

%D \macros
%D  {setMPlayer}
%D 
%D The following layer macro uses the positions that are 
%D registered by \METAPOST. 
%D
%D \starttypen
%D \definelayer[test] 
%D 
%D \setMPlayer [test] [somepos-1] {Whatever we want here!}
%D \setMPlayer [test] [somepos-2] {Whatever we need there!}
%D \setMPlayer [test] [somepos-3] {\externalfigure[cow.mps][width=2cm]}
%D 
%D \startuseMPgraphic{oeps}
%D   draw fullcircle scaled 10cm withcolor red ; 
%D   register ("somepos-1",2cm,3cm,center currentpicture) ; 
%D   register ("somepos-2",8cm,5cm,(-1cm,-2cm)) ; 
%D   register ("somepos-3",0cm,0cm,(-2cm,2cm)) ; 
%D \stopuseMPgraphic
%D 
%D \getMLlayer[test]{\useMPgraphic{oeps}}
%D \stoptypen
%D 
%D The last line is equivalent to 
%D
%D \starttypen
%D \framed
%D   [background={foreground,test},offset=overlay]
%D   {\useMPgraphic{oeps}}
%D \stoptypen

\def\setMPlayer%
  {\dotripleempty\dosetMPlayer}

\def\MPlayerwidth {\hsize}
\def\MPlayerheight{\vsize}

\def\dosetMPlayer[#1][#2][#3]%
  {\edef\MPlayerwidth {\MPw{#2}}%
   \edef\MPlayerheight{\MPh{#2}}%
   \setlayer[#1][\c!x=\MPx{#2},\c!y=\MPy{#2},\c!positie=\v!nee,#3]}

\def\getMPlayer
  {\dodoubleempty\dogetMPlayer}

\def\dogetMPlayer[#1][#2]%
  {\framed
     [\c!achtergrond={\v!voorgrond,#1},
      \c!kader=\v!uit,
      \c!offset=\v!overlay,#2]}

% Some day this (old) mechanism will be combined/integrated 
% in overlays 

\newskip\xposition  \newskip\yposition
\newskip\xdimension \newskip\ydimension
\newskip\xoffset    \newskip\yoffset

\newbox\positionbox

\def\startpositioning%
  {\bgroup
   \xposition \!!zeropoint   \yposition\!!zeropoint
   \xdimension\!!zeropoint   \ydimension\!!zeropoint
   \xoffset   \!!zeropoint   \yoffset   \!!zeropoint
   \hfuzz     \papierbreedte \vfuzz     \papierhoogte
   \setbox\positionbox\hbox\bgroup}

\def\stoppositioning%
  {\doifnot{\@@psoffset}{\v!ja}
     {\global\xoffset=\!!zeropoint
      \global\yoffset=\!!zeropoint}%
   \global\advance\xdimension by \xoffset
   \global\advance\ydimension by \yoffset
   \egroup
   \vbox to \ydimension
     {\vskip\yoffset
      \hbox to \xdimension
        {\hskip\xoffset
         \box\positionbox
         \hfill}
      \vfill}%
   \egroup}

\def\resetpositioning%
  {\getparameters[\??ps]
     [\c!status=\v!start,
      \c!eenheid=\s!cm,
      \c!factor=1,
      \c!schaal=1,
      \c!xfactor=\@@psfactor,
      \c!yfactor=\@@psfactor,
      \c!xschaal=\@@psschaal,
      \c!yschaal=\@@psschaal,
      \c!xstap=\v!absoluut,
      \c!ystap=\v!absoluut,
      \c!xoffset=\!!zeropoint,
      \c!yoffset=\!!zeropoint]}

\def\setuppositioning%
  {\resetpositioning
   \dodoubleargument\getparameters[\??ps]}

\def\calculateposition#1#2#3#4#5#6#7#8#9%
  {\setdimensionwithunit\scratchskip{#1}\@@pseenheid 
   \scratchskip=#8\scratchskip
   \scratchskip=#9\scratchskip
   \advance\scratchskip by #4\relax
   \doif{#2}{\v!relatief}%
     {\advance\scratchskip by #3%
      \let#4\!!zeropoint}%
   #3=\scratchskip\relax
   \doifnot{\@@psstatus}{\v!overlay}
     {\scratchskip=#5\relax
      \advance\scratchskip by #3\relax
      \ifdim#3<-#7\relax          \global#7=-#3\relax          \fi
      \ifdim\scratchskip>#6\relax \global#6=\scratchskip\relax \fi}}

\def\position%
  {\dosingleempty\doposition}

\def\doposition[#1]#2(#3,#4)%
  {\dowithnextbox
     {\bgroup
      \getparameters[\??ps][#1]%
      \dontcomplain
      \calculateposition{#3}\@@psxstap\xposition
        \@@psxoffset{\wd\nextbox}\xdimension\xoffset
        \@@psxschaal\@@psxfactor
      \scratchdimen=\ht\nextbox \advance\scratchdimen by \dp\nextbox
      \calculateposition{#4}\@@psystap\yposition
        \@@psyoffset\scratchdimen\ydimension\yoffset
        \@@psyschaal\@@psyfactor
      \vbox to \!!zeropoint % kan beter. 
        {\vskip\yposition
         \hbox to \!!zeropoint
           {\hskip\xposition
            \box\nextbox
            \hss}
         \vss}%
      \xdef\dopoppositioning%
        {\xposition=\the\xposition
         \yposition=\the\yposition
         \noexpand\def\noexpand\@@psxoffset{\@@psxoffset}%
         \noexpand\def\noexpand\@@psyoffset{\@@psyoffset}}%
      \egroup
      \dopoppositioning
      \ignorespaces}
   \hbox}

\resetpositioning

\setuppositioning 
  [\c!eenheid=\s!cm,
   \c!factor=1,
   \c!schaal=1,
   \c!xstap=\v!absoluut,
   \c!ystap=\v!absoluut,
   \c!offset=\v!ja,
   \c!xoffset=\!!zeropoint,
   \c!yoffset=\!!zeropoint]

\protect \endinput 
