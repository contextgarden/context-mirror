%D \module
%D   [       file=spec-tpd,
%D        version=1996.01.25,
%D          title=\CONTEXT\ Special Macros,
%D       subtitle=\PDFTEX,
%D         author=Hans Hagen,
%D           date=\currentdate,
%D      copyright={PRAGMA / Hans Hagen \& Ton Otten}]
%C
%C This module is part of the \CONTEXT\ macro||package and is
%C therefore copyrighted by \PRAGMA. Non||commercial use is
%C granted.

%D Being one of the first typographical systems able to support
%D advances \PDF\ support, \TEX\ is also one of the first
%D systems to produce high quality \PDF\ code directly. Thanks
%D to Han The Thanh c.s. the \TEX\ community can leap forward
%D once again. 
%D 
%D One important characteristic of \PDFTEX\ is that is can
%D produce standard \DVI\ code as well as \PDF\ code. This
%D enables us to use one format file to support both output
%D formats. 
%D 
%D All modules in this group use specials to tell drivers what
%D non||\TEX\ actions to take. Because from the \TEX\ point of
%D view, there is no difference between \DVI\ and \PDF, we
%D therefore only have to bend the \DVI\ driver support into
%D \PDF\ support. Technically spoken, specials no longer serve
%D a purpose, except from ending up as comment in the \PDF\
%D file. The core primitive in this module therefore is the
%D \PDFTEX\ primitive \type{\pdfliteral}. 
%D 
%D Before we continue we need to make sure if indeed those
%D \PDFTEX\ primitives are permitted. If no primitives are
%D available, we just stop reading any further. 

\ifx\pdfoutput\undefined \endinput \else \unprotect \fi 

%D Once we are sure that we're indeed supporting \PDFTEX, we
%D force \PDF\ output at the highest compression. For debugging
%D purposes one can set the compresslevel to~0. 

\pdfoutput       =1
\pdfcompresslevel=9

%D Now we have to make sure no other specials are supported, 
%D else \PDFTEX\ will keep on telling us that we're wrong. 

\unprotected \usespecials[\v!reset]

%D Just in case we mimmick specials, we have to make sure no 
%D default specials end up in the process. 

\let\defaultspecial=\gobbleoneargument

%D Having reset all the special support, we have to define 
%D all needed and possible support in this module. 

\startspecials[tpd]

%D This means that by saying 
%D
%D \starttypen
%D \usespecials[tpd]
%D \stoptypen
%D
%D we get ourselves \PDF\ output. 

%D \macros
%D   {dosetuppaper}
%D
%D If we don't set the paper size, \PDFTEX\ will certainly do 
%D it in a way we don't want, therefore we need: 

\def\dosetuppdftexpaper#1#2#3%
  {\global\pdfpagewidth =#2\relax  
   \global\pdfpageheight=#3\relax
   \global\let\dosetuppdftexpaper=\gobblethreearguments} 

\definespecial\dosetuppaper#1#2#3%
  {\dosetuppdftexpaper{#1}{#2}{#3}}

%D \macros
%D   {doinsertfile}
%D
%D Graphics are not part of \TEX\ and therefore not part of the
%D \DVI\ standard. \PDF\ on the other hand has several graphic
%D primitives. During the multi||step process \TEX\
%D $\rightarrow$ \DVI\ $\rightarrow$ \POSTSCRIPT\ $\rightarrow$
%D \PDF\ one can insert graphics using specials. In \PDFTEX\
%D however there is only one step! This means that \PDFTEX\
%D itself has to do the inclusion. 
%D 
%D At the moment \PDFTEX\ supports inclusion of bitmap \PNG\ 
%D graphics as well as not too complicated \PDF\ code. Using 
%D this last option, we are able to include both \METAPOST\ and 
%D \PDF\ output produced by \GHOSTSCRIPT. 
%D 
%D We fall back on the generic \CONTEXT\ module supp-pdf to
%D accomplish \PDF\ inclusion. The methods implemented there
%D are hooked into both the figure placement mechanisms of
%D \CONTEXT\ and the specials inclusion mechanism. 

\definespecial\doinsertfile#1#2#3#4#5#6#7#8#9% 
  {\bgroup
   \dodoinsertfile{tpd}{#1}{#2}{#3}{#4}{#5}{#6}{#7}{#8}{#9}%
   \egroup} 

%D The three methods supported for the moment are \type{mps}
%D for \METAPOST\ graphics, \type{pdf} for \GHOSTSCRIPT\ \PDF\
%D code, and \type{png} for bitmap graphics. 

\def\dotpdinsertmps#1#2#3#4#5#6#7#8%
  {\scratchdimen=#2pt \PointsToReal{.01\scratchdimen}\xscale
   \scratchdimen=#3pt \PointsToReal{.01\scratchdimen}\yscale
   \convertMPtoPDF{#1}\xscale\yscale}

\def\dotpdinsertpdf#1#2#3#4#5#6#7#8%
  {\beforesplitstring#1\at.\to\filename
   \scratchdimen=#2pt \PointsToReal{.01\scratchdimen}\xscale
   \scratchdimen=#3pt \PointsToReal{.01\scratchdimen}\yscale
   \convertPDFtoPDF{\filename.pdf}\xscale\yscale{#4sp}{#5sp}{#6sp}{#7sp}}

\def\dotpdinsertpng#1#2#3#4#5#6#7#8%
  {\pdfimage width #6sp height #7sp #1\relax}

%D Some surrogate \type{tif} and \type{eps} support is 
%D provided too. These work only when the size compatible 
%D \type{png} and \type{pdf} alternatives are present.

\def\dotpdinserttif#1#2#3#4#5#6#7#8%
  {\beforesplitstring#1\at.\to\filename
   \pdfimage width #6sp height #7sp \filename.png\relax}

\def\dotpdinserteps%
  {\dotpdinsertpdf}

%D \PDF\ supports the inclusion of video movies. In \CONTEXT\
%D we support these in a way similar to figure inclusion. 

\def\dotpdinsertmov#1#2#3#4#5#6#7#8%
  {\ScaledPointsToBigPoints{#6}\width
   \ScaledPointsToBigPoints{#7}\height
   \edef\pdf@@posterize{\ifcase#8 \or/Poster true\fi}%
   \pdfannotlink
     width #6sp
     height #7sp
     attr {/Border [0 0 0]}
     user {/Subtype /Movie
           /Movie <</F (#1) /Aspect [\width\space \height\space] \pdf@@posterize>>
           /A <</ShowControls false>>}%
   \pdfendlink}

%D \macros
%D  {doovalbox}
%D
%D For drawing ovals we use quite raw \PDF\ code. The next 
%D implementation does not differ that much from the one 
%D implemented in the \POSTSCRIPT\ driver. 

\def\dosomeovalcalc#1#2#3%
  {\dimen2=#1sp
   \advance\dimen2 by #2\relax
   \ScaledPointsToBigPoints{\number\dimen2}#3}

\definespecial\doovalbox#1#2#3#4#5#6#7%
  {\bgroup
   \dimen0=#4sp\divide\dimen0 by 2
   \dosomeovalcalc{0} {+\dimen0}\xmin
   \dosomeovalcalc{#1}{-\dimen0}\xmax
   \dosomeovalcalc{#2}{-\dimen0}\ymax
   \dosomeovalcalc{#3}{+\dimen0}\ymin
   \advance\dimen0 by #5sp
   \dosomeovalcalc{0} {+\dimen0}\xxmin
   \dosomeovalcalc{#1}{-\dimen0}\xxmax
   \dosomeovalcalc{#2}{-\dimen0}\yymax
   \dosomeovalcalc{#3}{+\dimen0}\yymin
   \dosomeovalcalc{#4}{0pt}\stroke
   \dosomeovalcalc{#5}{0pt}\radius
   \edef\dostroke{#6}%
   \edef\dofill{#7}%
   \vbox
     \bgroup
     \offinterlineskip
     \forgetall
     \hsize\!!zeropoint
     \vrule\!!width\!!zeropoint\!!height#2sp\!!depth#3sp\relax
     \pdfliteral
       {q
        \stroke\space w
        \xxmin\space \ymin\space  m
        \xxmax\space \ymin\space  l
        \xmax\space  \ymin\space  \xmax\space  \yymin\space y
        \xmax\space  \yymax\space l
        \xmax\space  \ymax\space  \xxmax\space \ymax\space  y
        \xxmin\space \ymax\space  l
        \xmin\space  \ymax\space  \xmin\space  \yymax\space y
        \xmin\space  \yymin\space l
        \xmin\space  \ymin\space  \xxmin\space \ymin\space  y
        \ifnum\dostroke=1 S \fi
        \ifnum\dofill  =1 f \fi
        Q}%
     \egroup
   \egroup}

%D \macros
%D   {dostartgraymode,dostopgraymode,
%D    dostartrgbcolormode,dostartcmykcolormode,dostartgraycolormode,
%D    dostopcolormode}
%D 
%D In \PDF\ there are two color states, one for strokes and one
%D for fills. This means that we have to set the color in a
%D rather redundant looking way. Unfortunately this makes the
%D \PDF\ file much larger than needed. 

\definespecial\dostartgraymode#1%
  {\pdfliteral{#1 g #1 G}}

\definespecial\dostopgraymode%
  {\pdfliteral{0 g 0 G}}

\definespecial\dostartrgbcolormode#1#2#3%
  {\pdfliteral{#1 #2 #3 rg #1 #2 #3 RG}}

\definespecial\dostartcmykcolormode#1#2#3#4%
  {\pdfliteral{#1 #2 #3 #4 k #1 #2 #3 #4 K}}

\definespecial\dostartgraycolormode#1%
  {\pdfliteral{#1 g #1 G}}

\definespecial\dostopcolormode%
  {\pdfliteral{0 g 0 G}}

%D \macros
%D  {dostartrotation,dostoprotation}
%D
%D Rotating some text can be accomplished by setting the first 
%D four elements of the transform matrix. We only support some 
%D fixed angles. The q's take care of grouping. 

\definespecial\dostartrotation#1%
  {\processaction
     [#1]
     [ 90=>\pdfliteral{q  0  1 -1  0 0 0 cm},
      180=>\pdfliteral{q -1  0  0 -1 0 0 cm},
      270=>\pdfliteral{q  0 -1  1  0 0 0 cm},
      360=>\pdfliteral{q  1  0  0  1 0 0 cm}]}

\definespecial\dostoprotation%
  {\pdfliteral{Q}}

%D \macros 
%D   {dosetupinteraction}
%D
%D Nothing special is needed to enable \PDF\ commands and 
%D interaction. We stick with a message.

\definespecial\dosetupinteraction%
  {\showmessage{\m!interactions}{21}{pdftex}}

%D \macros 
%D   {dostartthisisrealpage,dostartthisislocation
%D    dostartgotorealpage,dostartgotolocation}
%D 
%D The interactions macros are the core of this module. We
%D support both page destinations and named ones. 
%D 
%D {\em For the moment we use object number (that is, behind
%D the screens, the user uses his own numbers) destinations
%D instead of page ones. The latter works, but not 100\%.} 

\definespecial\dostartthisisrealpage#1%
  {\pdfdest num #1 fit}  % will be {} when page is ok

\definespecial\dostartthisislocation#1%
  {\ifusepagedestinations \else
     \setPDFdestination{#1}%
     \doifsomething{\PDFdestination}
       {\pdfdest name {\PDFdestination} fit}%
   \fi}

%D When going to a location, we obey the time and space saving 
%D boolean\type{\ifusepagedestination}. Names destinations are 
%D stripped and made robust. 

\definespecial\dostartgotolocation#1#2#3#4#5#6% url nog afhandelen 
  {\bgroup
   \doifelsenothing{#3}
     {\doifelsenothing{#4}
        {\!!doneafalse}
        {\doifparentfileelse{#4}
          {\!!doneafalse}
          {\!!doneatrue}}%
      \ifusepagedestinations 
        \scratchcounter=0#6\relax
        \edef\PDFdestination{\the\scratchcounter}%
        \pdfannotlink
          width #1sp  
          height #2sp
          depth 0pt
          attr {/Border [0 0 0]}
        % goto \if!!donea file {#4.pdf} \fi page \PDFdestination\space {/Fit} 
          goto \if!!donea file {#4.pdf} page 1 {/Fit}\else num \PDFdestination\space \fi 
        \pdfendlink
      \else
        \setPDFdestination{#5}%
        \doifsomething{\PDFdestination}
          {\pdfannotlink
             width #1sp  
             height #2sp
             depth 0pt
             attr {/Border [0 0 0]}
             goto \if!!donea file {#4.pdf} \fi name {\PDFdestination}%
           \pdfendlink}%
      \fi}
     {\doifelsenothing{#4}
        {\let\PDFfile=\empty
         \let\PDFdestination=\empty}
        {\edef\PDFfile{/#4}%
         \setPDFdestination{#5}%
         \doifsomething{\PDFdestination}
           {\edef\PDFdestination{\URLhash\PDFdestination}}}%
      \pdfannotlink 
        width #1sp  
        height #2sp
        depth 0pt
        attr {/Border [0 0 0]}
        user {/S /URI 
              /URI (#3\PDFfile\PDFdestination)}%
      \pdfendlink}%
    \egroup}

\definespecial\dostartgotorealpage#1#2#3#4#5%
  {\bgroup
   \doifelsenothing{#3}
     {\doifelsenothing{#4}
        {\!!doneafalse}
        {\doifparentfileelse{#4}
           {\!!doneafalse}
           {\!!doneatrue}}%
      \scratchcounter=0#5\relax
      \edef\PDFdestination{\the\scratchcounter}%
      \pdfannotlink
        width #1sp
        height #2sp
        depth 0pt
        attr {/Border [0 0 0]}
      % goto \if!!donea file {#4.pdf} \fi page \PDFdestination\space {/Fit} 
        goto \if!!donea file {#4.pdf} page 1 {/Fit} \else num \PDFdestination\space \fi 
      \pdfendlink}
     {\doifelsenothing{#4}
        {\let\PDFfile=\empty}
        {\edef\PDFfile{/#4}}%
      \pdfannotlink 
        width #1sp  
        height #2sp
        depth 0pt
        attr {/Border [0 0 0]}
        user {/S /URI 
              /URI (#3\PDFfile)}%
      \pdfendlink}%
   \egroup}

%D \macros 
%D   {dostarthide,dostophide}
%D 
%D Hiding parts of the document for printing is not yet
%D supported by \PDF\ and therefore \PDFTEX. 

\definespecial\dostarthide%
  {}

\definespecial\dostophide%
  {}

%D \macros
%D   {dosetupscreen}
%D 
%D Setting of the screen boundingbox involves some
%D calculations. Here we also take care of (non) full screen
%D startup. The dimensions are rounded. 

\definespecial\dosetupscreen#1#2#3#4#5%
  {\bgroup
   \!!widtha=#3sp
   \advance\!!widtha by #1sp
   \!!heighta=-#4sp
   \!!heightb=\pdfpageheight 
   \advance\!!heightb by -#2sp
   \advance\!!heighta by \!!heightb
   \ScaledPointsToWholeBigPoints{#1}\left
   \ScaledPointsToWholeBigPoints{\number\!!heighta}\bottom
   \ScaledPointsToWholeBigPoints{\number\!!widtha}\width
   \ScaledPointsToWholeBigPoints{\number\!!heightb}\height
   \expanded{\global\noexpand\pdfpagesattr=
     {/CropBox [\left\space\bottom\space\width\space\height]}}%
   \ifcase#5\else
     \pdfcatalog pagemode {/FullScreen}\relax
   \fi
   \egroup}

%D \macros
%D   {dostartexecutecommand}
%D 
%D \PDF\ viewers enable us to navigate using menus and shortcut
%D keys. These navigational tools can also be accessed by using
%D annotations. The next special takes care of inserting them. 

\definespecial\dostartexecutecommand#1#2#3#4%
  {\bgroup
   \processaction
     [#3]
     [     first=>\def\command{First},
        previous=>\def\command{Prev},
            next=>\def\command{Next},
            last=>\def\command{Last},
        backward=>\def\command{GoBack},
         forward=>\def\command{GoForward},
           print=>\def\command{Print},
            exit=>\def\command{Quit},
           close=>\def\command{Close},
            help=>\def\command{HelpUserGuide},
            swap=>\def\command{FullScreen},
      \s!unknown=>\let\command=\s!unknown]%
   \pdfannotlink
     width #1sp
     height #2sp
     depth 0pt
     attr {/Border [0 0 0]}
     user {/S /Named /N /\command}%
   \pdfendlink
   \egroup}

%D \macros
%D   {dosetupidentity}
%D 
%D Documents can be tagged with an application accessible title
%D and subtitle, the authorname, a date, the creator, keywords
%D etc. For the moment \PDFTEX\ only supports the first three
%D of these. 

\definespecial\dosetupidentity#1#2#3#4#5%
  {\pdfinfo title {#1} subject {#2} author {#3}\relax} % creator {#4}

%D \macros
%D   {dostartrunprogam}
%D
%D Although possible, running applications is not yet
%D implemented here. 
     
\definespecial\dostartrunprogram#1#2#3%
  {}

%D \macros
%D   {dostartgotoprofile, dostopgotoprofile,
%D    dobeginofprofile, doendofprofile}
%D 
%D \CONTEXT\ user profiles and version control fall back on
%D \PDF\ article threads. Unfortunately one cannot influence
%D the view yet in an (for me) acceptable way. 

\definespecial\dostartgotoprofile#1#2#3%
  {\pdfannotlink
     width #1sp
     height #2sp
     depth 0pt
     attr {/Border [0 0 0]}
     thread name {#3}%
   \pdfendlink}

\definespecial\dobeginofprofile#1#2#3%
  {\doifsomething{#1}
     {\pdfthread name {#1}}}

\definespecial\doendofprofile#1#2#3%
  {\pdfendthread}

%D \macros
%D  {doinsertbookmark}
%D
%D In \PDF\ bookmarks are the building blocks of a viewer 
%D provided sort of table of contents. \TEX\ has to provide 
%D the entry as well as the number of child entries. Strings 
%D need to be sanatized as good as possible to suit the default 
%D encoding. In \CONTEXT\ users can overrule this string by 
%D supplying an alternative one. 

\definespecial\doinsertbookmark#1#2#3#4#5% level sublevels text page open (1)
  {\sanitizePDFstring#3\to\bookmarktext
   \pdfoutline
     goto num #4 % why's page not accepted 
     \ifnum#2>0 count \ifcase#5-\fi#2\fi\space 
     {\bookmarktext}}

%D \macros
%D  {dostartobject,dostopobject,doinsertobject}
%D
%D Due to \PDF's object oriented character, we can include and 
%D reuse objects. These can be compared with \TEX's boxes. The 
%D \TEX\ counterpart is defined in the module \type{spec-dvi}. 
%D We don't use the dimensions here. 

% Forms can interfere with page dimension settings. Therefore
% calling macros can best postpone flushing. 

\definespecial\dostartobject#1#2#3#4%
  {\setbox\nextbox=\vbox\bgroup
     \def\dodostopobject%
       {\egroup
        \pdfform\nextbox
        \scratchcounter=\pdflastform
        \setxvalue{pdfform:#1}{\the\scratchcounter}}}

\definespecial\dostopobject%
  {\dodostopobject}

\definespecial\doinsertobject#1%
  {\expandafter\pdfrefform\csname pdfform:#1\endcsname\relax}

\stopspecials

\protect 

\endinput
