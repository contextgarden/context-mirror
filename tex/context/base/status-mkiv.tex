\setupbodyfont[dejavu,10pt]

\setuplayout
  [width=middle,
   height=middle,
   backspace=1cm,
   topspace=1cm,
   footer=0pt,
   header=1.25cm]

\setuphead
  [subject]
  [style=\bfa]

\setuppagenumbering
  [location=]

\setupheadertexts
  [\currentdate][MkIV cleanup Status / Page \pagenumber]

\starttext

\startsubject[title=Todo]

\startitemize[packed]
    \startitem currently the new namespace prefixes are not consistent but this
               will be done when we're satisfied with one scheme \stopitem
    \startitem there will be additional columns in the table, like for namespace
               so we need another round of checking then \stopitem
    \startitem the imp modules are not in the list and needs checking too \stopitem
    \startitem the s, x, m modules will be checked, redone and reorganized \stopitem
    \startitem the lua code will be cleaned up upgraded as some is quite old
               and experimental \stopitem
    \startitem we need a proper dependency tree and better defined loading order \stopitem
    \startitem all dotag.. will be moved to the tags_.. namespace \stopitem
    \startitem we need to check what messages are gone (i.e.\ clean up mult-mes) \stopitem
    \startitem some commands can go from mult-def (and the xml file) \stopitem
    \startitem check for setuphandler vs simplesetuphandler \stopitem
    \startitem all showcomposition etc can go (we can redo that in lua if needed) \stopitem
\stopitemize

\stopsubject

\startsubject[title=Status]

\startluacode

    local coremodules = dofile("status-mkiv.lua")

    if coremodules then

        local preloaded = coremodules.preloaded

        if preloaded then

            context.starttabulate { "|Tr|Tl|Tl|l|p|" }
            context.NC() -- context.bold("order")
            context.NC() context.bold("file")
            context.NC() context.bold("mark")
            context.NC() context.bold("status")
            context.NC() context.bold("comment")
            context.NC() context.NR()
            for i=1,#preloaded do
                local module = preloaded[i]
                local status = module.status
                context.NC() context(i)
                context.NC() context(module.filename)
                context.NC() context(module.marktype)
                context.NC() if status == "unknown" then context.italic(status) else context(status) end
                context.NC() context(module.comment)
                context.NC() context.NR()
            end
            context.stoptabulate()

        end

    end

\stopluacode

\stopsubject

\stoptext
