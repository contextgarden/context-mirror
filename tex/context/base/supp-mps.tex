%D \module
%D   [       file=supp-mps,
%D        version=1997.07.05,
%D          title=\CONTEXT\ Support Macros,
%D       subtitle=\METAPOST\ Inclusion,
%D         author=Hans Hagen,
%D           date=\currentdate,
%D      copyright={PRAGMA / Hans Hagen \& Ton Otten}]
%C
%C This module is part of the \CONTEXT\ macro||package and is
%C therefore copyrighted by \PRAGMA. Non||commercial use is
%C granted.

% E-tex   : a primitive that tells us that \write18 works
% E-tex   : \executeMetaPost filename
% ConTeXt : automatic flush at end of job
 
%D \METAPOST\ is John Hobbys alternative for \METAFONT\ and
%D produces superior \POSTSCRIPT\ code. In this module we
%D integrate \METAPOST\ support int \CONTEXT.  We offer two
%D tracks:
%D
%D \startopsomming
%D \som generating \METAPOST\ code, running this program from
%D      within \TEX\ using \type{\write18}, and importing the
%D      result
%D \som generating \METAPOST\ code, processing the code
%D      afterward, and importing the result in a second pass
%D \stopopsomming
%D
%D The first approach uses a non standard \TEX\ feature,
%D implemented in Web2c. I'm not going to discuss the pros and
%D cons of running programs from within others, but all
%D arguments against this can be overcome by implementing a
%D \TEX\ worthy primitive:
%D
%D \starttypen
%D \excuteMetaPost filename
%D \stoptypen
%D
%D Ok then, let's start:

\writestatus{loading}{Context Support Macros / MetaPost Inclusion}

\unprotect

%D \macros
%D   {startMPgraphic}
%D
%D From within \TEX\ one can execute \METAPOST\ code by putting
%D it between the two commands
%D
%D \starttypen
%D \startMPgraphic
%D \stopMPgraphic
%D \stoptypen
%D
%D This is implemented as:

\def\startMPgraphic#1\stopMPgraphic%
  {\startwritingMPgraphic
   \writeMPgraphic{#1}%
   \stopwritingMPgraphic}

%D \macros
%D   {startwritingMPgraphic,
%D    writeMPgraphic,
%D    stopwritingMPgraphic}
%D
%D If the writing process is divided into more steps, one can
%D use the components of this macro directly.
%D
%D \starttypen
%D \startwritingMPgraphic
%D ...
%D \writeMPgraphic{...}
%D ...
%D \writeMPgraphic{...}
%D ...
%D \stopwritingMPgraphic
%D \stoptypen

%D \macros
%D   {ifrunMPgraphics}
%D
%D These macros look a bit more complicated that one would
%D expect at first sight. This is due to the two ways of
%D processing these graphics, mentioned in a previous
%D paragraph. Which method is used, the direct or indirect
%D one, depends on a boolean.

\newif\ifrunMPgraphics

%D If set to true, one can do with a single pass, else one must
%D process the \METAPOST\ file \type{mpgraph} between two
%D succesive \TEX\ runs.

\def\MPgraphicfile{mpgraph}

%D When we run \METAPOST\ from within \TEX, each graphic is 
%D processed at once, which means that we reuse this file many 
%D times. When however the execution is delayed, all graphics 
%D are saved in a separate figure. The current graphic is 
%D characterized bij a number:

\newcounter\currentMPgraphic

%D \macros
%D   {ifreuseMPgraphics}
%D
%D If one want to reuse grapics, one can save much redundant
%D run time by setting the next switch to true. 

\newif\ifreuseMPgraphics

%D The three macros responsible for writing the graphic 
%D implement both schemes. 

%D \macro
%D   {MPinclusions}
%D
%D One can include for instance common input commands by 
%D assigning those to the token register:

\newtoks\MPinclusions

\def\writeMPgraphic%
  {\immediate\write\scratchwrite}

\def\startwritingMPgraphic%
  {\ifrunMPgraphics
     \ifreuseMPgraphics \else
       \doglobal\newcounter\currentMPgraphic 
     \fi
     \doglobal\increment\currentMPgraphic 
     \immediate\openout\scratchwrite=\MPgraphicfile.mp
     \immediate\write\scratchwrite{\the\MPinclusions}%
   \else
     \doglobal\increment\currentMPgraphic 
     \ifnum\currentMPgraphic=1
       \immediate\openout\scratchwrite=\MPgraphicfile.mp
       \immediate\write\scratchwrite{\the\MPinclusions}%
     \fi
   \fi
   \immediate\write\scratchwrite{beginfig(\currentMPgraphic);}%
   \global\let\flushMPgraphics\dodostopwritingMPgraphic
   \global\let\stopwritingMPgraphic=\dostopwritingMPgraphic}

\def\dostopwritingMPgraphic%
  {\immediate\write\scratchwrite{endfig;}%
   \ifrunMPgraphics
     \dodostopwritingMPgraphic
   \fi}

\def\dodostopwritingMPgraphic%
  {\ifnum\currentMPgraphic>0
     \immediate\write\scratchwrite{end.}%
     \immediate\closeout\scratchwrite
     \runMPgraphic{\MPgraphicfile}%
   \fi
   \global\let\flushMPgraphics=\relax}

\let\stopwritingMPgraphic=\relax
\let\flushMPgraphics     =\relax

%D \macros
%D   {flushMPgraphics}
%D
%D When we use the indirect method, all graphics are saved in
%D one file. This means that we cannot close this file after 
%D every \type{\stopMPgraphic}. Therefore we need to say:
%D
%D \starttypen
%D \flushMPgraphic
%D \stoptypen
%D 
%D else the file is closed without writing the \METAPOST\ end
%D command. One will notice this fast enough when in indirect 
%D mode. When using the direct mode this command is not 
%D implicitly needed, but ommiting it makes files less 
%D portable.

%D \macros
%D   {loadcurrentMPgraphic,
%D    placeMPgraphic}
%D
%D Once defined, we can call for this graphic by saying:
%D
%D \starttypen
%D \loadcurrentMPgraphic{setups}
%D \placeMPgraphic
%D \stoptypen
%D
%D This two stage insert permits some intermediate manipulations
%D of the graphic, which temporary saved in:

\newbox\MPgraphic

\def\loadcurrentMPgraphic#1%
  {\loadMPgraphic{\MPgraphicfile.\currentMPgraphic}{#1}}

\def\loadMPgraphic#1#2%
  {\setbox\MPgraphic=\hbox{\insertMPfile{#1}{#2}}}

\def\placeMPgraphic%
  {\box\MPgraphic}

%D \macros
%D   {startreusableMPgraphic, reuseMPgraphic, useMPbox}
%D 
%D One can use the next macro for defining graphics that are
%D to be reused. When the next switch is set, graphics are 
%D cached. 
 
\newif\ifuseMPbox % nog eens cyclische buffer

\long\def\startreusableMPgraphic#1#2\stopreusableMPgraphic%
  {\reuseMPgraphicstrue
   \doifundefined{MP:#1}
     {\startMPgraphic#2\stopMPgraphic
      \ifuseMPbox
        \ifx\setobject\undefined
          \newbox\somebox
          \global\setbox\somebox=\vbox
            {\loadMPgraphic{\MPgraphicfile.\currentMPgraphic}{}%
             \placeMPgraphic}%
          \global\setevalue{MP:#1}%
            {\copy\the\somebox\relax}%
        \else
          \setobject{MP:#1}
            \vbox
              {\loadMPgraphic{\MPgraphicfile.\currentMPgraphic}{}%
               \placeMPgraphic}%
          \global\setvalue{MP:#1}%
            {\getobject{MP:#1}}%
        \fi
      \else
        \global\setevalue{MP:#1}%
          {\noexpand\loadMPgraphic{\MPgraphicfile.\currentMPgraphic}{}%
           \noexpand\placeMPgraphic}%
      \fi}}

\def\reuseMPgraphic#1%
  {\getvalue{MP:#1}}

%D \macro
%D   {startuseMPgraphic,useMPgraphic}
%D
%D The every||time||it's||used original one is defined below.
%D This one makes sense when the graphic uses random numbers.

\def\startuseMPgraphic#1#2\stopuseMPgraphic%
  {\reuseMPgraphicstrue 
   \long\setvalue{MP:#1}%
     {\startMPgraphic#2\stopMPgraphic 
      \loadcurrentMPgraphic{}%
      \placeMPgraphic}}

\let\useMPgraphic=\reuseMPgraphic

%D We didn't yet define the macro responsible for processing 
%D the graphic from within \TEX. 

\def\runMPgraphic#1%
  {\ifrunMPgraphics
     \executeMETAPOST{#1}%
   \else 
     \message{[flush and process \MPgraphicfile.mp afterwards]}%
   \fi}

%D \macros
%D   {executeMetaPost, executeMETAPOST, executesystemcommand}
%D
%D With \type{\executeMETAPOST} being defined as:
 
\ifx\undefined\executeMetaPost
  \def\executeMETAPOST#1{\executesystemcommand{\executeMetaPost{#1}}}
\fi

%D There are two system dependant definitions: 

\ifx\undefined\executesystemcommand 
  \def\executesystemcommand#1{\immediate\write18{#1}}
\fi

\ifx\undefined\executeMetaPost
  \def\executeMetaPost#1{mpost #1}
\fi

%D \macros
%D   {insertMPfile}
%D
%D One can define this command in advance or redefine it after
%D loading this module. The same goes for the forward 
%D reference to the figure loading macro:

\ifx\undefined\insertMPfile 

  \def\insertMPfile#1#2%
      {\ifx\undefined\externalfigure 
         \message{[insert file #1 here]}%
       \else
         \externalfigure[#1][\c!type=eps,\c!methode=mps,#2]%
       \fi}

\fi

%D This macro takes {\em two} arguments, the second one can be
%D used to pass info to the inclusion macro. Some examples 
%D of its use can be found in the modules \type{supp-tpi} and 
%D \type{prag-log}. 

%D For some reason, \METAPOST\ needs the public domain \DVI\ to
%D \POSTSCRIPT\ converter \DVIPS. This symbiosis originates in
%D the need to include the fonts (glyphs) that \METAPOST\ uses
%D in the \POSTSCRIPT\ file. Driver independancy was one of my
%D prerequisites for using \METAPOST, so I decided to build
%D this kind of support myself. Personally I consider driver
%D dependancy a drawback for the dissemination of such a
%D package. The second part of this module more or less 
%D decouples \METAPOST\ and \DVIPS. 
%D 
%D The macros hereafter are copied from the module
%D \type{m-metapost}. After writing module \type{supp-pdf} I
%D added this method to the module named and after a while
%D decided to hook it into module \type{spec-yy}. Therefore 
%D they made it into a support module, but in a slightly 
%D different way. 

%D \macros
%D   {UseMetaPostGraphic, DontUseMetaPostGraphics}
%D   {}
%D 
%D The method we use is both robust and simple: one can do 
%D with calling the next macro with the filename as argument:
%D
%D \starttypen
%D \UseMetaPostGraphic{filename}
%D \stoptypen
%D
%D We can turn of this mechanism with: 
%D
%D \starttypen
%D \DontUseMetaPostGraphics
%D \stoptypen

\def\UseMetaPostGraphic#1%
  {\bgroup
   \message{[MP fonts #1]}%
   %\uncatcodespecials
   \endlinechar=-1
   \setMPspecials
   \obeyMPspecials
   \setbox0=\hbox
     {\hskip-\maxdimen
      \doprocessfile\scratchread{#1}\handleMPSline}%
   \smashbox0 
   \box0
   \egroup}

\def\DontUseMetaPostGraphics%
  {\let\UseMetaPostGraphic=\gobbleoneargument}

%D The characters are collected in a box and moved as far as
%D possible into the left margin. The resulting box has no
%D dimensions and can be prepended (appended) to the special
%D that handles the inclusion. The characters are in the file
%D but made invisible. 

%D We scan the graphics file for the \type{fshow} operator,
%D that is, lines that start with \type{(}. If found it
%D interprets the line, which looks like: 
%D
%D \starttypen
%D (string ... string) font size fshow
%D \stoptypen
%D 
%D Font definitions specified in the preamble are simply 
%D ignored. Only lines starting with \type{(} are interpreted.  

\def\dohandleMPSline#1#2\relax%
  {\if#1(%
     \expandafter\includeMPcharacters\fileline\relax
   \fi}

\def\handleMPSline%
  {\expandafter\dohandleMPSline\fileline\relax}

%D Before we start scanning for data, we first change some 
%D \CATCODES. The first set of macro's is copied from module 
%D \type{supp-pdf}. This scheme is a bit overdone for this 
%D module, but using the same macros saves us some memory.

\def\octalMPcharacter#1#2#3%
  {\char'#1#2#3\relax}

\bgroup
\catcode`\|=\@@comment
\catcode`\%=\@@active
\catcode`\[=\@@active
\catcode`\]=\@@active
\catcode`\{=\@@active
\catcode`\}=\@@active
\catcode`B=\@@begingroup
\catcode`E=\@@endgroup
\gdef\ignoreMPspecials|
  B\def%BE|
   \def[BE|
   \def]BE|
   \def{BE|
   \def}BEE     
\gdef\obeyMPspecials|
  B\def%B\char 37\relax E|
   \def[B\char 91\relax E|
   \def]B\char 93\relax E|
   \def{B\char123\relax E|
   \def}B\char125\relax EE   
\gdef\setMPspecials|
  B\catcode`\%=\@@active
   \catcode`\[=\@@active
   \catcode`\]=\@@active
   \catcode`\{=\@@active
   \catcode`\}=\@@active
   \catcode`\$=\@@letter
   \catcode`\_=\@@letter
   \catcode`\#=\@@letter
   \catcode`\^=\@@letter
   \catcode`\&=\@@letter
   \catcode`\|=\@@letter
   \catcode`\~=\@@letter
   \def\(B\char40\relax     E|
   \def\)B\char41\relax     E|
   \def\\B\char92\relax     E|
   \def\0B\octalMPcharacter0E|
   \def\1B\octalMPcharacter1E|
   \def\2B\octalMPcharacter2E|
   \def\3B\octalMPcharacter3E|
   \def\4B\octalMPcharacter4E|
   \def\5B\octalMPcharacter5E|
   \def\6B\octalMPcharacter6E|
   \def\7B\octalMPcharacter7E|
   \def\8B\octalMPcharacter8E|
   \def\9B\octalMPcharacter9EE
\egroup

%D The lines starting with \type{(} are interpreted and 
%D handled by 

\def\includeMPcharacters(#1) #2 #3 #4\relax%
  {\font\temp=#2 at #3bp\temp#1}

%D This method is both robust and reasonable fast. The only 
%D disadvantage is that when not embedded properly in the 
%D graphics inclusion macros, one has to load all graphics by
%D hand.  

%D Now let's see if things work all right and show the example 
%D files that are part of the \METAPOST\ distribution:
%D
%D \startregelcorrectie
%D \steluitlijnenin[midden]
%D \leavevmode
%D \startcombinatie[3*3]
%D   {\externfiguur[mp-exa-1][methode=mps,kader=aan,breedte=.2\hsize]} {}
%D   {\externfiguur[mp-exa-2][methode=mps,kader=aan,breedte=.2\hsize]} {}
%D   {\externfiguur[mp-exa-3][methode=mps,kader=aan,breedte=.2\hsize]} {}
%D   {\externfiguur[mp-exa-4][methode=mps,kader=aan,breedte=.2\hsize]} {}
%D   {\externfiguur[mp-exa-5][methode=mps,kader=aan,breedte=.2\hsize]} {}
%D   {\externfiguur[mp-exa-6][methode=mps,kader=aan,breedte=.2\hsize]} {}
%D   {\externfiguur[mp-exa-7][methode=mps,kader=aan,breedte=.2\hsize]} {}
%D   {\externfiguur[mp-exa-8][methode=mps,kader=aan,breedte=.2\hsize]} {}
%D   {\externfiguur[mp-exa-9][methode=mps,kader=aan,breedte=.2\hsize]} {}
%D \stopcombinatie
%D \stopregelcorrectie
%D
%D Here we used calls like: 
%D
%D \starttypen
%D \externfiguur[mp-exa-1][methode-mps,kader=aan,breedte=.2\hsize]
%D \stoptypen

\protect 

\endinput
