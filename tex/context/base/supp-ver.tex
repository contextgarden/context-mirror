%D \module
%D   [       file=supp-ver,
%D        version=1997.01.04,
%D          title=\CONTEXT\ Support Macros,
%D       subtitle=Verbatim,
%D         author=Hans Hagen,
%D           date=\currentdate,
%D      copyright={PRAGMA / Hans Hagen \& Ton Otten}]
%C
%C This module is part of the \CONTEXT\ macro||package and is
%C therefore copyrighted by \PRAGMA. Non||commercial use is 
%C granted. 

%D Because this module is quite independant of system macros,
%D it can be used as a stand||alone verbatim environment.

\ifx \undefined \writestatus \input supp-mis.tex \fi

%D Verbatim typesetting, especially of \TEX\ sources, is a
%D non||trivial task. This is a direct results of the fact that
%D characters can have \CATCODES\ other than~11 and such
%D characters needs a special treatment. What for instance is
%D \TEX\ supposed to do when it encounters a \type{$} or an
%D \type{#}? This module deals with these matters.

\writestatus{loading}{Context Support Macros / Verbatim}

%D The verbatim environment has some features, like coloring
%D \TEX\ text, seldom found in other environments. Especially
%D when the output of \TEX\ is viewed on an electronic medium,
%D coloring has a positive influence on the readability of
%D \TEX\ sources, so we found it very acceptable to dedicate
%D half of this module to typesetting \TEX\ specific character
%D sequences in color. In this module we'll also present some
%D macro's for typesetting inline, display and file verbatim.
%D The macro's are capable of handling \TAB\ too.
%D
%D This module shows a few tricks that are often overseen by
%D novice, like the use of the \TEX\ primitive \type{\meaning}.
%D First I'll show in what way the users are confronted with
%D verbatim typesetting. Because we want to be able to test for
%D symmetry and because we hate the method of closing down the
%D verbatim mode with some strange active character, we will
%D use the following construction for display verbatim:
%D
%D \starttypen
%D \starttyping
%D The Dutch word 'typen' stands for 'typing', therefore in the Dutch version
%D one will not find the word 'verbatim'.
%D \stoptyping
%D \stoptypen
%D
%D In \CONTEXT\ files can be typed with \type{\typefile} and
%D inline verbatim can be accomplished with \type{\type}. This
%D last command comes in many flavors:
%D
%D \starttypen
%D We can say \type<<something>> or \type{something}. The first one is a bit
%D longer but also supports slanted typing, which accomplished by typing
%D \type<<a <<slanted>> word>>. We can also use commands to enhance the text
%D \type<<with <</bf boldfaced>> text>>. Just to be complete, we decided
%D to accept also \LaTeX\ alike verbatim, which means that \type+something+
%D and \type|something| are valid commands too. Of course we want the grouped
%D alternatives to process \type{hello {\bf big} world}} with braces.
%D \stoptypen
%D
%D In the core modules, we will build this support on top of
%D this module. There these commands can be tuned with
%D accompanying setup commands. There we can enable commands,
%D slanted typing, control spaces, \TAB||handling and (here we
%D are:) coloring. We can also setup surrounding white space
%D and indenting. Here we'll only show some examples.

\unprotect

%D \macros
%D   {verbatimfont}
%D   {}
%D
%D When we are typesetting verbatim we use a non||proportional
%D (mono spaced) font. Normally this font is available by
%D calling \type{\tt}. In \CONTEXT\ this command does a
%D complete font||style switch. There we could have stuck with
%D \type{\tttf}.

\ifx \undefined \verbatimfont \def\verbatimfont {\tt} \fi

%D \macros
%D   {obeyedspace, obeyedtab, obeyedline, obeyedpage}
%D   {}
%D
%D We have followed Knuth in naming macros that make \SPACE,
%D \NEWLINE\ and \NEWPAGE\ active and assigning them
%D \type{\obeysomething}, but first we set some default values.

\def\obeyedspace {\hbox{ }}
\def\obeyedtab   {\obeyedspace}
\def\obeyedline  {\par}
\def\obeyedpage  {\vfill\eject}

%D \macros
%D   {controlspace,setcontrolspaces}
%D   {}
%D
%D First we define \type{\obeyspaces}. When we want visible
%D spaces (control spaces) we only have to adapt the definition
%D of \type{\obeyedspace} to:

\def\controlspace {\hbox{\char32}}

\bgroup
\catcode`\ =\@@active
\gdef\obeyspaces{\catcode`\ =\@@active\def {\obeyedspace}}
\gdef\setcontrolspaces{\catcode`\ =\@@active\def {\controlspace}}
\egroup

%D \macros
%D   {obeytabs, obeylines, obeypages,
%D    ignoretabs, ignorelines, ignorepages}
%D   {}
%D
%D Next we take care of \NEWLINE\ and \NEWPAGE\ and because we
%D want to be able to typeset listings that contain \TAB, we
%D have to handle those too. Because we have to redefine the
%D \NEWPAGE\ character locally, we redefine the meaning of
%D this (often already) active character.

\catcode`\^^L=\@@active \def^^L{\par}

\bgroup

\catcode`\^^I=\@@active
\catcode`\^^M=\@@active
\catcode`\^^L=\@@active

\gdef\obeytabs    {\catcode`\^^I=\@@active\def^^I{\obeyedtab}}
\gdef\obeylines   {\catcode`\^^M=\@@active\def^^M{\obeyedline}}
\gdef\obeypages   {\catcode`\^^L=\@@active\def^^L{\obeyedpage}}

\gdef\ignoretabs  {\catcode`\^^I=\@@active\def^^I{\obeyedspace}}
\gdef\ignorelines {\catcode`\^^M=\@@active\def^^M{\obeyedspace}}
\gdef\ignorepages {\catcode`\^^L=\@@active\def^^L{\obeyedline}}

\egroup

%D \macros
%D   {obeycharacters}
%D   {}
%D
%D We also predefine \type{\obeycharacters}, which will 
%D enable us to implement character||specific behavior, like 
%D colored verbatim.

\let\obeycharacters=\relax

%D \macros
%D   {settabskips}
%D   {}
%D
%D The macro \type{\settabskip} can be used to enable tab 
%D handling. Processing tabs is sometimes needed when one 
%D processes a plain \ASCII\ listing. Tab handling slows down 
%D verbatim typesetting considerably. 

\bgroup

\catcode`\^^I=\@@active

\gdef\settabskips%
  {\let\processverbatimline=\doprocesstabskipline
   \catcode`\^^I=\@@active
   \let^^I=\doprocesstabskip}

\egroup

%D \macros
%D   {processinlineverbatim}
%D   {}
%D
%D Although the inline verbatim commands presented here will be
%D extended and embedded in the core modules of \CONTEXT,
%D they can be used separately. Both grouped and character
%D alternatives are provided but \type{<<} and nested
%D braces are implemented in the core module. This commands
%D takes one argument: the closing command.
%D
%D \starttypen
%D \processinlineverbatim{\closingcommand}
%D \stoptypen
%D
%D One can define his own verbatim commands, which can be very
%D simple:
%D
%D \starttypen
%D \def\Verbatim {\processinlineverbatim\relax}
%D \stoptypen
%D
%D or a bit more more complex:
%D
%D \starttypen
%D \def\GroupedVerbatim%
%D   {\bgroup
%D    \dosomeusefullthings
%D    \processinlineverbatim\egroup}
%D \stoptypen
%D
%D Before entering inline verbatim mode, we take care of the
%D unwanted \TAB, \NEWLINE\ and \NEWPAGE\ characters and
%D turn them into \SPACE. We need the double \type{\bgroup}
%D construction to keep the closing command local.

\def\setupinlineverbatim%
  {\verbatimfont
   \let\obeytabs=\ignoretabs
   \let\obeylines=\ignorelines
   \let\obeypages=\ignorepages
   \setupcopyverbatim}

\def\doprocessinlineverbatim%
  {\ifx\next\bgroup
     \setupinlineverbatim
     \catcode`\{=\@@begingroup
     \catcode`\}=\@@endgroup
     \def\next{\let\next=}%
   \else
     \setupinlineverbatim
     \def\next##1{\catcode`##1=\@@endgroup}%
   \fi
   \next}

\def\processinlineverbatim#1%
  {\bgroup
   \localcatcodestrue % TeX processes paragraph's
   \def\endofverbatimcommand{#1\egroup}%
   \bgroup
   \aftergroup\endofverbatimcommand
   \futurelet\next\doprocessinlineverbatim}

%D \macros
%D   {processdisplayverbatim}
%D   {}
%D
%D The closing command is executed afterwards as an internal
%D command and therefore should not be given explicitly when
%D typesetting inline verbatim.
%D
%D We can define a display verbatim environment with the
%D command \type{\processdisplayverbatim} in the following way:
%D
%D \starttypen
%D \processdisplayverbatim{\closingcommand}
%D \stoptypen
%D
%D \noindent For instance, we can define a simple command like:
%D
%D \starttypen
%D \def\BeginVerbatim {\processdisplayverbatim{EndVerbatim}}
%D \stoptypen
%D
%D \noindent But we can also do more advance things like:
%D
%D \starttypen
%D \def\BeginVerbatim {\bigskip \processdisplayverbatim{\EndVerbatim}}
%D \def\EndVerbatim   {\bigskip}
%D \stoptypen
%D
%D When we compare these examples, we see that the backslash in
%D the closing command is optional. One is free in actually
%D defining a closing command. If one is defined, the command
%D is executed after ending verbatim mode.

\def\processdisplayverbatim#1%
  {\par
   \bgroup
   \escapechar=-1
   \xdef\verbatimname{\string#1}%
   \egroup
   \def\endofdisplayverbatim{\csname\verbatimname\endcsname}%
   \bgroup
   \parindent\!!zeropoint
   \ifdim\lastskip<\parskip
     \removelastskip
     \vskip\parskip
   \fi
   \parskip\!!zeropoint
   \processingverbatimtrue
   \linepartrue
   \expandafter\let\csname\verbatimname\endcsname=\relax
   \edef\endofverbatimcommand{\csname\verbatimname\endcsname}%
   \edef\endofverbatimcommand{\meaning\endofverbatimcommand}%
   \verbatimfont
   \setupcopyverbatim
   \ifskipfirstverbatimline 
     \let\doverbatimline=\relax
   \else
     \let\doverbatimline=\dodoverbatimline
   \fi
   \copyverbatimline}

%D \macros
%D   {ifskipfirstverbatimline} 
%D 
%D By default the rest of the first line is ignored. We can 
%D turn this feature off by saying:
%D 
%D \starttypen
%D \skipfirstverbatimlinefalse
%D \stoptypen

\newif\ifskipfirstverbatimline \skipfirstverbatimlinetrue 

%D We save the closing sequence in \type{\endofverbatimcommand}
%D in such a way that it can be compared on a line by line
%D basis. For the conversion we use \type{\meaning}, which
%D converts the line to non||expandable tokens. We reset
%D \type{\parskip}, because we don't want inter||paragraph
%D skips to creep into the verbatim source. Furthermore we
%D \type{\relax} the line||processing macro while getting the
%D rest of the first line. The initialization command
%D \type{\setupcopyverbatim} does just what we expect it to do:
%D it assigns all characters \CATCODE~11. Next we switch to
%D french spacing and call for obeyance.

\def\setupcopyverbatim%
  {\uncatcodecharacters
   \frenchspacing
   \obeyspaces
   \obeytabs
   \obeylines
   \obeycharacters}

%D \macros
%D   {ifeightbitcharacters,
%D    setcatcodes,uncatcodespecials,
%D    uncatcodecharacters}
%D   {}
%D
%D As its name says, \type{\uncatcodecharacters} resets the
%D \CATCODE\ of characters. When we use an upper bound of
%D 127 or 255, depending in \type{\ifeightbitcharacters}. By
%D counting down, we only have to use one counter. The 
%D macro \type{\setcatcodes} can be uses to set alternative 
%D values. The macro \type{\resetspecialcharacters} resets 
%D characters with special meanings. This macro is not used 
%D in the verbatim macros, but is best defined in this module. 

\def\doprocesscatcodes#1%
  {\ifeightbitcharacters
     \scratchcounter=255
   \else
     \scratchcounter=127
   \fi
   \loop
     \savecatcode
     #1\relax
     \advance\scratchcounter by -1
     \ifnum\scratchcounter>-1
   \repeat
   \let\savecatcode=\relax
   \let\restorecatcodes=\dorestorecatcodes}

\def\uncatcodespecials%
  {\doprocesscatcodes
     {\ifnum\catcode\scratchcounter=\@@letter\relax\else
        \catcode\scratchcounter=\@@other
      \fi}%
   \catcode`\   =\@@space
   \catcode`\^^L=\@@ignore
   \catcode`\^^M=\@@endofline   
   \catcode`\^^?=\@@ignore} 

\def\setcatcodes#1%
  {\doprocesscatcodes
     {\catcode\scratchcounter=#1}}

\def\uncatcodecharacters%
  {\setcatcodes\@@letter}

%D \macros
%D   {iflocalcatcodes,
%D    restorecatcodes,
%D    beginrestorecatcodes,endrestorecatcodes}
%D   {}
%D
%D We're not finished dealing \CATCODES\ yet. In \CONTEXT\ we 
%D use only one auxiliary file, which deals with tables of 
%D contents, registers, two pass tracking, references etc. This 
%D file, as well as files concerning graphics, is processed when
%D needed, which can be in the mid of typesetting verbatim. 
%D However, when reading in data in verbatim mode, we should 
%D temporary restore the normal \CATCODES, and that's exactly 
%D what the next macros do. Saving the catcodes can be 
%D disabled by saying \type{\localcatcodestrue}.

%D The previous macros call for \type{\savecatcode}, which is
%D implemented as: 

\newif\iflocalcatcodes

\def\savecatcode%
  {\iflocalcatcodes \else
     \expandafter\edef\csname @@cc@@\the\scratchcounter\endcsname%
       {\the\catcode\scratchcounter}%
   \fi}

%D It's counterpart is: 

\def\restorecatcode%
  {\expandafter\catcode\expandafter\scratchcounter\expandafter=
     \csname @@cc@@\the\scratchcounter\endcsname}

%D When we want to restore \CATCODES\ we call for 
%D \type{\restorecatcodes}, which default to \type{\relax} 

\let\restorecatcodes=\relax

%D or when we've saves things calls for: 

\def\dorestorecatcodes%
  {\iflocalcatcodes \else
     \doprocesscatcodes\restorecatcode
   \fi}

%D We also provide an alternative, that forces grouping 
%D when needed. An application of this macros can be found in
%D buffering data. 

\def\beginrestorecatcodes%
  {\ifx\restorecatcodes\relax
     \let\endrestorecatcodes=\relax
   \else
     \bgroup
     \let\beginrestorecatcodes=\bgroup
     \let\endrestorecatcodes=\egroup
   \fi}

%D The main copying routine of display verbatim does an
%D ordinary string||compare on the saved closing command and
%D the current line. The space after \type{#1} in the
%D definition of \type{\next} is essential! As a result of
%D using \type{\obeylines}, we have to use \type{%}'s after
%D each line but none after the first \type{#1}.

{\obeylines%
 \long\gdef\copyverbatimline#1
   {\ifx\doverbatimline\relax% gobble rest of the first line
      \let\doverbatimline=\dodoverbatimline%
      \def\next{\copyverbatimline}%
    \else%
       \def\next{#1 }%
       \ifx\next\emptyspace%
         \def\next%
           {\doemptyverbatimline{#1}%
            \copyverbatimline}%
       \else%
         \edef\next{\meaning\next}%
         \ifx\next\endofverbatimcommand%
           \def\next%
             {\egroup\endofdisplayverbatim}%
         \else%
           \def\next%
             {\doverbatimline{#1}%
              \copyverbatimline}%
         \fi%
       \fi%
   \fi%
   \next}}

%D The actual typesetting of a line is done by a separate
%D macro, which enables us to implement \TAB\ handling. The
%D \type{\do} and \type{\dodo} macros take care of the
%D preceding \type{\parskip}, while skipping the rest of the
%D first line. The \type{\relax} is used as an signal.

%D \macros
%D   {iflinepar}
%D   {}
%D
%D A careful reader will see that \type{\linepar} is reset.
%D This boolean can be used to determine if the current line is
%D the first line in a pseudo paragraph and this boolean is set
%D after each empty line.

\newif\iflinepar

\long\def\dodoverbatimline#1%
  {\leavevmode\the\everyline\strut\processverbatimline{#1}%
   \EveryPar{}%
   \lineparfalse
   \obeyedline\par}

%D \macros
%D   {obeyemptylines,verbatimbaselineskip}
%D   {}
%D
%D Empty lines in verbatim can lead to white space on top of
%D a new page. Because this is not what we want, we turn
%D them into vertical skips. This default behavior can be
%D overruled by:
%D
%D \starttypen
%D \obeyemptylines
%D \stoptypen
%D
%D Although it would cost us only a few lines of code, we
%D decided not to take care of multiple empty lines. When a
%D (display) verbatim text contains more successive empty
%D lines, this probably suits some purpose. When applicable, 
%D one can set the verbatim baselineskip.

\bgroup
\catcode`\^^L=\@@active  \gdef\emptypage  {^^L}
\catcode`\^^M=\@@active  \gdef\emptyline  {^^M}
                         \gdef\emptyspace { }
\egroup

\def\verbatimbaselineskip{\baselineskip}

\def\doemptyverbatimline%
  {\vskip\verbatimbaselineskip
   {\setbox0=\hbox{\the\everyline}}% 
   \linepartrue}

\def\obeyemptylines%
  {\def\doemptyverbatimline{\doverbatimline}}

%D \TEX\ does not offer \type{\everyline}, which is a direct
%D result of its advanced multi||pass paragraph typesetting
%D mechanism. Because in verbatim mode paragraphs and lines are
%D more or less equal, we can easily implement our own simple
%D \type{\everyline} support.

%D \macros
%D   {EveryPar, EveryLine}
%D   {}
%D
%D In this module we've reserved \type{\everypar} for the
%D things to be done with paragraphs and \type{\everyline} for
%D line specific actions. In \CONTEXT\ however, we use
%D \type{\everypar} for placing side- and columnfloats,
%D inhibiting indentation and some other purposes. In verbatim
%D mode, every line becomes a paragraph, which means that
%D \type{\everypar} is executed frequently. To be sure, the
%D user specific use of both \type{\everyline} and
%D \type{\everypar} is implemented by means of
%D \type{\EveryLine} and \type{\EveryPar}.
%D
%D We still have to take care of the \TAB. A \TAB\ takes eight
%D spaces and a \SPACE\ normally has a width of 0.5~em. Because
%D we can be halfway a tabulation, we must keep track of the
%D position. This takes time, especially when we print complete
%D files, therefore we \type{\relax} this mechanism by default.

\def\doprocesstabskip%
  {\obeyedspace % \hskip.5em  or  \hbox to .5em{}
   \ifdone
     \advance\scratchcounter by 1
     \let\next=\doprocesstabskip
     \donefalse
   \else\ifnum\scratchcounter>7
     \let\next=\relax
   \else
     \advance\scratchcounter 1
     \let\next=\doprocesstabskip
   \fi\fi
   \next}

\def\dodoprocesstabskipline#1#2\endoftabskipping%
  {\ifnum\scratchcounter>7
     \scratchcounter=1
     \donetrue
   \else
     \advance\scratchcounter 1
     \donefalse
   \fi
   \ifx#1\relax
     \let\next=\relax
   \else
     \def\next{#1\dodoprocesstabskipline#2\endoftabskipping}%
   \fi
   \next}

\let\endoftabskipping    = \relax
\let\processverbatimline = \relax

\def\doprocesstabskipline#1%
  {\bgroup
   \scratchcounter=1
   \dodoprocesstabskipline#1\relax\endoftabskipping
   \egroup}

%D \macros
%D   {processfileverbatim}
%D   {}
%D
%D The verbatim typesetting of files is done on a bit different
%D basis. This time we don't check for a closing command, but
%D look for \EOF\ and when we've met, we make sure it does not
%D turn into an empty line.
%D
%D \starttypen
%D \processfileverbatim{filename}
%D \stoptypen
%D
%D Typesetting a file in most cases results in more than one
%D page. Because we don't want problems with files that are
%D read in during the construction of the page, we set
%D \type{\ifprocessingverbatim}, so the output routine can
%D adapt its behavior. Originally we used 
%D \type{\scratchread}, but because we want to support nesting, 
%D we decided to use a separate input file.

\newif\ifprocessingverbatim

\newread\verbatiminput

\def\processfileverbatim#1%
  {\par
   \bgroup
   \parindent\!!zeropoint
   \ifdim\lastskip<\parskip
     \removelastskip
     \vskip\parskip
   \fi
   \parskip\!!zeropoint
   \processingverbatimtrue
   \linepartrue
   \uncatcodecharacters
   \verbatimfont
   \frenchspacing
   \obeyspaces
   \obeytabs
   \obeylines
   \obeypages
   \obeycharacters
   \openin\verbatiminput=#1%
   \def\doreadline%
     {\read\verbatiminput to \next
      \ifeof\verbatiminput
        % we don't want <eof> to be treated as <crlf>
      \else\ifx\next\emptyline
        \expandafter\doemptyverbatimline\expandafter{\next}%
      \else\ifx\next\emptypage
        \expandafter\doemptyverbatimline\expandafter{\next}%
      \else
        \expandafter\dodoverbatimline\expandafter{\next}%
      \fi\fi\fi
      \readline}%
   \def\readline%
     {\ifeof\verbatiminput
        \let\next=\relax
      \else
        \let\next=\doreadline
      \fi
      \next}%
   \readline
   \closein\verbatiminput
   \egroup
   \ignorespaces}

%D These macro's can be used to construct the commands we
%D mentioned in the beginning of this documentation. We leave
%D this to the fantasy of the reader and only show some \PLAIN\
%D \TEX\ alternatives for display verbatim and listings. We
%D define three commands for typesetting inline text, display
%D text and files verbatim. The inline alternative also accepts
%D user supplied delimiters.
%D
%D \starttypen
%D \type{text}
%D
%D \starttyping
%D ... verbatim text ...
%D \stoptyping
%D
%D \typefile{filename}
%D \stoptypen
%D
%D We can turn on the options by:
%D
%D \starttypen
%D \controlspacetrue
%D \verbatimtabstrue
%D \prettyverbatimtrue
%D \stoptypen
%D
%D Here is the implementation:

\newif\ifcontrolspace
\newif\ifverbatimtabs
\newif\ifprettyverbatim

\def\presettyping%
  {\ifcontrolspace
     \let\obeyspace=\setcontrolspace
   \fi
   \ifverbatimtabs
     \let\obeytabs=\settabskips
   \fi
   \ifprettyverbatim
     \let\obeycharacters=\setupprettytextype
   \fi}

\def\type%
  {\bgroup
   \presettyping
   \processinlineverbatim{\egroup}}

\def\starttyping%
  {\bgroup
   \presettyping
   \processdisplayverbatim{\stoptyping}}

\def\stoptyping%
  {\egroup}

\def\typefile#1%
  {\bgroup
   \presettyping
   \processfileverbatim{#1}%
   \egroup}

%D One can use the different \type{\obeysomething} commands to
%D influence the behavior of these macro's. We use for instance
%D \type{\obeycharacters} for making \type{/} an active
%D character when we want to include typesetting commands.
%D
%D We'll spend the remainder of this article on coloring the
%D verbatim text. At \PRAGMA\ we use the integrated environment
%D \TEXEDIT\ for editing and processing \TEX\
%D documents.\voetnoot{\TEXEDIT\ has been operative since
%D 1991.} This program also supports real time spell checking
%D and \TEX\ based file management. Although definitely not
%D exclusive, the programs cooperate nicely with \CONTEXT.
%D Because \TEX\ can be considered a tool for experts, we've
%D tried to put as less a burden on non||technical users as
%D possible. This is accomplished in the following ways:
%D
%D \startopsomming
%D
%D \som  We've added some trivial symmetry checking to
%D       \TEXEDIT. Sources are checked for the use of brackets,
%D       braces, begin||end and start||stop like constructions,
%D       with or without arguments.
%D
%D \som  Although \TEX\ is very tolerant to unformatted input,
%D       we stimulate users to make the \ASCII\ source as clean
%D       as possible. Many sources I've seen in distribution
%D       sets look so awful, that I sometimes wonder how people
%D       get them working. In our opinion, a good||looking
%D       source leads to less errors.
%D
%D \som  We use parameter driven setups and make the commands
%D       as tolerant as possible. We don't accept commands that
%D       don't look nice in \ASCII.
%D
%D \som  Finally ---I could have added some more--- we use
%D       color.
%D
%D \stopopsomming
%D
%D When in spell||checking||mode, the words spelled correctly
%D are shown in {\em green}, the unknown or wrongly spelled
%D words are in {\em red} and upto four categories of words,
%D for instance passive verbs and nouns, become {\em blue}
%D (or cyan) or {\em yellow}. Short and nearly always correct
%D words are in white (on a black screen). This makes
%D checking||on||the||fly very easy and convenient, especially
%D because we place the accents automatically.
%D
%D In \TEX||mode we show \TEX||specific tokens and sequences of
%D tokens in appropriate colors and again we use four colors.
%D We use those colors in a way that supports parameter driven
%D setups, table typesetting and easy visual checking of
%D symmetry. Furthermore the text becomes more readable.
%D
%D \bgroup
%D \chardef\ampersand =`\&
%D \chardef\comment   =`\%
%D \chardef\leftbrace =`\{
%D \chardef\rightbrace=`\}
%D
%D \plaatstabel{geen}
%D \starttabel[|l|l|]
%D \HL
%D \FC\bf color \MC\bf characters that are influenced    \LC\SR
%D \HL
%D \FC red      \MC\tt \leftbrace\ \rightbrace\
%D                     \string$                          \LC\FR
%D \FC green    \MC\tt \string\this\    \string\!!that\
%D                     \string\??these\ \string\@@those\ \LC\MR
%D \FC yellow   \MC\tt \string` \string' \string~
%D                     \string^ \string_ \ampersand\
%D                     \string/ \string+ \string-
%D                     \string| \comment\                \LC\MR
%D \FC blue     \MC\tt \string( \string) \string#
%D                     \string[ \string] \string"
%D                     \string< \string> \string=        \LC\LR
%D \HL
%D \stoptabel
%D \egroup
%D
%D Macro||definition and style files often look quite green,
%D because they contain many calls to macros. Pure text files
%D on the other hand are mostly white (on the screen) and color
%D clearly shows their structure.
%D
%D When I prepared the interactive \PDF\ manuals of \CONTEXT,
%D \TEXEDIT\ and \PPCHTEX\ (1995), I decided to include the
%D original source text of the manuals as an appendix. At every
%D chapter or (sub)section the reader can go to the
%D corresponding line in the source, just to see how things
%D were done in \TEX. Of course, the reader can jump from the
%D to corresponding typeset text too.
%D
%D Confronted with those long (boring) sources, I decided that
%D a colored output, in accordance with \TEXEDIT\ would be
%D nice. It would not only visually add some quality to the
%D manual, but also make the sources more readable.
%D
%D Apart from a lot of \CATCODE||magic, programming the color
%D macros was surprisingly easy. Although the macro's are
%D hooked into the standard \CONTEXT\ verbatim mechanism, they
%D are set up in a way that embedding them in another verbatim
%D environment is possible.
%D
%D We can turn on coloring by reassigning
%D \type{\obeycharacters}:
%D
%D \starttypen
%D \let\obeycharacters=\setupprettytextype
%D \stoptypen
%D
%D During pretty typesetting we can be in two states: {\em
%D command} and {\em parameter}. The first condition becomes
%D true if we encounter a backslash, the second state is
%D entered when we meet a~\type{#}.

\newif\ifintexcommand
\newif\ifintexparameter

%D \macros
%D   {splittexparameters}
%D   {}
%D
%D The mechanism described here, is meant to be used with
%D color. It is nevertheless possible to use different fonts
%D instead of distinctive colors. When using color, it's better
%D to end parameter mode after the \type{#}. When on the
%D other hand we use a slanted typeface for the hashmark, then
%D a slanted number looks better.

\newif\ifsplittexparameters   \splittexparameterstrue

%D \macros
%D   {splittexcontrols}
%D   {}
%D
%D With \type{\splittexcontrols} we can influence the way
%D control characters are processed in macro names. By default,
%D the \type{^^} part is uncolored. When this boolean is set to
%D false, they get the same color as the other characters.

\newif\ifsplittexcontrols     \splittexcontrolstrue

%D The next boolean is used for internal purposes only and
%D keeps track of the length of the name. Because two||character
%D sequences starting with a backslash are always seen as a
%D command.

\newif\iffirstintexcommand

%D We use a maximum of four colors because more colors will
%D distract too much. In the following table we show the
%D logical names of the colors, their color and $rgb$ values.
%D
%D \plaatstabel{geen}
%D \starttabel[|l|l|c|c|c|c|]
%D \HL
%D \FC\bf identifier  \MC\bf color \MC\bf r \MC\bf g \MC\bf b \MC\bf bw \LC\SR
%D \HL
%D \FC texprettyone   \MC red      \MC 0.9  \MC 0.0  \MC 0.0  \MC 0.30  \LC\FR
%D \FC texprettytwo   \MC green    \MC 0.0  \MC 0.8  \MC 0.0  \MC 0.45  \LC\MR
%D \FC texprettythree \MC yellow   \MC 0.0  \MC 0.0  \MC 0.9  \MC 0.60  \LC\MR
%D \FC texprettyfour  \MC blue     \MC 0.8  \MC 0.8  \MC 0.6  \MC 0.75  \LC\LR
%D \HL
%D \stoptabel

%D This following poor mans implementation of color is based on
%D PostScript. One can of course use grayscales too. In the
%D core modules these macros are redefined to using the color
%D mechanism present in \CONTEXT.

\def\setcolorverbatim%
  {\splittexparameterstrue
   \def\texprettyone   {.9 .0 .0 }       % red
   \def\texprettytwo   {.0 .8 .0 }       % green
   \def\texprettythree {.0 .0 .9 }       % blue
   \def\texprettyfour  {.8 .8 .6 }       % yellow
   \def\texbeginofpretty[##1]%
     {\special{ps:: \csname##1\endcsname setrgbcolor}}
   \def\texendofpretty%
     {\special{ps:: 0 0 0 setrgbcolor}}} % black

\def\setgrayverbatim%
  {\splittexparameterstrue
   \def\texprettyone   {.30 }            % gray
   \def\texprettytwo   {.45 }            % gray
   \def\texprettythree {.60 }            % gray
   \def\texprettyfour  {.75 }            % gray
   \def\texbeginofpretty[##1]%
     {\special{ps:: \csname##1\endcsname setgray}}
   \def\texendofpretty%
     {\special{ps:: 0 setgray}}}         % black

%D One can redefine these two commands after loading this
%D module. When available, one can also use appropriate
%D font||switch macro's. We default to color.

\setcolorverbatim

%D Here come the commands that are responsible for entering and
%D leaving the two states. As we can see, they've got much in
%D common.

\def\texbeginofcommand%
  {\texendofparameter
   \ifintexcommand
   \else
     \global\intexcommandtrue
     \global\firstintexcommandtrue
     \texbeginofpretty[texprettytwo]%
   \fi}

\def\texendofcommand%
  {\ifintexcommand
     \texendofpretty
     \global\intexcommandfalse
     \global\firstintexcommandfalse
   \fi}

\def\texbeginofparameter%
  {\texendofcommand
   \ifintexparameter
   \else
     \global\intexparametertrue
     \texbeginofpretty[texprettythree]%
   \fi}

\def\texendofparameter%
  {\ifintexparameter
     \texendofpretty
     \global\intexparameterfalse
   \fi}

%D We've got nine types of characters. The first type concerns
%D the grouping characters that become red and type seven takes
%D care of the backslash. Type eight is the most recently added
%D one and handles the control characters starting with
%D \type{^^}. In the definition part at the end of this module
%D we can see how characters are organized by type.

\def\ifnotfirstintexcommand#1%
  {\iffirstintexcommand
     \string#1%
     \texendofcommand
   \else}

\def\textypeone#1%
  {\ifnotfirstintexcommand#1%
     \texendofcommand
     \texendofparameter
     \texbeginofpretty[texprettyone]\string#1\texendofpretty
   \fi}

\def\textypetwo#1%
  {\ifnotfirstintexcommand#1%
     \texendofcommand
     \texendofparameter
     \texbeginofpretty[texprettythree]\string#1\texendofpretty
   \fi}

\def\textypethree#1%
  {\ifnotfirstintexcommand#1%
     \texendofcommand
     \texendofparameter
     \texbeginofpretty[texprettyfour]\string#1\texendofpretty
   \fi}

\def\textypefour#1%
  {\ifnotfirstintexcommand#1%
     \texendofcommand
     \texendofparameter
     \string#1%
   \fi}

\def\textypefive#1%
  {\ifnotfirstintexcommand#1%
     \texbeginofparameter
     \string#1%
   \fi}

\def\textypesix#1%
  {\ifnotfirstintexcommand#1%
     \ifintexparameter
       \ifsplittexparameters
         \texendofparameter
         \string#1%
       \else
         \string#1%
         \texendofparameter
       \fi
     \else
       \texendofcommand
       \string#1%
     \fi
   \fi}

\def\textypeseven#1%
  {\ifnotfirstintexcommand#1%
     \texbeginofcommand
     \string#1%
   \fi}

\def\textypeeight#1#2%
  {\texendofparameter
   \ifx#1#2%
     \ifsplittexcontrols
       \ifintexcommand
         \texendofcommand
         \string#1\string#1%
         \texbeginofcommand
       \else
         \string#1\string#2%
       \fi
     \else
       \string#1\string#1%
     \fi
   \else
     \ifintexcommand
       \firstintexcommandfalse
       \string#1#2%
     \else
       \textypethree#1#2%
     \fi
   \fi}

\def\textypenine#1%
  {\texendofparameter
   \global\firstintexcommandfalse
   \string#1}

%D We have to take care of the control characters we mentioned
%D before. We obey their old values but only after ending our
%D two states.

\def\texsetcontrols%
  {\global\let\oldobeyedspace = \obeyedspace
   \global\let\oldobeyedline  = \obeyedline
   \global\let\oldobeyedpage  = \obeyedpage
   \def\obeyedspace%
     {\texendofcommand
      \texendofparameter
      \oldobeyedspace}%
   \def\obeyedline%
     {\texendofcommand
      \texendofparameter
      \oldobeyedline}%
   \def\obeyedpage%
     {\texendofcommand
      \texendofparameter
      \oldobeyedpage}%
   \let\obeytabs=\ignoretabs}

%D Next comes the tough part. We have to change the \CATCODE\
%D of each character. These macro's are tuned for speed and
%D simplicity. When viewed in color they look quite simple.

\def\setupprettytextype%
  {\texsetcontrols
   \texsetspecialpretty
   \texsetalphabetpretty
   \texsetextrapretty}

%D When handling the lowercase characters, we cannot use
%D lowercased macro names. This means that we have to redefine
%D some well known macros, like \type{\bgroup}.

\def\texpresetcatcode%
  {\def\\##1%
     {\expandafter\catcode\expandafter`\csname##1\endcsname\@@active}}

\def\texsettypenine%
  {\def\\##1%
     {\def##1{\textypenine##1}}}

\bgroup
  \bgroup
    \gdef\texpresetalphapretty%
      {\texpresetcatcode
       \\A\\B\\C\\D\\E\\F\\G\\H\\I\\J\\K\\L\\M%
       \\N\\O\\P\\Q\\R\\S\\T\\U\\V\\W\\X\\Y\\Z}
    \texpresetalphapretty
    \gdef\texsetalphapretty%
      {\texpresetalphapretty
       \texsettypenine
       \\A\\B\\C\\D\\E\\F\\G\\H\\I\\J\\K\\L\\M%
       \\N\\O\\P\\Q\\R\\S\\T\\U\\V\\W\\X\\Y\\Z}
  \egroup
  \global\let\TEXPRESETCATCODE = \texpresetcatcode
  \global\let\TEXSETTYPENINE   = \texsettypenine
  \global\let\BGROUP           = \bgroup
  \global\let\EGROUP           = \egroup
  \global\let\GDEF             = \gdef
  \BGROUP
    \GDEF\TEXPRESETALPHAPRETTY%
      {\TEXPRESETCATCODE
       \\a\\b\\c\\d\\e\\f\\g\\h\\i\\j\\k\\l\\m%
       \\n\\o\\p\\q\\r\\s\\t\\u\\v\\w\\x\\y\\z}
    \TEXPRESETALPHAPRETTY
    \GDEF\TEXSETALPHAPRETTY%
      {\TEXPRESETALPHAPRETTY
       \TEXSETTYPENINE
       \\a\\b\\c\\d\\e\\f\\g\\h\\i\\j\\k\\l\\m%
       \\n\\o\\p\\q\\r\\s\\t\\u\\v\\w\\x\\y\\z}
  \EGROUP
  \gdef\texsetalphabetpretty%
    {\texsetalphapretty
     \TEXSETALPHAPRETTY}
\egroup

%D Macro names normally only may contain characters, but in
%D unprotected state we can also use the characters~\type{@},
%D \type{!} and~\type{?}. Of course they are only colored
%D (green) when they are part of a name.

\bgroup
  \gdef\texpresetextrapretty%
    {\texpresetcatcode
     \\?\\!\\@}
  \texpresetextrapretty
  \gdef\texsetextrapretty%
    {\texpresetextrapretty
     \texsettypenine
     \\?\\!\\@}
\egroup

%D Here comes the main specification  routine. In this macro we
%D also have to change the escape character to \type{!} and use
%D \type{X}, \type{Y} and \type{Z} for grouping and ignoring,
%D which makes the result a bit less readable. Plain \TEX\
%D defines \type{\+} as an outer macro, so we have to redefine
%D this one too.

\def\+{\tabalign}

\bgroup
  \gdef\texpresetspecialpretty%
    {\def\\##1{\catcode`##1\@@active}%
     \\\[\\\]\\\=\\\<\\\>\\\#\\\(\\\)\\\"%
     \\\$\\\{\\\}%
     \\\-\\\+\\\|\\\%\\\/\\\_\\\^\\\&\\\~\\\'\\\`%
     \\\.\\\,\\\:\\\;%
     \\\*%
     \\\1\\\2\\\3\\\4\\\5\\\6\\\7\\\8\\\9%
     \\\\}
  \catcode`\X=\the\catcode`\{
  \catcode`\Y=\the\catcode`\}
  \catcode`\Z=\the\catcode`\%
  \gdef\texsetsometypes%
    {\def\!##1##2{\def##1{##2{##1}}}}%
  XZ
   \catcode`\!=\@@escape
   !texpresetspecialpretty
   !gdef!texsetspecialpretty
     XZ
      !texpresetspecialpretty
      !texsetsometypes
      !! $ !textypeone   !! { !textypeone   !! } !textypeone
      !! [ !textypetwo   !! ] !textypetwo   !! ( !textypetwo   !! ) !textypetwo
      !! = !textypetwo   !! < !textypetwo   !! > !textypetwo   !! " !textypetwo
      !! - !textypethree !! + !textypethree !! / !textypethree
      !! | !textypethree !! % !textypethree !! ' !textypethree !! ` !textypethree
      !! _ !textypethree !! ^ !textypethree !! & !textypethree !! ~ !textypethree
      !! . !textypefour  !! , !textypefour  !! : !textypefour  !! ; !textypefour
      !! * !textypefour
      !! # !textypefive
      !! 1 !textypesix   !! 2 !textypesix   !! 3 !textypesix
      !! 4 !textypesix   !! 5 !textypesix   !! 6 !textypesix
      !! 7 !textypesix   !! 8 !textypesix   !! 9 !textypesix
      !! \ !textypeseven
      !! ^ !textypeeight
     YZ
  YZ
\egroup

%D This text was published in the \MAPS\ of the dutch \TEX\
%D users group \NTG. In that article, the verbatim part of the
%D text was set with the following commands for the examples:
%D
%D \starttypen
%D \def\starttypen% We simplify the \ConTeXt\ macro.
%D   {\bgroup
%D    \everypar{} % We disable some troublesome mechanisms.
%D    \advance\leftskip by 1em
%D    \processdisplayverbatim{\stoptypen}}
%D
%D \def\stoptypen%
%D   {\egroup}
%D \stoptypen
%D
%D The implementation itself was typeset with:
%D
%D \starttypen
%D \def\startdefinition%
%D   {\bgroup
%D    \everypar{} % Again we disable some troublesome mechanisms.
%D    \let\obeycharacters=\setupprettytextype
%D    \EveryPar{\showparagraphcounter}%
%D    \EveryLine{\showlinecounter}%
%D    \verbatimcorps
%D    \processdisplayverbatim{\stopdefinition}}
%D
%D \def\stopdefinition%
%D   {\egroup}
%D \stoptypen
%D
%D And because we have both \type{\EveryPar} and
%D \type{\EveryLine} available, we can implement a dual
%D numbering mechanism:
%D
%D \starttypen
%D \newcount\paragraphcounter
%D \newcount\linecounter
%D
%D \def\showparagraphcounter%
%D   {\llap
%D      {\bgroup
%D       \counterfont
%D       \hbox to 4em
%D         {\global\advance\paragraphcounter by 1
%D          \hss \the\paragraphcounter \hskip2em}%
%D       \egroup
%D       \hskip1em}}
%D
%D \def\showlinecounter%
%D   {\llap
%D      {\bgroup
%D       \counterfont
%D       \hbox to 2em
%D         {\global\advance\linecounter by 1
%D          \hss \the\linecounter}%
%D       \egroup
%D       \hskip1em}}
%D \stoptypen
%D
%D One may have noticed that the \type{\EveryPar} is only
%D executed once, because we consider each piece of verbatim
%D as one paragraph. When one wants to take the empty lines
%D into account, the following assignments are appropriate:
%D
%D \starttypen
%D \EveryLine
%D   {\iflinepar
%D      \showparagraphcounter
%D    \fi
%D    \showlinecounter}
%D \stoptypen
%D
%D In this case, nothing has to be assigned to \type{\EveryPar},
%D maybe except of just another extra numbering scheme. The
%D macros used to typeset this documentation are a bit more
%D complicated, because we have to take take 'long' margin
%D lists into account. When such a list exceeds the previous
%D pargraph we postpone placement of the paragraph number till
%D there's room. This way so it does not clash with the margin
%D words.

%D Normally such commands have to be embedded in a decent setup
%D structure, where options can be set at will.
%D
%D Now let's summarize the most important commands.
%D
%D \starttypen
%D \processinlineverbatim{\closingcommand}
%D \processdisplayverbatim{\closingcommand}
%D \processfileverbatim{filename}
%D \stoptypen
%D
%D We can satisfy our own specific needs with the following
%D interfacing macro's:
%D
%D \starttypen
%D \obeyspaces  \obeytabs  \obeylines  \obeypages  \obeycharacters
%D \stoptypen
%D
%D Some needs are fulfilled already with:
%D
%D \starttypen
%D \setcontrolspace  \settabskips  \setupprettytextype
%D \stoptypen
%D
%D lines can be enhanced with ornaments using:
%D
%D \starttypen
%D \everypar  \everyline  \iflinepar
%D \stoptypen
%D
%D and color support is implemented by:
%D
%D \starttypen
%D \texbeginofpretty[#1] ... \texendofpretty
%D \stoptypen
%D
%D We can influence the verbatim environment with the following
%D macro and booleans:
%D
%D \starttypen
%D \obeyemptylines  \splittexparameters...  \splittexcontrols...
%D \stoptypen
%D
%D The color support macro can be redefined by the user. The
%D parameter \type{#1} can be one of the four 'fixed'
%D identifiers {\em texprettyone}, {\em texprettytwo}, {\em
%D texprettythree} and {\em texprettyfour}. We have implemented a
%D more or less general PostScript color support mechanism,
%D using \type{specials}. One can toggle between color and
%D grayscale with:
%D
%D \starttypen
%D \setgrayverbatim  \setcolorverbatim
%D \stoptypen

%D \macros
%D   {permitshiftedendofverbatim}
%D   {}
%D
%D We did not mention one drawback of the mechanism described
%D here. The closing command must start at the first position
%D of the line. In \CONTEXT\ we will not have this drawback,
%D because we can test if the end command is a substring of the
%D current line. The testing is done by two of the support
%D macros, which of course are not available in a stand alone
%D application of this module.

\ifx \undefined \doifinstringelse \else

\def\processdisplayverbatim#1%
  {\par
   \bgroup
   \escapechar=-1
   \xdef\verbatimname{\string#1}%
   \egroup
   \def\endofdisplayverbatim{\csname\verbatimname\endcsname}%
   \bgroup
   \parindent\!!zeropoint
   \ifdim\lastskip<\parskip
     \removelastskip
     \vskip\parskip
   \fi
   \parskip\!!zeropoint
   \processingverbatimtrue
   \expandafter\let\csname\verbatimname\endcsname=\relax
   \expandafter\convertargument\csname\verbatimname\endcsname
     \to\endofverbatimcommand
   \verbatimfont
   \setupcopyverbatim
   \ifskipfirstverbatimline 
     \let\doverbatimline=\relax
   \else
     \let\doverbatimline=\dodoverbatimline
   \fi
   \copyverbatimline}

\let\doifendofverbatim=\doifelse 

\def\permitshiftedendofverbatim%
  {\let\doifendofverbatim=\doifinstringelse}

{\obeylines%
 \long\gdef\copyverbatimline#1
   {\ifx\doverbatimline\relax% gobble rest of the first line
      \let\doverbatimline=\dodoverbatimline%
      \def\next{\copyverbatimline}%
    \else%
       \convertargument#1 \to\next%
       \ifx\next\emptyspace%
         \def\next%
           {\doemptyverbatimline{#1}%
            \copyverbatimline}%
       \else%
         \doifendofverbatim{\endofverbatimcommand}{\next}%
           {\def\next%
              {\egroup\endofdisplayverbatim}}%
           {\def\next%
              {\doverbatimline{#1}%
               \copyverbatimline}}%
       \fi%
   \fi%
   \next}}

\fi

\protect

\endinput
