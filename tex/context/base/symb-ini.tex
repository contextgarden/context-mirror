%D \module
%D   [       file=symb-nav,
%D        version=1998.07.20,
%D          title=\CONTEXT\ Symbol Libraries,
%D       subtitle=Navigational Symbols,
%D         author=Hans Hagen,
%D           date=\currentdate,
%D      copyright={PRAGMA / Hans Hagen \& Ton Otten}]
%C
%C This module is part of the \CONTEXT\ macro||package and is
%C therefore copyrighted by \PRAGMA. See mreadme.pdf for 
%C details. 

%D The macros described here used to be part of the \type
%D {core-con} module. I decided to move them here when 
%D symbolsets saw the light. Let their light shine.  

\writestatus{loading}{Context Symbol Libraries / Initialization}

\unprotect

% TOBIAS 

\startmessages  dutch  library: symbols
  title: symbolen
      1: symboolset -- wordt geladen
\stopmessages

\startmessages  english  library: symbols
  title: symbols
      1: loading symbolset --
\stopmessages

\startmessages  german  library: symbols
  title: Symbole
      1: Lade Symboldatei --    
\stopmessages

\startmessages  czech  library: symbols
  title: symboly
      1: nacita se soubor symbolu --    
\stopmessages

\startmessages  italian  library: symbols
  title: simboli
      1: caricamento gruppo di simboli --
\stopmessages

\startmessages  norwegian  library: symbols
  title: symboler
      1: leser inn symbolsett --
\stopmessages

\startmessages  romanian  library: symbols
  title: simboluri
      1: se incarca setul de simboluri --
\stopmessages

%D \macros
%D   {definesymbol, symbol}
%D 
%D Converting numbers or levels into a character, romannumeral,
%D symbol or something else, is supported by many \CONTEXT\
%D commands. Therefore we need a mechanism for linking such
%D numbers to their counterparts. 
%D 
%D First we take care of symbols. These are for instance used
%D in enumerations and itemizations. We have: 
%D 
%D \showsetup{\y!definesymbol} 
%D \showsetup{\y!symbol} 
%D 
%D Symbols are simply linked to a tag. Such tags can be numbers
%D or strings. 
%D 
%D \starttypen
%D \definesymbol [1]       [$\bullet$]
%D \definesymbol [level 5] [$\star$]
%D \stoptypen

% ss:tag     -> symbol 
% ss:set:tag -> symbol out of set 
% sstag      -> list of symbols in set  

\def\dodefinesymbol[#1][#2]%
  {\ifx\currentsymbolset\empty
     \setvalue{\??ss:#1}{#2}%
   \else
     \doifundefinedelse{\??ss\currentsymbolset}
       {\let\currentsymbollist\empty}
       {\edef\currentsymbollist{\csname\??ss\currentsymbolset\endcsname}}%
     \addtocommalist{#1}\currentsymbollist
     \setvalue{\??ss:\currentsymbolset:#1}{#2}%
     \letvalue{\??ss\currentsymbolset}\currentsymbollist
   \fi}

\def\definesymbol%
  {\dodoubleargument\dodefinesymbol}

\unexpanded\def\symbol%      % This one always gobbles spaces, 
  {\dodoubleempty\dosymbol}  % so never change it again!

\def\dosymbol[#1][#2]% 
  {\ifsecondargument
     \doifdefinedelse{\??ss:#1:#2}
       {\dodosymbol{#1:#2}}
       {\doifdefinedelse{\??ss:#2}{\dodosymbol{#2}}{#2}}%
   \else\ifx\currentsymbolset\empty
     \doifdefinedelse{\??ss:#1}{\dodosymbol{#1}}{#1}%
   \else
     \doifdefinedelse{\??ss:\currentsymbolset:#1}
       {\dodosymbol{\currentsymbolset:#1}}
       {\doifdefinedelse{\??ss:#1}{\dodosymbol{#1}}{#1}}%
   \fi\fi}

\def\dodosymbol#1% \relax's prevent lookahead problems 
  {{\the\everysymbol\getvalue{\??ss:#1}\relax}\relax}

\newtoks\everysymbol

% % % % %
% this should go in symb-fig, to be loaded after core-fig

%D \macros
%D   {definefiguresymbol}
%D
%D To simplify defining figure symbols, we offer:
%D
%D \showsetup{\y!definefiguresymbol} 
%D
%D By default, such symbols scale along the current bodyfont
%D size.

\def\defaultsymbolfactor{10}

\def\dohandlefiguresymbol#1#2% 
  {\externalfigure[#1][\c!reset=\v!ja,\c!symbool=\v!ja,\c!hfactor=\defaultsymbolfactor,#2]}

\appendtoks \resetexternalfigures \to \everysymbol

\def\definefiguresymbol%
  {\dotripleempty\dodefinefiguresymbol}

\def\dodefinefiguresymbol[#1][#2][#3]%
  {\ifsecondargument
     \definesymbol[#1][\dohandlefiguresymbol{#2}{#3}]%
   \fi}

% but for the moment we keep it here
% % % % % % 

%\def\objectsymbol[#1]%
%  {\dopresetfieldsymbol{#1}\dogetfieldsymbol{#1}}

%D \macros
%D   {doifsymboldefinedelse}
%D
%D A handy private one: 

\def\doifsymboldefinedelse#1#2#3%
  {\ifx\currentsymbolset\empty
     \doifdefinedelse{\??ss:#1}{#2}{#3}%
   \else
     \doifdefinedelse{\??ss:\currentsymbolset:#1}
       {#2}{\doifdefinedelse{\??ss:#1}{#2}{#3}}%
   \fi}

%D \macros
%D   {setupsymbolset,startsymbolset}
%D 
%D From these macro definitions one can deduce that symbols can
%D be grouped in symbol sets: 
%D 
%D \starttypen
%D \startsymbolset [navigation 1]
%D   \definefiguresymbol [Next] [mp-symb.1] 
%D   \definefiguresymbol [Prev] [mp-symb.2]
%D \stopsymbolset
%D \stoptypen
%D 
%D Such a symbol can be typeset with:
%D 
%D \starttypen
%D \setupsymbolset[navigation 1]\symbol[Next]
%D \stoptypen
%D 
%D or simply:
%D 
%D \starttypen
%D \symbol[navigation 1][Next]
%D \stoptypen
%D 
%D Formally:
%D 
%D \showsetup{\y!setupsymbolset} 
%D \showsetup{\y!startsymbolset} 

\let\currentsymbolset\empty

\def\startsymbolset%
  {\localpushmacro\currentsymbolset
   \setupsymbolset}

\def\stopsymbolset%
  {\localpopmacro\currentsymbolset}

\def\setupsymbolset[#1]%
  {\def\currentsymbolset{#1}} 

%D \macros
%D   {showsymbolset}
%D
%D \showsetup{\y!showsymbolset} 

\def\doshowsymbolset[#1][#2]% looks like \showexternalfigureb
  {\vbox\bgroup
   \blanko
   \getparameters[\??ss][\c!n=5,#2]%
   \setupsymbolset[#1]%
   \doifdefined{\??ss\currentsymbolset}
     {\global\let\allfigures=\empty
      \doglobal\newcounter\figurecounter
      \setupcolors[\c!status=\v!start]% to prevent mps color conversion 
      \mindermeldingen
      \def\doshowsymbols% global needed due to grouping in alignment
        {\expanded{\globalprocesscommalist[\getvalue{\??ss\currentsymbolset}]\noexpand\docommando}}%
      \def\docommando##1%
        {\vbox
           {\forgetall
            \tttf 
            \halign
              {\hss\quad####\strut\quad\hss\cr
               \symbol[##1]\quad{\red\ruledhbox{\black\symbol[##1]}}\cr
               \tfx##1\cr}}%
         \doglobal\increment\figurecounter
         \ifnum\figurecounter=\@@ssn
           \doglobal\newcounter\figurecounter
           \def\next{\crcr\noalign{\vskip1ex}}%
         \else
           \def\next{&}%
         \fi
         \next}%
      \tabskip=\!!zeropoint \!!plus 1fill
      \halign to \hsize
        {&\hss##\hss\cr\doshowsymbols\crcr}}%
   \blanko
   \egroup}

\def\showsymbolset%
  {\dodoubleempty\doshowsymbolset}

%D \macros 
%D   {usesymbols}
%D
%D \showsetup{\y!usesymbols}

\def\dousesymbols#1%
  {\makeshortfilename[\f!symbolprefix#1]%
   \showmessage{\m!symbols}{1}{#1}%
   \startreadingfile
   \readsysfile{\shortfilename}{}{}%
   \stopreadingfile}

\def\usesymbols[#1]%
  {\processcommalist[#1]\dousesymbols}

%D As longs as symbols are linked to levels or numbers, we can 
%D also use the conversion mechanism, but in for instance the 
%D itemization macros, we prefer symbols because they can more 
%D easier be (partially) redefined. 

%D We predefine some common symbols and conversions that will
%D be understood by many commands. 

\definesymbol [\v!geen]   []
\definesymbol [bullet]    [\mathematics{\bullet}]
\definesymbol [dash]      [\mathematics{-}]
\definesymbol [star]      [\mathematics{\star}]
\definesymbol [triangle]  [\mathematics{\triangleright}]
\definesymbol [circle]    [\mathematics{\circ}]
\definesymbol [medcircle] [\hbox{\setsmallbodyfont\raise\!!onepoint\hbox{$\bigcirc$}}]
\definesymbol [bigcircle] [\mathematics{\bigcirc}]
\definesymbol [square]    [\hbox{\hsmash{$\sqcup$}$\sqcap$}]
\definesymbol [diamond]   [\mathematics{\diamond}]

\definesymbol [1] [{\symbol[bullet]}]
\definesymbol [2] [{\symbol[dash]}]
\definesymbol [3] [{\symbol[star]}]
\definesymbol [4] [{\symbol[triangle]}]
\definesymbol [5] [{\symbol[circle]}]
\definesymbol [6] [{\symbol[medcircle]}]
\definesymbol [7] [{\symbol[bigcircle]}]
\definesymbol [8] [{\symbol[square]}]

%D Special hyperlinks, namely those to pages or navigational 
%D properties, are associated with symbols. 

\definesymbol [\v!eerstepagina]      [\gotobegincharacter]
\definesymbol [\v!vorigepagina]      [\gobackwardcharacter]
\definesymbol [\v!volgendepagina]    [\goforwardcharacter]
\definesymbol [\v!laatstepagina]     [\gotoendcharacter]
\definesymbol [\v!eerstesubpagina]   [\gotobegincharacter]
\definesymbol [\v!vorigesubpagina]   [\gobackwardcharacter]
\definesymbol [\v!volgendesubpagina] [\goforwardcharacter]
\definesymbol [\v!laatstesubpagina]  [\gotoendcharacter]
\definesymbol [\v!PreviousJump]      [\gobackjumpcharacter]
\definesymbol [\v!NextJump]          [\goforjumpcharacter]
\definesymbol [\v!CloseDocument]     [\closecharacter]

\definesymbol [\v!eerste]   [{\symbol[\v!eerstepagina]}]
\definesymbol [\v!vorige]   [{\symbol[\v!vorigepagina]}]
\definesymbol [\v!volgende] [{\symbol[\v!volgendepagina]}]
\definesymbol [\v!laatste]  [{\symbol[\v!laatstepagina]}]
\definesymbol [\v!ergens]   [\gotosomewherecharacter]
\definesymbol [\v!nergens]  [\gonowherecharacter]

\definesymbol [\v!achteruit] [{\symbol[\v!vorigepagina]}]
\definesymbol [\v!vooruit]   [{\symbol[\v!volgendepagina]}]

%D The next two symbols (\symbol[P] and \symbol[S]) are 
%D variations in their math counterparts. The following ones 
%D {\em do} scale. 

\definesymbol[S][\getglyph{MathSymbol}{\char"78}]
\definesymbol[P][\getglyph{MathSymbol}{\char"7B}]

%D These symbols are taken from the Computer Moders Roman
%D symbol set or, when present, from the additional symbols of
%D the American Mathematical Society. Of course one can use 
%D his or her own symbols by redefining them. 

\def\dogotocharacter#1#2#3%
  {\ifx#1\undefined#2\else#3\fi}

\def\gotobegincharacter%
  {\hbox
     {\dogotocharacter\blacktriangleleft
        {\setbox0=\hbox{\mathematics{\triangleleft}}%
         \vrule\!!width.085ex\!!height1.075\ht0\!!depth\dp0
         \kern-.11ex\box0} 
        {\setbox0=\hbox{\mathematics{\blacktriangleleft}}%
         \setbox2=\hbox{\vrule\!!height\ht0\!!depth\dp0\!!width.25ex}%
         \hbox{\lower.03ex\box2\kern-.35ex\box0}}}}

\def\gotoendcharacter%
  {\hbox
     {\dogotocharacter\blacktriangleright
        {\setbox0=\hbox{\mathematics{\triangleright}}%
         \copy0\kern-.11ex 
         \vrule\!!width.085ex\!!height1.075\ht0\!!depth\dp0}  
        {\setbox0=\hbox{\mathematics{\blacktriangleright}}%
         \setbox2=\hbox{\vrule\!!height\ht0\!!depth\dp0\!!width.25ex}%
         \hbox{\box0\kern-.35ex\lower.03ex\box2}}}}

\def\gobackwardcharacter%
  {\mathematics{\dogotocharacter\blacktriangleright\triangleleft\blacktriangleleft}}

\def\goforwardcharacter%
  {\mathematics{\dogotocharacter\blacktriangleright\triangleright\blacktriangleright}}

\def\gonowherecharacter%
  {\mathematics{\bullet}}

%\def\gotosomewherecharacter% {} permits ^\...
%  {{\hbox{\hsmash{\gobackwardcharacter}\goforwardcharacter}}}

\def\gotosomewherecharacter%
  {{\hbox{\hsmash{\symbol[\v!vorige]}\symbol[\v!volgende]}}}

\unexpanded\def\closecharacter%
  {\dogotocharacter\boxtimes
     {\ruledhbox{\mathematics{\times}}}
     {\mathematics{\boxtimes}}}

\def\goforjumpcharacter%
  {\hbox{\goforwardcharacter \kern-.5em\goforwardcharacter}}

\def\gobackjumpcharacter%
  {\hbox{\gobackwardcharacter\kern-.5em\gobackwardcharacter}}

% temporarily here

% gejat van Knuth (zie \copyright, p356)

\def\omcirkeld#1%
  {{\ooalign{\hfil\raise0.07ex\hbox{{\tfx#1}}\hfil\crcr\mathhexbox20D}}}

\def\copyright
  {\omcirkeld{c}}

\protect \endinput
