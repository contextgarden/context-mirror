%D \module
%D   [       file=supp-new,
%D        version=1997.01.03,
%D          title=\CONTEXT\ Support Macros,
%D       subtitle=New Ones,
%D         author=Hans Hagen,
%D           date=\currentdate,
%D      copyright={PRAGMA / Hans Hagen \& Ton Otten}]
%C
%C This module is part of the \CONTEXT\ macro||package and is
%C therefore copyrighted by \PRAGMA. Non||commercial use is 
%C granted. 

\unprotect

\def\DoMod #1by#2to#3%
  {\scratchcounter=#1\relax
   \divide\scratchcounter by #2\relax
   \multiply\scratchcounter by #2\relax
   #3=#1\relax
   \advance#3 by -\scratchcounter}

\def\DoDiv #1by#2to#3%
  {#3=#1\relax
   \divide#3 by #2\relax}

\def\dounprotected#1\par%
  {#1\protect}

\def\unprotected%
  {\unprotect\dounprotected}

%D \pagina
%D \starttypen
%D \def\obeyhyphens% % after fontswitch
%D   {\def\obeyedspace%
%D      {\hyphenchar\font=45
%D       \spaceskip.5em\!!plus.25em\!!minus.25em\relax%
%D       \def\obeyedspace{ }}}
%D \stoptypen

%D Standaard kan een spatie (zoals ~) uitrekken. Dit is in
%D overzichten niet altijd de bedoeling, vandaar:

\def\fixedspace%
  {\hskip\fontdimen2\font\relax}

%\def\ExpandSecondAfter#1#2#3%
%  {\!!toksa={#2}%
%   \edef\!!stringa{#3}%
%   \edef\expanded%
%     {\noexpand#1{\the\!!toksa}{\!!stringa}}%
%   \expanded}
%
%\def\ExpandThirdAfter#1#2#3#4%
%  {\!!toksa={#2}%
%   \!!toksb={#3}%
%   \edef\!!stringa{#4}%
%   \edef\expanded%
%     {\noexpand#1{\the\!!toksa}{\the\!!toksb}{\!!stringa}}%
%   \expanded}

%\def\indirect#1#2#3%
%  {\@EA#1\@EA#2\@EA{\@EA#3\csname\s!do\string#2\endcsname}%
%   \@EA#1\csname\s!do\string#2\endcsname}
%
%\def\doubleemptied#1#2#3%
%  {\indirect#1#2\dodoublempty}
%
%\indirect\def\stelietsin\dodoubleempty[#1][#2]%
%  {...}
%
%\doubleemptied\def\stelietsin[#1][#2]%
%  {...}

% in mult-set
%
%\def\defaultsetup{def}
%
%\def\selectdefaultsetup#1#2%
%  {\writestatus{setup}{choose #1 setupfile}%
%   \bgroup
%   \endlinechar=-1
%   \global\read16 to \usersetup
%   \egroup
%   \ifx\usersetup\empty
%     \let\usersetup=\defaultsetup
%   \fi
%   \readfile{#2\usersetup}{}{}%
%   \writestatus{setup}{loading #1 setupfile #2\usersetup}}

\protect

\endinput
