%D \module
%D   [       file=xtag-ini,
%D        version=2000.12.20,
%D          title=\CONTEXT\ XML Support,
%D       subtitle=Initialization,
%D         author=Hans Hagen,
%D           date=\currentdate,
%D      copyright={PRAGMA / Hans Hagen \& Ton Otten}]
%C
%C This module is part of the \CONTEXT\ macro||package and is
%C therefore copyrighted by \PRAGMA. See mreadme.pdf for
%C details.

\beginTEX 
  \writestatus{XML}{sorry, XML is only supported in (pdf)etex}
  \def\startXMLdefinitions{\gobbleuntil\stopXMLdefinitions}
  \endinput 
\endTEX 

\beginETEX XML

  \bgroup \obeylines

    \gdef\startXMLdefinitions% 
      {\bgroup\obeylines\dostartXMLdefinitions} 

    \gdef\dostartXMLdefinitions #1 
      {\egroup\doifsomething{#1}{\writestatus{XML}{loading #1 support}}}

    \global\let\stopXMLdefinitions\relax

  \egroup

\endETEX

\writestatus{loading}{Context XML Macros (ini)}

%D Remark: some hard coded character things will be replaced
%D by named glyphs as soon as the upgraded encoding modules
%D are released. At that moment, unicode support will be
%D provided in accordance with the normal support in \CONTEXT.

%D Like it or not, this module deals with angle bracketed
%D input. Processing \XML\ alike input in \CONTEXT\ has been
%D possible since 1994, but several methods ran in parallel
%D and were implemented in modules like \type {m-sgml}.
%D
%D There is no one optimal solution for processing \XML\ data.
%D The oldest method was based on a very simple preprocessor
%D written in \PERL: \type {<command>} was converted into
%D \type {\begSGML[command]} and optional parameters were
%D passed.
%D
%D A second method is to use a \PERL\ or \XSL\ transformation
%D script that produces \CONTEXT\ commands. This method is
%D much slower, mainly because the whole document is read into
%D memory and a document tree is to be build. The advantage is
%D that processing of the resulting document is fast.
%D
%D The third method uses a basic parser written in the \TEX\
%D language, and apart from a few pitfalls, this method is
%D clean and efficient, but not always robust. Because errors
%D in the input are not catched on forhand, processing in
%D \TEX\ may fail due to errors. But, given that a document
%D can be validated on forehand, this is no big problem.
%D
%D Each method has it's advantage, but especially the third
%D method puts some demands on \CONTEXT, since no interference
%D between the parser and the core commands may occur. What
%D method is used, depends on the situation.
%D
%D All three methods introduce some problems in interfacing to
%D core \CONTEXT\ functionality. This is due to the fact that
%D sometimes we want to typeset content directly, while on
%D other cases we just want to pick up data for later usage,
%D either or not using delimited arguments. And, when moving
%D data around, there is always the expansion problem to deal
%D with.
%D
%D In order to be able to incorporate \XML\ style definitions
%D into basic \TEX\ styles, we will provide some basic
%D functionality in the core itself.

%D We will be dealing with elements, which means that we have
%D to take care of \type {<this>} and \type {</that>}, but
%D also with \type {<such/>} and \type {<so />}. In due time
%D this module will deal with all these animals in a
%D convenient way. In some cases the upper and lowercase
%D alternatives need to be dealt with, although this is not
%D realy needed since XML is case sensitive.
%D
%D We also have to handle entities, like \type {&you;} and
%D \type {&me;}. These are quite easy to deal with and need to
%D be hooked into the encoding and abbreviation mechanisms.
%D
%D And then there are the parameters to be taken care of. Here
%D we meet \type {key="value"} but also \type {key='eulav'}
%D and even the spacy \typ {key = "value"}.
%D
%D Since we have to handlers for each element and entity, we
%D will create a few namespaces. Special care has to be
%D given to preformated code.

\unprotect

\def\@@XML            {XML:}
\def\@@XMLentity      {\@@XML ent}
\def\@@XMLelement     {\@@XML ele}
\def\@@XMLvariable    {\@@XML var}
\def\@@XMLvalue       {\@@XML val}
\def\@@XMLpars        {\@@XML par}
\def\@@XMLdata        {\@@XML dat}
\def\@@XMLcode        {\@@XML cod}
\def\@@XMLinstruction {\@@XML ins}
\def\@@XMLmap         {\@@XML map}
\def\@@XMLlist        {\@@XML lst}

\newtoks\XMLtoks
\newtoks\XMLresetlist

\chardef\XMLargumentmode=0

\newif\ifignoreXMLcase
\newif\ifignoreXMLspaces
\newif\iffixedXMLfont

%D \macros
%D   {compound}
%D
%D We will overload the already active \type {|} so we have
%D to save its meaning in order to be able to use this handy
%D macro.
%D
%D \starttypen
%D so test\compound{}test can be used instead of test||test
%D \stoptypen

\let\docompound=| \def\compound#1{\docompound#1|}

%D We will implement the parser by making a few characters
%D active. For that reason we also have to save their
%D original meaning. The core handlers are \type
%D {\doXMLentity} and \type {\doXMLelement}.

%D \macros
%D   {enableXML}
%D
%D The macro \type {\enableXML} will be used to turn on the
%D parser. This means that after that, \TEX\ commands starting
%D with a backslash will not longer be read as such. There is
%D a way around this, but for convenience \TEXEXEC\ will take
%D care of processing raw \XML\ files in a transparant way.

\bgroup
\catcode`\*=\@@comment
\catcode`\.=\@@escape
.catcode`.B=.@@begingroup
.catcode`.E=.@@endgroup

.catcode`.&=.@@active .gdef.letterampersand  B.string&E
.catcode`.<=.@@active .gdef.letterless       B.string<E

.catcode`.#=.@@active .gdef.letterhash       B.string#E
.catcode`.$=.@@active .gdef.letterdollar     B.string$E
.catcode`.%=.@@active
.catcode`.\=.@@active .gdef.letterbackslash  B.string\E
.catcode`.^=.@@active .gdef.letterhat        B.string^E
.catcode`._=.@@active .gdef.letterunderscore B.string_E
.catcode`.{=.@@active .gdef.letterbgroup     B.string{E
.catcode`.}=.@@active .gdef.letteregroup     B.string}E
.catcode`.|=.@@active .gdef.letterbar        B.string|E
.catcode`.~=.@@active .gdef.lettertilde      B.string~E

.gdef.enableXMLexpansion              
  B.def<B.doXMLelementE.let&=.doXMLentityE

.gdef.disableXMLexpansion  
  B.unexpanded.def<B.doXMLelementE.let&=.doXMLentityE

.gdef.enableXML*
  B.catcode`.!=.@@other*
   .catcode`.?=.@@other*
   .catcode`.&=.@@active .let&=.doXMLentity*
   .catcode`.<=.@@active .unexpanded.def<B.doXMLelementE*
   .catcode`.>=.@@other*
   .catcode`.#=.@@active .def#B&tex-hash;E*
   .catcode`.$=.@@active .def$B&tex-dollar;E*
   .catcode`.%=.@@active .def%B&tex-percent;E*
   .catcode`.\=.@@active .def\B&tex-backslash;E*
   .catcode`.^=.@@active .def^B&tex-hat;E*
   .catcode`._=.@@active .def_B&tex-underscore;E*
   .catcode`.{=.@@active .def{B&tex-leftbrace;E*
   .catcode`.}=.@@active .def}B&tex-rightbrace;E*
   .catcode`.|=.@@active .def|B&tex-bar;E*
   .catcode`.~=.@@other* active .def~B&tex-tilde;E*
   .relax* needed for successive .if's  
  E

.gdef.enableXMLelements*
  B.catcode`.<=.@@active .unexpanded.def<B.doXMLelementE*
   .catcode`.>=.@@other*
   .relax* needed for successive .if's  
  E

.egroup

%D An element can be singular or paired. A singular element is
%D called an empty element. The following definitions are
%D equivalent:
%D
%D \starttypen
%D <eerste></eerste>  <eerste/>  <eerste />
%D \stoptypen
%D
%D Empty elements can have arguments too. Conforming the
%D standard, each key must have a value. These are separated
%D by an \type {=} sign and the value is delimited by either
%D \type {"} or \type {'}. There may be spaces around the
%D equal sign.
%D
%D \starttypen
%D <eerste a= "b" c ="d" />  <eerste a = "b" c="d"/>
%D \stoptypen
%D
%D Officially the following definition is not valid:
%D
%D \starttypen
%D <eerste>some text</eerste>  <eerste/>  <eerste />
%D \stoptypen
%D
%D Although we can handle both cases independently, this is
%D seldom needed.
%D
%D Processing instructions are identified by a~\type {?} and are
%D like empty elements.
%D
%D \starttypen
%D <?doel a="b" c="d"?> <?doel a="b" c="d" ?>
%D \stoptypen
%D
%D Comment is formatted as follows.
%D
%D \starttypen
%D <!-- comment -->
%D \stoptypen
%D
%D Verbatim code inits purest form is called \type {CDATA} and
%D is embedded in the following ugly and therefore recognizable
%D way:
%D
%D \starttypen
%D <![CDATA[
%D Dit is nogal verbatim !
%D Dit is nogal verbatim !
%D Dit is nogal verbatim !
%D ]]>
%D \stoptypen

%D The parser is implemented as a multi||step macro. Because
%D \type {!} and \type {?} should be picked up correctly, we
%D need to define a few macros in unprotected mode!
%D
%D Because \XML\ is defined with some restrictions in mind,
%D parsing the elements is not that complicated. First we have
%D to determine if we're dealing with a comment or processing
%D instruction. We need a bit of grouping because we have to
%D mess up with catcodes. We probably have to treat a few
%D more catcode and first character cases. We need to use 
%D \type {\begingroup} here, otherwise we get funny spaces in 
%D math.

\protect

\long\def\doXMLelement#1%
  {\begingroup % maybe tab and space needs some treatment too: \catcode`\ =10 % \@@space
   \catcode`\^^M=10
   \if#1!\let\next \xdoXMLelement \else
   \if#1?\let\next \ydoXMLelement \else
         \let\next \zdoXMLelement \fi\fi
   \next#1}

%D By using a few {\expandafter}'s we can us a \type {\next}
%D construction. We could speed the first char test up a bit
%D by using an installer and something \typ {\getvalue
%D {#1doXMLelement}} (todo). 

\long\def\doXMLelement#1%
  {\begingroup % maybe tab and space needs some treatment too: \catcode`\ =10 % \@@space
   \catcode`\^^M=10\relax
   \if#1!\expandafter                         \xdoXMLelement \else
   \if#1?\expandafter\expandafter\expandafter \ydoXMLelement \else
         \expandafter\expandafter\expandafter \zdoXMLelement \fi\fi
   #1}

%D The (yet experimental) \type {CDATA} parser is implemented
%D on top of the verbatim environment.

\long\def\xdoXMLelement !#1 % !-- --> or !xyz >
  {\endgroup
   \doifelse{#1}{--}
     {\long\def\nextelement{\gobbleuntil{-->}}}
     {\doifelse{#1}{[CDATA[}
        {\long\def\nextelement{\skipfirstverbatimlinefalse
                               \processtaggeddisplayverbatim{]]>}}}
        {\long\def\nextelement{\gobbleuntil{>}}}}%
    \nextelement}

%D In our case, processing instructions are only needed if
%D we want specific \CONTEXT\ support. This may be useful in
%D applications where the data is generated by an
%D application. We will implement a \CONTEXT\ code handler
%D later.

\long\def\ydoXMLelement#1 #2?>% ?target ?>
  {\endgroup\dodoXMLprocessor{#1}{#2}}

%D The normal elements are handled by \type {\dodoXMLelement}.

\long\def\zdoXMLelement#1>%
  {\endgroup\dodoXMLelement#1 >}

%D Now we switch to unprotected mode again.

\unprotect

%D The processing instructions handler is implemented as
%D follows.

\long\def\dodoXMLprocessor#1%
  {\ifundefined{\@@XMLinstruction:#1}%
     \let\next\gobbleoneargument
   \else
     \def\next{\getvalue{\@@XMLinstruction:#1}}%
   \fi
   \next}

\long\def\defineXMLprocessor[#1]#2% watch the ?
  {\long\setvalue{\@@XMLinstruction:?#1}{#2}}

%D As an example, we implement a \CONTEXT\ code handler:

\defineXMLprocessor[context]        {\contextXMLcommand}
\defineXMLprocessor[context-command]{\contextXMLcommand}

\def\contextXMLcommand#1%
  {\pushmacro\disableXML
   \def\disableXML{\global\let\afterXMLprocessor\empty}%
   \global\let\afterXMLprocessor\enableXML
   \setnormalcatcodes\scantokens{#1}\afterXMLprocessor
   \popmacro\disableXML}

\defineXMLprocessor[context-directive]{\contextXMLdirective}

\def\contextXMLdirective#1%
  {\docontextXMLdirective#1 dummy dummy dummy\end}

\def\docontextXMLdirective#1 #2 #3 #4\end% class var value
  {\setvalue{\@@XMLvariable:#1:#2}{#3}}

\defineXMLprocessor[context-message]{\contextXMLmessage}

\def\contextXMLmessage#1%
  {\writestatus{xml}{#1}}

\def\setnormalcatcodes%
  {\catcode`\!=\@@other       \catcode`\?=\@@other
   \catcode`\&=\@@alignment   \catcode`\<=\@@other
   \catcode`\#=\@@parameter   \catcode`\$=\@@mathshift
   \catcode`\%=\@@comment     \catcode`\\=\@@escape
   \catcode`\^=\@@superscript \catcode`\_=\@@subscript
  %\catcode`\|=\@@active      \catcode`\~=\@@active
   \catcode`\{=\@@begingroup  \catcode`\}=\@@endgroup}

\let\disableXML\setnormalcatcodes

%D Given the previous definition, and given that \ETEX\ is
%D used, we can now say:
%D
%D \starttypen
%D <?context {\bf Start Of Some \TeX\ Text} ?>
%D \stoptypen
%D
%D A non||\ETEX\ solution is also possible, using buffers,
%D but for the moment we assume that \ETEX\ is used.

%D Next we will implement the normal element handler.

\let\currentXMLarguments\empty
\let\currentXMLelement  \empty

\newtoks\everyXMLelement

\long\def\dodoXMLelement#1 #2>%
  {\def\!!stringa{#2}%
   \def\!!stringb{/ }%
   \ifx\!!stringa\empty
     \let\currentXMLarguments\empty
     \def\currentXMLelement{#1}%
     \the\everyXMLelement
   \else\ifx\!!stringa\!!stringb
     \let\currentXMLarguments\empty
     \def\currentXMLelement{#1/}%
     \the\everyXMLelement
   \else
     \def\currentXMLelement{#1}%
     \def\currentXMLarguments{#2}%
     \the\everyXMLelement
    %\getXMLarguments\currentXMLelement{#2}%
     \dogetXMLarguments\currentXMLelement#2>%
   \fi \fi
   \executeXMLelement\currentXMLelement}

\def\executeXMLelement#1%
  {\getvalue{\@@XMLelement:#1}}

\newif\ifXMLrawentities

% \bgroup
% 
% \catcode`<=\@@active
% 
% \gdef\defineXMLentity[#1]#2%
%   {\unspaceargument#1\to\ascii
%    \long\setvalue{\@@XMLelement:ent:\@EA\firstofoneargument\ascii/}{#2}}
% 
% \gdef\doXMLentity#1;%
%   {\ifXMLrawentities#1\else\executeXMLentity{#1}\fi}
% 
% \gdef\executeXMLentity#1%
%   {<ent:#1/>}
% 
% \gdef\getXMLentity#1% 
%   {\getvalue{\@@XMLelement:ent:#1/}}
% 
% \gdef\doifXMLentityelse#1#2#3%
%   {\ifundefined{\@@XMLelement:ent:#1/}#3\else#2\fi}
% 
% \egroup

\gdef\defineXMLentity[#1]#2%
  {\unspaceargument#1\to\ascii
   \long\setvalue{\@@XMLentity:\@EA\firstofoneargument\ascii}{#2}}

% we need to be able to do: 
%
% \defineXMLentity[amp] {\FunnyAmp} \def\FunnyAmp#1;{\getXMLentity{#1}}
%
% \defineXMLentity [pound] {(why not use euro's?)}
% 
% \startXMLdata
% test &amp;pound; test
% \stopXMLdata
%
% so we need an ifless implementation of:

\gdef\doXMLentity#1;%
  {\ifXMLrawentities
     \expandafter\firstofoneargument
   \else
     \expandafter\executeXMLentity
   \fi{#1}}

\def\executeXMLentity#1% internal ! ! ! 
  {\getXMLentity{#1}}

\def\expandedXMLentity#1% 
  {\getvalue{\@@XMLentity:#1}}

\unexpanded\def\getXMLentity#1% 
  {\getvalue{\@@XMLentity:#1}}

\gdef\doifXMLentityelse#1#2#3%
  {\ifundefined{\@@XMLentity:#1}#3\else#2\fi}

% \long\def\getXMLarguments#1#2%
%   {\dogetXMLarguments{#1}#2>}
%
% \long\def\dogetXMLarguments#1%
%   {\XMLtoks\emptytoks
%    \def\@@XMLclass{#1}%
%    \let\dodoparseXMLarguments\doparseXMLarguments
%    \doparseXMLarguments}
% 
% \def\dosetXMLargument#1%
%   {\setvalue{\@@XMLvariable:\@@XMLclass:\@@XMLname}{#1}%
%   %\message{[\@@XMLname=#1]}%
%    \let\dodoparseXMLarguments\doparseXMLarguments
%    \dodoparseXMLarguments}

% see \defineXML... commands: 
%
% [key=val]          => \presetXMLarguments{element} => default key/vals
% [blabla]           => \theXMLarguments{blabla}     => user key/vals
% [blabla] [key=val] => \presetXMLarguments{element} => default key/vals
%                       \theXMLarguments{blabla}     => user key/vals
%
% <element key="val"> stored in case of [blabla] else set as \XMLpar
%
% see m-steps for an example of usage 

\long\def\getXMLarguments#1#2%
  {\dogetXMLarguments{#1}#2>}

\long\def\dogetXMLarguments#1%
  {\XMLtoks\emptytoks
   \ifcsname\@@XMLmap:#1\endcsname
     \let\dosetXMLargument\dosetXMLargumentB
   \else
     \def\@@XMLclass{#1}%
     \let\dosetXMLargument\dosetXMLargumentA
   \fi
   \let\dodoparseXMLarguments\doparseXMLarguments
   \doparseXMLarguments}

\def\dosetXMLargumentA#1%
  {\setvalue{\@@XMLvariable:\@@XMLclass:\@@XMLname}{#1}%
   \let\dodoparseXMLarguments\doparseXMLarguments
  %\message{[\@@XMLclass][\@@XMLname=#1]}\wait
   \dodoparseXMLarguments}

\def\dosetXMLargumentB#1%
  {\setevalue{\@@XMLmap:\@@XMLmapmap}%
     {\@EA\ifx\csname\@@XMLmap:\@@XMLmapmap\endcsname\empty\else
        \csname\@@XMLmap:\@@XMLmapmap\endcsname,%
      \fi 
      \@@XMLname=#1}%
   \let\dodoparseXMLarguments\doparseXMLarguments
  %\message{[\@@XMLprefix][\@@XMLname=#1]}\wait
   \dodoparseXMLarguments}

\appendtoks
  \resetXMLarguments\currentXMLelement
\to \everyXMLelement

\def\resetXMLarguments#1%
  {\ifcsname\@@XMLmap:#1\endcsname
    \@EA\let\@EA\@@XMLmapmap\csname\@@XMLmap:#1\endcsname
    \@EA\let\csname\@@XMLmap:\@@XMLmapmap\endcsname\empty
  \fi}

\def\theXMLarguments#1%
  {\ifcsname\@@XMLmap:#1\endcsname\csname\@@XMLmap:#1\endcsname\fi}

\long\def\doparseXMLarguments#1% space goes ok
  {\if#1>%
     \let\dodoparseXMLarguments\empty
   \else\if#1=%
     \edef\@@XMLname{\the\XMLtoks}%
     \XMLtoks\emptytoks
   \else\if#1"%
     \let\dodoparseXMLarguments\dodoparseXMLargumentsD
   \else\if#1'%
     \let\dodoparseXMLarguments\dodoparseXMLargumentsS
   \else\if#1/%
     \edef\currentXMLelement{\currentXMLelement/}%
   \else
     \@EA\XMLtoks\@EA{\the\XMLtoks#1}%
   \fi\fi\fi\fi\fi
   \dodoparseXMLarguments}

\def\dodoparseXMLargumentsD#1"{\dosetXMLargument{#1}}
\def\dodoparseXMLargumentsS#1'{\dosetXMLargument{#1}}

%D The previous macros were the basic parser and their working
%D is left to the imagination of the reader. These macros
%D will be improved.

\bgroup

\catcode`<=\@@active

\long\gdef\dododefineXMLsingular#1#2%
  {\long\setvalue{\@@XMLelement:#1/}{#2}}

\long\gdef\dododefineXMLcommand#1#2%
  {\long\setvalue{\@@XMLelement:#1/}{#2}%
   \long\setvalue{\@@XMLelement:#1}{#2}}

\long\gdef\dododefineXMLgrouped#1#2%
  {\long\setvalue{\@@XMLelement:#1}{\groupedcommand{#2}{}\bgroup}%
   \long\setvalue{\@@XMLelement:/#1}{\egroup}}

\long\gdef\dododefineXMLargument#1#2% watch the {} around ##1
  {\long\setvalue{\@@XMLelement:#1/}{#2{}}%
   \long\setvalue{\@@XMLelement:#1}##1</#1>{#2{##1}}}

\long\gdef\dododefineXMLignore#1%
  {\long\setvalue{\@@XMLelement:#1/}{}%
   \long\setvalue{\@@XMLelement:#1}##1</#1>{}}

\long\gdef\dododefineXMLpickup#1#2#3%
  {\long\setvalue{\@@XMLelement:#1/}{#2#3}%
   \long\setvalue{\@@XMLelement:#1}##1</#1>{#2##1#3}}

\long\gdef\dododefineXMLenvironment#1#2#3%
  {\long\setvalue{\@@XMLelement:#1/}{#2#3}% % genereert evt relax
   \long\setvalue{\@@XMLelement:#1}{#2}%
   \long\setvalue{\@@XMLelement:/#1}{#3}}

\long\gdef\dododefineXMLpush#1%
  {\long\setvalue{\@@XMLelement:#1/}{\long\setvalue{\@@XMLdata:#1}{}}%
   \long\setvalue{\@@XMLelement:#1}##1</#1>{\long\setvalue{\@@XMLdata:#1}{##1}}}

\long\gdef\dododefineXMLenvironmentpush#1#2#3%
  {\long\setvalue{\@@XMLelement:#1/}{#2\long\setvalue{\@@XMLdata:#1}{}#3}%
   \long\setvalue{\@@XMLelement:#1}##1</#1>{#2\long\setvalue{\@@XMLdata:#1}{##1}#3}}

\long\gdef\dododefineXMLprocess#1%
  {\long\setvalue{\@@XMLelement:#1/}{}%
   \long\setvalue{\@@XMLelement:#1}{}%
   \long\setvalue{\@@XMLelement:/#1}{}}

\long\gdef\dododefineXMLnestedenvironment#1#2#3%
  {\long\setvalue{\@@XMLelement:#1}{\getXMLgroupedenvironment{#1}{#2}{#3}}}

\long\gdef\dododefineXMLnestedargument#1#2%
  {\long\setvalue{\@@XMLelement:#1}{\getXMLgroupedargument{#1}{#2}}}

\egroup

%D The high level definition macros.

\def\defineXMLsingular       {\dotripleempty\dodefineXMLsingular}
\def\defineXMLcommand        {\dotripleempty\dodefineXMLcommand}
\def\defineXMLgrouped        {\dotripleempty\dodefineXMLgrouped}
\def\defineXMLargument       {\dotripleempty\dodefineXMLargument}
\def\defineXMLignore         {\dotripleempty\dodefineXMLignore}
\def\defineXMLpickup         {\dotripleempty\dodefineXMLpickup}
\def\defineXMLenvironment    {\dotripleempty\dodefineXMLenvironment}
\def\defineXMLpush           {\dotripleempty\dodefineXMLpush}
\def\defineXMLenvironmentpush{\dotripleempty\dodefineXMLenvironmentpush}
\def\defineXMLprocess        {\dotripleempty\dodefineXMLprocess}

% goes for all types 

\def\defineXMLnested           {\dotripleempty\dodefineXMLnestedenvironment}
\def\defineXMLnestedenvironment{\dotripleempty\dodefineXMLnestedenvironment}
\def\defineXMLnestedargument   {\dotripleempty\dodefineXMLnestedargument}

\long\def\dodefineXMLsingular[#1][#2][#3]#4%
  {\defineXMLmethod\dododefineXMLsingular{#1}{#2}{#3}{#4}{}}

\long\def\dodefineXMLcommand[#1][#2][#3]#4%
  {\defineXMLmethod\dododefineXMLcommand{#1}{#2}{#3}{#4}{}}

\long\def\dodefineXMLgrouped[#1][#2][#3]#4%
  {\defineXMLmethod\dododefineXMLgrouped{#1}{#2}{#3}{#4}{}}

\long\def\dodefineXMLargument[#1][#2][#3]#4%
  {\defineXMLmethod\dododefineXMLargument{#1}{#2}{#3}{#4}{}}

\long\def\dodefineXMLignore[#1][#2][#3]%
  {\defineXMLmethod\dododefineXMLignore{#1}{#2}{#3}{}{}}

\long\def\dodefineXMLpickup[#1][#2][#3]#4#5%
  {\defineXMLmethod\dododefineXMLpickup{#1}{#2}{#3}{#4}{#5}}

\long\def\dodefineXMLenvironment[#1][#2][#3]#4#5%
  {\defineXMLmethod\dododefineXMLenvironment{#1}{#2}{#3}{#4}{#5}}

\long\def\dodefineXMLpush[#1][#2][#3]%
  {\defineXMLmethod\dododefineXMLpush{#1}{#2}{#3}{}{}}

\long\def\dodefineXMLenvironmentpush[#1][#2][#3]#4#5%
  {\defineXMLmethod\dododefineXMLenvironmentpush{#1}{#2}{#3}{#4}{#5}}

\long\def\dodefineXMLprocess[#1][#2][#3]%
  {\defineXMLmethod\dododefineXMLprocess{#1}{#2}{#3}{}{}}

\long\def\dodefineXMLnestedenvironment[#1][#2][#3]#4#5%
  {\defineXMLmethod\dododefineXMLnestedenvironment{#1}{#2}{#3}{#4}{#5}}

\long\def\dodefineXMLnestedargument[#1][#2][#3]#4%
  {\defineXMLmethod\dododefineXMLnestedargument{#1}{#2}{#3}{#4}{}}

% [key=val]          => \presetXMLarguments{element} => default key/vals
% [blabla]           => \theXMLarguments{blabla}     => user key/vals
% [blabla] [key=val] => \presetXMLarguments{element} => default key/vals
%                       \theXMLarguments{blabla}     => user key/vals

\long\def\defineXMLmethod#1#2#3#4#5#6% command element [map] [parlst] begin end
  {\ifsecondargument
     \setXMLarguments{#2}{#3}{#4}% 
   \fi
   \ifignoreXMLcase
     \lowercasestring#2\to\ascii \@EA#1\@EA{\ascii}{#5}{#6}%
     \uppercasestring#2\to\ascii \@EA#1\@EA{\ascii}{#5}{#6}%
   \else
     #1{#2}{#5}{#6}%
   \fi}

\def\setXMLarguments#1#2#3% element [tag] settings
  {\doifassignmentelse{#2}
     {\setvalue{\@@XMLpars:#1}{\getrawparameters[\@@XMLvariable:#1:][#2]}}
     {\setvalue{\@@XMLmap :#1}{#2}% later we can init vars by this name 
      \doifsomething{#3}{\setvalue{\@@XMLpars:#1}{\getrawparameters[#2][#3]}}}} 

\def\presetXMLarguments#1%
  {\getvalue{\@@XMLpars:#1}} 

\prependtoks
  \presetXMLarguments\currentXMLelement
\to \everyXMLelement

\def\doifXMLdataelse#1#2#3% % \relax too, so no etex
% wrong 
% {\expandafter\ifx\csname\@@XMLdata:#1\endcsname\relax
% slow 
% {\ifundefined{\@@XMLdata:#1}% 
% etex
  {\unless\ifcsname\@@XMLdata:#1\endcsname
     #3%
   \else\expandafter\ifx\csname\@@XMLdata:#1\endcsname\empty
     #3%
   \else\expandafter\ifx\csname\@@XMLdata:#1\endcsname\relax
     #3%
   \else
     #2%
   \fi\fi\fi}

\def\XMLpop#1% one level 
% wrong 
% {\expandafter\ifx\csname\@@XMLdata:#1\endcsname\relax\else
%    \csname\@@XMLdata:#1\endcsname
%  \fi}
% slow, hm was not commented  
% {\ifundefined{\@@XMLdata:#1}\else\getvalue{\@@XMLdata:#1}\fi}
% etex 
  {\ifcsname\@@XMLdata:#1\endcsname\csname\@@XMLdata:#1\endcsname\fi}

\def\XMLpopdata#1% see m-steps for usage 
  {\unless\ifcsname\@@XMLdata:#1\endcsname
   \else\expandafter\ifx\csname\@@XMLdata:#1\endcsname\empty
   \else\expandafter\ifx\csname\@@XMLdata:#1\endcsname\relax
   \else
     \@EA\@EA\@EA\XMLdata\@EA\@EA\@EA{\csname\@@XMLdata:#1\endcsname}%
   \fi\fi\fi}

\def\XMLappend#1#2%
  {\edef\!!stringa{\@@XMLdata:#1}%
   \doifXMLdataelse{#1}%
     {\@EA\@EA\@EA\setvalue\@EA\@EA\@EA\!!stringa\@EA\@EA\@EA
        {\csname\!!stringa\endcsname#2}}
     {\setvalue\!!stringa{#2}}}

\def\XMLprepend#1#2%
  {\edef\!!stringa{\@@XMLdata:#1}%
   \doifXMLdataelse{#1}%
     {\@EA\@EA\@EA\setvalue\@EA\@EA\@EA\!!stringa\@EA\@EA\@EA
        {#2\csname\!!stringa\endcsname}}
     {\setvalue\!!stringa{#2}}}

\def\XMLerase#1%
  {\letvalue{\@@XMLdata:#1}\empty}

\def\XMLassign#1%
  {\setvalue{\@@XMLdata:#1}}

\def\defXMLstring#1#2%
%  {\@EA\convertcommand\csname\@@XMLdata:#2\endcsname\to#1}
  {\bgroup
   \let\getXMLentity\firstofoneargument
   \xdef\@@XML@@string{\csname\@@XMLdata:#2\endcsname}%
   \egroup                                                
   \@EA\convertcommand\@@XML@@string\to#1}
   
\def\XMLshow#1%
  {\showvalue{\@@XMLdata:#1\endcsname}}

\def\XMLunspace#1%
  {\ifcsname\@@XMLdata:#1\endcsname
     \setevalue{\@@XMLdata:#1}%
       {\@EA\@EA\@EA\dounspaced\csname\@@XMLdata:#1\endcsname\end}%
   \fi}

\def\defXMLlowerclean#1% lowercase ! evt tzt upper too
  {\bgroup
   \lccode`\#=32\lccode`\$=32\lccode`\%=32\lccode`\\=32\lccode`\^=32
   \lccode`\_=32\lccode`\{=32\lccode`\}=32\lccode`\|=32\lccode`\~=32
   \@EA\lowercase\@EA{\@EA\xdef\@EA#1\@EA{#1}}%
   \egroup}

\def\processXMLparelse#1#2#3#4%
  {\processaction
     [\XMLpar{#1}{#2}{}]
     [#3,\s!unknown=>{#4},\s!default={#4}]}

%D We can pick up key|/|value pairs, but we still need a way
%D to process these.

\def\mapXMLvalue#1#2#3% td align center -> middle
  {\setvalue{\@@XMLvalue:#1:#2:#3}}

% \def\XMLvar#1#2#3% td align center
%   {\ifundefined{\@@XMLvariable:#1:#2}%
%      \XMLval{#1}{#2}{#3}%
%    \else
%      \XMLval{#1}{#2}{\getvalue{\@@XMLvariable:#1:#2}}%
%    \fi}
% 
% \def\XMLval#1#2#3%
%   {\ifundefined{\@@XMLvalue:#1:#2}%
%      #3%
%    \else
%      \getvalue{\@@XMLvalue:#1:#2}%
%    \fi}
% 
% \def\XMLpar#1#2#3%
%   {\ifundefined{\@@XMLvariable:#1:#2}%
%      #3%
%    \else
%      \getvalue{\@@XMLvariable:#1:#2}%
%    \fi}
%
% speedup 

\def\XMLvar#1#2#3% td align center
  {\ifcsname\@@XMLvariable:#1:#2\endcsname
     \XMLval{#1}{#2}{\csname\@@XMLvariable:#1:#2\endcsname}%
   \else
     \XMLval{#1}{#2}{#3}% evt inline code 
   \fi}

\def\XMLval#1#2#3%
  {\ifcsname\@@XMLvalue:#1:#2\endcsname
     \csname\@@XMLvalue:#1:#2\endcsname
   \else
     #3%
   \fi}

\def\XMLpar#1#2#3%
  {\ifcsname\@@XMLvariable:#1:#2\endcsname
     \csname\@@XMLvariable:#1:#2\endcsname
   \else
     #3%
   \fi}

% so far for speedup 

\defineXMLsingular [begingroup] {\begingroup}
\defineXMLsingular [endgroup]   {\endgroup}

\def\XMLstr#1%
  {{\enableXML\scantokens{#1}\unskip}}

\def\XMLstr#1% test 
  {\scantokens{\begingroup\enableXML#1<endgroup/>}}

%\def\XMLstrpar#1#2#3%
%  {{\enableXML
%    \ifundefined{\@@XMLvariable:#1:#2}%
%      \scantokens{#3}%
%    \else
%      \scantokens\@EA\@EA\@EA
%        {\csname\@@XMLvariable:#1:#2\endcsname}\unskip
%    \fi}}

\def\XMLstrpar#1#2#3% test 
  {\ifundefined{\@@XMLvariable:#1:#2}%
     \scantokens{\begingroup\enableXML#3<endgroup/>}%
   \else
     \scantokens\@EA\@EA\@EA{\@EA\begingroup\@EA\enableXML
       \csname\@@XMLvariable:#1:#2\endcsname<endgroup/>}%
   \fi}

\def\doifXMLvarelse#1#2#3#4% geen etex, \relax too
  {\expandafter\ifx\csname\@@XMLvariable:#1:#2\endcsname\relax#4\else
   \expandafter\ifx\csname\@@XMLvariable:#1:#2\endcsname\empty#4\else#3\fi\fi}

\def\doifXMLvalelse#1#2#3#4% geen etex, \relax too
  {\expandafter\ifx\csname\@@XMLvalue:#1:#2\endcsname\relax#4\else
   \expandafter\ifx\csname\@@XMLvalue:#1:#2\endcsname\empty#4\else#3\fi\fi}

\let\doifXMLparelse\doifXMLvarelse

\def\dogotoXML%
  {\ifx\nexttoken<%
     \expandafter\nexttoken
   \else
     \expandafter\gotoXML
   \fi}

\def\gotoXML%
  {\afterassignment\dogotoXML\let\nexttoken=}

%D Saves tokens and typing. 

\def\XMLownvar        {\XMLvar        \currentXMLelement}
\def\XMLownval        {\XMLval        \currentXMLelement}
\def\XMLownpar        {\XMLpar        \currentXMLelement}
\def\XMLownstrpar     {\XMLstrpar     \currentXMLelement}
\def\doifXMLownvarelse{\doifXMLvarelse\currentXMLelement}
\def\doifXMLownvalelse{\doifXMLvalelse\currentXMLelement}
\def\doifXMLownparelse{\doifXMLparelse\currentXMLelement}

%D

\long\def\startXMLcode[#1] #2 \stopXMLcode
  {\setgvalue{\@@XMLcode:#1}{\startXMLdata#2\stopXMLdata}}

\def\getXMLcode[#1]% \expandXMLcode
  {\getvalue{\@@XMLcode:#1}}

% \long\def\startXMLdata#1\stopXMLdata%
%   {\begingroup\enableXML\scantokens{#1}\endgroup}
%
% \defineXMLentity[tex-backslash] {\catchXMLpar} 
%
% \def\catchXMLpar#1#2#3
%   {\if#1p\if#2a\if#3r\ifmmode\else\endgraf\fi
%    \else\texescape\fi\else\texescape\fi\else\texescape\fi}

\long\def\startXMLdata
  {\begingroup
   \catcode`\^^I=\@@space
   \catcode`\^^M=\@@space
   \catcode`\^^L=\@@space
   \dostartXMLdata}
 
% \long\def\dostartXMLdata#1\stopXMLdata
%   {\enableXML\scantokens{#1}\endgroup}

\long\def\dostartXMLdata#1\stopXMLdata
  {\enableXML\scantokens{#1}\ifhmode\unskip\unskip\fi\endgroup}

\unexpanded\def\XMLdata#1% % \unexpanded added 22/5/2001
  {\begingroup
   \enableXML\scantokens{#1}\ifhmode\unskip\unskip\fi
   \endgroup}

\unexpanded\def\XMLdata#1% % grouping changed 20/5/2001
  {\scantokens{\begingroup\enableXML#1<endgroup/>}}
  
%D

\def\bXMLs{\ifignoreXMLspaces\ignorespaces\fi}
\def\eXMLs{\ifignoreXMLspaces\ifhmode\unskip\fi\fi}

\protect

% \defineXMLcommand{placeindex/}
%   {\placeindex[criterium=all]}
%
% \defineXMLargument{index}
%   {\index[\XMLvar{index}{key}{}]}

%D Here we implement the handling of preformatted code.

\unprotect

\def\startXMLpreformatted#1%
  {\startpacked
   #1%
   \fixedXMLfonttrue
   \obeylines
   \obeyspaces
   \setbox\scratchbox=\hbox{x}%
   \edef\obeyedspace{\noindent\noexpand\kern\the\wd\scratchbox}}

\def\stopXMLpreformatted#1%
  {\stoppacked}

%D

\def\XMLinput{\enableXML\input} \global\let\inputXML\XMLinput

% options

\def\processXMLfile       #1{\enableXML\processfile{#1}}
\def\processXMLfilegrouped#1{{\enableXML\processfile{#1}\relax\ifmmode\else\par\fi}}

% partially defined here

\fetchruntimecommand\showXMLfile  {\f!xtagprefix\s!run}
\fetchruntimecommand\showXMLbuffer{\f!xtagprefix\s!run}

\fetchruntimecommand\showXMLtxt   {\f!xtagprefix\s!run}
\fetchruntimecommand\showXMLpar   {\f!xtagprefix\s!run}
\fetchruntimecommand\showXMLlin   {\f!xtagprefix\s!run}
\fetchruntimecommand\showXMLwrd   {\f!xtagprefix\s!run}
\fetchruntimecommand\showXMLemp   {\f!xtagprefix\s!run}

%D \type
%D   {processXMLbuffer}
%D
%D For illustrative purposes, we need to be able to reuse
%D definitions, which is why we implement a buffer processor
%D here. The macro \type {\processXMLbuffer} behaves like
%D any buffer processor.

\def\processXMLbuffer%
  {\dosingleempty\doprocessXMLbuffer}

\def\doprocessXMLbuffer[#1]%
  {\doifelsenothing{#1}
     {\doprocessXMLbuffer[\jobname]}
     {\begingroup
      \def\dodoprocessXMLbuffer##1%
        {\enableXML\processXMLfile{\TEXbufferfile{##1}}}%
      \processcommalist[#1]\dodoprocessXMLbuffer
      \endgroup}}

%D Loading specific modules takes place with \type
%D {\useXMLfilters}.

\def\useXMLfilter[#1]%
  {\processcommalist[#1]\douseXMLfilter}

\def\douseXMLfilter#1%
  {\doifundefined{\c!file\f!xtagprefix#1}
     {\setvalue{\c!file\f!xtagprefix#1}{}%
      \makeshortfilename[\f!xtagprefix#1]%
      \writestatus{xml}{loading module #1}% will be \showmessage
      \startreadingfile
      \readsysfile{\shortfilename}{}{}%
      \stopreadingfile}}

%D Temporarily here.

\newtoks\groupedtoks

\bgroup

\catcode`\<=\@@active

\long\unexpanded\gdef\getXMLgrouped#1#2#3%
  {\groupedtoks\emptytoks
   \convertargument<#1>\to\xxascii
   \convertargument<#1 \to\yyascii
   \newcounter\groupedlevel
   \long\def\dogetgrouped##1</#1>%
     {\appendtoks##1\to\groupedtoks
      \convertargument##1\to\ascii
      \doloop
        {\@EA\@EA\@EA\aftersplitstring\@EA\ascii\@EA\at\xxascii\to\ascii
         \ifx\ascii\empty
           \exitloop
         \else
           \increment\groupedlevel
         \fi}%
      \convertargument##1\to\ascii
      \doloop
        {\@EA\@EA\@EA\aftersplitstring\@EA\ascii\@EA\at\yyascii\to\ascii
         \ifx\ascii\empty
           \exitloop
         \else
           \increment\groupedlevel
         \fi}%
      \ifnum\groupedlevel>0
        \decrement\groupedlevel
        \appendtoks</#1>\to\groupedtoks
      \else
        \edef\dogetgrouped{\noexpand#2\the\groupedtoks\noexpand#3}%
      \fi
      \dogetgrouped}%
   \dogetgrouped}

%D Cleaner but hardly faster unless big strings are passed.

\long\gdef\docountXMLgrouped#1\end#2\end
  {\long\def\dosplitstring##1#2##2@@##3\end%
     {\def\ascii{##2}%
      \ifx\ascii\empty \else
        \advance\scratchcounter 1
        \dosplitstring##2@@#2@@\end
      \fi}%
   \dosplitstring#1@@#2@@\end}

\long\unexpanded\gdef\getXMLgrouped#1#2#3%
  {\groupedtoks\emptytoks
   \scratchcounter=0
   \long\def\dogetgrouped##1</#1>%
     {\appendtoks##1\to\groupedtoks
      \docountXMLgrouped##1\end<#1>\end
      \docountXMLgrouped##1\end<#1 \end
      \ifcase\scratchcounter
        \def\dogetgrouped{\@EA#2\the\groupedtoks#3}%
      \else
        \advance\scratchcounter -1
        \appendtoks</#1>\to\groupedtoks
      \fi
      \dogetgrouped}%
   \dogetgrouped}

%D More versatile.

\long\unexpanded\gdef\getXMLgroupedenvironment#1#2#3%
  {\def\dodogetgrouped{\@EA#2\the\groupedtoks#3}%
   \getXMLgrouped{#1}}

\long\unexpanded\gdef\getXMLgroupedargument#1#2%
  {\def\dodogetgrouped{\@EA#2\@EA{\the\groupedtoks}}%
   \getXMLgrouped{#1}}

\long\unexpanded\gdef\getXMLgrouped#1%
  {\groupedtoks\emptytoks
   \scratchcounter=0
   \long\def\dogetgrouped##1</#1>%
     {\appendtoks##1\to\groupedtoks
      \docountXMLgrouped##1\end<#1>\end
      \docountXMLgrouped##1\end<#1 \end
      \ifcase\scratchcounter
        \let\dogetgrouped\dodogetgrouped
      \else
        \advance\scratchcounter -1
        \appendtoks</#1>\to\groupedtoks
      \fi
      \dogetgrouped}%
   \dogetgrouped}

\egroup

% {pre}{pos}{before}{after}
%
%\unexpanded\def\getgrouped#1#2#3#4%
%  {\groupedtoks\emptytoks
%   \convertargument#1\to\xxascii
%   \newcounter\groupedlevel
%   \def\dogetgrouped##1#2%
%     {\appendtoks##1\to\groupedtoks
%      \convertargument##1\to\ascii
%      \doloop
%        {\@EA\@EA\@EA\aftersplitstring\@EA\ascii\@EA\at\xxascii\to\ascii
%         \ifx\ascii\empty
%           \exitloop
%         \else
%           \increment\groupedlevel
%         \fi}%
%      \ifnum\groupedlevel>0
%        \decrement\groupedlevel
%        \appendtoks#2\to\groupedtoks
%      \else
%        \edef\dogetgrouped{\noexpand#3\the\groupedtoks\noexpand#4}%
%      \fi
%      \dogetgrouped}%
%   \dogetgrouped}

% interesting and fully expandable

\def\XMLifequalelse#1#2#3#4#5%
  {\ifundefined{\@@XMLvariable:#1:#2}%
     #5%
   \else
     \@EA\@EA\@EA\@@ifequal\csname\@@XMLvariable:#1:#2\endcsname
     \relax\@@and#3\relax\@@then#4\@@else#5\@@fi
   \fi}

\def\expifequalelse#1#2#3#4%
  {\@@ifequal#1\relax\relax\@@and#2\relax\relax\@@then#3\@@else#4\@@fi}

\def\@@ifequal#1#2\@@and#3#4\@@then#5\@@else#6\@@fi%
  {\ifx#1\relax
     \ifx#3\relax#5\else#6\fi
   \else
     \ifx#3\relax#6\else\@@ifequal#2\@@and#4\@@then#5\@@else#6\@@fi\fi
   \fi}

\protect \endinput
