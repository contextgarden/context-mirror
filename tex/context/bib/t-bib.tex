%D \module
%D   [       file=t-bib,
%D        version=2005.01.04,
%D          title=\CONTEXT\ Publication Module,
%D       subtitle=Publications,
%D         author=Taco Hoekwater,
%D           date=\currentdate,
%D      copyright=Public Domain]
%C
%C Donated to the public domain.

\usemodule[bibltx] % bibtex files can contain latex left-overs

%D Slightly adapted by HH (2005/01). For \BIBTEX\ relate questions,
%D contact Taco, if you expect interference with core macros, bother
%D Hans. This file will be cleaned up in due time.
%D
%D Documentation and additional resources can be found at
%D Taco's website: \url{tex.aanhet.net}.

%D \subject{DONE}
%D
%D \startitemize
%D \item minor bugfixes today (25/6)
%D \item add finalnamesep support for Oxford comma (17/9)
%D \item add \insert... for: doi, eprint, howpublished (17-19/9)
%D \item minor fix (26/5/2004):
%D \item add author definition (and associated system variable)
%D \stopitemize

%D \subject{TODO}
%D
%D \startitemize
%D \item export \type {\citation{<cited item>}}
%D \item export a \type {\bibalternative{lastpubsep}} from BST instead of 'and'
%D \item don't reset [numbercommand] in \type {\setuppublication} by default
%D \item sort out different APS versions: PR A/B/L vs. RPM
%D \item (implement sub-options? / \type {\setupcitationstyle} ?)
%D \item worry about localization at same time
%D \item add hyperlinking to the doi->URL and \type {\cite}->list, in general
%D \item don't load the whole lot, but filter entries instead
%D \stopitemize

\unprotect

%D A few new shortcuts:

\definesystemvariable  {pv}  % PublicationVariable
\definesystemvariable  {pb}  % PuBlication
\definemessageconstant {bib}
\definefileconstant    {bibextension} {bbl}

%D Some user information messages.

\startmessages all library: bib
    title: publications
    1: file -- not found, unknown style ignored
    2: file -- not found, don't forget to run bibtex
    3: wrote a new auxiliary file \jobname.aux
    4: loading database from --
    5: warning: cite argument -- on \the\inputlineno
    6: loading formatting style from --
\stopmessages

%D Some constants for the multi-lingual interface

\startconstants        dutch                english

             database: database             database
             sorttype: sorttype             sorttype
             compress: compress             compress
             autohang: autohang             autohang
       %       author: author               author
            artauthor: artauthor            artauthor
               editor: editor               editor
      authoretallimit: authoretallimit      authoretallimit
   artauthoretallimit: artauthoretallimit   artauthoretallimit
      editoretallimit: editoretallimit      editoretallimit
    authoretaldisplay: authoretaldisplay    authoretaldisplay
 artauthoretaldisplay: artauthoretaldisplay artauthoretaldisplay
    editoretaldisplay: editoretaldisplay    editoretaldisplay
       authoretaltext: authoretaltext       authoretaltext
    artauthoretaltext: artauthoretaltext    artauthoretaltext
       editoretaltext: editoretaltext       editoretaltext
          totalnumber: totalnumber          totalnumber
         firstnamesep: firstnamesep         firstnamesep
               vonsep: vonsep               vonsep
            juniorsep: juniorsep            juniorsep
           surnamesep: surnamesep           surnamesep
          lastnamesep: lastnamesep          lastnamesep
         finalnamesep: finalnamesep         finalnamesep
              namesep: namesep              namesep
               pubsep: pubsep               pubsep
           lastpubsep: lastpubsep           lastpubsep
           refcommand: refcommand           refcommand
           samplesize: samplesize           samplesize

\stopconstants

\startvariables       dutch                 english

                      german                czech
                      italian               romanian
               title: titel                 title
                      titel                 titul
                      titolo                titlu
               short: kort                  short
                      kurz                  short
                      short                 short
                cite: cite                  cite
                      cite                  cite
                      cite                  cite
                 bbl: bbl                   bbl
                      bbl                   bbl
                      bbl                   bbl
                 bib: bib                   bib
                      bib                   bib
                      bib                   bib
              author: auteur                author
                      autor                 autor
                      autore                autor

\stopvariables

%D The text string for the publication list header

\setupheadtext[en][pubs=References]
\setupheadtext[nl][pubs=Literatuur]
\setupheadtext[de][pubs=Literatur]

%D \macros{bibdoifelse}
%D
%D Here is a really small helper that is used a lot (which is why it
%D makes sense to use \type{\ifx} instead of \type{\doif..}).
%D This test is used in all the typesetting commands
%D (\type{\insert...}) we will encounter later.

\def\bibdoifelse#1%
  {\ifx#1\relax
     \@EA\secondoftwoarguments
   \else\ifx#1\empty
     \@EAEAEA\secondoftwoarguments
   \else
     \@EAEAEA\firstoftwoarguments
   \fi\fi}

\def\bibdoif#1%
  {\ifx#1\relax
     \@EA\gobbleoneargument
   \else\ifx#1\empty
     \@EAEAEA\gobbleoneargument
   \else
     \@EAEAEA\firstofoneargument
   \fi\fi}

\def\bibdoifnot#1%
  {\ifx#1\relax
     \@EA\firstofoneargument
   \else\ifx#1\empty
     \@EAEAEA\firstofoneargument
   \else
     \@EAEAEA\gobbleoneargument
   \fi\fi}

%D Bibtex settings separated out

%D No point in writing the aux file if there is no database...

\def\setupbibtex{\dosingleempty\dosetupbibtex}

\def\dosetupbibtex[#1]%
  {\let\@@pbdatabase\empty
   \let\@@pbsort    \empty
   \getparameters[\??pb][#1]
   \expanded{\processaction[\@@pbsort]}
        [      \v!no=>\def\bibstyle{cont-no},
           \v!author=>\def\bibstyle{cont-au},
            \v!title=>\def\bibstyle{cont-ti},
            \v!short=>\def\bibstyle{cont-ab},
          \s!default=>\def\bibstyle{cont-no},
          \s!unknown=>\def\bibstyle{cont-no}]%
   \ifx\@@pbdatabase\empty\else \writeauxfile \fi}

%D \macros{writeauxfile}
%D
%D Unfortunately, \BIBTEX\ is not the best configurable program
%D around. The names of the commands it parses as well as the \type{.aux}
%D extension to the file name are both hardwired.
%D
%D This means \CONTEXT\ has to write a \LATEX-style auxiliary file, yuk!
%D The good news is that it can be rather short. We'll just ask
%D \BIBTEX\ to output the entire database(s) into the \type{bbl} file.
%D
%D The \type{\bibstyle} command controls how the \type{bbl} file will
%D be sorted. The possibilities are:
%D
%D \startitemize[packed]
%D \item by author (+year, title): cont-au.bst
%D \item by title  (+author, year): cont-ti.bst
%D \item by short key as in abbrev.bst: cont-ab.bst
%D \item not sorted at all: cont-no.bst
%D \stopitemize

\def\writeauxfile
  {\openout \scratchwrite \jobname.aux
   \write   \scratchwrite {\string\citation{*}}%
   \write   \scratchwrite {\string\bibstyle{\bibstyle}}%
   \write   \scratchwrite {\string\bibdata{\@@pbdatabase}}%
   \closeout\scratchwrite
   \showmessage\m!bib{3}{}}

%D \macros{ifsortbycite,iftypesetall,ifautohang,ifbibcitecompress}
%D
%D The module needs some new \type{\if} statements.

%D Default sort order of the reference list is by citation.

\newif\ifsortbycite        \sortbycitetrue

%D By default, only referenced publications are typeset

\newif\iftypesetall        \typesetallfalse

%D Hanging indentation of the publication list
%D will not adjust itself according to the width of the label.

\newif\ifautohang          \autohangfalse

%D Cite lists are compressed, if possible.

\newif\ifbibcitecompress   \bibcitecompresstrue

\def\setuppublications
  {\dosingleargument\dosetuppublications}

\def\bibleftnumber#1%
  {#1\hfill~}

\def\dosetuppublications[#1]%
  {\getparameters
     [\??pb]
     [\c!numbercommand=\bibleftnumber,\c!alternative=,#1]%
   \doifsomething\@@pbalternative
     {\readsysfile{bibl-\@@pbalternative.tex}
        {\showmessage\m!bib{6}{bibl-\@@pbalternative}\let\@@pbalternative\empty}
        {\showmessage\m!bib{1}{bibl-\@@pbalternative}\let\@@pbalternative\empty}}%
   \doifnothing{\@@pbalternative}
     {\processaction
        [\@@pbcriterium]
        [    \v!all=>\typesetalltrue,
         \s!unknown=>\typesetallfalse]%
      \processaction
        [\@@pbautohang]
        [    \v!yes=>\autohangtrue,
         \s!unknown=>\autohangfalse]%
      \processaction
        [\@@pbsorttype]
        [   \v!cite=>\sortbycitetrue,
             \v!bbl=>\sortbycitefalse,
         \s!default=>\sortbycitetrue,
         \s!unknown=>\sortbycitefalse]%
      \processaction
        [\@@pbnumbering]
        [    \v!yes=>\let\@@pbinumbercommand\firstofoneargument,
              \v!no=>\let\@@pbinumbercommand\gobbleoneargument,
           \v!short=>\def\@@pbinumbercommand##1{\@@pbs},
             \v!bib=>\def\@@pbinumbercommand##1{\@@pbn},
         \s!unknown=>\let\@@pbinumbercommand\firstofoneargument]%
      \processaction
        [\@@pbrefcommand]
        [\s!default=>\edef\@@citedefault{\@@pbrefcommand},
         \s!unknown=>\edef\@@citedefault{\@@pbrefcommand}]}}

% initialize

\def\@@pbrefcommand{num}

%D how to load the references:

\appendtoks \dousepublications\jobname \to \everystarttext

%D \macros{usepublications}
%D
%D We need \type{\usereferences} so that it is possible to
%D refer to page and/or appearance number for publications
%D in the other document.

\def\usepublications[#1]%
  {\usereferences[#1]\processcommalist[#1]\dousepublications}

\def\dousepublications#1%
  {\readsysfile{#1.\f!bibextension}
     {\showmessage\m!bib{4}{#1.\f!bibextension}}
     {\showmessage\m!bib{2}{#1.\f!bibextension}}}

%D \macros{setuppublicationlist}
%D
%D This will be the first command in (\BIBTEX-generated) \type{bbl}
%D files. `samplesize' is a sample value (in case of \BIBTEX-generated
%D files, this will be the longest `short' key). `totalnumber'
%D is the total number of entries that will follow in this
%D file.

%D Both values are only needed for the label calculation
%D if `autohang' is `true', so by default the command is
%D not even needed, and therefore I saw no need to give
%D it it's own system variable and it just re-uses \type{pb}.

\def\setuppublicationlist
  {\dosingleempty\dosetuppublicationlist}

\definelist[pubs][pubs]

\def\dosetuppublicationlist[#1]%
  {\getparameters
     [@@pvdata]
     [\c!samplesize={AA99},\c!totalnumber={99},#1]% for sample & totalnumber & firstnamesep etc.
   \setuplist
     [pubs]
     [\c!alternative=a,\c!pagenumber=\v!no,#1]}

\def\setuppublicationlayout[#1]%
  {\setvalue{@@pvdata#1}}

%D \macros{bibalternative}
%D
%D A nice little shorthand that will be used so we don't have to
%D key in the weird \type{\@@pv} parameter names all the time.

\def\bibalternative#1%
  {\getvalue{\??pv\@@currentalternative#1}}

%D \macros{simplebibdef,bibcommandlist}
%D
%D \type{\simplebibdef} defines \type{bib@#1}, which in turn will
%D use one argument that is stored in \type{@@pb@#1}.
%D
%D \type{\simplebibdef} also defines \type{insert#1}, which can be
%D used in the argument of \type{\setuppublicationlayout} to fetch
%D one of the \type{@@pb@} data entries. \type{insert#1} then has
%D three arguments: \type{#1} are commands to be executed before the
%D data, \type{#2} are commands to be executed after the data, and
%D \type{#3} are commands to be executed if the data is not found.

%D \type{\bibcommandlist} is the list of commands that is affected
%D by this approach. Later on, it will be used to do a series
%D of assignments from \type{#1} to \type{bib@#1}: e.g
%D \type{\title} becomes \type{\bib@title} when used within
%D a publication.

\def\simplebibdef#1% hh: funny expansion ?
  {\@EA\long\@EA\def\csname bib@#1\endcsname##1%
     {\setvalue{\??pb @#1}{##1}\ignorespaces}%
      \@EA\def\csname insert#1\endcsname##1##2##3%
        {\@EA\bibdoifelse
           \@EA{\csname @@pb@#1\endcsname}%
           {##1\csname @@pb@#1\endcsname##2}%
           {##3}}}

\def\bibcommandlist
  {arttitle,title,journal,notes,
   volume,issue,pubname,city,country,
   type,organization,institute,series,thekey,
   edition,month,pubyear,note,annotate,pages,
   keyword,keywords,comment,abstract,names,size,
   issn,isbn,chapter,eprint,doi,howpublished}

\processcommacommand[\bibcommandlist]\simplebibdef

\def\newbibfield[#1]%
  {\simplebibdef{#1}%
   \edef\bibcommalist{\bibcommandlist,#1}}

%D \macros{bib@crossref}
%D
%D \type{\crossref} is used in database files to point to another
%D entry. Because of this special situation, it has to be defined
%D separately. Since this command will not be seen until at
%D \type{\placepublications}, it may force extra runs. The same is
%D true for \type{\cite} commands inside of publications.

\def\bib@crossref#1%
  {\setvalue{\??pb @crossref}{#1}\ignorespaces}

\def\insertcrossref#1#2#3%
  {\bibdoifelse{\@@pb@crossref}
     {#1\@EA\cite\@EA[\@@pb@crossref]#2}
     {#3}}

%D \macros{complexbibdef,specialbibinsert}
%D
%D The commands \type{\artauthor}, \type{\author} and
%D \type{\editor} are more complex than the other commands.
%D Their argument lists have this form:
%D
%D \type{\author[junior]{firstnames}[inits]{von}{surname}}
%D
%D (bracketed stuff should become optional someday)
%D
%D And not only that, but there also might be more than one of each of
%D these commands. This is why a special command is needed to insert
%D them, as well as one extra counter for each command.

%D All of these \type{\@EA}'s and \type{\csnames} make this code
%D look far more complex than it really is. For example, the argument
%D \type{author} defines the macro \type{\bib@author} to do two
%D things: increment the counter \type{\author@num} (let's say to 2)
%D and next store it's arguments in the macro \type{\@@pb@author2}.
%D And it defines \type{\insertauthors} to expand into
%D \starttypen
%D \specialbibinsert{author}{\author@num}{<before>}{<after>}{<not>}
%D \stoptypen

% hh: use a context counter instead, more options

\def\complexbibdef#1%
  {\@EA\newcounter\csname #1@num\endcsname
   \@EA\def\csname bib@#1\endcsname[##1]##2[##3]##4##5%
     {\@EA\increment\csname #1@num\endcsname
      \setvalue{\??pb @#1\csname #1@num\endcsname}%
        {{##2}{##4}{##5}{##3}{##1}}\ignorespaces}%
   \@EA\def\csname insert#1s\endcsname##1##2##3%
     {\specialbibinsert{#1}{\csname #1@num\endcsname}{##1}{##2}{##3}}}

\processcommalist[author,artauthor,editor]\complexbibdef

%D Another level of indirection is needed to control the
%D typesetting of all of these arguments, which explains the usage
%D of \type{\tempa} below.

%D There is some sneaky stuff with \type{\tempa} and \type{\tempb}
%D going on here to resolve the \type{\csname}'s. It probably could
%D be done a little bit more elegant, but it works. The basic goal
%D is to get the command that will actually typeset the name into
%D the macro \type{\tempb}, and to make sure that that command will actually
%D recieve five arguments (see the definition of
%D e.g. \type{\invertedauthor} below).

%D There is a conflict between `author' and the predefined interface
%D variable `auteur'. The old version is overruled `auteur' is
%D overruled by the systemconstant definition at the top of this file!

%D The increment/decrement trick on \type{\scratchcounter} is needed
%D to decide what name the last one is.

\newcount\etallimitcounter
\newcount\etaldisplaycounter
\newcount\todocounter

\def\specialbibinsert#1#2#3#4#5%
  {\bgroup
   \ifnum#2>\zerocount
     \letcscsname\tempa\csname @@pvdata#1\endcsname
     \def\tempb{\@EA\tempa}%
     \etallimitcounter  =0\bibalternative{#1etallimit}\relax
     \etaldisplaycounter=0\bibalternative{#1etaldisplay}\relax
     \ifnum #2>\etallimitcounter
       \todocounter\etaldisplaycounter
       % just in case ...
	   \ifnum\todocounter>\etallimitcounter
         \todocounter\etallimitcounter
       \fi
     \else
       \todocounter#2\relax
     \fi
     \scratchcounter\zerocount
     \ifnum\todocounter>\zerocount
       #3%
       \doloop
         {\ifnum \scratchcounter < \todocounter
            \advance\scratchcounter \plusone
            \ifnum \scratchcounter = \todocounter
              \@EA\tempb \csname @@pb@#1\the\scratchcounter\endcsname
              \ifnum\etallimitcounter<#2 \bibalternative{#1etaltext}\fi #4%
            \else
              \@EA\tempb \csname @@pb@#1\the\scratchcounter\endcsname
              \advance\scratchcounter \plusone
              \ifnum \scratchcounter = \todocounter
                 \ifnum \todocounter > \plustwo
                    \bibalternative\c!finalnamesep
                 \else
                    \bibalternative\c!lastnamesep
                 \fi
              \else
                 \bibalternative\c!namesep
              \fi
              \advance\scratchcounter \minusone
            \fi
          \else
            \exitloop
          \fi}%
      \else
        #5%
      \fi
   \else
     #5%
   \fi
   \egroup}

%D \macros{invertedauthor,normalauthor,invertedshortauthor,normalshortauthor}
%D
%D Just some commands that can be used in \type{\setuppublicationparameters}
%D If you want to write an extension to the styles, you might
%D as well define some of these commands yourself.
%D
%D The argument liust has been reordered here, and the meanings
%D are:
%D
%D {\obeylines\parskip0pt
%D \type{#1} firstnames
%D \type{#2} von
%D \type{#3} surname
%D \type{#4} inits
%D \type{#5} junior
%D }
%D

\def\invertedauthor#1#2#3#4#5%
  {\bibdoif{#2}{#2\bibalternative\c!vonsep}%
   #3\bibalternative\c!surnamesep
   \bibdoif{#5}{#5\bibalternative\c!juniorsep}%
   \bibdoif{#1}{#1\unskip}}

\def\normalauthor#1#2#3#4#5%
  {\bibdoif{#1}{#1\bibalternative\c!firstnamesep}%
   \bibdoif{#2}{#2\bibalternative\c!vonsep}%
   #3\bibalternative\c!surnamesep
   \bibdoif{#5}{#5\unskip}}

\def\invertedshortauthor#1#2#3#4#5%
  {\bibdoif{#2}{#2\bibalternative\c!vonsep}%
   #3\bibalternative\c!surnamesep
   \bibdoif{#5}{#5\bibalternative\c!juniorsep}%
   \bibdoif{#4}{#4\unskip}}

\def\normalshortauthor#1#2#3#4#5%
  {\bibdoif{#4}{#4\bibalternative\c!firstnamesep}%
   \bibdoif{#2}{#2\bibalternative\c!vonsep}%
   #3\bibalternative\c!surnamesep
   \bibdoif{#5}{#5\unskip}}

%D \macros{clearbibitem,clearbibitemtwo,bibitemdefs}
%D
%D These are used in \type{\typesetapublication} to do
%D initializations and cleanups.

\def\clearbibitem#1{\setvalue{\??pb @#1}{}}%

\def\clearbibitemtwo#1%
  {\letvalue{#1@num}\!!zerocount
   \scratchcounter\plusone
   \doloop
     {\doifdefinedelse{\??pb @#1\the\scratchcounter}
        {\letvalue{\??pb @#1\the\scratchcounter}\empty
         \advance\scratchcounter\plusone}%
        {\exitloop}}}

\def\bibitemdefs#1{\setvalue{#1}{\csname bib@#1\endcsname}}

%D \macros{startpublication}
%D
%D We are coming to the end of this module, to the macros that
%D do typesetting and read the \type{bbl} file.
%D
%D The stuff between \type{\startpublication} ... \type{\stoppublication}
%D is simply stored into a macro: either
%D \type{\publist} (potentially huge) or separate macros
%D for each of them, depending on whether
%D or not we do \type{\sortbycite}, as explained above.

\newcounter\bibcounter

%D Just a \type{\dosingleempty} is the most friendly
%D of doing this: there need not even be an argument
%D to \type{\startpublication}. Of course, then there
%D is no key either, and it had better be an
%D article (otherwise the layout will be all screwed up).

\def\startpublication{\dosingleempty\dostartpublication}
\def\stoppublication {}

%D Only specifying the key in the argument is also
%D legal. In storing this stuff into macros, some trickery with
%D token registers is needed to fix the expansion problems. Even so,
%D this appears to not always be 100\% safe, so people are
%D urgently advised to use \ETEX\ instead of traditional \TEX.
%D
%D In \ETEX, all expansion problems are conviniently solved by
%D the primitive \type{\protected}. To put that another way:
%D
%D It's not a bug in this module if it does not appear in \ETEX!

\long\def\dostartpublication[#1]#2\stoppublication%
  {\increment\bibcounter
   \bgroup
   \bgroup\honorunexpanded
   \doifassignmentelse{#1}%
     {\egroup\getparameters[\??pb][k=,t=article,n=,s=,a=,y=,o=,#1]}%
     {\egroup\getparameters[\??pb][k=#1,t=article,n=,s=,a=,y=,o=]}%
     \toks0={\ignorespaces #2}%
     \@EA\toks\@EA2\@EA{\@@pba}%
     \@EA\toks\@EA4\@EA{\@@pbs}%
     \setxvalue{pbd-\@@pbk}%
       {{\the\toks2}% \@@pba (author)
        {\@@pby}%
        {\the\toks4}% \@@pbs (short)
        {\@@pbn}%
        {\@@pbt}%
        {\the\toks0}% (data)
        {\@@pbo}}%    doi
    \xdef\allrefs{\allrefs,\@@pbk}%
    \egroup}

% intialization of the order-list:

\let\allrefs\empty

% how to get stuff from a pbd-*** macro:

\def\restorebibdata#1%
  {\def\@@pbk{#1}%
   \@EA\dorestorebibdata\csname pbd-#1\endcsname}

\def\dorestorebibdata#1%
  {\@EA\dodorestorebibdata#1{}{}{}{}{}{}{}\relax}

\def\dodorestorebibdata#1#2#3#4#5#6#7#8\relax%
  {\def\@@pba{#1}%
   \def\@@pby{#2}%
   \def\@@pbs{#3}%
   \def\@@pbn{#4}%
   \def\@@pbt{#5}%
   \def\@@pbd{#6}%
   \def\@@pbo{#7}}

%D The \writeutility trick is dodgy at best, but it is needed
%D to make sure that \placepublications\stoptext works as
%D advertised. (without the \immediate, there *has* to be at
%D least one page break between \placepublications and \stoptext)
%D

\def\preinitializepubslist
  {\let\bibcounter\!!zerocount
   \ifsortbycite
     \processcommacommand[\publist]\sortwritepublist
     \glet\publist\empty
     \iftypesetall
       \processcommacommand[\allrefs]\writepublist
     \fi
   \else
     \iftypesetall
       \processcommacommand[\allrefs]\writepublist
     \else
       \processcommacommand[\allrefs]\writereferredpublist
     \fi
   \fi}

\def\initializepubslist
  {\edef\@@pbnumbering{\@@pbnumbering}%
   \ifautohang
     \ifx\@@pbnumbering\v!short
       \setbox\scratchbox\hbox{\@@pbnumbercommand{\csname @@pvdata\c!samplesize\endcsname}}%
     \else\iftypesetall
       \setbox\scratchbox\hbox{\@@pbnumbercommand{\csname @@pvdata\c!totalnumber\endcsname}}%
     \else
       \setbox\scratchbox\hbox{\@@pbnumbercommand{\numreferred}}%
     \fi\fi
     \edef\samplewidth{\the\wd\scratchbox}%
     \setuplist[pubs][\c!width=\samplewidth,\c!distance=0pt]%
     \def\@@pblimitednumber##1{\hbox to \samplewidth{\@@pbnumbercommand{##1}}}%
   \else
     \def\@@pblimitednumber##1{\hbox{\@@pbnumbercommand{##1}}}%
   \fi
   \ifx\@@pbnumbering\v!no
     \setuplist[pubs][\c!numbercommand=,\c!textcommand=\outdented]
   \else
     \setuplist[pubs][\c!numbercommand=\@@pblimitednumber,\c!textcommand=]
   \fi
   \forgetall % bugfix 2005/03/18
}

\def\outdented#1% move to supp-box ?
  {\hskip -\hangindent
   \strut#1\strut}

%D The full list of publications

\def\completepublications
  {\dosingleempty\docompletepublications}

\def\docompletepublications[#1]%
  {\preinitializepubslist
   \ifcase\bibcounter\else % HERE it said "\or" instead of "\else"
     \initializepubslist
     \let\bibcounter\!!zerocount
     \completelist[pubs][\c!criterium=all,#1]%
   \fi}

%D And the portion with the entries only.

\def\placepublications
  {\dosingleempty\doplacepublications}

\def\doplacepublications[#1]%
  {\preinitializepubslist
   \ifcase\bibcounter\else % HERE it said "\or" instead of "\else"
     \initializepubslist
     \let\bibcounter\!!zerocount
     \placelist[pubs][\c!criterium=\v!all,#1]%
   \fi}

\def\dowritebiblist#1#2%
  {\restorebibdata{#2}%
   \edef\pbnumbercommand{\@@pbinumbercommand{#1}}%
   \@EA\dodowritebiblist\@EA{\pbnumbercommand}{\typesetapublication{#2}}}

\def\dodowritebiblist
  {\writetolist[pubs]}

\def\writepublist#1%
  {\doifnotempty{#1}
     {\increment\bibcounter
      \@EA\dowritebiblist\@EA{\bibcounter}{#1}}}

\def\writereferredpublist#1%
  {\doifnotempty{#1}
    {\doifreferredelse{#1}
       {\increment\bibcounter
        \@EA\dowritebiblist\@EA{\bibcounter}{#1}}{}}}

\def\sortwritepublist#1%
  {\doifnotempty{#1}
     {\removefromcommalist{#1}\allrefs
      \increment\bibcounter
      \@EA\dowritebiblist\@EA{\bibcounter}{#1}}}

%D \subonderwerp{What's in a publication}
%D

\def\typesetapublication
  {\doglobal\increment\bibcounter
   \dotypesetapublication}

\def\dotypesetapublication#1%
  {\bgroup
   \def\@@currentalternative{data}%
   \restorebibdata{#1}%
   \processcommacommand[\bibcommandlist,crossref]\clearbibitem
   \processcommalist   [artauthor,author,editor]\clearbibitemtwo
   \processcommacommand[\bibcommandlist]\bibitemdefs
   \processcommalist   [artauthor,author,editor,crossref]\bibitemdefs
   \expanded{\reference[\@@pbk]{\bibcounter}}%
   \@@pbd % execute data
   \bibalternative{\@@pbt}% do typesetting
   \egroup}

%D An afterthought

\def\maybeyear#1{}

%D \onderwerp{Citations}

%D \macros{cite,bibref}
%D
%D The indirection with \type{\dobibref} allows \LATEX\ style
%D \type{\cite} commands with a braced argument (these might appear
%D in included data from the \type{.bib} file).

\def\cite
  {\doifnextcharelse{[}
     {\dodoubleempty\docite}
     {\dobibref}}

\def\dobibref#1%
  {\docite[#1]}

\def\docite#1[#2]#3[#4]%
  {\ifsecondargument
     \def\@@currentalternative{#2}%
     \expanded{\processaction[\csname @@pv#2compress\endcsname]}
       [    \v!yes=>\bibcitecompresstrue,
             \v!no=>\bibcitecompressfalse,
        \s!default=>\bibcitecompresstrue,
        \s!unknown=>\bibcitecompresstrue]%
     \getvalue{bib#2ref}[#4]%
   \else
     \expanded{\processaction[\csname @@pv\@@citedefault compress\endcsname]}
       [    \v!yes=>\bibcitecompresstrue,
             \v!no=>\bibcitecompressfalse,
        \s!default=>\bibcitecompresstrue,
        \s!unknown=>\bibcitecompresstrue]%
     \edef\@@currentalternative{\@@citedefault}%
     \getvalue{bib\@@citedefault ref}[#2]%
   \fi}

\def\setupcite#1[#2]#3[#4]%
  {\def\getciteargs##1{\getparameters[\??pv##1][#4]}%
   \processcommalist[#2]\dosetupcite}

\def\dosetupcite#1%
  {\getciteargs{#1}}

%D \macros{numreferred,doifreferredelse,addthisref,publist}
%D
%D The interesting command here is \type{\addthisref}, which maintains
%D the global list of references.
%D
%D \type{\numreferred} is needed to do automatic calculations on
%D the label width, and \type{\doifreferredelse} will be used
%D to implement \type{criterium=cite}.

\newcounter\numreferred

\long\def\doifreferredelse#1{\doifdefinedelse{pbr-#1}}

\def\addthisref#1%
  {\doifundefined{pbr-#1}
     {\setgvalue{pbr-#1}{a}%
      \doglobal\increment\numreferred
      \appended\gdef\publist{,#1}}}

\let\publist\empty

%D \macros{doifbibreferencefoundelse}
%D
%D Some macros to fetch the information provided by
%D \type{\startpublication}.

\def\doifbibreferencefoundelse#1#2#3%
  {\doifdefinedelse{pbd-#1}
     {\restorebibdata{#1}#2}
     {\showmessage\m!bib{5}{#1 is unknown}#3}}

%D \macros{ixbibauthoryear,thebibauthors,thebibyears}
%D
%D If compression of \type{\cite}'s argument expansion is on,
%D the macros that deal with authors and years call this internal
%D command to do the actual typesetting.
%D
%D Two entries with same author but with different years may
%D be condensed into ``Author (year1,year2)''. This is about the
%D only optimization that makes sense for the (author,year)
%D style of citations (years within one author have to be unique
%D anyway so no need to test for that, and ``Author1, Author2 (year)''
%D creates more confusion than it does good).
%D
%D In the code below,
%D the macro \type{\thebibauthors} holds the names of the alternative
%D author info fields for the current list. This is a commalist,
%D and \type{\thebibyears} holds the (collection of) year(s) that go with
%D this author (possibly as a nested commalist).
%D
%D There had better be an author for all cases, but there
%D does not have to be year info always. \type{\thebibyears} is
%D pre-initialized because this makes the insertion macros simpler.
%D
%D In `normal' \TeX, of course there are expansion problems again.

\def\gobble#1{\def#1##1{##1}}

\def\sanitizeaccents{\processcommalist[\',\`,\",\.,\c,\d,\~,\=]\gobble}

\def\ixbibauthoryear#1#2#3#4%
  {\bgroup
   \sanitizeaccents
   \gdef\ixlastcommand  {#4}%
   \gdef\ixsecondcommand{#3}%
   \gdef\ixfirstcommand {#2}%
   \glet\thebibauthors  \empty
   \glet\thebibyears    \empty
   \glet\theauthorssize \empty
   \getcommalistsize[#1]%
   \ifbibcitecompress
     \dorecurse\commalistsize{\xdef\thebibyears{\thebibyears,}}%
     \processcommalist[#1]\docompressbibauthoryear
   \else
     \processcommalist[#1]\donormalbibauthoryear
   \fi
   \getcommacommandsize[\thebibauthors]%
   \xdef\theauthorssize{\commalistsize}%
   \egroup
   \dobibauthoryear}

%D \macros{dodobibauthoryear}
%D
%D This macro only has to make sure that the lists
%D \type{\thebibauthors} and \type{\thebibyears} are printed.

\def\dobibauthoryear
  {\scratchcounter\zerocount
   \getcommacommandsize[\thebibauthors]%
   \@EA\processcommalist\@EA[\thebibauthors]\dodobibauthoryear}

\def\dodobibauthoryear#1%
  {\advance\scratchcounter\plusone
   \edef\wantednumber{\the\scratchcounter}%
   \getfromcommacommand[\thebibyears][\wantednumber]%
   \def\AU{#1}% brr
   \@EA\def\@EA\YR\@EA{\commalistelement}%
   \ifnum\scratchcounter=\plusone
     \ixfirstcommand
   \else\ifnum \scratchcounter=\commalistsize\relax
     \ixlastcommand
   \else
     \ixsecondcommand
   \fi\fi}

%D \macros{donormalbibauthoryear}
%D
%D Now we get to the macros that fill the two lists.
%D The `simple' one really is quite simple.

\def\donormalbibauthoryear#1%
  {\addthisref{#1}%
   \xdef\myauthor{Xxxxxxxxxx}%
   \xdef\myyear{0000}%
   \doifbibreferencefoundelse{#1}
     {\@EA\gdef\@EA\myauthor\@EA{\@@pba}%
      \@EA\gdef\@EA\myyear  \@EA{\@@pby}}
     {}%
   \@EA\doglobal\@EA\addtocommalist\@EA{\myauthor}\thebibauthors
   \@EA\doglobal\@EA\addtocommalist\@EA{\myyear  }\thebibyears}

%D \macros{docompressbibauthoryear}
%D
%D So much for the easy parts. Nothing at all will be done if
%D the reference is not found or the reference does not contain
%D author data. No questions marks o.s.s. (to be fixed later)

\def\docompressbibauthoryear#1%
  {\addthisref{#1}%
   \xdef\myauthor{Xxxxxxxxxx}%
   \xdef\myyear  {0000}%
   \doifbibreferencefoundelse{#1}
     {\@EA\gdef\@EA\myauthor\@EA{\@@pba}%
      \@EA\gdef\@EA\myyear  \@EA{\@@pby}}
     {}%
    \ifx\myauthor\empty\else
      \checkifmyauthoralreadyexists
      \findmatchingyear
    \fi}

%D two temporary counters. One of these two can possibly be replaced
%D by \type{\scratchcounter}.

\newcount\bibitemcounter
\newcount\bibitemwanted

%D The first portion is simple enough: if this is the very first author
%D it is quite straightforward to add it. \type{\bibitemcounter} and
%D \type{\bibitemwanted} are needed later to insert the year
%D information in the correct item of \type{\thebibyears}

\def\checkifmyauthoralreadyexists
  {\doifemptyelsevalue{thebibauthors}
     {\global\bibitemcounter\plusone
      \global\bibitemwanted \plusone
      \@EA\gdef\@EA\thebibauthors\@EA{\myauthor}} % hh: one level
     {\getcommacommandsize[\thebibauthors]%
      \global\bibitemwanted\zerocount
      \global\bibitemcounter\commalistsize
      \processcommacommand[\thebibauthors]\docomparemyauthor}}

%D The outer \type{\ifnum} accomplishes the addition of
%D a new author to \type{\thebibauthors}. The messing about with
%D the two counters is again to make sure that \type{\thebibyears}
%D will be updated correctly.If the author {\it was} found,
%D the counters will stay at their present values and everything
%D will be setup properly to insert the year info.

\def\docomparemyauthor#1%
  {\global\advance\bibitemwanted \plusone
   \def\mytempc{#1}%
   \ifnum\bibitemwanted = \commalistsize\relax
     \ifx\mytempc\myauthor \else
       \global\advance\bibitemwanted \plusone
       \global\bibitemcounter\bibitemwanted\relax
       \@EA\doglobal\@EA\addtocommalist\@EA{\myauthor}\thebibauthors
       \quitcommalist
     \fi
   \else
     \ifx\mytempc\myauthor
       \quitcommalist
     \fi
   \fi}

%D This macro should be clear now.

\def\findmatchingyear
  {\edef\wantednumber{\the\bibitemwanted}%
   \getfromcommacommand[\thebibyears][\wantednumber]%
   \ifx\commalistelement\empty
     \edef\myyear{{\myyear}}%
   \else
     \edef\myyear{{\commalistelement, \myyear}}%
   \fi
   \edef\newcommalistelement{\myyear}%
   \doglobal\replaceincommalist \thebibyears \wantednumber}

%D \macros{bibauthoryearref,bibauthoryearsref,bibauthorref,bibyearref}
%D
%D Now that all the hard work has been done, these are simple.
%D \type{\ixbibauthoryearref} stores the data in the macros
%D \type{\AU} and \type{\YR}.

\def\bibauthoryearref[#1]%
 {\ixbibauthoryear{#1}%
   {{\AU}\bibalternative\c!inbetween
    \bibalternative\v!left{\YR}\bibalternative\v!right}
   {\bibalternative\c!pubsep{\AU}\bibalternative\c!inbetween
    \bibalternative\v!left  {\YR}\bibalternative\v!right}
   {\bibalternative\c!lastpubsep{\AU}\bibalternative\c!inbetween
    \bibalternative\v!left      {\YR}\bibalternative\v!right}}

\def\bibauthoryearsref[#1]%
  {\bibalternative\v!left
   \ixbibauthoryear{#1}
     {{\AU}\bibalternative\c!inbetween{\YR}}
     {\bibalternative\c!pubsep    {\AU}\bibalternative\c!inbetween{\YR}}
     {\bibalternative\c!lastpubsep{\AU}\bibalternative\c!inbetween{\YR}}%
   \bibalternative\v!right}

\def\bibauthorref[#1]%
  {\bibalternative\v!left
   \ixbibauthoryear{#1}%
    {{\AU}}
    {\bibalternative\c!pubsep    {\AU}}
    {\bibalternative\c!lastpubsep{\AU}}%
   \bibalternative\v!right}

\def\bibyearref[#1]%
  {\bibalternative\v!left
   \ixbibauthoryear{#1}%
      {{\YR}}
      {\bibalternative\c!pubsep    {\YR}}
      {\bibalternative\c!lastpubsep{\YR}}%
   \bibalternative\v!right}

%D ML problems:

%D \macros{bibshortref,bibkeyref,bibpageref,bibtyperef,bibnumberref}
%D
%D There is hardly any point in trying to compress these. The only
%D thing that needs to be done is making sure that
%D the separations are inserted correctly. And that is
%D what \type{\refsep} does.

\newif\iffirstref

\def\refsep{\iffirstref\firstreffalse\else\bibalternative\c!pubsep\fi}

\def\bibshortref[#1]%
  {\bibalternative\v!left
   \firstreftrue\processcommalist[#1]\dobibshort
   \bibalternative\v!right}

\def\dobibshort#1%
  {\addthisref{#1}\refsep
   \doifbibreferencefoundelse{#1}{\@@pbs}{??}}


\def\bibnumberref[#1]%
  {\bibalternative\v!left
   \firstreftrue\processcommalist[#1]\dobibnumbered
   \bibalternative\v!right}

\def\dobibnumbered#1%
  {\addthisref{#1}\refsep
   \doifbibreferencefoundelse{#1}{\@@pbn}{??}}

\def\bibkeyref[#1]%
  {\bibalternative\v!left
   \firstreftrue\processcommalist[#1]\dobibkeyref
   \bibalternative\v!right}

\def\dobibkeyref#1%
  {\addthisref{#1}\refsep#1}

\def\bibdoiref[#1]%
  {\bibalternative\v!left
   \firstreftrue\processcommalist[#1]\dobibdoiref
   \bibalternative\v!right}

\def\dobibdoiref#1%
  {\addthisref{#1}\refsep#1}

\def\bibtyperef[#1]%
  {\bibalternative\v!left
   \firstreftrue\processcommalist[#1]\dobibtyperef
   \bibalternative\v!right}

\def\dobibtyperef#1%
  {\addthisref{#1}\refsep
   \doifbibreferencefoundelse{#1}{\@@pbt}{??}}

\def\bibpageref[#1]%
  {\bibalternative\v!left
   \firstreftrue\processcommalist[#1]\dobibpageref
   \bibalternative\v!right}

\def\dobibpageref#1%
  {\addthisref{#1}\refsep\at[#1]}

\def\bibdataref[#1]%
  {\bibalternative\v!left
   \firstreftrue\processcommalist[#1]\dobibdata
   \bibalternative\v!right}

\def\dobibdata#1%
  {\addthisref{#1}\refsep
   \doifbibreferencefoundelse{#1}{\dotypesetapublication{#1}}{??}}

%D \macros{bibnumref}
%D
%D It makes sense to try and compress the argument list of
%D \type{\bibnumref}. There are two things involved: the actual
%D compression, and a sort routine. The idea is to store the
%D found values in a new commalist called \type{\therefs}.

%D But that is not too straight-forward, because \type{\in} is
%D not expandable,
%D so that the macro \type{\expandrefs} is needed.

\def\expandrefs#1%
  {\doifreferencefoundelse{#1}
     {\@EA\doglobal\@EA\addtocommalist\@EA{\reftypet}\therefs }
     {\showmessage\m!bib{5}{#1 unknown}%
      \doglobal\addtocommalist{0}\therefs}}

%D But at least the actual sorting code is simple (note that sorting
%D a list with exactly one entry fails to return anything, which
%D is why the \type{\ifx} is needed).

\ifx\compresscommacommandnrs\undefined
  \usemodule[list]
  \let\compresscommacommandnrs\compresscommacommand
\fi

\def\bibnumref[#1]%
  {\bibalternative\v!left
   \penalty\!!tenthousand
   \processcommalist[#1]\addthisref
   \firstreftrue
   \ifbibcitecompress
     \glet\therefs\empty
     \processcommalist[#1]\expandrefs
     \sortcommacommand[\therefs]\donumericcompare
     \ifx\empty\sortedcommalist\else
       \let\therefs\sortedcommalist
     \fi
     \compresscommacommandnrs[\therefs]%
   % \message{\meaning\therefs, \meaning\compressedlist}%
     \processcommacommand[\compressedlist]\verysimplebibnumref
   \else
     \processcommalist[#1]\dosimplebibnumref
   \fi
   \bibalternative\v!right}

%D Here is the simple case first:

\def\dosimplebibnumref  #1{\refsep\in[#1]}
\def\verysimplebibnumref#1{\doverysimplebibnumref#1}

\def\doverysimplebibnumref#1#2%
  {\refsep
   \ifcase#1\relax ??\else
     \def\tempa{#2}\ifx\empty\tempa#1\else#1\bibalternative\c!inbetween#2\fi
   \fi}

%D And some defaults are loaded from bibl-apa:

% hh: shouldn't those bibl files be made international ?

\setuppublications
  [\c!alternative=apa]

%D \completepublications

\protect \endinput
